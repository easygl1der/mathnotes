\subsection{Trace and Norm}

\begin{definition}[trace, norm]
If $L / K$ is a finite Galois extension with Galois group $G$, the characteristic polynomial of $\alpha \in L$ is $\chi_{\alpha, L / K}(X)=\prod_{\sigma \in G}(X-\sigma(\alpha))$. In particular,
\[
\operatorname{Tr}_{L / K}(\alpha)=\sum_{\sigma \in G} \sigma(\alpha), \quad \mathrm{N}_{L / K}(\alpha)=\prod_{\sigma \in G} \sigma(\alpha) .
\]
\end{definition}
\subsubsection{Example 1}

If $K=\mathbf{Q}$ and $L=\mathbf{Q}(\sqrt{d})$ for a nonsquare $d$ in $\mathbf{Q}^{\times}$, the two elements of $\operatorname{Gal}(L / K)$ are determined by $\sigma_1(\sqrt{d})=\sqrt{d}$ and $\sigma_2(\sqrt{d})=-\sqrt{d}$, so
\[
\operatorname{Tr}_{\mathbf{Q}(\sqrt{d}) / \mathbf{Q}}(a+b \sqrt{d})=(a+b \sqrt{d})+(a-b \sqrt{d})=2 a
\]
and
\[
\mathbf{N}_{\mathbf{Q}(\sqrt{d}) / \mathbf{Q}}(a+b \sqrt{d})=(a+b \sqrt{d})(a-b \sqrt{d})=a^2-d b^2 .
\]
Also
\[
\chi_{a+b \sqrt{d}, \mathbf{Q}(\sqrt{d}) / \mathbf{Q}}(X)=(X-(a+b \sqrt{d}))(X-(a-b \sqrt{d}))=X^2-2 a X-\left(a^2-d b^2\right) .
\]
\subsubsection{Example 2}

For $\alpha \in \mathbf{F}_{p^n}$,
\[
\operatorname{Tr}_{\mathbf{F}_{p^n} / \mathbf{F}_p}(\alpha)=\alpha+\alpha^p+\cdots+\alpha^{p^{n-1}} \text { and } \mathrm{N}_{\mathbf{F}_{p^n} / \mathbf{F}_p}(\alpha)=\alpha \alpha^p \cdots \alpha^{p^{n-1}} .
\]
\paragraph{Galois Group of \texorpdfstring{$\mathbf{F}_{p^n} / \mathbf{F}_p$}{mathbfF_p^n / mathbfF_p}}

The Galois group of $\mathbf{F}_{p^n} / \mathbf{F}_p$ is a cyclic group of order $n$, \textbf{generated by the Frobenius automorphism $\sigma$}, where $\sigma(x) = x^p$.
\[
\operatorname{Gal}(\mathbf{F}_{p^n} / \mathbf{F}_p) \cong \mathbb{Z} / n \mathbb{Z}
\]
\subsection{Relations among Galois Groups}

\begin{theorem}[Theorem 6.1]
Let $L_1$ and $L_2$ be \underline{Galois} over $K$.
a) The embedding
\[
\operatorname{Gal}\left(L_1 L_2 / K\right) \hookrightarrow \operatorname{Gal}\left(L_1 / K\right) \times \operatorname{Gal}\left(L_2 / K\right)
\]given by $\sigma \mapsto\left(\left.\sigma\right|_{L_1},\left.\sigma\right|_{L_2}\right)$ is an isomorphism \textbf{if and only if} $L_1 \cap L_2=K$. In particular, $\left[L_1 L_2: K\right]=\left[L_1: K\right]\left[L_2: K\right]$ if and only if $L_1 \cap L_2=K$.
b) The image of the embedding in part a is the set of compatible pairs of automorphisms: $\left\{\left(\tau_1, \tau_2\right) \in \operatorname{Gal}\left(L_1 / K\right) \times \operatorname{Gal}\left(L_2 / K\right): \tau_1=\tau_2\right.$ on $\left.L_1 \cap L_2\right\}$.\label{426887}
\end{theorem}

\begin{proof}
We omit the proof.
\end{proof}

\begin{example}
Let $L_1=\mathbf{Q}(\sqrt{2}, \sqrt{3})$ and $L_2=\mathbf{Q}(\sqrt[4]{2}, i)$. Both are Galois over $\mathbf{Q}$, we know their Galois groups, and $L_1 \cap L_2=\mathbf{Q}(\sqrt{2})$. Define $\tau_1 \in \operatorname{Gal}\left(L_1 / \mathbf{Q}\right)$ and $\tau_2 \in \operatorname{Gal}\left(L_2 / \mathbf{Q}\right)$ by the conditions
\[
\tau_1(\sqrt{2})=-\sqrt{2}, \quad \tau_1(\sqrt{3})=\sqrt{3}, \quad \tau_2(\sqrt[4]{2})=i \sqrt[4]{2}, \quad \tau_2(i)=i
\]These agree on $\sqrt{2}$ since $\tau_2(\sqrt{2})=\tau_2(\sqrt[4]{2})^2=-\sqrt{2}$, so there is a unique $\sigma \in \operatorname{Gal}\left(L_1 L_2 / \mathbf{Q}\right)$ that restricts to $\tau_1$ on $L_1$ and $\tau_2$ on $L_2$.
\end{example}
\begin{example}
If $K=\mathbf{Q}, L_1=\mathbf{Q}(\sqrt[3]{2})$, and $L_2=\mathbf{Q}(\omega \sqrt[3]{2})$, then $L_1 \cap L_2=K$ but $\left[L_1 L_2: K\right]=6 \neq\left[L_1: K\right]\left[L_2: K\right]$. See the field diagram below on the left.
\end{example}
If one of $L_1/K$ and $L_2/K$ is Galois, \cref{426887} still holds.

\begin{theorem}[Theorem 6.6]
Let $L / K$ and $F / K$ be finite extensions with $L / K$ a Galois extension.
	\begin{itemize}
		\item The extension $L F / F$ is finite Galois and $\operatorname{Gal}(L F / F) \cong \operatorname{Gal}(L / L \cap F)$ by restriction. In particular,
	\end{itemize}
\[
[L F: F]=[L: L \cap F] \quad \text { and } \quad[L F: K]=\frac{[L: K][F: K]}{[L \cap F: K]}
\]so $[L F: K]=[L: K][F: K]$ if and only if $L \cap F=K$.
	\begin{itemize}
		\item The sets of intermediate fields $\{M: F \subset M \subset L F\}$ and $\left\{M^{\prime}: L \cap F \subset M^{\prime} \subset L\right\}$ are in bijection by $M \mapsto L \cap M$, with inverse $M^{\prime} \mapsto M^{\prime} F$.
	\end{itemize}
In particular, every field between $F$ and $L F$ has the form $F(\alpha)$ where $\alpha \in L$, and if $M$ and $M^{\prime}$ correspond by the bijection then $M / F$ is Galois if and only if $M^{\prime} / L \cap F$ is Galois, in which case $\operatorname{Gal}(M / F) \cong \operatorname{Gal}\left(M^{\prime} / L \cap F\right)$ by restriction.\label{f158fe}
\end{theorem}

The extension $\mathbf{Q}(i, \sqrt[4]{2}) / \mathbf{Q}$ is Galois, with Galois group $D_4$. If we translate this extension by $\mathbf{Q}(i, \sqrt[3]{2})$ (which is not Galois over $\mathbf{Q}$ ), we get the Galois extension $\mathbf{Q}(i, \sqrt[3]{2}, \sqrt[4]{2}) / \mathbf{Q}(i, \sqrt[3]{2})$. What is its Galois group? By \cref{f158fe},
\[
\mathrm{Gal}(\mathbf{Q}(i,\sqrt[3]{ 2 },\sqrt[4]{ 2 })/\mathbf{Q}(i,\sqrt[3]{ 2 }))\cong \mathrm{Gal}(\mathbf{Q}(i,\sqrt[4]{ 2 })/F)
\]
where $F=\mathbf{Q}(i,\sqrt[4]{ 2 })\cap \mathbf{Q}(i,\sqrt[3]{ 2 })$. Obviously, $\mathbf{Q}(i)\subset F$, and since $[\mathbf{Q}(i,\sqrt[4]{ 2 }):\mathbf{Q}(i)]=4$, $[\mathbf{Q}(i,\sqrt[3]{ 2 }):\mathbf{Q}(i)]=3$, we have $F=\mathbf{Q}(i)$.

\subsection{Cyclotomic Extension}

\begin{definition}
For a positive integer $n \in \mathbb{N}$,
\[
\mu_n:=\{n \text {th roots of unity in } \mathbb{C}\}=\left\langle\zeta_n\right\rangle \cong \mathbf{Z}_n
\]where $\zeta_n=e^{2 \pi \mathbf{i} / n}$.
Define $\mathbb{Q}\left(\mu_n\right)=\mathbb{Q}\left(\zeta_n\right) \subseteq \mathbb{C}$; it is a finite field extension of $\mathbb{Q}$, called the \textbf{$n$th cyclotomic extension} of $\mathbb{Q}$.
A \textbf{primitive $n$th root of unity} is a generator of $\mu_n$; it is equal to $\zeta_n^a$ for some $a \in \mathbf{Z}_n^{\times}$. Define
\[
\Phi_n(x):=\prod_{a \in \mathbf{Z}_n^{\times}}\left(x-\zeta_n^a\right)
\]it is called the \textbf{$n$th cyclotomic polynomial}.
\end{definition}
\begin{example}
We have $\Phi_1(x)=x-1, \Phi_2(x)=x+1, \Phi_3(x)=x^2+x+1$.
\end{example}
\begin{lemma}
We have
\[
x^n-1=\prod_{d \mid n} \Phi_d(x)
\]Each $\Phi_n(x)$ is a polynomial of degree $\varphi(n)$ with coefficients in $\mathbb{Z}$.
\end{lemma}
\begin{proof}
The first equality is easy:
\begin{equation}
x^n-1=\prod_{b \in \mathbf{Z}_n}\left(x-\zeta_n^b\right)=\prod_{d \mid n} \prod_{i \in \mathbf{Z}_d^{\times}}\left(x-\zeta_n^{d i}\right)=\prod_{d \mid n} \Phi_d(x) .
\label{910936}
\end{equation}

We will prove that $\Phi_n(x)$ has coefficients in $\mathbb{Z}$ and its coefficients have gcd $=1$. Assume that this has been proved for smaller $n$. Then \cref{910936} and Gauss' lemma implies that $\Phi_n(x)$ has coefficients in $\mathbb{Z}$ and has coefficients' gcd $=1$.
\end{proof}

\subsection{Generating a Composite Field with a Sum}

\begin{theorem}[Theorem 7.1]
If $K$ has characteristic 0 and $K(\alpha, \beta) / K$ is a finite extension such that $K(\alpha) / K$ and $K(\beta) / K$ are both Galois and $K(\alpha) \cap K(\beta)=K$, then $K(\alpha, \beta)=K(\alpha+\beta)$.\label{561cb5}
\end{theorem}

\begin{proof}
Let $H=\operatorname{Gal}(K(\alpha, \beta) / K(\alpha+\beta))$. We will show this group is trivial.

Pick $\sigma \in H$, so $\sigma(\alpha+\beta)=\alpha+\beta$. Therefore
\[
\sigma(\alpha)-\alpha=\beta-\sigma(\beta)
\]
Since $K(\alpha)$ and $K(\beta)$ are Galois over $K, \sigma(\alpha) \in K(\alpha)$ and $\sigma(\beta) \in K(\beta)$, so $\sigma(\alpha)-\alpha \in K(\alpha)$ and $\beta-\sigma(\beta) \in K(\beta)$. This common difference is therefore in $K(\alpha) \cap K(\beta)=K$. Write $\sigma(\alpha)-\alpha=t$, so
\[
\sigma(\alpha)=\alpha+t, \quad \sigma(\beta)=\beta-t
\]
Applying $\sigma$ repeatedly, $\sigma^j(\alpha)=\alpha+j t$ for all integers $j$. Choose $j \geq 1$ such that $\sigma^j$ is the identity (for instance, let $j=[K(\alpha, \beta): K]$ ). Then $\alpha=\alpha+j t$, so $j t=0$. Since we are in characteristic 0 and $j$ is a positive integer, we must have $t=0$, so $\sigma(\alpha)=\alpha$ and $\sigma(\beta)=\beta$. Therefore $\sigma$ is the identity on $K(\alpha, \beta)$.
\end{proof}

\begin{theorem}[Theorem 7.6]
If $K$ has characteristic 0 and $K(\alpha, \beta) / K$ is a finite extension such that $[K(\alpha, \beta): K]=[K(\alpha): K][K(\beta): K]$ then $K(\alpha, \beta)=K(\alpha+\beta)$.\label{2e605f}
\end{theorem}

\begin{remark}
The degree hypothesis in \cref{2e605f} is equivalent to $K(\alpha)\cap K(\beta)=K$ when one of $K(\alpha)$ or $K(\beta)$ is Galois over $K$, so \cref{561cb5} is also true when only one of $K(\alpha)$ or $K(\beta)$ is Galois over $K$.
\end{remark}
\begin{example}
\cref{2e605f} implies $\mathbb{Q}(\sqrt[3]{2}, \omega)=\mathbb{Q}(\sqrt[3]{2}+\omega)$ and $\mathbb{Q}(\sqrt[4]{2}, i)=\mathbb{Q}(\sqrt[4]{2}+i)$.
\end{example}
\begin{example}
We know $\mathbb{Q}(\sqrt[3]{2}, \omega \sqrt[3]{2}) \neq \mathbb{Q}(\sqrt[3]{2}+\omega \sqrt[3]{2})$ and this example does not fit  \cref{2e605f} since $[\mathbb{Q}(\sqrt[3]{2}, \omega \sqrt[3]{2}): \mathbb{Q}]=6$ while $[\mathbb{Q}(\sqrt[3]{2}): \mathbb{Q}][\mathbb{Q}(\omega \sqrt[3]{2}): \mathbb{Q}]=9$.
\end{example}
\begin{example}
Letting $r$ and $r^{\prime}$ be two roots of $X^4+8 X+12$, we can't decide if $\mathbb{Q}\left(r, r^{\prime}\right)$ equals $\mathbb{Q}\left(r+r^{\prime}\right)$ from  \cref{2e605f} since $\left[\mathbb{Q}\left(r, r^{\prime}\right): \mathbb{Q}\right]=12$ and $[\mathbb{Q}(r): \mathbb{Q}]\left[\mathbb{Q}\left(r^{\prime}\right): \mathbb{Q}\right]=16$. The two fields are not the same since $\mathbb{Q}\left(r+r^{\prime}\right)$ has degree 6 over $\mathbb{Q}$.
\end{example}
\begin{example}
Does $\mathbb{Q}\left(\sqrt[4]{2}, \zeta_8\right)=\mathbb{Q}\left(\sqrt[4]{2}+\zeta_8\right)$ ? Since $\left[\mathbb{Q}\left(\sqrt[4]{2}, \zeta_8\right): \mathbb{Q}\right]=8$ (by Example 1.4) and $[\mathbb{Q}(\sqrt[4]{2}): \mathbb{Q}]\left[\mathbb{Q}\left(\zeta_8\right): \mathbb{Q}\right]=16$, we can't answer this with  \cref{2e605f}. Since $\mathbb{Q}\left(\sqrt[4]{2}, \zeta_8\right)=\mathbb{Q}(\sqrt[4]{2}, i)$, you can check the Galois orbit of $\sqrt[4]{2}+\zeta_8$ has size 8 , so in fact $\mathbb{Q}\left(\sqrt[4]{2}, \zeta_8\right)=\mathbb{Q}\left(\sqrt[4]{2}+\zeta_8\right)$. (The minimal polynomial of $\sqrt[4]{2}+\zeta_8$ over $\mathbb{Q}$ is $X^8-8 X^5-2 X^4+$ $16 X^3+32 X^2+24 X+9$.) Thus the degree hypothesis of  \cref{2e605f} is sufficient to imply $K(\alpha, \beta)=K(\alpha+\beta)$ in characteristic 0 , but it is not necessary.
\end{example}
\subsection{The Inverse Galois Problem}

\begin{theorem}[Theorem 8.1]
Every finite group is the Galois group of some finite Galois extension in any characteristic.
\end{theorem}
\begin{theorem}[Theorem 8.2]
If every finite group can be realized as a Galois group over $\mathbb{Q}$ then every finite group can be realized as a Galois group over any finite extension of $\mathbb{Q}$.
\end{theorem}
\subsection{What Next?}

There are two important aspects of field extensions that are missing by a study of Galois theory of finite extensions, and we briefly address them:

\begin{enumerate}
	\item Galois theory for infinite extensions
	\item transcendental extensions
\end{enumerate}

An example of an infinite Galois extension of $\mathbf{Q}$ is
\[
\mathbf{Q}(\mu_{p^{\infty}})=\bigcup_{n\geq 1}\mathbf{Q}(\mu_{p^{n}})
\]
the union of all $p$ -th power cyclotomic extensions of $\mathbf{Q}$, where $p$ is a fixed prime.
