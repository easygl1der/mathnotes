\section{Simplicity of \texorpdfstring{$A_n$}{A_n}}

\href{https://kconrad.math.uconn.edu/blurbs/grouptheory/Ansimple.pdf}{Ansimple.pdf}

Some examples of groups:

\begin{itemize}
	\item A finte group of prime order is simple.
	\item A finite abelian gorup $G$ not of prime order is not simple.
	\begin{itemize}
		\item Let prime $p\mid \lvert G \rvert$, by Cauchy's theorem, $G$ contains a subgroup of order $p$, which is abelian\footnote{As $G$ is abelian} thus normal in $G$.
	\end{itemize}
	\item When $n\geq3$, $S_n$ is not simple since $A_n\lhd S_n$ with index 2.
\end{itemize}

A meaningful result is as follows.

\begin{theorem}
For $n\geq5$, $A_n$ is simple.
\end{theorem}
\begin{proof}
See \href{https://kconrad.math.uconn.edu/blurbs/grouptheory/Ansimple.pdf}{Ansimple.pdf}, where 5 proofs are listed.
\end{proof}

Next we show 3 useful lemmas and a theorem.

\begin{lemma}
For $n \geq 3, A_n$ is generated by 3-cycles. For $n \geq 5, A_n$ is generated by permutations of type $(2,2)$.
\end{lemma}
The 3-cyclics in $S_n$ are all conjugate in $S_n$, since permutations of cycle type in $S_n$ are conjugate. Are 3-cycles conjugate in $A_n$?

\begin{lemma}
For $n \geq 5$, all 3-cycles in $A_n$ are conjugate in $A_n$.
\end{lemma}
\begin{lemma}
For $n \geq 5$, the only nontrivial proper normal subgroup of $S_n$ is $A_n$. In particular, the only subgroup of $S_n$ with index 2 is $A_n$.\label{c45dad}
\end{lemma}

\begin{corollary}
By \cref{c45dad} , for $n \geq 5$ each homomorphic image of $S_n$ not isomorphic to $S_n$ has order 1 or 2. So there is no surjective homomorphism $S_n \rightarrow \mathbb{Z} /(m)$ for $m \geq 3$.
\end{corollary}
\subsection{The index of subgroup of \texorpdfstring{$S_n$}{S_n}}

\begin{theorem}
For $n \geq 5$, no subgroup of $S_n$ has index strictly between 2 and $n$. Moreover, each subgroup of index $n$ in $S_n$ is isomorphic to $S_{n-1}$.\label{93da7c}
\end{theorem}

\begin{proof}
Let $H$ be a proper subgroup of $S_n$ and let $m:=\left[S_n: H\right]$, so $m \geq 2$. If $m=2$ then $H=A_n$ by Lemma 2.4. If $m<n$ then we will show $m=2$. The left multiplication action of $S_n$ on $S_n / H$ gives a group homomorphism
\[
\varphi: S_n \rightarrow \operatorname{Sym}\left(S_n / H\right) \cong S_m
\]
By hypothesis $m<n$, so $\varphi$ is not injective. Let $K$ be the kernel of $\varphi$, so $K \subset H$ and $K$ is nontrivial. Since $K \triangleleft S_n$, Lemma 2.4 says $K=A_n$ or $S_n$. Since $K \subset H$, we get $H=A_n$ or $S_n$, which implies $m=2$. Therefore we can't have $2<m<n$.

Now let $H$ be a subgroup of $S_n$ with index $n$. Consider the left multiplication action of $S_n$ on $S_n / H$. This is a homomorphism $\ell: S_n \rightarrow \operatorname{Sym}\left(S_n / H\right)$. Since $S_n / H$ has order $n$, $\operatorname{Sym}\left(S_n / H\right)$ is isomorphic to $S_n$. The kernel of $\ell$ is a normal subgroup of $S_n$ that lies in $H$ (why?). Therefore the kernel has index at least $n$ in $S_n$. Since the only normal subgroups of $S_n$ are $1, A_n$, and $S_n$, the kernel of $\ell$ is trivial, so $\ell$ is an isomorphism. What is the image $\ell(H)$ in $\operatorname{Sym}\left(S_n / H\right)$ ? Since $g H=H$ if and only if $g \in H, \ell(H)$ is the group of permutations of $S_n / H$ that fixes the "point" $H$ in $S_n / H$. The subgroup fixing a point in a symmetric group isomorphic to $S_n$ is isomorphic to $S_{n-1}$. Therefore $H \cong \ell(H) \cong S_{n-1}$.
\end{proof}

\begin{corollary}
Let $F$ be a field. If $f \in F\left[X_1, \ldots, X_n\right]$ and $n \geq 5$, the number of different polynomials we get from $f$ by permuting its variables is either 1, 2, or at least $n$.
\end{corollary}
\begin{proof}
Letting $S_n$ act on $F\left[X_1, \ldots, X_n\right]$ by permutations of the variables, the polynomials we get by permuting the variables of $f$ is the $S_n$-orbit of $f$. The size of this orbit is $\left[S_n: H\right]$, where $H=\operatorname{Stab}_f=\left\{\sigma \in S_n: \sigma f=f\right\}$. By Theorem 2.7, this index is either 1, 2, or at least $n$.
\end{proof}

\subsection{The index of subgroup of \texorpdfstring{$A_n$}{A_n}}

\begin{corollary}
For $n \geq 5$, each proper subgroup of $A_n$ has index at least $n$.
\end{corollary}
\begin{note}
The proof is similar to \cref{93da7c}
\end{note}
\begin{proof}
Let $H$ be a proper subgroup of $A_n$, with index $m>1$. Consider the left multiplication action of $A_n$ on $A_n / H$. This gives a group homomorphism
\[
\varphi: A_n \rightarrow \operatorname{Sym}\left(A_n / H\right) \cong S_m
\]
Let $K$ be the kernel of $\varphi$, so $K \subset H$ (why?) and $K \triangleleft A_n$. By simplicity of $A_n, K$ is trivial. Therefore $A_n$ injects into $S_m$, so $(n!/ 2) \mid m$ !, which implies $n \leq m$.
\end{proof}
