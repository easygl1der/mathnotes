\section{Sylow 定理习题}

\href{https://aryamanmaithani.github.io/alg/groups/sylow-exercises/}{Sylow exercises | Aryaman Maithani}

\href{https://kconrad.math.uconn.edu/blurbs/grouptheory/sylowapp.pdf}{sylowapp.pdf}

\href{https://kconrad.math.uconn.edu/blurbs/}{Expository papers by K. Conrad}

Dummit\&Foote 4.5

\begin{note}
It should be pointed out that the definition of cosets does not require normality, but the definition of quotient group requires. What's the difference? The former (as set) admits disjoint orbits, but $aH\cdot bH$ need not be $ab\cdot H$. The latter (as group) admits both.
\end{note}
\subsection{Definitions}

The \textbf{affine group}
\[
\mathrm{Aff}(\mathbf{Z}_{5})\coloneqq \left\{ \begin{pmatrix}
a & b \\
0 & 1
\end{pmatrix}:a\in \mathbf{Z}_{5}^{\times},b\in \mathbf{Z}_{5} \right\}
\]
It's defined by the transformation
\[
x\mapsto ax+b\qquad \mathrm{mod}\ 5
\]
\[
\begin{pmatrix}
x' \\
1 
\end{pmatrix}
=\begin{pmatrix}
a & b \\
0 & 1
\end{pmatrix}\begin{pmatrix}
x  \\
1
\end{pmatrix}
\]
The \textbf{Heisenberg group} over $\mathbb{Z} / p \mathbb{Z}$, denoted by $\operatorname{Heis}(\mathbb{Z} / p \mathbb{Z})$, is the set of $3 \times 3$ upper triangular matrices of the form:
\[
\left(\begin{array}{lll}
1 & x & z \\
0 & 1 & y \\
0 & 0 & 1
\end{array}\right)
\]
where:

\begin{itemize}
	\item $x, y, z \in \mathbb{Z} / p \mathbb{Z}$, meaning they are elements of the finite field $\mathbb{Z} / p \mathbb{Z}$ (the integers modulo $p$ ).
	\item The group operation is matrix multiplication.
\end{itemize}

\subsection{CONSEQUENCES OF THE SYLOW THEOREMS}

\href{https://kconrad.math.uconn.edu/blurbs/grouptheory/sylowapp.pdf}{sylowapp.pdf}
\begin{figure}[H]
\centering
\includegraphics[width=\textwidth]{Sylow-theorems-2025040815.png}
% \caption{}
\label{}
\end{figure}
\begin{figure}[H]
\centering
\includegraphics[width=\textwidth]{1-Sylow-theorems-2025040815.png}
% \caption{}
\label{}
\end{figure}
\begin{figure}[H]
\centering
\includegraphics[width=\textwidth]{Sylow-theorems-2025040816.png}
% \caption{}
\label{}
\end{figure}
\begin{figure}[H]
\centering
\includegraphics[width=\textwidth]{1-Sylow-theorems-2025040816.png}
% \caption{}
\label{}
\end{figure}
\begin{figure}[H]
\centering
\includegraphics[width=\textwidth]{2-Sylow-theorems-2025040816.png}
% \caption{}
\label{}
\end{figure}
\begin{figure}[H]
\centering
\includegraphics[width=\textwidth]{3-Sylow-theorems-2025040816.png}
% \caption{}
\label{}
\end{figure}

\subsection{Exercises from Dummit\&Foote}

\href{https://aryamanmaithani.github.io/alg/groups/sylow-exercises/}{Sylow exercises | Aryaman Maithani}

\begin{exercise}
Prove that if $P \in Syl_p(G)$ and $H$ is a subgroup of $G$ containing $P$, then $P \in Syl_p(H)$. Give an example to show that, in general, a Sylow $p$-subgroup of a subgroup of $G$ need not be a Sylow $\boldsymbol{p}$-subgroup of $\boldsymbol{G}$.
\end{exercise}
\begin{proof}
We are given $P \leq H \leq G$ with $|P|=p^\alpha$ and $|G|=p^\alpha m$ such that $p \nmid  m$. As $|P|||H|$, we get that $| H \mid=\boldsymbol{p}^\alpha k$ for some $k \in \mathbb{N}$.
Moreover, $|H|||G|$ and thus, $k| m$. As $p \nmid  m$, we get that $p \nmid  k$.
Thus, $P$ is a Sylow $\boldsymbol{p}$-subgroup of $\boldsymbol{H}$.

For the second part, consider $G=\mathbb{Z}_4$ and $H=\{\overline{0}, \overline{2}\}$. Then $P=H$ is a Sylow 2subgroup of $\boldsymbol{H}$ but not one of $\boldsymbol{G}$.
\end{proof}

\begin{exercise}
Prove that if $H$ is a subgroup of $G$ and $Q \in Syl_p(H)$ then $g Q g^{-1} \in Syl_p\left(g H g^{-1}\right)$ for all $g \in G$.
\end{exercise}
\begin{proof}
We first note that $\boldsymbol{g Q g ^ { - 1 }}$ is indeed a subgroup of $\boldsymbol{g H} \boldsymbol{g}^{-1}$.
The assertion follows from the fact that $\left|g Q q^{-1}\right|=|Q|$ and $\left|\boldsymbol{g H} g^{-1}\right|=|H|$.
\end{proof}

\begin{exercise}
Use Sylow's theorem to prove Cauchy's theorem.
\end{exercise}
\begin{proof}
Recall that Cauchy's theorem says that if $G$ is a finite group such that $p||G|$ for some prime $p$, then $G$ has an element of order $p$.

To prove this, let $G$ and $p$ be as above and $P$ be a Sylow $p$-subgroup of $G$. (Existence of $P$ is given by the Sylow theorems.)

Let $G=p^\alpha m$ with usual meanings. Then, $|P|=p^\alpha$.

Consider $x \in P$ such that $x \neq 1$. Then, the order of $x$ is $p^\beta$ for some $1 \leq \beta \leq \alpha$. Consider $y=x^{p^{\beta-1}}$. Then, order of $y$ is $p$. (How? It is clear that $y^p=1$. Thus, the only other possibility for the order is 1 but that is not possible since $y \neq 1$ as $p^{\beta-1}<p^\beta$, the order of $\boldsymbol{x}$.)
\end{proof}

\begin{exercise}
Exhibit all Sylow 2-subgroups and Sylow 3-subgroups of $D_{12}$ and $S_3 \times S_3$.
\end{exercise}
\begin{proof}
$D_{12}$ :
Note that $\left|D_{12}\right|=12=2^2 \cdot 3$. Thus, the Sylow 2-subgroup(s) will have order 4 and Sylow 3-subgroup(s) will have order 3 .

Sylow 3:
Note that a Sylow 3-subgroup must necessarily be isomorphic to $\mathbb{Z}_3$. Since \textbackslash{}underline\{all elements not of the form $r^k$ have order 2\}, we just find powers of $r$ which have order 3. These turn out to be $r^2$ and $r^4$. As $\left\langle r^2\right\rangle=\left\langle r^4\right\rangle$, we have a unique Sylow 3-subgroup:
\[
\left\langle r^2\right\rangle
\]
Sylow 2:
Note that no element of $D_{12}$ has order 4. (Any such element would have to be a power of $r$ but 4 does not divide 6, the order of $\boldsymbol{r}$.)
Thus, any Sylow 2-subgroup must be isomorphic to $\mathbb{Z}_2 \times \mathbb{Z}_2$.
Every element of this group must have order 2. $r^3$ is the only power of $r$ which has order 2. Thus, the other two elements must be of the form $s r^k$. Noting that the multiplication of two such elements is:
\[
s r^k \cdot s r^{k^{\prime}}=r^{k^{\prime}-k}
\]
we get that $\left|k-k^{\prime}\right|=3$. This gives us three such groups which are all the Sylow 2subgroups:
\[
\underbrace{\left\langle s, s r^3\right\rangle}_{\left\langle s\right\rangle\times \left\langle r^3\right\rangle},\underbrace{\left\langle s r, s r^4\right\rangle}_{\left\langle s r\right\rangle\times \left\langle r^3\right\rangle},\underbrace{\left\langle s r^2, s r^5\right\rangle}_{\left\langle s r^2\right\rangle\times \left\langle r^3\right\rangle} .
\]
(It can be checked that all of these are distinct subgroups.)

$S_3 \times S_3:$
The order of the group is $36=2^2 \cdot 3^2$. Thus, the Sylow 2-subgroup(s) will have order 4 and Sylow 3-subgroup(s) will have order 9.

Sylow 2:
Note that there are three subgroups of order 2 of $S_3:\langle(12)\rangle,\langle(13)\rangle,\langle (23)\rangle$. Let these be $H_1, H_2, H_3$. (Note that these are indeed distinct.)
Then, the nine products:
\[
H_i \times H_j \quad i, j \in\{1,2,3\}
\]
are subgroups of $S_3 \times S_3$ have order 4. This gives us nine Sylow 2-subgroups. However, by the Sylow theorems, $n_2 \mid 9$ and thus, there can't be any more.

Sylow 3:
Let $P$ be a Sylow 3-subgroup. Note that no element of $S_3 \times S_3$ has order 9 . Thus, every non-identity element of $\boldsymbol{P}$ must have order 3 . Now, note that the order of an element $(\sigma, \tau) \in S_3 \times S_3$ is the lcm of the orders of $\sigma$ and $\tau$. This gives us that there are 8 elements of order 3 . These elements together with the identity do form a subgroup and moreover, there can't be any other. Thus, there is a unique Sylow 3-subgroup which is:
\[
\langle(123)\rangle \times\langle(123)\rangle .
\]
\end{proof}

\begin{exercise}
Show that a Sylow $p$-subgroup of $D_{2 n}$ is cyclic and normal for every odd prime $p$.
\end{exercise}
\begin{proof}
Let $2 n=p^\alpha m$ as usual. As $p$ is odd and $2 n$ is even, we must have that $\boldsymbol{m}=\mathbf{2} \boldsymbol{m}^{\prime}$ for some $\boldsymbol{m} \in \mathbb{N}$. Thus, $\boldsymbol{n}=\boldsymbol{p}^\alpha \boldsymbol{m}^{\prime}$ with $\boldsymbol{p} \nmid  \boldsymbol{m}^{\prime}$.

Note that $r$ has order $n$ and hence, $\boldsymbol{r}^{m^{\prime}}$ has order $\boldsymbol{p}^\alpha$. Hence, $\left\langle\boldsymbol{r}^{m^{\prime}}\right\rangle$ is Sylow p-subgroup of $D_{2 n}$

We now show that is normal. To do this, it is enough to work with the generators of $D_{2 n}$. Clearly, $r\left(r^{m^{\prime}}\right) r^{-1} \in\left\langle r^{m^{\prime}}\right\rangle$. Also, $s\left(r^{m^{\prime}}\right) s^{-1}=s^2 r^{-m^{\prime}}=r^{-m^{\prime}} \in\left\langle r^{m^{\prime}}\right\rangle$. Thus, $\left\langle r^{m^{\prime}}\right\rangle$ is normal.

As all Sylow p-subgroups are conjugates, we get that the above Sylow p-subgroup is the unique Sylow p-subgroup. As it is cyclic and normal, we are done.
\end{proof}

\begin{exercise}
Exhibit all Sylow 3-subgroups of $\boldsymbol{A}_4$ and all Sylow 3-subgroups of $S_4$.
\end{exercise}
\begin{proof}
$A_4$ : Clearly, $\boldsymbol{n}_{\mathbf{3}} \mid 4$. We exhibit 4 such Sylow 3-subgroups now:
\[
\langle(123)\rangle,\langle(124)\rangle,\langle(134)\rangle,\langle(234)\rangle .
\]
(Note that these are indeed distinct subgroups.)

These were easy to find as a Sylow 3-subgroup must have order 3 and thus, must be isomorphic to $\mathbb{Z}_3$. It was then a matter of finding elements of order 3 .

$S_4$ : Same as earlier. Note that there are 8 elements of order 3. If we consider all the 8 subgroups generated by them, we see that we get \textbackslash{}underline\{only four distinct ones\}, listed above.
\end{proof}

\begin{exercise}
Exhibit all Sylow 2-subgroups of $S_4$ and find elements of $S_4$ which conjugate one of these into each of the others.
\end{exercise}
\begin{proof}
A little bit of experimenting with elements of order 2 and 4 gives the following subgroups:
\[
\langle(1234),(13)\rangle,\langle(1243),(14)\rangle,\langle(1324),(12)\rangle .
\]
It can be verified that these are distinct. (For example, the 2 cycles listed in the generators don't appear in any of the other subgroups.) It can also be verified that all of these have order 8. Moreover, there can't be any more as $\boldsymbol{n}_2 \mid 3$.

As for the conjugation question, (34) conjugates the first to the second and (24) the second to the third.\textbackslash{}footnote\{Note that the conjugate of $D_{2n}$ works by transforming the places of the elements.\}
\end{proof}

\begin{exercise}
Exhibit two distinct Sylow 2-subgroups of $S_5$ and an element of $S_5$ that conjugates one into the other.
\end{exercise}
\begin{proof}
Note that $\left|S_5\right|=2^3 \cdot 15$ and thus, any two of the Sylow 2-subgroups listed earlier will work again. (With the understanding that the elements are now elements of $S_5$)
\end{proof}

\begin{lemma}
The order of $\mathrm{GL}_2(\mathbb{F}_q)$ is $(q^2-1)(q^2-q)$.
\end{lemma}
\begin{proof}
Consider the choice of the first column. There are $q^2-1$ choices for the first column. The second column must be a vector \textbackslash{}underline\{independent\} of the first column. There are $q^2-q$ choices for the second column.
\end{proof}

\begin{exercise}
Exhibit all Sylow 3-subgroups of $S L_2\left(\mathbb{F}_3\right)$.
\end{exercise}
\begin{proof}
Note that the order of $G L_2\left(\mathbb{F}_3\right)$ is $\left(3^2-1\right)\left(3^2-3\right)=48$. It can be easily shown that the order of $S L_2$ is half of that.
Thus, $n_3=1$ or 4 . Also, note that any Sylow 3-subgroup will have order 3 and hence, is isomorphic to $\mathbb{Z}_3$.
It is easy to find two distinct subgroups of order 3 by observing the following matrices to have order 3:
\[
\left(\begin{array}{ll}
1 & 1 \\
0 & 1
\end{array}\right),\left(\begin{array}{ll}
1 & 0 \\
1 & 1
\end{array}\right)
\]
Also, these matrices generate distinct subgroups. Thus, $n_3>1$ which forces $n_3=4$. \textbackslash{}textbf\{We may now conjugate the above matrices to get the other two.\} In any case, we are left with the following:
\[
\left\langle\left(\begin{array}{ll}
1 & 1 \\
0 & 1
\end{array}\right)\right\rangle,\left\langle\left(\begin{array}{ll}
0 & 1 \\
1 & 1
\end{array}\right)\right\rangle,\left\langle\left(\begin{array}{ll}
0 & 1 \\
2 & 2
\end{array}\right)\right\rangle,\left\langle\left(\begin{array}{ll}
2 & 1 \\
2 & 0
\end{array}\right)\right\rangle .
\]
\end{proof}

\begin{exercise}
Prove that the subgroup of $S L_2\left(\mathbb{F}_3\right)$ generated by $\left(\begin{array}{cc}0 & -1 \\ 1 & 0\end{array}\right)$ and $\left(\begin{array}{cc}1 & 1 \\ 1 & -1\end{array}\right)$ is the unique Sylow 2-subgroup of $S L_2\left(\mathbb{F}_3\right)$.
\end{exercise}
\begin{proof}
Hmm.
\end{proof}

\begin{exercise}
Prove that a group $G$ of order 351 has a normal Sylow $p$ -subgroup for some prime $p$ dividing its order.
\end{exercise}
\begin{proof}
Note that $312=3^3 \cdot 13$. By the Sylow theorems, $n_{13} \mid 27$ and $n_{13} \equiv 1 \bmod$ 13. This forces $n_{13}=1,27$. If $n_{13}=1$, then we are done. Let us assume that $n_{13} \neq 1$. Thus, $n_{13}=27$.

Thus, there are 27 Sylow 13 -subgroups of $G$. \textbf{As 13 is a prime, this forces the intersection of two distinct Sylow 13 -subgroups to be trivial.} Thus, the number of elements having order 13 equals $27(13-1)=3^3(12)$.

This leaves us with $3^3$ remaining elements which are not part of any Sylow 13-subgroup. Now, by Sylow Theorems, we know that $n_3 \geq 1$. However, no non-identity element can be part of a Sylow 3 as well as a Sylow 13 -subgroup. Thus, \textbf{the remaining $3^3$ elements form the unique Sylow 3 -subgroup} which gives us that $n_3=1$ and thus, we are done.
\end{proof}

\begin{exercise}
Let $|G|=p q r$, where $p, q$, and $r$ are primes with $p<q<r$. Prove that G has a normal Sylow subgroup for either $p,q$ or $r$.
\end{exercise}
\begin{proof}
Let $p, q$, and $r$ be distinct prime numbers. WLOG, we may assume that $p<q<r$. Let $G$ be a group with order $p q r$.

We shall be appealing to the Sylow theorems without mentioning it explicitly. If any of $n_p, n_q$, or $n_r$ is equal to 1 , then we know that $G$ is not simple. For sake of contradiction assume that each of the above is strictly greater than 1. As $n_r \mid p q$ and $p, q<r$, we get that $n_r=p q$. (Since we have assumed that $n_r>1$.) Thus, there are $p q$ Sylow- $r$ subgroups of $G$.

Now, note that each such Sylow- $r$ subgroup has order $r$, a prime and thus, the intersection of two distinct Sylow- $r$ subgroups must be trivial, id est, (1). Thus, the number of elements having order $r$ equals $o_r=p q(r-1)$.

Now, $n_q>1$ and $n_q \mid p r$. Thus, $n_q \in\{p, r, p r\}$. However, $n_q \equiv 1 \bmod q$ and thus, $n_q \neq p$. This gives us that $n_q \geq r$. Thus, the number of elements having order $q$ equals $o_q \geq r(q-1)$.

Lastly, similar argument as earlier gives us that $o_p \geq q(p-1)$. Note that $o_r, o_q$, and $o_p$ are counting distinct non-identity elements and thus,
\[
\begin{aligned}
|G|  & \geq o_r+o_q+o_p+1 \\
 &  \geq p q(r-1)+r(q-1)+q(p-1) \\
&  \geq p q r+r q-r-q+1 \\
 & =p q r+\underbrace{(r-1)(q-1)}_{>0} \\
 & >p q r 
\end{aligned}
\]
Thus, we have a contradiction as $|G|=p q r$ and we are done!
\end{proof}

\begin{exercise}
Prove that if $|G|=105$, then $G$ has a normal Sylow 5-subgroup and a normal Sylow 7 subgroup.
\end{exercise}
\begin{proof}
Assume not. Then, $n_5>1$ and $n_7>1$.

Note that $105=3 \cdot 5 \cdot 7$. The Sylow theorems force $n_5=21$ and $n_7=15$.

Note that the intersection of any two of these $21+15$ Sylow subgroups must be trivial (by Lagrange's Theorem). Thus, their union contains exactly
\[
21(5-1)+15(7-1)+1=175 \text { elements. }
\]
This is clearly a contradiction. (As $175>105$.)
\end{proof}

\begin{exercise}
Prove that if $|G|=6545$ then $G$ is not simple.
\end{exercise}
\begin{proof}
Note that $6545=5 \cdot 7 \cdot 11 \cdot 17$. For the sake of contradiction, let us assume that $G$ is not simple. Thus, $n_5, n_7, n_{11}$, and $n_{17}$ are all strictly greater than 1 . Now, looking at the possibilities for each by considering the factors of the "complimentary part", we get that
\[
n_5=11, n_7=85, n_{11}=594, n_{17}=35
\]
Note that the intersection of any two of these Sylow subgroups must be trivial (by Lagrange's Theorem). Thus, their union contains exactly
\[
11(5-1)+85(7-1)+594(11-1)+35(17-1)+1=7055 \text { elements. }
\]
This is clearly a contradiction. (As $7055>6545$.)
\end{proof}

\begin{exercise}
Prove that if $|G|=1365$ then $G$ is not simple.
\end{exercise}
\begin{proof}
Note that $1365=3 \cdot 5 \cdot 7 \cdot 13$. For the sake of contradiction, let us assume that $G$ is not simple. Thus, $n_3, n_7$, and $n_{13}$ are all strictly greater than 1
.
Now, looking at the possibilities for each by considering the factors of the "complimentary part", we get that
\[
n_3 \geq 7, n_7 \geq 15, n_{13}=105
\]
Note that the intersection of any two of these Sylow subgroups must be trivial (by Lagrange's Theorem). Thus, their union contains at least
\[
7(3-1)+15(7-1)+105(13-1)>1365 \text { elements. }
\]
This is clearly a contradiction.
\end{proof}

\begin{exercise}
Prove that if $\lvert G \rvert=132$ then $G$ is not simple.
\end{exercise}
\begin{proof}
Let $G$ be a group with order 132 . We show that $G$ is not simple.

Note that $|G|=2^2 \cdot 3 \cdot 11$. Let us assume that $G$ is simple and arrive at a contradiction. By simplicity, we know that $n_{11}>1, n_3>1$, and $n_2>1$.

By Sylow Theorem (3), it is forced that $n_{11}=12$. Also, $n_3 \geq 4$ and $n_2 \geq 3$.

Note the following:

\begin{enumerate}
	\item Intersection of any two Sylow- 11 subgroups is trivial.
	\item Intersection of any two Sylow-3 subgroups is trivial.
	\item Intersection of any Sylow- $\boldsymbol{p}$ subgroup with any Sylow- $\boldsymbol{q}$ subgroup is trivial.
\end{enumerate}

The above facts follow by considering the fact that the intersection would be a subgroup of the two bigger subgroups and would have to divide their orders.

Now, if we consider the union of all the Sylow-11 and Sylow- 3 subgroups, it contains at least $12(11-1)+4(3-1)+1=129$ elements. Thus, the remaining elements are at most 3 . However, we do need at least 3 more elements to form a Sylow- 2 subgroup.

These 3 elements, along with the identity must form the unique Sylow- 2 subgroup of order 4 . However, we have reached a contradiction as we get that $n_2=1$.

Thus, we are done!

\end{proof}

\begin{exercise}
Prove that if $|G|=462$ then $G$ is not simple.
\end{exercise}
\begin{proof}
Note that $462=11 \cdot 42$. Considering the factors of 42 , we see that $n_{11}$ is forced to be 1 .
\end{proof}

\begin{exercise}
Prove that if $G$ is a group of order 231 then $Z(G)$ contains a Sylow 11-subgroup of $G$ and a Sylow 7-subgroup is normal in $G$.
\end{exercise}
\begin{proof}
Note that $231=3 \cdot 7 \cdot 11$. The restrictions from Sylow Theorems force that $n_7=n_{11}=1$. In particular, this shows that $G$ has a (unique) Sylow 7 -subgroup which is normal in it.

Now, we show that the Sylow 11-subgroup is contained in $Z(G)$. Let us denote this subgroup by $P$. Note that $P$ is normal. Consider the action of $G$ on $P$ given by conjugation, that is, $(g, p) \mapsto g p g^{-1}$.

This induces a homomorphism $\Phi: G \rightarrow \operatorname{Aut}(P)$. Note that $P=(\mathbb{Z} / 11 \mathbb{Z})$ is cyclic and thus, $\operatorname{Aut}(P)=(\mathbb{Z} / 11 \mathbb{Z})^{\times}=(\mathbb{Z} / 10 \mathbb{Z})$. (The last equality may be manually verified by noting that $\overline{2}$ is a generator for $(\mathbb{Z} / 11 \mathbb{Z})^{\times}$ which has 10 elements.)

Thus, $G / \operatorname{ker} \Phi$ is isomorphic to a subgroup of $\mathbb{Z}_{10}$ (First Isomorphism Theorem). In particular, $|G / \operatorname{ker} \Phi| \mid 10$. Looking at the divisors of $|G|$, we see that this forces $|G / \operatorname{ker} \Phi|=1$ or $G=\operatorname{ker} \Phi$.

This means that given any $g \in G$, the map $\Phi(g)$ is the identity map. In other words, $\Phi(g)(p)=p$ for all $g \in G$ and $p \in P$. The above is equivalent to $g p g^{-1}=p$ or $g p=p g$ for all $g \in G$ and $p \in P$. Thus, $P \leq Z(G)$, by the definition of the center.

Remark. We didn't really require the fact that $\mathbb{Z}_{11}^{\times} \cong \mathbb{Z}_{10}$. Only $\left|\mathbb{Z}_{11}^{\times}\right|=10$ would've been sufficient (which is more straightforward as well).
\end{proof}

\begin{exercise}
Let $G$ be a group of order 105. Prove that if a Sylow 3-subgroup of $G$ is normal then $G$ is abelian.
\end{exercise}
\begin{proof}
Let $N$ be the Sylow 3-subgroup.

Claim 1: $N \leq Z(G)$.

Note that $\left|G / C_G(N)\right|$ must divide both $|\operatorname{Aut}(N)|=2$ and $|G|=105$. This forces $G=C_G(N)$ or $N \leq Z(G)$.
(We have used the fact that $N$ is normal in the above by considering the $\Phi$ defined similarly in the previous exercise.)

Claim 2: $G / N$ is cyclic.

Note that $G / N$ has order $35=5 \cdot 7$. Noting that $5 \nmid  6$ proves the claim. (See this.)

Claim 3: $G / Z(G)$ is cyclic.

By the third isomorphism theorem, we have that
\[
G / Z(G) \cong(G / N) /(Z(G) / N)
\]
(Note that Claim 1 (and the fact that $N \triangleleft G$ ) tells us that $Z(G) / N$ makes sense.) By the previous claim, $G / N$ is cyclic. The claim follows as quotients of a cyclic group are cyclic.

Claim 4: $G$ is abelian.

This is a consequence of the above claim.

\end{proof}

\begin{lemma}
$\mathrm{Aut}(\mathbb{Z}_{3}\times \mathbb{Z}_{3})$ is isomorphic to $\mathrm{GL}(2,\mathbb{F}_{3})$.
\end{lemma}
\begin{proof}
Consider $\mathbb{Z}_{3}\times \mathbb{Z}_{3}$ as a 2-dimension vector in $\mathbb{F}_{3}^2$. Then $\mathrm{Aut}(\mathbb{Z}_{3}\times \mathbb{Z}_{3})$ is a nondegenerate transform above $\mathbb{F}_{3}^2$.
\end{proof}

\begin{exercise}
Let $G$ be a group of order 315 which has a normal Sylow 3-subgroup. Prove that $Z(G)$ contains a Sylow 3-subgroup of $G$ and deduce that $G$ is abelian.
\end{exercise}
\begin{proof}
Let $P$ be the normal Sylow 3-subgroup. (It must necessarily be unique.) Consider the action of $G$ on $P$ given by conjugation. (This is an action as $P$ is normal.) Consider the induced homomorphism $\Phi: G \rightarrow \operatorname{Aut}(P)$.

Now, note that $P$ is a group of order 9 . Thus, it is isomorphic to either $\mathbb{Z}_9$ or $\mathbb{Z}_3 \times \mathbb{Z}_3$. Accordingly, $\operatorname{Aut}$ has order either 6 or 48.

As $G / \operatorname{ker} \Phi \cong \Phi(G)$ by the first isomorphism theorem, we get that $|G / \operatorname{ker} \Phi|$ divides 48. (6 divides 48 anyway.)

Also, $|G / \operatorname{ker} \Phi|||G|$. Thus, $| G / \operatorname{ker} \Phi \mid$ is either 1 or 3.

If $|G / \operatorname{ker} \Phi|=3$, then $|\operatorname{ker} \Phi|=315 / 3=105$. However, note that $P \leq \operatorname{ker} \Phi=C_G(P)$ and hence $|P|\left|\left|C_G(P)\right|\right.$ which is a contradiction as $9 \nmid  105$. Hence, we get that $G / \operatorname{ker} \Phi=(1)$ or $G=\operatorname{ker} \Phi$.

The above shows that the image of $\Phi$ is just the identity map. In other words, $g p g^{-1}=p$ for all $p \in P$ and $g \in G$. This is precisely what it means for $P \leq Z(G)$.

The rest of the solution is now the same as the earlier one. We note that $G / P$ has order 35 and hence, is cyclic.

We also have
\[
G / Z(G) \cong(G / P) /(Z(G) / P)
\]
and thus, $G / Z(G)$ is cyclic. This proves that $G$ is abelian.
\end{proof}

\begin{exercise}
Let $G$ be a group of order 1575 . Prove that if a Sylow 3 -subgroup of $G$ is normal then a Sylow 5-subgroup and a Sylow 7 -subgroup are normal. In this situation prove that $G$ is abelian.
\end{exercise}
\begin{proof}
By the same arguments as earlier, we get that $G$ is abelian. Now, by the Sylow Theorems, there do exist Sylow 5 and 7-subgroups. As $G$ is abelian, they must be normal.
\end{proof}

\begin{exercise}
How many elements of order 7 must there be in a simple group of order $168 ?$
\end{exercise}
\begin{proof}
Note that $168=7 \cdot 24$. The Sylow Theorems force $n_7$ to be either 1 or 8 . If $n_7=1$, then there is a unique Sylow 7 -subgroup which must be normal. This contradicts the assumption that the group is simple.

Thus, $n_7=8$. This means that every Sylow 7 -subgroup is isomorphic to $\mathbb{Z}_7$. In particular, every non-identity has order 7.

Moreover, two distinct Sylow 7-subgroups can only intersect trivially, by Lagrange's theorem. Thus, the total number of order 7 elements contained in these Sylow subgroups is
\[
8(7-1)=48
\]
\textbf{Now, we show that there are no more.} That is, any order 7 element is contained in some Sylow 7-subgroup.\footnote{This is important.}

Let $x$ be any order 7 element. Then, $\langle x\rangle$ is a 7-group and thus, must be contained in some Sylow 7 -subgroup, as desired. (In fact, $\langle x\rangle$ is itself a Sylow 7 -subgroup.)
\end{proof}

\begin{exercise}
For $p=2,3$, and 5 , find $n_p\left(A_5\right)$ and $n_p\left(S_5\right)$.
\end{exercise}
\begin{proof}
$A_5$ :
Note that Sylow Theorems force $n_3 \in\{1,10\}$ and $n_5 \in\{1,6\}$.

As $A_5$ is simple, $n_p$ cannot be 1 for any $p$ and hence, $n_3=10$ and $n_5=6$.

Now, we show that $n_2=5$. Note that $\langle(12)(34),(13)(24)\rangle$ is a Sylow 2-subgroup. (The way I found this was by taking a Sylow 2-subgroup from $Q_8$. And intersecting it with $A_5$.) Note that its elements are
\[
\{1,(12)(34),(13)(24),(14)(23)\}
\]
That is, identity along with all possible elements of the cycle type 1-2-2 formed using the numbers $1,2,3,4$.

Recalling that all Sylow 2-subgroups are conjugates and also recalling how conjugates in $S_n$ look, we see that any Sylow 2-subgroup is of this form:
\[
\{1,(a b)(c d),(a c)(b d),(a d)(b c)\}
\]
(Where $a, b, c, d$ are distinct elements of $\{1,2,3,4,5\}$.)

This shows us that picking any four numbers from $\{1, \ldots, 5\}$ uniquely defines Sylow 2 subgroup. (In the sense, that there's no over-counting or under-counting.) Thus, the number of such subgroups is $\binom{5}{4}=5$.

To conclude, we have, for $A_5$ :
\[
n_2=5, n_3=10, n_5=6
\]
$S_5$ :
\[
\left|S_5\right|=120=2^3 \cdot 3 \cdot 5
\]
For $n_3$, we have the possibilities 1,10 , and 40 . Note that it can't be 1 as any Sylow 3 subgroup of $A_5$ is again a Sylow 3-subgroup of $S_5$ and we already had 10 there.

Now, if $n_3$ were 40 , then there would be a total of 80 elements of order 3 . However, it can be verified that only elements of the cycle type ( $a b c$ ) have order 3 and there are only 20 of those. Thus, $n_3=10$.

For $n_5$, the discussion is even simpler as there are no new possibilities and we have $n_5=6$ as before.

As before, we have $n_2 \leq 15$. We show that $n_2=15$ by demonstrating fifteen such subgroups. Let $a, b, c, d \in\{1,2,3,4,5\}$ be distinct. Assume that $a$ is the smallest amongst the four.

Consider the subgroup:
\[
\langle(a b c d),(a c)\rangle
\]
It can be verified that the above does have 8 elements. Moreover, to determine such a subgroup, we first need to pick 4 elements from $\{1, \ldots, 5\}$ and then from those 4 , pick an element to be paired with the smallest.

It can be verified that different such pickings will give different subgroups. (Simply by writing explicitly the elements of the above subgroup.)

This gives us $5 \times 3=15$ different subgroups and hence, we are done.
To conclude, we have, for $S_5$ :
\[
n_2=15, n_3=10, n_5=6
\]
\end{proof}

\section{Discussion of the intersection of Sylow 3 subgroups, when \texorpdfstring{$n_3\neq1$}{n_3neq1}}

\href{https://aryamanmaithani.github.io/alg/groups/simple/90/}{Order 90 | Aryaman Maithani}

\href{https://aryamanmaithani.github.io/alg/groups/simple/144/}{Order 144 | Aryaman Maithani}

\subsection{Group with order 180 is not simple}

\href{https://aryamanmaithani.github.io/alg/groups/simple/180/}{Order 180 | Aryaman Maithani}

\begin{exercise}
Let $G$ be a group with order 180. Show that $G$ is not simple.
\end{exercise}
\begin{proof}
Assume that $G$ is simple, note that $\lvert G \rvert=2^{2}\cdot3^2\cdot5$; by Sylow III theorem, $n_3\in \{ 4,10 \}$, $n_5\in \{ 6,36 \}$.

Claim that $n_3\neq4$. Suppose not, $n_3=4$, then there is a homomorphism
\[
\varphi:G\to \mathrm{Aut}(\mathrm{Syl}_{3}(G))\cong S_4\qquad g\mapsto \mathrm{Ad}_{g}
\]
recall that $\mathrm{Syl}_{p}(G)\coloneqq \{ \text{Sylow }p\text{-subgroups of }G \}$. By Sylow II theorem, $\varphi$ is transitive, thus $\ker\varphi\neq G$. As $G$ is simple, $\ker\varphi$ is trivial, otherwise the nontrivial $\ker\varphi\lhd G$; thus $G$ is not simple. So $\ker\varphi=\{ e \}$, then
\[
G\cong G/\ker\varphi\cong\mathrm{Im}\varphi\leq S_4\Rightarrow 180= \lvert G \rvert \mid \lvert S_4 \rvert =24\Rightarrow180\leq 24
\]
which is a contradiction.

Claim that $n_5\neq6$. Suppose not, $n_5=6$, then there is a homomorphism
\[
\psi:G\to \mathrm{Aut}(\mathrm{Syl}_{5}(G))\cong S_6\qquad g\mapsto \mathrm{Ad}_{g}
\]
Similarly, $\ker \psi=\{ e \}$, then
\[
G\cong G/\ker \psi \cong \mathrm{Im}\psi\leq S_6
\]
We have $[S_6:G]=4$, strictly between $2$ and $n=6$, which is impossible by the following lemma.

\begin{lemma}
For $n \geq 5$, no subgroup of $S_n$ has index strictly between 2 and $n$. Moreover, each subgroup of index $n$ in $S_n$ is isomorphic to $S_{n-1}$.\label{14333f}
\end{lemma}

Applying the idea in the proof of \cref{14333f}, we consider the left multiplication action of $S_6$ on $S_6/G$,
\[
\Phi:S_6\to \mathrm{Aut}(S_6/G)\cong S_4\qquad g\mapsto \ell_{g}
\]
Then
\[
S_6/\ker \Phi \cong \mathrm{Im}\Phi\leq S_4
\]
Clearly, $\ker \Phi\lhd S_6$, then $\ker \Phi$ is trivial or $\ker \Phi=A_6$. Since $\ker \Phi \subset G$ (by the definition of $\Phi$), we have $[S_6:\ker \Phi]\geq[S_6:G]=4$, thus $\ker \Phi$ must be $\{ e \}$. Then
\[
180=\lvert S_6 \rvert =\lvert S_6/\ker \Phi \rvert =\lvert \mathrm{Im}\Phi \rvert \mid \lvert S_4 \rvert =24 
\]
which is a contradiction. Hence $n_5\neq6$.

Note that the intersection of any two Sylow-5 subgroups is trivial, since $P_1\cap P_2$ is subgroup of $P_1$, then $\lvert P_1\cap P_2 \rvert \mid5$; $P_1\cap P_2=\{ e \}$.

\begin{remark}
When $p^2\mid \lvert G \rvert$, the intersection of Sylow $p$ -subgroups may be nontrivial. A counterexample is $G=S_3\times \mathbf{Z}_{2}$; consider $P_1=\{ (e,0),(e,1),((12),0),((12),1) \}$, $P_2=\{ (e,0),(e,1),((13),0),((13),1) \}$, which are Sylow 2-subgroups of $G$; but $P_1\cap P_2=\{ (e,0),(e,1) \}$.
\end{remark}
Thus we now arrive the conclusion that $n_3=10$ and $n_5=36$. Then there are two cases,

\begin{itemize}
	\item Intersection of any two Sylow-3 subgroups is trivial.
	\item Intersection of some two Sylow-3 subgroups is non-trivial.
\end{itemize}

In the first case,

\begin{itemize}
	\item The number of elements of order 9 is $10\times(9-1)=80$.
	\item The number of elements of order 5 is $36\times(5-1)=144$.
\end{itemize}

$80+144>180$; this is a contradiction.

In the second case, let $P_1$ and $P_2$ have nontrivial intersection. Note that $\lvert P_1 \rvert=\lvert P_2 \rvert=9$ and thus $P=P_1\cap P_2$ must contain exactly 3 elements. As groups of $p^2$ order are abelian, we have $P_1\cup P_2\subset N_{G}(P)\eqqcolon N$. By the definition of group, the set $P_1P_2\coloneqq \{ p_1p_2:p_1\in P_1,p_2\in P_2 \}$ is contained in $N$, as $P_1\subset N$, $P_2\subset N$. This gives
\[
\lvert N \rvert \geq \lvert P_1P_2 \rvert =\frac{\lvert P_1 \rvert \cdot \lvert P_2 \rvert  }{\lvert P_1\cap P_2 \rvert }=27
\]
Thus the coset
\[
\lvert G/N \rvert =\frac{\lvert G \rvert }{\lvert N \rvert }\leq \frac{180}{27}<7\implies n\coloneqq  \lvert G/N  \rvert\leq 6 
\]
Let $G$ act on $G/N$ by left multiplication,
\[
\phi:G\to \mathrm{Aut}(G/N )\cong S_n\qquad g\mapsto \ell_{g}
\]
Since $G$ is simple, $\ker \phi$ is trivial; since the action is transitive, $\ker \phi\neq G$; thus $\ker \phi=\{ e \}$.
\[
G\cong G/\ker \phi \cong \mathrm{Im}\phi \leq S_n\implies 180= \lvert G \rvert \mid \lvert S_n \rvert =n!
\]
Therefore $n=6$. But by \cref{14333f}, $[S_n:\mathrm{Im}\phi]\neq4$, as $\mathrm{Im}\phi$ is a subgroup of $S_n$, $n=6\geq5$.

\end{proof}

\section{Group with order 36 has either a normal 2-Sylow or 3-Sylow subgroup}

\begin{note}
This topic involves extensive discussion and various techniques in group theory.
\end{note}
Let $|G|=36$. Then $n_3=1$ or 4 . We will show that if $n_3=4$ then $n_2=1$. Assume $n_3=4$ and $n_2>1$. Since $n_2>1, G$ has no subgroup of size 18 (it would have index 2 and therefore be normal, so a 3-Sylow subgroup of it would be normal in $G$ by Lemma 5.8, which contradicts $n_3>1$ ). Since $n_2>1, G$ is nonabelian. Our goal is to get a contradiction. We will try to count elements of different orders in $G$ and find the total comes out to more than 36 elements. That will be our contradiction.

Let $Q$ be a 3-Sylow in $G$, so $[G: Q]=4$. Left multiplication of $G$ on $G / Q$ gives a homomorphism $G \rightarrow \operatorname{Sym}(G / Q) \cong S_4$. Since $|G|>\left|S_4\right|$, the kernel $K$ is nontrivial. Since $K \subset Q$, either $|K|=3$ or $K=Q$. Since $Q \nsubseteq G, Q$ does not equal $K$, so $|K|=3$.

Since $K \triangleleft G$, we can make $G$ act on $K$ by conjugations. This is a homomorphism $G \rightarrow \operatorname{Aut}(K) \cong \mathbb{Z} /(2)$. If this homomorphism is onto (that is, some element of $G$ conjugates on $K$ in a nontrivial way) then the kernel is a subgroup of $G$ with size 18 , which $G$ does not have. So the conjugation action of $G$ on $K$ is trivial, which means every element of $G$ commutes with the elements of $K$, so $K \subset Z(G)$. Then $3||Z(G)|$, so the size of $Z(G)$ is one of the numbers in $\{3,6,9,12,18,36\}$. Since $G$ is nonabelian and a group is abelian when the quotient by its center is cyclic, $|Z(G)|$ can't be 12,18 , or 36 . Since $n_3>1$ there is no normal subgroup of size 9 , so $|Z(G)| \neq 9$. If $|Z(G)|=6$ then the product set $Z(G) Q$ is a subgroup of size 18 , a contradiction. So we must have $|Z(G)|=3$, which means $Z(G)=K$.

Now we start counting elements with various orders. The center is a 3 -subgroup of $G$, so by the conjugacy of 3 -Sylow subgroups every 3 -Sylow subgroup contains $K$. Each pair ofdifferent 3-Sylow subgroups have $K$ as their intersection, so we can count the total number of elements of 3 -power order: $|K|+n_3 \cdot(9-3)=27$.

Let $g \in G$ have order 2. Then $K\langle g\rangle$ is a subgroup of order 6 by Lemma 4.1, and it is is abelian and in fact cyclic since $K \subset Z(G)$. The cyclic group $K\langle g\rangle$ has a unique element of order 2, which must be $g$. Therefore when $g$ and $g^{\prime}$ are different elements of order 2 in $G$, the groups $K\langle g\rangle$ and $K\left\langle g^{\prime}\right\rangle$ have $K$ has their intersection. So each element of order 2 in $G$ provides us with 2 new elements of order 6 . Let $n$ be the number of elements of order 2 in $G$, so there are at least $2 n$ elements of order 6 , giving at least $3 n$ elements in total with order 2 or 6 . Since we already found 27 elements with 3-power order (including the identity), $3 n \leq 36-27$, or $n \leq 3$. We can get an inequality on $n$ in the other direction: $n \geq 2$. Indeed, no element of order 2 lies in $Z(G)=K$, so some conjugate of an element of order 2 is a second element of order 2 . Thus $n \geq 2$.

Since $\left|\left\{g \in G: g^2=e\right\}\right|$ is even (by McKay's proof of Cauchy's theorem) and this number is $1+n, n$ is odd, so $n=3$. Therefore $G$ has 3 elements of order 2 , so at least $3 n=9$ elements of order 2 or 6 . Adding this to 27 from before gives $9+27=36=|G|$, so each element of $G$ has 3-power order or order 2 or 6 . In particular, the 2-Sylow subgroup of $G$ is isomorphic to $\mathbb{Z} /(2) \times \mathbb{Z} /(2)$ (no elements of order 4 in $G)$. Then different 2-Sylow subgroups meet at most in a group of order 2 , which gives us 5 elements of order 2 from both subgroups. We saw before that there are only 3 elements of order 2 . This is a contradiction.
