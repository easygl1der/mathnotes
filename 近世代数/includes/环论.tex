\section{环论}

\begin{theorem}
设 $(R,+, \cdot)$ 为含 $1 \neq 0$ 的结合环, $a, b \in R$ .若 $a+b=b a$ ,且关于 $x$ 的方程
\[
\left\{\begin{array}{l}
x^2-\left(a x^2+x^2 a\right)+a x^2 a=1 \\
x+a-(a x+x a)+a x a=1
\end{array}\right.
\]在 $R$ 中有解.证明:$a b=b a$ .
\end{theorem}
\begin{proof}
1)首先注意到
\[
\begin{aligned}
& \left\{\begin{array}{l}
x^2-\left(a x^2+x^2 a\right)+a x^2 a=1 \\
x+a-(a x+x a)+a x a=1
\end{array}\right. \\
& \Leftrightarrow\left\{\begin{array}{c}
(1-a) x^2(1-a)=1 \\
(1-a) x(1-a)=1-a
\end{array}\right.
\end{aligned}
\]
结果有:
\[
\begin{aligned}
& (1-a) x=(1-a) x\left\{(1-a) x^2(1-a)\right\} \\
= & (1-a) x(1-a) x^2(1-a)=(1-a) x^2(1-a)=1 \\
& x(1-a)=(1-a) x^2(1-a) x(1-a) \\
= & (1-a) x^2 \cdot(1-a) x(1-a)=(1-a) x^2(1-a)=1
\end{aligned}
\]
因此有 $1-a$ 可逆且 $(1-a)^{-1}=x$

2)现在考虑 $(1-b)(1-a)$ ,则有 $(1-b)(1-a)=1-a-b+b a=1$ ,结合前面所证 $1-a$ 可逆,因此得 $(1-a)^{-1}=1-b$ .进而有
\[
1=(1-a)(1-b)=1-a-b+a b=1-b a+a b
\]
亦即 $a b=b a$ .

\end{proof}

\begin{exercise}[Kaplansky]
含幺环中某元若有多于一个右逆,则它必然有无限多个右逆.
\end{exercise}
\begin{proof}
设 $a\in R$ 有多于一个右逆,则它是左零因子,设理想
\[
I\coloneqq \{ x\in R:ax=0 \}
\]
只需要证明:$I$ 是无限集. 这是因为若 $b$ 是 $a$ 的一个右逆,则 $a(b+I)=0$,$b+x, \forall x\in I$ 也是 $a$ 的右逆.

考虑反证,假设 $I$ 是有限集 $\{x_0, x_1,\dots,x_n \}$,($x_0=0$) 于是 $ax_k=0, k\in \{ 1,\dots,n \}$. 考虑 $x_ka$,它们显然也在 $I$ 内,但 $x_ka\neq x_{l}a$,两两不同,故 $\{ x_1a,\dots,x_na \}$ 是 $\{ x_1,\dots,x_n \}$ 的一个置换.

取 $b$ 为 $a$ 的一个右逆,因为 $a$ 是零因子,故 $ba\neq1$(若 $b$ 是 $a$ 的左右逆,则 $a$ 为 unit,不是零因子)从而 $ba-1\neq0$,但是 $a(ba-1)=a-a=0$,故 $ba-1\in I$. 于是 $ba-1=x_i\neq0$ for some $i\neq0$. 同时 $x_ib=(ba-1)b=0$,$x_i=x_ja$ for some $j\neq0$. 于是
\[
0=x_ib=x_jab=x_j\neq 0
\]
矛盾!

\end{proof}

\begin{exercise}[注意环的乘法不包含可逆性]
设 $D$ 为整环,$m$ 和 $n$ 为互素的正整数,$a, b\in D$. 如果 $a^{m}=b^{m}$,$a^{n}=b^{n}$. 求证:$a=b$.
\end{exercise}
\begin{proof}
不妨设 $a\neq0,b\neq0$,由于 $(m,n)=1$,故存在整数 $s,t$ 使得 $sm+tn=1$,如果 $s\geq0$ 则 $t\leq0$ 故
\[
a\cdot a^{sm}=a\cdot b^{sm}=a\cdot b^{1-tn}=ba\cdot b^{-tn}=ba\cdot a^{-tn}= ba ^{1-tn}=b a^{sm}
\]
由于 $D$ 是整环,$(a-b)a^{sm}=0$,$a\neq0$ 不是零因子,所以 $a-b$ 是零因子,故为 0. 也就是 $a=b$.
\end{proof}
