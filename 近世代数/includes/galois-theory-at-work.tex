\section{Galois-theory-at-work}

\subsection{CMC 高}

\begin{exercise}
证明:
	\begin{enumerate}
		\item 存在恰有 $q$ 个元素的域 $\mathbb{F}_q$ ;
		\item 如果 $K$ 是一个 $q$ 元域,则 $K \cong \mathbb{F}_q$ ;
		\item 设 $L \supset \mathbb{F}_q$ 是有限域扩张,则映射 $\eta_q: L \rightarrow L, x \mapsto x^q$ 是域自同构,且满足
	\end{enumerate}
\[
\eta_q(a)=a, \quad \forall a \in \mathbb{F}_q ;
\]	\begin{enumerate}
		\item 域扩张 $L \supset \mathbb{F}_q$ 的 $\operatorname{Galois}$ 群 $\operatorname{Gal}\left(L / \mathbb{F}_q\right)$ 是由 $\eta_q$ 生成的循环群;
		\item $L^*=L \backslash\{0\}$ 关于 $L$ 中乘法成为一个循环群.特别地,存在 $\alpha \in L$ 使 $L=\mathbb{F}_q[\alpha]$ .
	\end{enumerate}
\end{exercise}
\begin{proof}

\begin{enumerate}
	\item Let $\mathbb{F}_q$ be the splitting field of the polynomial $f(x)=x^q-x \in \mathbb{F}_p[x]$. Since $f^{\prime}(x)=-1$, $f(x)=x^q-x$ has exactly $q$ roots in the splitting field $\mathbb{F}_q$, denoted as $R_f=\left\{\alpha_1, \cdots, \alpha_q\right\} \subset \mathbb{F}_q$. Directly verifying that $R_f$ is a subfield, we conclude that $R_f=\mathbb{F}_q$.
	\item If $\operatorname{Char}(K)=p_1,\left[K: \mathbb{F}_{p_1}\right]=m$, then $|K|=p_1^m=q$, which implies $p_1=p, m=n$. Thus, $K$ is also an $n$-th degree extension of $\mathbb{F}_p$, similar to $\mathbb{F}_q$. Since $K^*$ forms a group of order $q-1$ under multiplication, we have $u^{q-1}=1$ for all $u \in K^*$. This means that the field $K$ consists of the $q$ roots of $f(x)=x^q-x \in \mathbb{F}_p[x]$ in $K$. Therefore, $K$ and $\mathbb{F}_q$ are both splitting fields of the same polynomial, implying $K \cong \mathbb{F}_q$.
	\item From the proof of conclusion (1), we know that elements in $\mathbb{F}_q$ satisfy $x^q=x$, i.e., $\eta_q(a)=a\left(\forall a \in \mathbb{F}_q\right)$. Because $q \alpha=0(\forall \alpha \in L)$, we have $(\alpha+\beta)^q=\alpha^q+\beta^q(\forall \alpha, \beta \in L)$. It's easy to prove that $\eta_q$ is a field isomorphism.
	\item Let $L \supset \mathbb{F}_q$ be an $m$-th degree extension; then $L$ is a field with $q^m$ elements. From the proof of conclusion (2), we know that $L$ is a splitting field of the separable polynomial $f(x)=x^{q^m}-x \in \mathbb{F}_q[x]$, so $\left|\operatorname{Gal}\left(L / \mathbb{F}_q\right)\right|=\left[L: \mathbb{F}_q\right]=m$. On the other hand, for any $0<k<m, L \xrightarrow{\eta_q^k} L$ is not an identity map (otherwise, the polynomial $x^{q^k}-x$ would have $q^m$ roots in $L$, which contradicts $q^k<q^m$), so $\operatorname{Gal}\left(L / \mathbb{F}_q\right)$ is a cyclic group generated by $\eta_q$.
	\item Let $\alpha \in L^*$ be an element of maximal order $N$ in the finite abelian group $L^*$. We claim that
\[
\beta^N=1, \quad \forall \beta \in L^* .
\]Indeed, suppose the order of $\beta$ is $N^{\prime}$. If $\beta^N \neq 1$, then $N^{\prime} \nmid  N$. Let $d=\left(N, N^{\prime}\right)$ be the \textbf{greatest common divisor} of $N$ and $N^{\prime}$.
\[
d=p_1^{k_1} p_2^{k_2} \cdots p_s^{k_s}
\]is the irreducible factorization, $d_1=\prod_{p_i \nmid  \frac{N^{\prime}}{d}} p_i^{k_i}, d_2=\frac{d}{d_1}$. Then $\left(\frac{N^{\prime}}{d_1}, \frac{N}{d_2}\right)=1$, the order of $\beta^{d_1}$ is $\frac{N^{\prime}}{d_1}$, and the order of $\alpha^{d_2}$ is $\frac{N}{d_2}$. It is easy to prove that the order of $\alpha^{d_2} \beta^{d_1}$ is $\frac{N}{d_2} \cdot \frac{N^{\prime}}{d_1}=N \cdot \frac{N^{\prime}}{d}>N$, which contradicts the choice of $N$. Therefore, all elements in $L^*$ are roots of the equation $x^N-1=0$, so $\left|L^*\right|=q^m-1 \leq N$, which means $L^*$ is a cyclic group generated by $\alpha \in L^*$.
\end{enumerate}

\end{proof}

\subsection{\texorpdfstring{$\mathbb{Q}(\sqrt[4]{ 2 },\zeta_8)$}{mathbbQ(sqrt[4] 2 ,zeta_8)}}

\href{https://kconrad.math.uconn.edu/blurbs/galoistheory/galoisapp.pdf}{galoisapp.pdf}

$K = \mathbb{Q}(\sqrt[4]{ 2 }, \zeta_8)$ is a Galois extension of $\mathbb{Q}$ because:

\begin{enumerate}
	\item $\zeta_8 = e^{2\pi i / 8} = \frac{\sqrt{2}}{2} + \frac{\sqrt{2}}{2}i$, and $\sqrt{2} = \zeta_8 + \zeta_8^{-1}$, $i = \zeta_8^2$.
	\item $\sqrt[4]{ 2 }$ is a root of $x^4-2$, and all roots of $x^4-2$ are in $K$ because $i \in K$.
	\item $\zeta_8$ is a root of $x^4+1$, and all roots of $x^4+1$ are in $K$.
	\item $K$ is a splitting field of $(x^4-2)(x^4+1)$ over $\mathbb{Q}$, making it a normal extension.
	\item Since $\mathbb{Q}$ has characteristic 0, $K$ is a separable extension of $\mathbb{Q}$.
	\item Therefore, $K$ is a Galois extension of $\mathbb{Q}$.
\end{enumerate}

\begin{definition}[normal extension]
An algebraic field extension $E/F$ is a \textbf{normal extension} if every irreducible polynomial in $F[x]$ that has a root in $E$ factors completely into linear factors in $E[x]$.
\end{definition}
\begin{definition}[separable extension]
An algebraic extension $E/F$ is a \textbf{separable extension} if every element of $E$ is separable over $F$, i.e., is a root of a separable polynomial in $F[x]$.
\end{definition}
\begin{definition}[separable polynomial]
A polynomial is called \textbf{separable} if all its roots are distinct.
\end{definition}
\textbf{Task:} why splitting field of $(x^4-2)(x^4+1)$ over $\mathbb{Q}$ is separable.

\textbf{Output:}
The splitting field of $(x^4-2)(x^4+1)$ over $\mathbb{Q}$ is separable because we are in characteristic 0. More generally, if $K$ is a field of characteristic 0 and $f(x) \in K[x]$, then the splitting field of $f(x)$ over $K$ is separable. This is because every irreducible polynomial over a field of characteristic 0 is separable.

In this case, the splitting field $L$ is an extension of $\mathbb{Q}$. Since $\mathbb{Q}$ has characteristic 0, $L$ also has characteristic 0. The polynomials $x^4-2$ and $x^4+1$ have distinct roots. The roots of $x^4-2$ are $\sqrt[4]{2}, i\sqrt[4]{2}, -\sqrt[4]{2}, -i\sqrt[4]{2}$. The roots of $x^4+1$ are $e^{i\pi/4}, e^{i3\pi/4}, e^{i5\pi/4}, e^{i7\pi/4}$. Since these are all distinct, the extension is separable.

\begin{theorem}
Every irreducible polynomial over a field of characteristic 0 is \textbf{separable}.
\end{theorem}
\textbf{Counterexample:}

Let $K = \mathbb{F}_p(t)$ be the field of rational functions in one variable over $\mathbb{F}_p$. Then $f(x) = x^p - t \in K[x]$ is irreducible by Eisenstein's criterion. However, $f'(x) = px^{p-1} = 0$, so $\gcd(f(x), f'(x)) = f(x)$. Thus, $f(x)$ is not separable because a polynomial is separable if and only if it has no repeated roots in its splitting field, which is equivalent to the polynomial and its derivative being coprime (i.e., their greatest common divisor is 1). Since $\gcd(f(x), f'(x)) = f(x) \neq 1$, $f(x)$ has repeated roots and is therefore not separable.


The field extension $[\mathbb{Q}(\sqrt[4]{2}, \zeta_8):\mathbb{Q}]$ has degree 8, not 16, because $\mathbb{Q}(\sqrt[4]{2}, \zeta_8) = \mathbb{Q}(\sqrt[4]{2}, i)$. The Galois group $\operatorname{Gal}(\mathbb{Q}(\sqrt[4]{2}, \zeta_8) / \mathbb{Q})$ has at most 16 automorphisms, determined by $\sigma(\zeta_8) = \zeta_8^a$ and $\sigma(\sqrt[4]{2}) = i^b \sqrt[4]{2}$, but the choices of $a$ and $b$ are not independent. Since $\zeta_8 + \zeta_8^{-1} = \sqrt{2} = \sqrt[4]{2}^2$, $\mathbb{Q}(\sqrt{2})$ is a common subfield of both $\mathbb{Q}(\zeta_8)$ and $\mathbb{Q}(\sqrt[4]{2})$.

The effect of $\sigma \in \operatorname{Gal}(\mathbb{Q}(\sqrt[4]{2}, \zeta_8) / \mathbb{Q})$ on $\sqrt[4]{2}$ partially determines it on $\zeta_8$, and conversely: $(\sigma(\sqrt[4]{2}))^2 = \sigma(\zeta_8) + \sigma(\zeta_8)^{-1}$, which gives the relation
\begin{equation}
(-1)^b = \frac{\zeta_8^a + \zeta_8^{-a}}{\sqrt{2}}
\label{503e51}
\end{equation}

This tells us that if $a \equiv 1,7 \bmod 8$ then $(-1)^b=1$, so $b \equiv 0,2 \bmod 4$, while if $a \equiv 3,5 \bmod 8$ then $(-1)^b=-1$, so $b \equiv 1,3 \bmod 4$. For example, $\sigma$ can't both fix $\sqrt[4]{2}(b=0)$ and send $\zeta_8$ to $\zeta_8^3(a=3)$ because \cref{503e51}  would not hold.


\begin{theorem}
Let $L_1$ and $L_2$ be Galois over $K$. There is an injective homomorphism
\[
\operatorname{Gal}\left(L_1 L_2 / K\right) \hookrightarrow \operatorname{Gal}\left(L_1 / K\right) \times \operatorname{Gal}\left(L_2 / K\right)
\]given by $\sigma \mapsto\left(\left.\sigma\right|_{L_1},\left.\sigma\right|_{L_2}\right)$. In particular, if $L_1 / K$ and $L_2 / K$ are finite abelian extensions then $L_1 L_2$ is a finite abelian extension of $K$.
\end{theorem}
\begin{proof}
We omit this proof
\end{proof}

\begin{proposition}
Consider a finite extension $L / K . L / K$ is normal if and only if any of the following equivalent conditions hold:
	\begin{enumerate}
		\item Every irreducible polynomial in $K[x]$ that has at least one root in $L$ splits completely into linear factors over $L$.
		\item $L$ is the splitting field of some polynomial in $K[x]$.
		\item Every $K$ -embedding of $L$ into an algebraic closure $\bar{K}$ of $K$ has its image contained in $L$. That is, if $\sigma: L \hookrightarrow \bar{K}$ is a field homomorphism such that $\sigma(k)=k$ for all $k \in K$, then $\sigma(L) \subseteq L$.
	\end{enumerate}
\end{proposition}
\textbf{Task:} The Galois group of $\left(X^2-2\right)\left(X^2-3\right)$ over $\mathbf{Q}$ is $\mathbf{Z} / 2 \mathbf{Z} \times \mathbf{Z} / 2 \mathbf{Z}$. Its Galois group over $\mathbf{R}$ is trivial. Why trivial?

\textbf{Answer:}

The Galois group of a polynomial over a field $F$ is the group of automorphisms of the splitting field of the polynomial that fix $F$.

In this case, we are considering the polynomial $p(x) = (x^2 - 2)(x^2 - 3)$ over $\mathbf{R}$.
The roots of $x^2 - 2$ are $\pm\sqrt{2}$, and the roots of $x^2 - 3$ are $\pm\sqrt{3}$.  Since $\sqrt{2}$ and $\sqrt{3}$ are both real numbers, all four roots $\sqrt{2}, -\sqrt{2}, \sqrt{3}, -\sqrt{3}$ are in $\mathbf{R}$. Thus, the splitting field of $p(x)$ over $\mathbf{R}$ is $\mathbf{R}$ itself.

The Galois group consists of automorphisms of the splitting field that fix the base field. In this case, we seek automorphisms of $\mathbf{R}$ that fix $\mathbf{R}$. The only such automorphism is the identity map. Therefore, the Galois group is trivial.

\begin{figure}[H]
\centering
\includegraphics[width=\textwidth]{Galois-theory-at-work-2025041800.png}
% \caption{}
\label{}
\end{figure}


\begin{theorem}[Theorem 4.10]
Let $f(X) \in K[X]$ be a separable polynomial of degree $n$.
(a) If $f(X)$ is irreducible in $K[X]$ then its Galois group over $K$ has order divisible by $n$.
(b) The polynomial $f(X)$ is irreducible in $K[X]$ if and only if its Galois group over $K$ is a transitive subgroup of $S_n$.
\end{theorem}
\begin{proof}
(a) For a root $r$ of $f(X)$ in $K, [K(r): K]=n$ is a factor of the degree of the splitting field over $K$, which is the size of the Galois group over $K$.

(b) First suppose $f(X)$ is irreducible. For two roots $r_i$ and $r_j$ of $f(X)$, we can write $r_j=\sigma(r_i)$ for some $\sigma$ in the Galois group of $f(X)$ over $K$. Therefore the Galois group, as a subgroup of $S_n$, sends $i$ to $j$, so it is a transitive subgroup. Now suppose $f(X)$ is reducible (so $n \geq 2$). It is a product of distinct irreducibles since it is separable. Let $r_i$ and $r_j$ be roots of different irreducible factors of $f(X)$. These irreducible factors are the minimal polynomials of $r_i$ and $r_j$ over $K$. For any $\sigma$ in the Galois group of $f(X)$ over $K, \sigma(r_i)$ has the same minimal polynomial over $K$ as $r_i$, so we can't have $\sigma(r_i)=r_j$. Therefore, as a subgroup of $S_n$, the Galois group of $f(X)$ does not send $i$ to $j$, so it is not a transitive subgroup of $S_n$.
\end{proof}

\begin{definition}[transitive subgroup]
A subgroup $G$ of $S_n$ is called \textbf{transitive} if for every $i, j \in\{1,2, \ldots, n\}$, there exists a $g \in G$ such that $g(i)=j$.
\end{definition}
\begin{theorem}
Let $f(X) \in \mathbf{Q}[X]$ be an irreducible polynomial of prime degree $p$ with all but two roots in $\mathbf{R}$. The Galois group of $f(X)$ over $\mathbf{Q}$ is isomorphic to $S_p$.\label{1bb574}
\end{theorem}

\begin{proof}
Let $L=\mathbf{Q}\left(r_1, \ldots, r_p\right)$ be the splitting field of $f(X)$ over $\mathbf{Q}$. The permutations of the $r_i$ 's by $\operatorname{Gal}(L / \mathbf{Q})$ provide an embedding $\operatorname{Gal}(L / \mathbf{Q}) \hookrightarrow S_p$ and $\# \operatorname{Gal}(L / \mathbf{Q})$ is divisible by $p$ by Theorem 4.10, so $\operatorname{Gal}(L / \mathbf{Q})$ contains an element of order $p$ by Cauchy's theorem. In $S_p$, the only permutations of order $p$ are $p$ -cycles (why?). So the image of $\operatorname{Gal}(L / \mathbf{Q})$ in $S_p$ contains a $p$ -cycle.

We may take $L$ to be a subfield of $\mathbf{C}$, since $\mathbf{C}$ is algebraically closed. Complex conjugation restricted to $L$ is a member of $\operatorname{Gal}(L / \mathbf{Q})$. Since $f(X)$ has only two non-real roots by hypothesis, \textbf{complex conjugation transposes two of the roots of $f(X)$ and fixes the others}. Therefore $\operatorname{Gal}(L / \mathbf{Q})$ contains a transposition of the roots of $f(X)$. (This is the reason for the hypothesis about all but two roots being real.)

\textbf{We now show the only subgroup of $S_p$ containing a $p$ -cycle and a transposition is $S_p$}, so $\operatorname{Gal}(L / \mathbf{Q}) \cong S_p$. By suitable labeling of the numbers from 1 to $p$, we may let 1 be a number moved by the transposition, so our subgroup contains a transposition $\tau=(1 a)$. Let $\sigma$ be a $p$ -cycle in the subgroup. As a $p$ -cycle, $\sigma$ acts on $\{1,2, \ldots, p\}$ by a single orbit, so some $\sigma^i$ with $1 \leq i \leq p-1$ sends 1 to $a$ : $\sigma^i=(1 a \ldots)$. This is also a $p$ -cycle, because $\sigma^i$ has order $p$ in $S_p$ and all elements of order $p$ in $S_p$ are $p$ -cycles, so writing $\sigma^i$ as $\sigma$ and suitably reordering the numbers $2, \ldots, p$ (which replaces our subgroup by a conjugate subgroup), we may suppose our subgroup of $S_p$ contains the particular transposition (12) and the particular $p$ -cycle $(12 \ldots p)$. For $n \geq 2$, it is a theorem in group theory that the particular transposition (12) and $n$ -cycle $(12 \ldots n)$ generate $S_n$, so our subgroup is $S_p$.
\end{proof}

\textbf{Task:} Why? In $S_p$, the only permutations of order $p$ are $p$ -cycles

\textbf{Output:}
The order of a permutation written in cycle notation is the \textbf{least common multiple} of the lengths of its cycles. To get order $p$, where $p$ is prime, we need the least common multiple of the cycle lengths to be $p$. Since $p$ is prime, the cycle lengths must be $1$ or $p$. Since the lengths must add up to $p$, we must have one cycle of length $p$ and all the other cycles of length $1$. Thus, we have a $p$-cycle.

\textbf{Task:} why? in group theory that the particular transposition (12) and $n$ -cycle $(12 \ldots n)$ generate $S_n$.

\textbf{Output:}
The symmetric group $S_n$ is the group of all permutations of $n$ elements. We want to show that the transposition (1 2) and the $n$-cycle (1 2 ... $n$) generate $S_n$.

Let $\tau = (1\ 2)$ and $\sigma = (1\ 2\ \dots\ n)$. We want to show that any permutation in $S_n$ can be written as a product of $\tau$ and $\sigma$.

First, we can generate any transposition of the form $(1\ i)$ for $i = 2, \dots, n$. We have
\[
\sigma^{i-2} \tau \sigma^{-(i-2)} = (1\ i)
\]
For example,

\begin{itemize}
	\item $i = 2: \sigma^0 \tau \sigma^0 = (1\ 2)$
	\item $i = 3: \sigma^1 \tau \sigma^{-1} = (2\ 3)(1\ 2)(1\ n) \dots (2\ 3) = (1\ 3)$
	\item $i = 4: \sigma^2 \tau \sigma^{-2} = (3\ 4)(2\ 3)(1\ 2)(1\ n) \dots (3\ 4) = (1\ 4)$
\end{itemize}

So we can generate any transposition of the form $(1\ i)$.

Next, we can generate any transposition $(i\ j)$ for $1 \le i, j \le n$. We have
\[
(i\ j) = (1\ i)(1\ j)(1\ i)
\]
Since we can generate any $(1\ i)$ and $(1\ j)$, we can generate any $(i\ j)$.

Finally, since any permutation can be written as a product of transpositions, and we can generate any transposition $(i\ j)$ using $\tau$ and $\sigma$, we can generate any permutation in $S_n$ using $\tau$ and $\sigma$.

Therefore, the transposition (1 2) and the $n$-cycle (1 2 ... $n$) generate $S_n$.


\begin{example}[4.13]
The polynomial $X^3-X-1$ is irreducible in $\mathbf{Q}[X]$ since it is irreducible mod 2 or since it is a cubic without any rational roots. It has one real root (approximately 1.3247), and one root of a cubic is all but two roots, so its Galois group over $\mathbf{Q}$ is isomorphic to $S_3$.
\end{example}
\begin{example}[4.14]
The polynomials $X^3-3 X-1$ and $X^3-4 X-1$ are both irreducible in $\mathbf{Q}[X]$ since they are cubics without any rational roots. Each polynomial has three real roots (check!), so we can't use Theorem 4.11 to determine their Galois groups over $\mathbf{Q}$.
\end{example}
\begin{example}[4.15]
The quintic polynomial $X^5-X-1$ is irreducible in $\mathbf{Q}[X]$ since it is irreducible mod 3. It has one real root, so Theorem 4.11 does not tell us the Galois group.
\end{example}
\begin{example}[4.16]
The quintic polynomial $X^5-4 X-1$ is irreducible in $\mathbf{Q}[X]$ since it is irreducible mod 3. It has three real roots, which is all but two roots, so its Galois group over $\mathbf{Q}$ is isomorphic to $S_5$.
\end{example}
\subsection{The Galois Group of \texorpdfstring{$\mathbb{Q}(\sqrt[n]{ a },\zeta _n)$}{mathbbQ(sqrt[n] a ,zeta _n)} over \texorpdfstring{$\mathbb{Q}$}{mathbbQ}}

In general, if $\theta=\sqrt[n]{  a}$ with $n\in \mathbb{N}$ and $a\in \mathbb{Q}$ such that for any divisor $m$ of $n$, $\sqrt[m]{ a }\not\in \mathbb{Q}$, then the normal closure of $\mathbb{Q}(\theta)$ is $K=\mathbb{Q}(\theta,\zeta _n)$. The Galois group $\mathrm{Gal}\left( K/\mathbb{Q} \right)$ is always a subgroup of semidirect product $\mathbf{Z}_n\rtimes \mathbf{Z}_n^{\times}$.

In "majority" cases, the Galois group $\mathrm{Gal}\left( K/\mathbb{Q} \right)$ is isomorphic to $\mathbf{Z}_n\rtimes \mathbf{Z}_n^{\times}$.

\begin{note}
$\mathbf{Z}_n$ is a normal subgroup of $\mathbf{Z}_n\rtimes \mathbf{Z}^{\times}_n$, which corresponds to that $\mathbb{Q}(\zeta _n)$ is a Galois extension of $\mathbb{Q}$. On the other hand, $\mathbf{Z}^{\times}_n$ is not a normal subgroup and thus $\mathbb{Q}(\sqrt[n]{ a })$ is not normal over $\mathbb{Q}$.
\end{note}
\begin{definition}[transitive]
A subgroup $G$ of the symmetric group $S_n$ is said to be \textbf{transitive} if for every $i, j \in \{1, 2, \dots, n\}$, there exists a $g \in G$ such that $g(i) = j$. In other words, for any two elements in the set $\{1, 2, \dots, n\}$, there is a permutation in the subgroup that maps the first element to the second.
\end{definition}

\subsection{Discriminant}

\begin{definition}[discriminant]
For a nonconstant $f(X) \in K[X]$ of degree $n$ that factors over a splitting field as
\[
f(X)=c\left(X-r_1\right) \cdots\left(X-r_n\right),
\]the \textbf{discriminant} of $f(X)$ is defined to be
\[
\operatorname{disc} f=\prod_{i<j}\left(r_j-r_i\right)^2 .
\]
\end{definition}
The number disc $f$ is nonzero if $f(X)$ is separable and is 0 if $f(X)$ is not separable.

When $f$ is separable, $\mathrm{disc}\ f$ is a symmetric polynomial in the $r_i$ 's, so it is fixed by $\mathrm{Gal}(K(r_1,\dots,r_n)/K)$ and therefore $\mathrm{disc }\ f\in K$ by Galois theory.

\begin{theorem}
Let $K$ not have characteristic 2 and let $f(X)$ be a separable \underline{cubic} in $K[X]$ with a root $r$ and discriminant $D$. The splitting field of $f(X)$ over $K$ is $K(r, \sqrt{D})$.
Note we are not assuming $f(X)$ is irreducible here.
\end{theorem}
\begin{theorem}[Theorem 4.23]
Let $f(X) \in K[X]$ be a \textbf{separable polynomial} of degree $n$. If $K$ does not have characteristic 2, the embedding of the Galois group of $f(X)$ over $K$ into $S_n$ as permutations of the roots of $f(X)$ \underline{has image in $A_n$} if and only if disc $f$ is a square in $K$.
\end{theorem}
\begin{proof}
设 $\delta=\prod_{i<j}\left(r_j-r_i\right) \neq 0$,因此 $\delta \in K\left(r_1, \ldots, r_n\right)$ 且 $\delta^2=\operatorname{disc} f \in K$。因此,$\operatorname{disc} f$ 是 $K$ 中的一个平方,当且仅当 $\delta \in K$。

对于任何 $\sigma \in \operatorname{Gal}\left(K\left(r_1, \ldots, r_n\right) / K\right)$,设 $\varepsilon_\sigma= \pm 1$ 是其作为 $r_i$ 的排列的符号。根据排列符号的定义之一,
\[
\sigma(\delta)=\prod_{i<j}\left(\sigma\left(r_j\right)-\sigma\left(r_i\right)\right)=\varepsilon_\sigma \prod_{i<j}\left(r_j-r_i\right)=\varepsilon_\sigma \delta
\]
因此 $\sigma(\delta)= \pm \delta$。由于 $\delta \neq 0$ 且 $K$ 没有特征 $2, \delta \neq-\delta$。当且仅当 $\varepsilon_\sigma=1$ 时,我们有 $\sigma \in A_n$,因此当且仅当 $\sigma(\delta)=\delta$ 时,$\sigma \in A_n$。因此,$f(X)$ 在 $K$ 上的伽罗瓦群在 $A_n$ 中当且仅当 $\delta$ 被伽罗瓦群固定,这与 $\delta \in K$ 相同。
\end{proof}

\begin{theorem}[Theorem 4.25]
Let $K$ not have characteristic 2 and let $f(X)$ be a separable irreducible cubic in $K[X]$.
(a) If disc $f$ is a square in $K$ then the Galois group of $f(X)$ over $K$ is isomorphic to $A_3$.
(b) If disc $f$ is not a square in $K$ then the Galois group of $f(X)$ over $K$ is isomorphic to $S_3$.\label{a1e730}
\end{theorem}

\begin{remark}
If the characteristic of the field is 2, then $1=-1$, which means every permutation is even since $\mathrm{sgn}(\sigma)=\pm1=1$.
\end{remark}
\subsubsection{Discriminants in low-degree cases}

In low-degree cases, explicit formulas for discriminants of some trinomials are
\[
\begin{aligned}
\operatorname{disc}\left(X^2+a X+b\right) & =a^2-4 b, \\
\operatorname{disc}\left(X^3+a X+b\right) & =-4 a^3-27 b^2, \\
\operatorname{disc}\left(X^4+a X+b\right) & =-27 a^4+256 b^3, \\
\operatorname{disc}\left(X^5+a X+b\right) & =256 a^5+3125 b^4 .
\end{aligned}
\]
\subsection{Determine the Galois group}

\begin{theorem}[Dedekind]
Let $f(X) \in \mathbf{Z}[X]$ be monic irreducible over $\mathbf{Q}$ of degree $n$. For any prime $p$ not dividing disc $f$, let the monic irreducible factorization of $f(X) \bmod p$ be
\[
f(X) \equiv \pi_1(X) \cdots \pi_k(X) \bmod p
\]and set $d_i=\operatorname{deg} \pi_i(X)$, so $d_1+\cdots+d_k=n$. The Galois group of $f(X)$ over $\mathbf{Q}$, viewed as a subgroup of $S_n$, contains a permutation of type $\left(d_1, \ldots, d_k\right)$.\label{ef1229}
\end{theorem}

The nicest proof of \cref{ef1229}  uses algebraic number theory and is beyond the scope of these notes.

\begin{example}
We compute the Galois group of $X^4-X-1$ over $\mathbf{Q}$ using \cref{ef1229} .
This polynomial is irreducible $\bmod 2$, so it is irreducible over $\mathbf{Q}$. Let its roots be $r_1, r_2, r_3, r_4$. The extension $\mathbf{Q}\left(r_1\right) / \mathbf{Q}$ has degree 4 , so the Galois group of $X^4-X-1$ over $\mathbf{Q}$ has order divisible by 4 . Since the Galois group embeds into $S_4$, its size is either 4 , 8,12 , or 24 . The discriminant of $X^4-X-1$ is -283 , which is not a rational square, so the Galois group is not a subgroup of $A_4$. This eliminates the possibility of the Galois group having order 12 , because the only subgroup of $S_4$ with order 12 is $A_4$. (Quite generally, the only subgroup of index 2 in $S_n$ is $A_n$ for $n \geq 2$.) There are subgroups of $S_4$ with orders 4, 8 , and (of course) 24 outside of $A_4$, so no other size but 12 is eliminated yet.
Using \cref{ef1229}  with $p=7$,
\[
X^4-X-1 \equiv(X+4)\left(X^3+3 X^2+2 X+5\right) \bmod 7
\]This is an irreducible factorization, so \cref{ef1229}  says that the Galois group of $X^4-X-1$ over $\mathbf{Q}$ contains a permutation of the roots with cycle type $(1,3)$, which means the Galois group has order divisible by 3 , and that proves the Galois group is $S_4$.
\end{example}
\textbf{Task:} why implies irreducibility? We compute the Galois group of $X^4-X-1$ over $\mathbf{Q}$ using \cref{ef1229}. This polynomial is irreducible $\bmod 2$, so it is irreducible over $\mathbf{Q}$.

A polynomial $f(x)$ is \textbf{irreducible over $\mathbf{Q}$} if it cannot be factored into non-constant polynomials with rational coefficients. Reducing a polynomial modulo a prime $p$ can be used to prove irreducibility over $\mathbf{Q}$ because if $f(x)$ factors over $\mathbf{Q}$, then it must also factor modulo $p$ for almost all primes $p$. Thus, if $f(x) \bmod p$ is irreducible for some prime $p$, then $f(x)$ must be irreducible over $\mathbf{Q}$.

In this specific case, since $X^4 - X - 1$ is irreducible modulo 2, it follows that $X^4 - X - 1$ is irreducible over $\mathbf{Q}$. This is a consequence of the fact that if $X^4 - X - 1$ were reducible over $\mathbf{Q}$, then it would also be reducible modulo 2, which contradicts the given information that it is irreducible modulo 2.

\subsubsection{Example}

Let's determine the Galois group of $X^5-X-1$ over $\mathbf{Q}$, which was left unresolved in Example 4.15. Its irreducible factorization mod 2 is
\[
X^5-X-1=\left(X^2+X+1\right)\left(X^3+X^2+1\right) \bmod 2 .
\]
Because the polynomial is irreducible over $\mathbf{Q}, 5$ divides the size of the Galois group. From the mod 2 factorization, the Galois group contains a permutation of the roots with cycle type $(2,3)$, which has order 6 , so the Galois group has size divisible by $5 \cdot 6=30$. Since the Galois group is a subgroup of $S_5$, its size is either 30,60 , or 120 .

It turns out that there is no subgroup of $S_5$ with order 30 and the only subgroup of order 60 is $A_5$. The discriminant of $f(x)$ is $2869=19 \cdot 151$, which is not a rational square, so the Galois group is not in $A_5$ by Theorem 4.23. Therefore the Galois group is $S_5$.

\paragraph{Why divisible by \texorpdfstring{$30$}{30}}

We want to understand why the size of the Galois group $G$ of the polynomial $f(X) = X^5 - X - 1$ over the rational numbers $\mathbf{Q}$ must be divisible by $5 \times 6 = 30$.

This conclusion is derived from two key facts about the polynomial and fundamental theorems in Galois theory and group theory.

\begin{enumerate}
	\item Divisibility by 5
	\begin{itemize}
		\item \textbf{Reason:} The polynomial $f(X) = X^5 - X - 1$ is \textbf{irreducible} over $\mathbf{Q}$.
		\item \textbf{Theorem:} For an \textbf{irreducible} polynomial of degree $n$ over a field $F$, the order of the Galois group of its splitting field over $F$ is divisible by $n$. Since $f(X)$ is irreducible and has degree $n=5$, the size of its Galois group $G = \text{Gal}(K/\mathbf{Q})$ must be divisible by 5. Equivalently, the Galois group $G$ acts \textbf{transitively} on the 5 roots, so $|G|$ is divisible by 5.
	\end{itemize}
	\item Divisibility by 6
	\begin{itemize}
		\item \textbf{Reason:} The factorization of $f(X)$ modulo 2 is given as:
\[
X^5 - X - 1 \equiv (X^2 + X + 1)(X^3 + X^2 + 1) \pmod 2
\]Both factors, $X^2+X+1$ and $X^3+X^2+1$, are irreducible over the field $\mathbf{F}_2$.
		\item \textbf{Theorem (Dedekind's Theorem / Frobenius Elements):} If a polynomial with integer coefficients factors modulo a prime $p$ into distinct irreducible factors of degrees $n_1, n_2, \dots, n_k$, then its Galois group $G$ contains an element $\sigma$ having the cycle structure $(n_1, n_2, \dots, n_k)$. Thus, the Galois group $G$, seen as a subgroup of $S_5$, must contain a permutation $\sigma$ with cycle structure $(2, 3)$.
		\item The order of permutation $\sigma$ is $\text{lcm}(2, 3) = 6$. Since $G$ contains an element $\sigma$ of order 6, the order of the group $|G|$ must be divisible by 6.
	\end{itemize}
\end{enumerate}

\begin{theorem}[Lagrange's Theorem]
In any finite group, the order of any element must divide the order of the group.
\end{theorem}
\begin{proof}
By considering the partition of a finite group $G$ into disjoint left cosets of a subgroup $H=\left< g \right>$, we can deduce that the order of $H$ divides the order of $G$, and consequently, the order of any element $g$ in $G$ must also divide the order of $G$.
\end{proof}

\subsubsection{Some Strong Corollaries}

From \cref{1bb574} and \cref{ef1229}, we have

\begin{corollary}
Let $f(X) \in \mathbf{Z}[X]$ be monic irreducible over $\mathbf{Q}$ of prime degree $p$. If there is a prime number $\ell$ not dividing disc $f$ such that $f(X) \bmod \ell$ has all but two roots in $\mathbf{F}_{\ell}$, then the Galois group of $f(X)$ over $\mathbf{Q}$ is isomorphic to $S_p$.\label{8c6166}
\end{corollary}

\begin{proof}
The proof of  \cref{1bb574}  can be used again except for the step explaining why the Galois group of $f(X)$ over $\mathbf{Q}$ contains a transposition. In  \cref{1bb574}  this came from the use of complex conjugation to transpose two non-real roots, assuming there are only two non-real roots. We aren't assuming that anymore. By hypothesis the factorization of $f(X) \bmod \ell$ has all linear factors except for one quadratic irreducible factor. Therefore Theorem 4.29 says the Galois group contains a permutation of the roots with cycle type $(1,1, \ldots, 1,2)$, which is a transposition in $S_p$.
\end{proof}

If we seek an analogue of \cref{1bb574} for a Galois group to be isomorphic to $A_p$, using 3 -cycles in place of transpositions, there is no analogue since an irreducible polynomial over $\mathbf{Q}$ can't have all but three roots in $\mathbf{R}$ (the number of non-real roots is always even). But $f(X) \bmod \ell$ could have all but three roots in $\mathbf{F}_{\ell}$ for some $\ell$. This suggests the next result.

\begin{corollary}
Let $f(X) \in \mathbf{Z}[X]$ be monic irreducible over $\mathbf{Q}$ of prime degree $p \geq 3$ with disc $f$ a perfect square. If there is a prime number $\ell$ not dividing disc $f$ such that $f(X) \bmod \ell$ has all but three roots in $\mathbf{F}_{\ell}$, then the Galois group of $f(X)$ over $\mathbf{Q}$ is isomorphic to $A_p$.\label{0cc046}
\end{corollary}

\begin{proof}
Let $G$ be the Galois group, so $G$ is a subgroup of $A_p$ since $\operatorname{disc} f$ is a square. The Galois group has order divisible by $p$, so it contains a $p$ -cycle. From the factorization of $f(X) \bmod \ell$ and \cref{ef1229} , $G$ contains a 3 -cycle. It is a theorem of C. Jordan that for any prime $p \geq 3$, any $p$ -cycle and any 3 -cycle in $S_p$ generate $A_p$, so $G \cong A_p$.
\end{proof}

\begin{example}
The polynomial $X^5+20 X+16$ has discriminant $2^{16} 5^6$. It is irreducible $\bmod 3$, so it's irreducible over $\mathbf{Q}$. Modulo 7, its irreducible factorization is
\[
X^5+20 X+16 \equiv(X-4)(X-5)\left(X^3+2 X^2+5 X+5\right) \bmod 7 .
\]This has all but three roots in $\mathbf{F}_7$, so the Galois group of $X^5+20 X+16$ over $\mathbf{Q}$ is isomorphic to $A_5$.
\end{example}
\begin{remark}
It is a hard theorem of Chebotarev that the sufficient conditions for $f(X)$ to have Galois group $S_p$ in \cref{8c6166}  and $A_p$ in \cref{0cc046}  are also necessary in a strong sense: if $f(X) \in \mathbf{Z}[X]$ is monic irreducible of prime degree $p$ with Galois group over $\mathbf{Q}$ isomorphic to $S_p$ (resp., $A_p$ ) then there are infinitely many primes $\ell$ not dividing $\operatorname{disc} f$ such that
\end{remark}
\paragraph{Decomposing \texorpdfstring{$X^3 - X - 1 \pmod{5}$}{X^3 - X - 1 pmod5}}

To decompose $X^3 - X - 1 \pmod{5}$, we want to find its roots in $\mathbb{Z}_5 = \{0, 1, 2, 3, 4\}$. We can test each element:

\begin{itemize}
	\item For $x = 0$, $0^3 - 0 - 1 = -1 \equiv 4 \pmod{5}$.
	\item For $x = 1$, $1^3 - 1 - 1 = -1 \equiv 4 \pmod{5}$.
	\item For $x = 2$, $2^3 - 2 - 1 = 8 - 2 - 1 = 5 \equiv 0 \pmod{5}$.
	\item For $x = 3$, $3^3 - 3 - 1 = 27 - 3 - 1 = 23 \equiv 3 \pmod{5}$.
	\item For $x = 4$, $4^3 - 4 - 1 = 64 - 4 - 1 = 59 \equiv 4 \pmod{5}$.
\end{itemize}

Since $x = 2$ is a root, $(X - 2)$ is a factor. Then we can perform polynomial division to find the other factor.
Note that $-2 \equiv 3 \pmod{5}$, so we can write $(X - 2) = (X + 3)$.

$X^3 - X - 1 = (X - 2)(X^2 + 2X + 3)$ in $\mathbb{Z}_5[X]$.

Now we consider the quadratic $X^2 + 2X + 3 \pmod{5}$. We want to see if it can be factored further. We can check its discriminant:
$\Delta = b^2 - 4ac = 2^2 - 4(1)(3) = 4 - 12 = -8 \equiv 2 \pmod{5}$.
Since $2$ is not a quadratic residue modulo $5$ (i.e., there is no $x$ such that $x^2 \equiv 2 \pmod{5}$), the quadratic has no roots in $\mathbb{Z}_5$. The quadratic residues modulo 5 are $0^2 \equiv 0$, $1^2 \equiv 1$, $2^2 \equiv 4$, $3^2 \equiv 9 \equiv 4$, $4^2 \equiv 16 \equiv 1 \pmod{5}$.

Thus, $X^2 + 2X + 3$ is irreducible over $\mathbb{Z}_5$.

Therefore, the decomposition of $X^3 - X - 1 \pmod{5}$ is $(X - 2)(X^2 + 2X + 3)$.

\subsection{Methods to Verify Polynomial Irreducibility in a Field}

Verifying whether a polynomial is irreducible in a given field is a fundamental problem in algebra. Here are several methods and theorems that can be used:

\subsubsection{Eisenstein's Criterion}

\begin{theorem}[Eisenstein's Criterion]
Let $f(x) = a_n x^n + a_{n-1} x^{n-1} + \dots + a_1 x + a_0$ be a polynomial with integer coefficients. If there exists a prime number $p$ such that:
	\begin{itemize}
		\item $p \mid a_i$ for all $0 \leq i < n$,
		\item $p \nmid  a_n$,
		\item $p^2 \nmid  a_0$,
	\end{itemize}
then $f(x)$ is irreducible over $\mathbb{Q}$.
\end{theorem}
This criterion directly proves irreducibility over $\mathbb{Q}$ by checking divisibility conditions of the polynomial's coefficients by a prime number.

\subsubsection{Reduction Modulo \texorpdfstring{$p$}{p}}

\begin{itemize}
	\item Choose a prime $p$ such that the degree of the polynomial $f(x)$ with integer coefficients remains the same when reduced modulo $p$.
	\item If the reduced polynomial is irreducible over $\mathbb{Z}_p$, then $f(x)$ is irreducible over $\mathbb{Q}$.
	\item If the reduced polynomial is reducible, \underline{no conclusion} can be made about the irreducibility of $f(x)$ over $\mathbb{Q}$.
\end{itemize}

This method involves reducing the polynomial modulo a prime $p$ and checking for irreducibility in the finite field $\mathbb{Z}_p$.

\subsubsection{Rational Root Theorem}

\begin{itemize}
	\item For a polynomial $f(x) = a_n x^n + a_{n-1} x^{n-1} + \dots + a_0$ with integer coefficients, any rational root must have the form $\frac{p}{q}$, where $p$ divides $a_0$ and $q$ divides $a_n$.
	\item A polynomial of degree 2 or 3 with no rational roots is irreducible over $\mathbb{Q}$.
\end{itemize}

This theorem helps find potential rational roots, and if none exist for polynomials of degree 2 or 3, it confirms irreducibility.

\subsubsection{Adjoining Roots}

\begin{itemize}
	\item If $f(x)$ is a polynomial over a field $F$, and $\alpha$ is a root of $f(x)$ in some extension field, then $f(x)$ is irreducible if and only if $F(\alpha) = F[x]/(f(x))$ has degree equal to the degree of $f(x)$.
\end{itemize}

This method checks irreducibility by examining the degree of the field extension formed by adjoining a root of the polynomial.

\subsubsection{Polynomial Decomposition Algorithms}

\begin{itemize}
	\item Algorithms like Berlekamp's algorithm determine if a polynomial is irreducible over a finite field.
\end{itemize}

These algorithms provide computational methods to test for irreducibility, especially over finite fields.

\subsection{Powerful Theorems}

\subsubsection{Gauss's Lemma}

\begin{lemma}[Gauss's Lemma]
If a polynomial with integer coefficients can be factored into two polynomials with rational coefficients, then it can be factored into two polynomials with integer coefficients.
\end{lemma}
\paragraph{Proof Sketch:}

Suppose that $f(x)$ is a polynomial with integer coefficients. Assume that it can be factored into two polynomials with rational coefficients, say $f(x) = A(x)B(x)$, where $A(x)$ and $B(x)$ have rational coefficients.

\begin{enumerate}
	\item \textbf{Clear denominators:} Multiply $A(x)$ and $B(x)$ by suitable integers $a$ and $b$ respectively, such that $aA(x)$ and $bB(x)$ have integer coefficients. Thus, we have $ab f(x) = aA(x) \cdot bB(x)$, where $aA(x)$ and $bB(x)$ are polynomials with integer coefficients.
	\item \textbf{Remove common factors:} Divide $aA(x)$ and $bB(x)$ by the greatest common divisor of their coefficients. Let $A_1(x) = \frac{aA(x)}{c_1}$ and $B_1(x) = \frac{bB(x)}{c_2}$, where $c_1$ is the greatest common divisor of the coefficients of $aA(x)$ and $c_2$ is the greatest common divisor of the coefficients of $bB(x)$. Then $A_1(x)$ and $B_1(x)$ are primitive polynomials (polynomials with integer coefficients whose coefficients have a greatest common divisor of 1). We then have $ab f(x) = c_1 c_2 A_1(x) B_1(x)$.
	\item \textbf{Reduce to primitive polynomials:} Let $C = \frac{ab}{c_1 c_2}$. Then $C f(x) = A_1(x) B_1(x)$. Since $A_1(x)$ and $B_1(x)$ have integer coefficients, their product $A_1(x) B_1(x)$ also has integer coefficients. Furthermore, $f(x)$ has integer coefficients. Hence $C$ must be a rational number such that $C f(x)$ has integer coefficients.
	\item \textbf{Show $C$ is an integer:}  Since $A_1(x)$ and $B_1(x)$ are primitive polynomials, their product $A_1(x)B_1(x)$ is also a primitive polynomial. This is a key step and is often proven separately. Now, since $C f(x) = A_1(x) B_1(x)$ and $f(x)$ is a polynomial with integer coefficients, and $A_1(x)B_1(x)$ is primitive, it follows that $C$ must be an integer.
	\item \textbf{Conclude the factorization:} Since $C$ is an integer and $C f(x) = A_1(x) B_1(x)$, we have a factorization of $f(x)$ into two polynomials with integer coefficients: $f(x) = \frac{1}{C} A_1(x) B_1(x)$, where $\frac{1}{C} A_1(x)$ and $B_1(x)$ have integer coefficients after adjusting for the constant factor. We can then rewrite $f(x) = A_2(x) B_1(x)$ where $A_2(x) = \frac{1}{C}A_1(x)$.
\end{enumerate}

Thus, $f(x)$ can be factored into two polynomials with integer coefficients.

\begin{remark}
This lemma relates \textbf{irreducibility} over $\mathbb{Z}$ to \textbf{irreducibility} over $\mathbb{Q}$, ensuring that if a polynomial is irreducible over the integers, it is also irreducible over the rationals.
\end{remark}
\subsubsection{Hilbert's Irreducibility Theorem}

\begin{theorem}[Hilbert's Irreducibility Theorem]
If $f(x_1, \dots, x_n, y)$ is an irreducible polynomial in $\mathbb{Q}[x_1, \dots, x_n, y]$, then there exist infinitely many tuples $(a_1, \dots, a_n)$ of rational numbers such that $f(a_1, \dots, a_n, y)$ is irreducible in $\mathbb{Q}[y]$.
\end{theorem}
This theorem is used in advanced settings to prove the existence of irreducible polynomials by specializing variables.

\subsubsection{Capelli's Lemma}

\begin{lemma}[Capelli's Lemma]
Let $K$ be a field and $f(x)$ be an irreducible polynomial over $K$. Let $\alpha$ be a root of $f(x)$. If $g(x) \in K(\alpha)[x]$, then $g(x)$ is irreducible over $K(\alpha)$ if and only if $f(x)$ is irreducible over $K(\alpha)$.
\end{lemma}
Given an irreducible polynomial $f(x)$ over a field $K$ and a root $\alpha$ of $f(x)$, this lemma relates the irreducibility of another polynomial $g(x)$ over $K(\alpha)$ to the irreducibility of $f(x)$ over $K(\alpha)$.

\subsubsection{Irreducibility over Finite Fields}

\begin{theorem}[Irreducibility over Finite Fields]
Let $f(x)$ be a polynomial of degree $n$ over $\mathbb{F}_q$. Then $f(x)$ is irreducible if and only if $f(x)$ divides $x^{q^{n}}-x$ and $\gcd(f(x), x^{q^{k}}-x)=1$ for all $k$ with $1\leq k < n$.
\end{theorem}
A polynomial $f(x)$ of degree $n$ over a finite field $\mathbb{F}_q$ is irreducible if and only if $f(x)$ divides $x^{q^n} - x$ and $\gcd(f(x), x^{q^k} - x) = 1$ for all $k$ with $1 \leq k < n$.
