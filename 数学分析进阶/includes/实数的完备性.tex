\section{实数的完备性}

44.证明下列命题是等价的:
(I)Dedekind 分割原理;
(II)确界存在原理;
(III)单调有界原理;
(IV)区间套定理;
(V)有限覆盖定理;
(VI)聚点原理;
(VII)有界数列必有收敛子列;
(VIII)Cauchy 收敛准则.

证明:

$(\mathrm{I}) \Rightarrow(\mathrm{II})$:只要证明,若数集 $D$ 有上界,则 $D$ 必有唯一的上确界。
$1^{\circ}$ 若 $D$ 有最大值 $M$,则 $M$ 就是 $D$ 的上确界。
$2^{\circ}$ 设 $D$ 为无限集且无最大值。做实数集 $\mathbb{R}$ 的分割 $(X, Y)$,其中 $Y$ 为 $D$ 的一切上界所组成之集,$X$ 为 $Y$ 的补集。于是
(i)$X \neq \emptyset, Y \neq \emptyset$,
(ii)$X \cap Y=\emptyset$,
(iii)任取 $x \in X$,若存在 $y \in Y$ 使 $y \leqslant x$,则 $x$ 成为 $D$ 的一个上界,从而 $x \in Y$,这与(ii)矛盾。故对任意 $y \in Y, x \in X$,均有 $x<y$。

因此,$(X, Y)$ 是一个 Dedekind 分割。由分割原理,存在唯一 $\alpha \in \mathbb{R}$,使对任意 $x \in X, y \in Y$ 有
\[
x \leqslant \alpha<y \text { 或 } x<\alpha \leqslant y \text{.}
\]
因 $\alpha$ 不可能是 $X$ 的最大值(若 $\alpha$ 是 $X$ 的最大值,则 $\alpha \in Y$,从而 $X \cap Y \neq \emptyset$,矛盾!)故对一切 $x \in X$,均有 $x<\alpha$,即 $x \leqslant \alpha<y$ 不能成立。这样,只有 $x<\alpha \leqslant y$ 成立,即 $\alpha$ 是 $Y$ 中的最小者,也就是 $D$ 的上界中的最小者,故 $D$ 有唯一的上确界 $\alpha$。

$(\mathrm{II}) \Rightarrow(\mathrm{III})$:不妨假设 $\left\{x_n\right\}$ 单调增加且有上界。据确界存在原理,$\left\{x_n\right\}$ 必有上确界 $a$,即
\[
\sup \left\{x_n\right\}=a
\]
兹证 $a$ 就是 $\left\{x_n\right\}$ 当 $n \rightarrow \infty$ 时的极限。
事实上,由上确界定义知对任意 $\varepsilon>0$,存在 $n_0$,使
\[
a-\varepsilon<x_{n_0} \leqslant a
\]
再据 $\left\{x_n\right\}$ 单调增加,可知对任何 $n>n_0$ 均有
\[
x_{n_0} \leqslant x_n \leqslant a
\]
因此,当 $n>n_0$ 时有 $\left|x_n-a\right|<\varepsilon$,即
\[
\lim_{n \rightarrow \infty} x_n=a
\]
$(\mathrm{III}) \Rightarrow(\mathrm{IV})$:设闭区间列 $\left\{\left[a_n, b_n\right]\right\}$ 满足条件:
(i)$\left[a_{n+1}, b_{n+1}\right] \subset\left[a_n, b_n\right](n=1,2, \cdots)$,
(ii)$\lim_{n \rightarrow \infty}\left(b_n-a_n\right)=0$。
由(i)知数列 $\left\{a_n\right\}$ 单调增加且有上界,故 $\left\{a_n\right\}$ 收敛。令
\[
\lim_{n \rightarrow \infty} a_n=\sup \left\{a_n\right\}=a \geqslant a_n \quad(n=1,2, \cdots)
\]
又因对每一 $n$, $b_n$ 是 $\left\{a_n\right\}$ 的上界,且 $a$ 是 $\left\{a_n\right\}$ 的最小上界,故
\[
a \leqslant b_n \quad(n=1,2, \cdots).
\]
由(1),(2)可知 $a_n \leqslant a \leqslant b_n(n=1,2, \cdots)$,即
\[
a \in\left[a_n, b_n\right] \quad(n=1,2, \cdots).
\]
若还存在实数 $b$ 使
\[
b \in\left[a_n, b_n\right] \quad(n=1,2, \cdots),
\]
则 $0 \leqslant|a-b| \leqslant b_n-a_n \rightarrow 0(n \rightarrow \infty)$,故 $a=b$。

$(\mathrm{IV}) \Rightarrow(\mathrm{V})$:设 $\mathscr{D}$ 是闭区间 $[a, b]$ 的一个开覆盖。假如 $[a, b]$ 不能被 $\mathscr{D}$ 中任何有限个开集覆盖,将 $[a, b]$ 等分为两个区间,则其中至少有一个区间不能被 $\mathscr{D}$ 中任何有限个开集覆盖,记此区间为 $\left[a_1, b_1\right]$。再等分 $\left[a_1, b_1\right]$,同样至少有一个不能被 $\mathscr{D}$ 中任何有限个开集覆盖,记此区间为 $\left[a_2, b_2\right]$。如此继续下去,得到一列闭区间
\[
\left[a_1, b_1\right],\left[a_2, b_2\right], \cdots,\left[a_n, b_n\right], \cdots,
\]
适合
(i)任何一个 $\left[a_n, b_n\right]$ 都不能被 $\mathscr{D}$ 中任何有限个开集覆盖。
(ii)$\left[a_{n+1}, b_{n+1}\right] \subset\left[a_n, b_n\right] \subset[a, b] \quad(n=1,2, \cdots)$。
(iii)$\lim_{n \rightarrow \infty}\left(b_n-a_n\right)=0$。
据区间套定理,存在唯一实数 $c \in\left[a_n, b_n\right](n=1,2, \cdots)$,且
\[
\lim_{n \rightarrow \infty} a_n=\lim_{n \rightarrow \infty} b_n=c
\]
因 $\mathscr{D}$ 覆盖了 $[a, b]$,故 $\mathscr{D}$ 中至少有一个开集从而至少有一个开区间 $(\alpha, \beta)$,使得
\[
c \in(\alpha, \beta)
\]
由极限性质知存在 $n_0$,当 $n>n_0$ 时有
\[
\alpha<a_n<b_n<\beta,
\]
即 $\left[a_n, b_n\right] \subset(\alpha, \beta)$。因此,$(\alpha, \beta)$ 覆盖了 $\left[a_n, b_n\right]\left(n>n_0\right)$。这与(i)发生矛盾。

$(\mathrm{V}) \Rightarrow(\mathrm{VI})$:设 $D$ 为有界的无限集。令
\[
a=\inf D, \quad b=\sup D,
\]
则 $D \subset[a, b]$。假如 $D$ 没有聚点,那么对任意 $x \in[a, b]$,存在 $x$ 的邻域 $U\left(x, \delta_x\right)=\left(x-\delta_x, x+\delta_x\right)$,使 $U\left(x, \delta_x\right)$ 中至多含有 $D$ 的有限个点,即 $U\left(x, \delta_x\right) \cap D$ 是有限集。显然,当 $x$ 走遍 $[a, b]$ 时,这些邻域就覆盖了 $[a, b]$,即
\[
[a, b] \subset \bigcup_{x \in[a, b]} U\left(x, \delta_x\right)
\]
据有限覆盖定理,存在有限个邻域 $U\left(x_1, \delta_1\right), \cdots, U\left(x_n, \delta_n\right)$,它们足以覆盖 $[a, b]$,即
\[
[a, b] \subset \bigcup_{i=1}^n U\left(x_i, \delta_i\right)
\]
从而 $D \subset \bigcup_{i=1}^n U\left(x_i, \delta_i\right)$。由此得到 $D$ 为有限集,此为矛盾。因此,$D$ 必有聚点。

$(\mathrm{VI})\Rightarrow(\mathrm{VII})$:设 $\left\{x_n\right\}$ 是有界无穷数列。若 $\left\{x_n\right\}$ 是由有限个实数重复出现而构成的数列,则至少有一个数 $c$ 要重复出现无穷多次。设 $c$ 重复出现的项为 $n_1, n_2, \cdots$,则
\[
\lim_{k \rightarrow \infty} x_{n_k}=c
\]
即 $\left\{x_{n_k}\right\}$ 是 $\left\{x_n\right\}$ 的一个收敛子列。
现设 $\left\{x_n\right\}$ 确由无穷多个不同的实数组成,则此数列为一有界无穷集。据聚点原理,$\left\{x_n\right\}$ 至少有一聚点 $c$。于是,对任何 $k$,$\left(c-\frac{1}{k}, c+\frac{1}{k}\right)$ 中必含有 $\left\{x_n\right\}$ 的无穷多项,从而在 $\left(c-\frac{1}{k}, c+\frac{1}{k}\right)$ 中可以选出 $\left\{x_n\right\}$ 的一个项 $x_{n_k}$ 使 $x_{n_k} \neq c$。因 $k$ 是任意正整数,故得 $\left\{x_n\right\}$ 的一个子列 $\left\{x_{n_k}\right\}$,使得 $x_{n_k} \in\left(c-\frac{1}{k}, c+\frac{1}{k}\right)$,因而
\[
\lim_{k \rightarrow \infty} x_{n_k}=c
\]
$(\mathrm{VII}) \Rightarrow(\mathrm{VIII})$:必要性是显然的。
兹证充分性。据条件,对 $\varepsilon=1$,存在 $n_0$,当 $n, m>n_0$ 时有
\[
\left|x_n-x_m\right|<1
\]
于是,
\[
\left|x_n\right| \leqslant\left|x_n-x_{n_0+1}\right|+\left|x_{n_0+1}\right|<1+\left|x_{n_0+1}\right|.
\]
令
\[
M=\max \left\{\left|x_1\right|, \cdots,\left|x_{n_0}\right|, 1+\left|x_{n_0+1}\right|\right\},
\]
则 $\left|x_n\right| \leqslant M(n=1,2, \cdots)$,故 $\left\{x_n\right\}$ 有界。因此存在收敛子列 $\left\{x_{n_k}\right\}$,设
\[
\lim_{k \rightarrow \infty} x_{n_k}=c.
\]
于是由不等式
\[
\left|x_n-c\right| \leqslant\left|x_n-x_{n_k}\right|+\left|x_{n_k}-c\right|
\]
可知,$\lim_{n \rightarrow \infty} x_n=c$。

$(\mathrm{VIII}) \Rightarrow(\mathrm{I})$:设 $(X, Y)$ 是全体实数 $\mathbb{R}$ 的任意一个分割。因 $X \neq \emptyset, Y \neq \emptyset$,故可任取 $a_1 \in X, b_1 \in Y$,则 $b_1>a_1$。将 $\left[a_1, b_1\right]$ 等分为二,若分点 $\frac{a_1+b_1}{2} \in X$ 就取右半区间并记为 $\left[a_2, b_2\right]$;若 $\frac{a_1+b_1}{2} \in Y$,则取左半区间并记为 $\left[a_2, b_2\right]$。总之,$a_2 \in X, b_2 \in Y$。如此继续下去,得到闭区间列 $\left\{\left[a_n, b_n\right]\right\}$,满足

(i)$\left[a_{n+1}, b_{n+1}\right] \subset\left[a_n, b_n\right](n=1,2, \cdots)$,
(ii)$\lim_{n \rightarrow \infty}\left(b_n-a_n\right)=0$,
(iii)$a_n \in X, b_n \in Y(n=1,2, \cdots)$。
由(i),(ii)可知数列
\[
a_1, b_1, a_2, b_2, \cdots, a_n, b_n, \cdots
\]
是 Cauchy 数列,因而收敛。设其极限为 $c \in \mathbb{R}$。若 $c \in X$,可证 $c$ 必为 $X$ 的最大值。事实上,假如存在 $x \in X$ 而有 $c<x$,取正数 $\varepsilon=x-c$,则
\[
(c-\varepsilon, c+\varepsilon) \subset X
\]
由(iii),每个 $b_n \notin(c-\varepsilon, c+\varepsilon)$,这与 $a_1, b_1, a_2, b_2, \cdots, a_n, b_n, \cdots$ 收敛于 $c$ 发生矛盾。因此,$c$ 为 $X$ 的最大值。此时 $Y$ 显然无最小值。类似地可证,若 $c$ 在 $Y$ 中,则 $c$ 必是 $Y$ 的最小值,此时 $X$ 无最大值。

注:本题中八个等价的命题从不同的角度刻画了实数的连续性(又称完备性)。
