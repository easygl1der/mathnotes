\section{数值积分误差估计}

好的,这是一份关于数值积分主要公式的备考资料,包括公式、误差分析、代数精度以及简要的推导思路。

数值积分备考资料

\subsection{引言}

数值积分(Numerical Quadrature)用于逼近定积分 $\int_a^b f(x) \, dx$ 的值,特别是当被积函数 $f(x)$ 的原函数很难或无法用初等函数表达时,或者当 $f(x)$ 是以离散数据点形式给出时。

\subsection{一、牛顿-柯特斯公式 (Newton-Cotes Formulas)}

牛顿-柯特斯公式的基本思想是用一个次数为 $n$ 的插值多项式 $P_n(x)$ 来近似 $f(x)$, 然后计算插值多项式的积分。
\[
\int_a^b f(x) \, dx \approx \int_a^b P_n(x) \, dx
\]
节点 $x_i$ 在积分区间 $[a,b]$ 内等距分布。

\subsubsection{1. 梯形公式 (Trapezoidal Rule)}

使用1次插值多项式(连接 $(a, f(a))$ 和 $(b, f(b))$ 的直线)。

\paragraph{基本梯形公式 (2个节点: \texorpdfstring{$x_0=a, x_1=b$}{x_0=a, x_1=b})}

\begin{itemize}
	\item \textbf{估计值:}
\[
T(f) = \frac{b-a}{2} [f(a) + f(b)]
\]	\item \textbf{误差分析 (截断误差):}
\[
E_T(f) = \int_a^b f(x) \, dx - T(f) = -\frac{(b-a)^3}{12} f''(\xi), \quad \xi \in (a,b)
\]	\item \textbf{代数精度:} 1 (对于次数 $\le 1$ 的多项式,结果精确)
	\item \textbf{简要推导思路:}
	\begin{enumerate}
		\item 用线性插值多项式 $P_1(x) = f(a) + \frac{f(b)-f(a)}{b-a}(x-a)$ 近似 $f(x)$。
		\item 计算 $\int_a^b P_1(x) \, dx$ 得到 $T(f)$。
		\item 误差项可以通过泰勒展开或插值多项式的误差公式推导。对于 $f(x) - P_1(x) = \frac{f''(\eta_x)}{2!}(x-a)(x-b)$,积分后得到。
	\end{enumerate}
\end{itemize}

\paragraph{复合梯形公式 (n+1个节点, n个子区间)}

将 $[a,b]$ 分成 $n$ 个等长的子区间 $[x_i, x_{i+1}]$, 每个子区间长度 $h = \frac{b-a}{n}$,$x_i = a+ih$。

\begin{itemize}
	\item \textbf{估计值:}
\[
T_n(f) = h \left[ \frac{1}{2}f(x_0) + \sum_{i=1}^{n-1} f(x_i) + \frac{1}{2}f(x_n) \right]
\]	\item \textbf{误差分析:}
\[
E_{T_n}(f) = -\frac{(b-a)h^2}{12} f''(\xi) = -\frac{(b-a)^3}{12n^2} f''(\xi), \quad \xi \in (a,b)
\]	\item \textbf{简要推导思路:}
将各子区间上的基本梯形公式误差累加起来。$\sum_{i=0}^{n-1} -\frac{h^3}{12} f''(\xi_i) = -\frac{h^3}{12} \sum f''(\xi_i)$。根据中值定理,存在 $\xi \in (a,b)$ 使得 $n f''(\overline{\xi}) = \sum f''(\xi_i)$,则误差为 $-\frac{h^3}{12} n f''(\overline{\xi}) = -\frac{(b-a)h^2}{12}f''(\overline{\xi})$。
\end{itemize}

\subsubsection{2. 辛普森公式 (Simpson's Rule)}

使用2次插值多项式。通常考虑区间 $[a,b]$ 和中点 $m=(a+b)/2$。

\paragraph{基本辛普森公式 (3个节点: \texorpdfstring{$x_0=a, x_1=(a+b)/2, x_2=b$}{x_0=a, x_1=(a+b)/2, x_2=b})}

令 $h = (b-a)/2$(即每个小子区间的长度)。

\begin{itemize}
	\item \textbf{估计值:}
\[
S(f) = \frac{h}{3} [f(a) + 4f(\frac{a+b}{2}) + f(b)] = \frac{b-a}{6} [f(a) + 4f(\frac{a+b}{2}) + f(b)]
\]	\item \textbf{误差分析:}
\[
E_S(f) = \int_a^b f(x) \, dx - S(f) = -\frac{h^5}{90} f^{(4)}(\xi) = -\frac{(b-a)^5}{2880} f^{(4)}(\xi), \quad \xi \in (a,b)
\]	\item \textbf{代数精度:} 3 (对于次数 $\le 3$ 的多项式,结果精确)。比预期的2次要高,这是因为选取节点的方式对称,使得三次项的误差积分为0。
	\item \textbf{简要推导思路:}
	\begin{enumerate}
		\item 用过点 $(a, f(a))$, $(\frac{a+b}{2}, f(\frac{a+b}{2}))$, $(b, f(b))$ 的二次插值多项式 $P_2(x)$ 近似 $f(x)$。
		\item 计算 $\int_a^b P_2(x) \, dx$ 得到 $S(f)$。
		\item 误差项可以通过更高阶的泰勒展开(Peano核方法)或插值多项式误差 $\int_a^b \frac{f^{(3)}(\eta_x)}{3!} (x-a)(x-\frac{a+b}{2})(x-b)dx$ 推导,但更标准的方法是使用Hermite插值或特定技巧显示其为 $f^{(4)}$。一个常用方法是先对 $f(x)$ 在 $(a+b)/2$ 处泰勒展开到4阶。
	\end{enumerate}
\end{itemize}

\paragraph{复合辛普森公式 (n个子区间,n必须为偶数, n+1个节点)}

将 $[a,b]$ 分成 $n$ 个等长的子区间($n$ 为偶数),每个子区间长度 $h = \frac{b-a}{n}$,$x_i = a+ih$。相当于应用 $n/2$ 次基本辛普森公式。

\begin{itemize}
	\item \textbf{估计值:}
\[
S_n(f) = \frac{h}{3} [f(x_0) + 4f(x_1) + 2f(x_2) + 4f(x_3) + \dots + 2f(x_{n-2}) + 4f(x_{n-1}) + f(x_n)]
\]	\item \textbf{误差分析:}
\[
E_{S_n}(f) = -\frac{(b-a)h^4}{180} f^{(4)}(\xi) = -\frac{(b-a)^5}{180n^4} f^{(4)}(\xi), \quad \xi \in (a,b)
\]	\item \textbf{简要推导思路:}
\end{itemize}

累加 $n/2$ 个基本辛普森公式的误差。每个基本辛普森公式作用的区间长度是 $2h$。所以单个误差是 $-\frac{(2h)^5}{2880}f^{(4)}(\xi_i) = -\frac{h^5}{90}f^{(4)}(\xi_i)$。总共 $n/(2h) \times (Length Of One Simpson Application) = (b-a)/(2h)$ 个这样的应用。

总误差是
\[
\sum_{k=1}^{n/2} -\frac{(2h)^5}{2880} f^{(4)}(\xi_k) = \frac{n}{2} \left( -\frac{32h^5}{2880} f^{(4)}(\overline{\xi}) \right) = \frac{n}{2} \left( -\frac{h^5}{90} f^{(4)}(\overline{\xi}) \right) = -\frac{nh^5}{180} f^{(4)}(\overline{\xi}) = -\frac{(b-a)h^4}{180} f^{(4)}(\overline{\xi})
\]
\subsection{二、高斯求积公式 (Gaussian Quadrature)}

高斯求积公式的目标是选择节点 $x_i$ 和权重 $w_i$ 使得求积公式对于尽可能高次的多项式是精确的。
对于 $n$ 个节点,高斯求积公式可以达到 $2n-1$ 的代数精度。
通用形式:
\[
\int_a^b \omega(x) f(x) \, dx \approx \sum_{i=1}^n w_i f(x_i)
\]
其中 $\omega(x)$ 是权函数。节点 $x_i$ 是与权函数 $\omega(x)$ 在区间 $[a,b]$ 上相关的 $n$ 次正交多项式的根。

\begin{itemize}
	\item 高斯-勒让德求积 (Gauss-Legendre Quadrature)
这是最常见的高斯求积,其中权函数 $\omega(x)=1$,积分区间通常标准化为 $[-1, 1]$。
\[
\int_{-1}^1 f(x) \, dx \approx \sum_{i=1}^n w_i f(x_i)
\]	\begin{itemize}
		\item \textbf{节点 $x_i$:} 是 $n$ 次勒让德多项式 $P_n(x)$ 的根。
		\item \textbf{权重 $w_i$:} $w_i = \frac{2}{(1-x_i^2)[P_n'(x_i)]^2}$。
		\item \textbf{估计值 (例子):}
		\begin{itemize}
			\item $n=1$: $x_1=0, w_1=2$. $\int_{-1}^1 f(x)dx \approx 2f(0)$ (中点公式)
			\item $n=2$: $x_{1,2}=\pm \frac{1}{\sqrt{3}}, w_{1,2}=1$. $\int_{-1}^1 f(x)dx \approx f(-\frac{1}{\sqrt{3}}) + f(\frac{1}{\sqrt{3}})$
			\item $n=3$: $x_1=0, x_{2,3}=\pm\sqrt{\frac{3}{5}}$. $w_1=\frac{8}{9}, w_{2,3}=\frac{5}{9}$. $\int_{-1}^1 f(x)dx \approx \frac{8}{9}f(0) + \frac{5}{9}f(-\sqrt{\frac{3}{5}}) + \frac{5}{9}f(\sqrt{\frac{3}{5}})$
		\end{itemize}
		\item \textbf{误差分析:}
\[
E_G(f) = \frac{2^{2n+1}(n!)^4}{(2n+1)[(2n)!]^3} f^{(2n)}(\xi) = \frac{f^{(2n)}(\xi)}{(2n)!} \int_{-1}^1 \prod_{i=1}^n (x-x_i)^2 dx, \quad \xi \in (-1,1)
\](对于一般的区间 $[a,b]$,需要做线性变换 $x = \frac{b-a}{2}t + \frac{a+b}{2}$,然后误差项会包含 $(\frac{b-a}{2})^{2n+1}$ 因子)
$E_G(f) = \frac{(b-a)^{2n+1}(n!)^4}{(2n+1)[(2n)!]^3 (\text{const})} f^{(2n)}(\xi)$
		\item \textbf{代数精度:} $2n-1$
		\item \textbf{简要推导思路:}
		\begin{enumerate}
			\item 目标:选择 $n$ 个节点 $x_i$ 和 $n$ 个权重 $w_i$ (共 $2n$ 个自由参数),使得公式对次数尽可能高的多项式精确。
			\item 令 $f(x)$ 是一个次数 $\le 2n-1$ 的多项式。作带余除法 $f(x) = q(x) \phi_n(x) + r(x)$,其中 $\phi_n(x)$ 是 $n$ 次正交多项式(其根为 $x_i$),$\deg(q), \deg(r) \le n-1$。
			\item $\int_a^b \omega(x)f(x)dx = \int_a^b \omega(x)q(x)\phi_n(x)dx + \int_a^b \omega(x)r(x)dx = \int_a^b \omega(x)r(x)dx$ (由于 $\phi_n(x)$ 与所有次数 < n 的多项式正交)。
			\item $\sum w_i f(x_i) = \sum w_i (q(x_i)\phi_n(x_i) + r(x_i)) = \sum w_i r(x_i)$ (因为 $x_i$ 是 $\phi_n(x)$ 的根)。
			\item 问题转化为选择 $w_i$ 使得 $\int_a^b \omega(x)r(x)dx = \sum w_i r(x_i)$ 对所有次数 $\le n-1$ 的 $r(x)$ 精确。这可以通过构造一个 $n-1$ 次的插值多项式 $L(x)$ 使得 $L(x_i)=r(x_i)$,然后令 $w_i = \int_a^b \omega(x) l_i(x) dx$ (其中 $l_i(x)$ 是拉格朗日基函数)。
		\end{enumerate}
	\end{itemize}
\end{itemize}

\subsection{三、切比雪夫求积公式 (Chebyshev Quadrature)}

高斯-切比雪夫求积 (Gauss-Chebyshev Quadrature)

这是高斯求积的一种特殊情况,权函数 $\omega(x) = \frac{1}{\sqrt{1-x^2}}$,积分区间 $[-1,1]$。
\[
\int_{-1}^1 \frac{f(x)}{\sqrt{1-x^2}} \, dx \approx \sum_{i=1}^n w_i f(x_i)
\]
\begin{itemize}
	\item \textbf{节点 $x_i$}: 是 $n$ 次第一类切比雪夫多项式 $T_n(x)$ 的根: $x_i = \cos\left(\frac{(2i-1)\pi}{2n}\right)$, for $i=1, \dots, n$.
\end{itemize}

\begin{itemize}
	\item \textbf{权重 $w_i$}: 非常特殊,所有权重都相等 $w_i = \frac{\pi}{n}$.
	\item \textbf{估计值:}
\[
\int_{-1}^1 \frac{f(x)}{\sqrt{1-x^2}} \, dx \approx \frac{\pi}{n} \sum_{i=1}^n f\left(\cos\left(\frac{(2i-1)\pi}{2n}\right)\right)
\]	\item \textbf{误差分析:}
\[
E_{GC}(f) = \frac{2\pi}{2^{2n}(2n)!} f^{(2n)}(\xi), \quad \xi \in (-1,1)
\]	\item \textbf{代数精度:} $2n-1$
	\item \textbf{简要推导思路:} 同高斯求积一般思路,使用切比雪夫多项式作为正交多项式系。
\end{itemize}

下面给出误差的推导

设 $f(x)$ 在 $[-1,1]$ 上充分光滑,则可以将其展开为 Chebyshev 多项式级数:
\[
f(x) = \sum_{k=0}^{\infty} a_k T_k(x)
\]
其中 $T_k(x)$ 是第 $k$ 阶 Chebyshev 多项式, $a_k$ 是对应的系数。

将上述展开式代入积分:
\[
\int_{-1}^1 \frac{f(x)}{\sqrt{1-x^2}} dx = \int_{-1}^1 \frac{\sum_{k=0}^{\infty} a_k T_k(x)}{\sqrt{1-x^2}} dx = \sum_{k=0}^{\infty} a_k \int_{-1}^1 \frac{T_k(x)}{\sqrt{1-x^2}} dx
\]
利用 Chebyshev 多项式的正交性:
\[
\int_{-1}^1 \frac{T_k(x) T_j(x)}{\sqrt{1-x^2}} dx = \begin{cases} 0, & k \neq j \\ \pi, & k = j = 0 \\ \frac{\pi}{2}, & k = j \neq 0 \end{cases}
\]
可以得到
\[
\int_{-1}^1 \frac{T_k(x)}{\sqrt{1-x^2}} dx = \begin{cases} \pi, & k = 0 \\ 0, & k \neq 0 \end{cases}
\]
因此,
\[
\int_{-1}^1 \frac{f(x)}{\sqrt{1-x^2}} dx = a_0 \pi
\]
另一方面,考虑数值积分公式:
\[
\frac{\pi}{n} \sum_{i=1}^n f\left(\cos\left(\frac{(2i-1)\pi}{2n}\right)\right) = \frac{\pi}{n} \sum_{i=1}^n \sum_{k=0}^{\infty} a_k T_k\left(\cos\left(\frac{(2i-1)\pi}{2n}\right)\right)
\]
利用 Chebyshev 多项式的离散正交性:
\[
\sum_{i=1}^n T_k\left(\cos\left(\frac{(2i-1)\pi}{2n}\right)\right) T_j\left(\cos\left(\frac{(2i-1)\pi}{2n}\right)\right) = \begin{cases} 0, & k \neq j, k+j < 2n \\ \frac{n}{2}, & k = j \neq 0, k+j < 2n \\ n, & k = j = 0 \\ \end{cases}
\]
可以得到
\[
\frac{\pi}{n} \sum_{i=1}^n T_k\left(\cos\left(\frac{(2i-1)\pi}{2n}\right)\right) = \begin{cases} \pi, & k = 0 \\ 0, & 0 < k < 2n \end{cases}
\]
因此,
\[
\frac{\pi}{n} \sum_{i=1}^n f\left(\cos\left(\frac{(2i-1)\pi}{2n}\right)\right) = \pi \sum_{k=0}^{\infty} a_k \frac{1}{n} \sum_{i=1}^n T_k\left(\cos\left(\frac{(2i-1)\pi}{2n}\right)\right) = \pi a_0 + \pi \sum_{k=2n, 4n, \dots} a_k
\]
所以误差为
\[
E_{GC}(f) = \int_{-1}^1 \frac{f(x)}{\sqrt{1-x^2}} dx - \frac{\pi}{n} \sum_{i=1}^n f\left(\cos\left(\frac{(2i-1)\pi}{2n}\right)\right) = - \pi \sum_{k=2n, 4n, \dots} a_k
\]
当 $f$ 充分光滑时,有
\[
|a_k| \leq \frac{M}{k!}
\]
其中 $M$ 为常数。因此,误差可以表示为
\[
E_{GC}(f) = \frac{2\pi}{2^{2n}(2n)!} f^{(2n)}(\xi), \quad \xi \in (-1,1)
\]
其中 $\xi \in (-1,1)$.

\subsection{四、其他常见公式 (提及)}

\begin{itemize}
	\item 中点公式 (Midpoint Rule): $n=1$ 的开型牛顿-柯特斯公式,或 $n=1$ 的高斯-勒让德公式(经过调整)。
$\int_a^b f(x) dx \approx (b-a) f(\frac{a+b}{2})$. 误差: $\frac{(b-a)^3}{24}f''(\xi)$。代数精度: 1.
	\item 龙贝格积分 (Romberg Integration): 对复合梯形公式使用理查森外推法,逐步提高精度。
	\item 自适应求积 (Adaptive Quadrature): 根据被积函数在不同子区间的行为自动调整步长或子区间划分,以在满足给定精度要求的情况下最小化计算量。
\end{itemize}

\subsection{五、代数精度总结}

\begin{table}[h]
	\centering
	\begin{tabular}{|c|c|c|c|}
		\hline
		公式名称 & 基本型节点数 & 代数精度 & 复合型误差阶 ($h$) \\
		\hline
		梯形公式 & 2 & 1 & $O(h^2)$ \\
		\hline
		辛普森公式 & 3 & 3 & $O(h^4)$ \\
		\hline
		高斯-勒让德 ($n$点) & $n$ & $2n-1$ & (通常不直接用复合) \\
		\hline
		高斯-切比雪夫 ($n$点) & $n$ & $2n-1$ & (通常不直接用复合) \\
		\hline
		(狭义)切比雪夫($n$点) & $n$ & $n$ & (若存在) \\
		\hline
		中点公式 & 1 & 1 & $O(h^2)$ (复合型) \\
		\hline
	\end{tabular}
\end{table}
\subsection{误差分析推导关键点}

\begin{itemize}
	\item \textbf{牛顿-柯特斯:}
	\begin{itemize}
		\item 基于插值多项式的误差公式: $f(x) - P_n(x) = \frac{f^{(n+1)}(\eta_x)}{(n+1)!} \prod_{i=0}^n (x-x_i)$。
		\item 积分 $\int_a^b (f(x)-P_n(x))dx$。如果 $\int_a^b \prod (x-x_i) dx \neq 0$,误差主项包含 $f^{(n+1)}$。
		\item 对于对称节点(如辛普森,中点),有时积分 $\int_a^b \prod (x-x_i) dx = 0$,使得误差阶比预期高一阶(即 $f^{(n+2)}$)。这需要使用更高阶的埃尔米特(Hermite)插值或Peano核方法。
		\item \textbf{Peano核方法:} 误差 $E(f) = \int_a^b K(t) f^{(k+1)}(t) dt$ 对于所有次数 $\le k$ 的多项式 $P_k(x)$,$E(P_k)=0$。$K(t) = \frac{1}{k!} E_x[(x-t)_+^k]$。
	\end{itemize}
	\item \textbf{高斯求积:}
	\begin{itemize}
		\item 利用正交多项式的性质。关键在于 $n$ 次正交多项式 $\phi_n(x)$ 与所有次数小于 $n$ 的多项式关于权函数 $\omega(x)$ 正交。
		\item 误差基于 $2n-1$ 次Hermite插值多项式 $H_{2n-1}(x)$,它在节点 $x_i$ 处与 $f(x)$ 和 $f'(x)$ 都一致 (但高斯公式本身不使用 $f'$ 的值)。实际推导更直接:构造一个在节点 $x_i$ 处取值为 $f(x_i)$ 的 $n-1$ 次多项式 $L_{n-1}(x)$。那么 $f(x)-L_{n-1}(x)$ 在 $x_i$ 处为0。误差 $E(f) = \int_a^b \omega(x) (f(x)-L_{n-1}(x)) dx$。 $f(x)-L_{n-1}(x) = \frac{f^{(n)}(\eta_x)}{n!} \prod (x-x_i)$。但高斯公式误差是关于 $f^{(2n)}$ 的,这是因为节点 $x_i$ 的特殊选择(正交多项式的根)。
		\item 一个更标准的误差推导是考虑一个在 $x_i$ 处与 $f$ 值相同的 $2n-1$ 次Hermite插值多项式 $H(x)$,它在 $x_i$ 处满足 $H(x_i)=f(x_i)$ 和 $H'(x_i)=f'(x_i)$(尽管公式不直接用 $f'$)。积分 $H(x)$ 得到高斯公式的权重。误差 $\int \omega(x)(f(x)-H(x))dx = \int \omega(x) \frac{f^{(2n)}(\eta)}{(2n)!} \prod (x-x_i)^2 dx$。
	\end{itemize}
\end{itemize}

这份资料应该能帮助你复习数值积分的主要内容。祝你考试顺利!
