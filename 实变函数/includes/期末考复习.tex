\section{实变函数期末考试复习大纲}

\subsection{第一部分:集合论基础与基数理论}

\subsubsection{集合的基数与可数性}
\begin{itemize}
    \item 可数集与不可数集的定义与判定
    \item Cantor对角线方法
    \item 实数集的不可数性证明
    \item 集合基数的比较:$|\mathbb{N}| < |\mathbb{R}|$
    \item 连续统假设简介
\end{itemize}

\subsubsection{集合的极限}
\begin{itemize}
    \item 集合列的上极限和下极限:$\limsup_{n \to \infty} A_n$ 和 $\liminf_{n \to \infty} A_n$
    \item 集合列收敛性的判定
    \item Borel-Cantelli引理及其应用
\end{itemize}

\subsection{第二部分:测度论}

\subsubsection{Lebesgue外测度}
\begin{itemize}
    \item Lebesgue外测度的定义:$m^*(E) = \inf\{\sum_{k=1}^{\infty}|I_k| : E \subset \cup_{k=1}^{\infty} I_k\}$
    \item 外测度的基本性质:单调性、可数次可加性
    \item 外测度的计算方法
    \item 外测度与集合运算的关系
\end{itemize}

\subsubsection{可测集与Lebesgue测度}
\begin{itemize}
    \item Carathéodory条件:$m^*(T) = m^*(T \cap E) + m^*(T \cap E^c)$
    \item 可测集类$\mathcal{M}$的性质
    \item $\sigma$-代数的概念
    \item Borel集与Lebesgue可测集的关系
    \item 测度的可数可加性
    \item 测度的连续性:上连续性和下连续性
\end{itemize}

\subsubsection{特殊集合的构造}
\begin{itemize}
    \item Cantor集的构造及其性质(零测度、不可数、完全集)
    \item Vitali集(不可测集的例子)
    \item 零测度的第二纲集的构造
    \item Fat Cantor集的构造
\end{itemize}

\subsection{第三部分:可测函数}

\subsubsection{可测函数的定义与判定}
\begin{itemize}
    \item 可测函数的定义:逆像可测性
    \item 可测函数的等价刻画
    \item 简单函数的逼近定理
    \item 连续函数与可测函数的关系
\end{itemize}

\subsubsection{可测函数的运算}
\begin{itemize}
    \item 可测函数的四则运算
    \item 可测函数的复合
    \item 可测函数列的极限
    \item 上确界、下确界的可测性
\end{itemize}

\subsubsection{可测函数列的收敛性}
\begin{itemize}
    \item 几乎处处收敛
    \item 依测度收敛(测度意义下收敛)
    \item 一致收敛
    \item 各种收敛性之间的关系
    \item Egorov定理
    \item Lusin定理
\end{itemize}

\subsection{第四部分:积分理论}

\subsubsection{简单函数的积分}
\begin{itemize}
    \item 简单函数的定义
    \item 简单函数积分的定义
    \item 简单函数积分的基本性质
\end{itemize}

\subsubsection{非负可测函数的积分}
\begin{itemize}
    \item 非负可测函数积分的定义
    \item 积分的单调性
    \item Levi单调收敛定理(Monotone Convergence Theorem)
    \item Fatou引理
\end{itemize}

\subsubsection{一般可测函数的积分}
\begin{itemize}
    \item 可积函数的定义
    \item 积分的线性性
    \item Lebesgue控制收敛定理(Dominated Convergence Theorem)
    \item 积分号下取极限的条件
\end{itemize}

\subsubsection{Riemann积分与Lebesgue积分}
\begin{itemize}
    \item Riemann可积函数的刻画
    \item Lebesgue定理:Riemann可积当且仅当不连续点集零测度
    \item 两种积分的比较
\end{itemize}

\subsection{第五部分:微分与积分}

\subsubsection{单调函数的可微性}
\begin{itemize}
    \item 单调函数几乎处处可微
    \item 导数的可积性
    \item Vitali覆盖引理
\end{itemize}

\subsubsection{有界变差函数}
\begin{itemize}
    \item 有界变差函数的定义
    \item Jordan分解定理
    \item 绝对连续函数
    \item Newton-Leibniz公式的推广
\end{itemize}

\subsubsection{Lebesgue微分定理}
\begin{itemize}
    \item Lebesgue点的定义
    \item 几乎处处的点都是Lebesgue点
    \item Hardy-Littlewood极大函数
\end{itemize}

\subsection{第六部分:$L^p$空间理论}

\subsubsection{$L^p$空间的定义与基本性质}
\begin{itemize}
    \item $L^p$范数的定义:$\|f\|_p = \left(\int |f|^p\right)^{1/p}$
    \item Hölder不等式和Minkowski不等式
    \item $L^p$空间的完备性
    \item $L^p$空间之间的包含关系
\end{itemize}

\subsubsection{$L^p$空间中的收敛性}
\begin{itemize}
    \item $L^p$收敛与其他收敛性的关系
    \item 控制收敛定理在$L^p$中的应用
    \item $L^p$空间中的稠密性结果
\end{itemize}

\subsubsection{$L^2$空间(Hilbert空间)}
\begin{itemize}
    \item 内积的定义
    \item 正交性和正交系
    \item Bessel不等式和Parseval等式
    \item 完全正交系
    \item 傅里叶级数理论基础
\end{itemize}

\subsection{第七部分:乘积测度与Fubini定理}

\subsubsection{乘积测度的构造}
\begin{itemize}
    \item 矩形的测度
    \item 乘积$\sigma$-代数
    \item 乘积测度的定义
\end{itemize}

\subsubsection{Fubini定理与Tonelli定理}
\begin{itemize}
    \item Tonelli定理(非负函数情形)
    \item Fubini定理(可积函数情形)
    \item 积分次序交换的条件
    \item 多重积分的计算
\end{itemize}

\subsection{常见考试题型与解题策略}

\subsubsection{证明题常见类型}
\begin{enumerate}
    \item \textbf{可测性证明}:利用可测函数的定义或等价刻画
    \item \textbf{测度计算}:使用外测度定义或测度性质
    \item \textbf{收敛性分析}:区分不同收敛概念,选择合适定理
    \item \textbf{积分计算与估计}:应用控制收敛定理、Fatou引理等
    \item \textbf{反例构造}:构造满足特定条件的函数或集合
\end{enumerate}

\subsubsection{计算题重点}
\begin{enumerate}
    \item 特殊集合(如Cantor集)的测度计算
    \item 简单函数的积分计算
    \item $L^p$范数的计算
    \item 利用Fubini定理计算二重积分
    \item 函数列极限的$L^p$收敛性判定
\end{enumerate}

\subsubsection{应用题要点}
\begin{enumerate}
    \item 概率论中的应用(Borel-Cantelli引理)
    \item 傅里叶分析中的应用($L^2$理论)
    \item 偏微分方程中的应用(Sobolev空间预备知识)
    \item 泛函分析中的应用(Banach空间理论)
\end{enumerate}

\subsection{重要定理与公式总结}

\subsubsection{核心定理}
\begin{itemize}
    \item \textbf{Carathéodory定理}:可测集的刻画
    \item \textbf{Egorov定理}:几乎处处收敛与一致收敛
    \item \textbf{Lusin定理}:可测函数的连续逼近
    \item \textbf{Levi单调收敛定理}
    \item \textbf{Lebesgue控制收敛定理}
    \item \textbf{Fatou引理}
    \item \textbf{Fubini-Tonelli定理}
    \item \textbf{Lebesgue微分定理}
\end{itemize}

\subsubsection{重要不等式}
\begin{itemize}
    \item Hölder不等式:$\|fg\|_1 \leq \|f\|_p \|g\|_q$,$\frac{1}{p} + \frac{1}{q} = 1$
    \item Minkowski不等式:$\|f+g\|_p \leq \|f\|_p + \|g\|_p$
    \item Jensen不等式
    \item Chebyshev不等式
\end{itemize}

\subsection{复习建议}

\begin{enumerate}
    \item \textbf{理解概念}:重点理解测度、可测性、积分的本质含义
    \item \textbf{掌握定理}:熟练掌握各大定理的条件、结论和应用场合
    \item \textbf{练习计算}:多做测度计算、积分计算的练习题
    \item \textbf{构造反例}:学会构造满足特定条件的函数和集合
    \item \textbf{联系应用}:了解实变函数在其他数学分支中的应用
\end{enumerate}
