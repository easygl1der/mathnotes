\section{Topological preliminaries}

\subsection{Definition of topology}

\begin{figure}[H]
\centering
\includegraphics[width=\textwidth]{Topological preliminaries-20250322.png}
% \caption{}
\label{}
\end{figure}

\subsection{General definitions of closed, closure, compact, neighborhood, Hausdorff space, locally compact, separated and connected sets}

\begin{figure}[H]
\centering
\includegraphics[width=\textwidth]{1-Topological preliminaries-20250322.png}
% \caption{}
\label{}
\end{figure}
\begin{figure}[H]
\centering
\includegraphics[width=\textwidth]{2-Topological preliminaries-20250322.png}
% \caption{}
\label{}
\end{figure}
Open sets remains to be open under arbitrary unions, while closed sets remains to be closed under arbitrary intersections. Imagine the special case of intervals to remember it.

The definition of compact is not technique to prove a set compact but a property.

\begin{figure}[H]
\centering
\includegraphics[width=\textwidth]{10-Topological preliminaries-20250322.png}
% \caption{}
\label{}
\end{figure}
\begin{figure}[H]
\centering
\includegraphics[width=\textwidth]{11-Topological preliminaries-20250322.png}
% \caption{}
\label{}
\end{figure}

Another definition of closure:
\begin{figure}[H]
\centering
\includegraphics[width=\textwidth]{16-Topological preliminaries-20250322.png}
% \caption{}
\label{}
\end{figure}

\subsection{Metric definition of neighborhood, limit point, closed, interior point, open, complement, perfect, bounded, dense}

\begin{figure}[H]
\centering
\includegraphics[width=\textwidth]{6-Topological preliminaries-20250322.png}
% \caption{}
\label{}
\end{figure}
\begin{figure}[H]
\centering
\includegraphics[width=\textwidth]{8-Topological preliminaries-20250322.png}
% \caption{}
\label{}
\end{figure}

\begin{definition}[relatively open]
Suppose $E\subset Y\subset X$, where $X$ is a metric space. Motivated by the idea that $Y$ can also be a metric space, we say $E$ is \textbf{open relative} to $Y$ if to each $p\in E$ there is associated an $r>0$ such that $q\in E$ whenever $d(p,q)<r$ and $q\in Y$.
\end{definition}
A set may be open relative to $Y$ without being an open subset of $X$, e.g. $(a,b)\subset \mathbb{R}\subset \mathbb{R}^{2}$.

\begin{theorem}
Suppose $Y\subset X$. A subset $E$ of $Y$ is open relative to $Y$ iff $E=Y\cap G$ for some open subset $G$ of $X$.\label{27f07d}
\end{theorem}

\cref{27f07d}  can be another definition of \textbf{relatively open}.

\subsection{Some properties in metric spaces}

\begin{itemize}
	\item Every neighborhood is an open set.
	\item If $p$ is a limit point of a set $E$, then every neighborhood of $p$ contains infinitely many points of $E$.
	\item A finite point set has no limit points.
	\item Closed subsets of compact sets are compact.
	\item Perfect set in $\mathbb{R}^{k}$ is uncountable.
	\item If $\{ K_{\alpha} \}$ is a collection of compact subsets of a metric space $X$ such that the intersection of every finite subcollection of $\{ K_{\alpha} \}$ is nonempty, then $\bigcap K_{\alpha}$ is nonempty.
	\item If $\{ K_n \}$ is a decreasing sequence of nonempty compact sets, then $\bigcap_{n=1}^{\infty}K_n$ is not empty.
\end{itemize}

\subsection{Examples}

\begin{figure}[H]
\centering
\includegraphics[width=\textwidth]{9-Topological preliminaries-20250322.png}
% \caption{}
\label{}
\end{figure}

\subsection{Heine-Borel theorem}

\begin{theorem}[Heine-Borel theorem]
The compact subsets of a enclidean space $\mathbb{R}^{n}$ are precisely those that are closed and bounded.
\end{theorem}
Moreover, the theorem is true for any locally compact Hausdorff space. Note that metric spaces are locally compact Hausdorff space.

\begin{theorem}[Closed subsets of compact sets are compact.]
Suppoes $K$ is compact and $F$ is closed, in a topological space $X$. If $F\subset K$ then $F$ is compact.
\end{theorem}
\begin{proof}
If $\{ V_{\alpha} \}$ is an open cover of $F$ and $W=F^{c}$ then $W\cup \bigcup_{\alpha}V_{\alpha}$ covers $X$; hence there is a finite collection $\{ V_{\alpha _i} \}$ such that
\[
K\subset W\cup V_{\alpha_1}\cup\dots \cup V_{\alpha _n}
\]
Then $F\subset V_{\alpha_1}\cup\dots \cup V_{\alpha _n}$.
\end{proof}

\begin{corollary}
If $A\subset B$ and if $B$ has compact closure, so does $A$.
\end{corollary}
\begin{theorem}
$X$ hausdorff, $K\subset X$, $K$ compact, and $p\in K^{c}$. Then there are open sets $U$ and $W$ such that $p\in U$, $K\subset W$, and $U\cap W=\varnothing$.
\end{theorem}
\begin{proof}
If $q\in K$, the Hausdorff separation axiom implies the existence of disjoint open sets $U_{q}$ and $V_{q}$ such that $p\in U_{q}$ and $q\in V_{q}$. Since $K$ is compact, there are points $q_1,\dots, q_n\in K$ such that
\[
K\subset V_{q_1}\cup\dots \cup V_{q_n}
\]
Our requirements are then satisfied by the sets
\[
U=U_{q_1}\cap\dots \cap U_{q_n}\qquad W=V_{q_1}\cup\dots \cup V_{q_n}
\]
\end{proof}

\begin{corollary}
Compact subsets of Hausdorff spaces are closed. (the inverse is not true.)
\end{corollary}
\begin{corollary}
If $F$ closed, $K$ compact in a Hausdorff space, then $F\cap K$ is compact.
\end{corollary}
\begin{figure}[H]
\centering
\includegraphics[width=\textwidth]{4-Topological preliminaries-20250322.png}
% \caption{}
\label{}
\end{figure}
Use the definition of compact set.

\begin{figure}[H]
\centering
\includegraphics[width=\textwidth]{5-Topological preliminaries-20250322.png}
% \caption{}
\label{}
\end{figure}
Use the definition of compact set.

\section{General properties}

See Royden Chapter 11.

\subsection{Bases and subbases}

\begin{figure}[H]
\centering
\includegraphics[width=\textwidth]{12-Topological preliminaries-20250322.png}
% \caption{}
\label{}
\end{figure}
\begin{figure}[H]
\centering
\includegraphics[width=\textwidth]{13-Topological preliminaries-20250322.png}
% \caption{}
\label{}
\end{figure}
\begin{figure}[H]
\centering
\includegraphics[width=\textwidth]{15-Topological preliminaries-20250322.png}
% \caption{}
\label{}
\end{figure}

\subsection{Separation properties}

\begin{itemize}
	\item \textbf{The Tychonoff Separation Property}:
	\begin{itemize}
		\item For each two points $u$ and $v$ in $X$, there is a neighborhood of $u$ that does not contain $v$ and a neighborhood of $v$ that does not contain $u$.
	\end{itemize}
	\item \textbf{The Hausdorff Separation Property}:
	\begin{itemize}
		\item Each two points in $X$ can be separated by disjoint neighborhoods.
	\end{itemize}
	\item \textbf{The Regular Separation Property}:
	\begin{itemize}
		\item The Tychonoff separation property holds and, moreover, each closed set and point not in the set can be separated by disjoint neighborhoods.
	\end{itemize}
	\item \textbf{The Normal Separation Property}:
	\begin{itemize}
		\item The Tychonoff separation property holds and, moreover, each two disjoint closed sets can be separated by disjoint neighborhoods.
	\end{itemize}
\end{itemize}

\begin{figure}[H]
\centering
\includegraphics[width=\textwidth]{17-Topological preliminaries-20250322.png}
% \caption{}
\label{}
\end{figure}

\begin{proposition}
A topological space $X$ is a Tychonoff space iff every set consisting of a single point is closed.
\end{proposition}
\begin{proposition}
Every metric space is normal.
\end{proposition}
\begin{proposition}
Let $X$ be Tychonoff. Then $X$ is normal iff whenever $\mathcal{U}$ is a neighborhood of a closed subset $F$ of $X$, there is another neighborhood of $F$ whose closure is contained in $\mathcal{U}$, that is, there is an open set $\mathcal{O}$ for which
\[
F\subseteq \mathcal{O}\subseteq  \overline{\mathcal{O}}\subseteq \mathcal{U}
\]
\end{proposition}
\subsection{Countablity and Separability}

\begin{figure}[H]
\centering
\includegraphics[width=\textwidth]{18-Topological preliminaries-20250322.png}
% \caption{}
\label{}
\end{figure}
\begin{figure}[H]
\centering
\includegraphics[width=\textwidth]{19-Topological preliminaries-20250322.png}
% \caption{}
\label{}
\end{figure}

In a topological space that is not first countable, it is possible for a point to be a point of closure of a set and yet no sequence in the set converges to the point.

\begin{figure}[H]
\centering
\includegraphics[width=\textwidth]{20-Topological preliminaries-20250322.png}
% \caption{}
\label{}
\end{figure}
\begin{figure}[H]
\centering
\includegraphics[width=\textwidth]{21-Topological preliminaries-20250322.png}
% \caption{}
\label{}
\end{figure}

\subsection{Strong, weak topology, induced topology, homeomorphism}

\begin{figure}[H]
\centering
\includegraphics[width=\textwidth]{22-Topological preliminaries-20250322.png}
% \caption{}
\label{}
\end{figure}
\begin{figure}[H]
\centering
\includegraphics[width=\textwidth]{23-Topological preliminaries-20250322.png}
% \caption{}
\label{}
\end{figure}
\begin{figure}[H]
\centering
\includegraphics[width=\textwidth]{24-Topological preliminaries-20250322.png}
% \caption{}
\label{}
\end{figure}

\subsection{Compact and Sequentially Compact topological spaces}

\begin{figure}[H]
\centering
\includegraphics[width=\textwidth]{25-Topological preliminaries-20250322.png}
% \caption{}
\label{}
\end{figure}
In view of the definition of the subspace topology, a subset $K$ of $X$ is compact provided every covering of $K$ by a collection of open subsets of $X$ has a finite subcover.

\begin{definition}[finite intersection property]
A collection of sets is said to be \textbf{finite intersection property} provided every finite subcollection has nonempty intersection.
\end{definition}
\begin{figure}[H]
\centering
\includegraphics[width=\textwidth]{26-Topological preliminaries-20250322.png}
% \caption{}
\label{}
\end{figure}

\begin{figure}[H]
\centering
\includegraphics[width=\textwidth]{27-Topological preliminaries-20250322.png}
% \caption{}
\label{}
\end{figure}

\begin{figure}[H]
\centering
\includegraphics[width=\textwidth]{28-Topological preliminaries-20250322.png}
% \caption{}
\label{}
\end{figure}

\begin{figure}[H]
\centering
\includegraphics[width=\textwidth]{29-Topological preliminaries-20250322.png}
% \caption{}
\label{}
\end{figure}

\begin{figure}[H]
\centering
\includegraphics[width=\textwidth]{30-Topological preliminaries-20250322.png}
% \caption{}
\label{}
\end{figure}

\begin{figure}[H]
\centering
\includegraphics[width=\textwidth]{31-Topological preliminaries-20250322.png}
% \caption{}
\label{}
\end{figure}

Additionally, homeomorphism requires continuous inverse, so hypothesis upon space is neccessary.

\begin{figure}[H]
\centering
\includegraphics[width=\textwidth]{32-Topological preliminaries-20250322.png}
% \caption{}
\label{}
\end{figure}

Regard a compact set as a compact topological space...
\begin{figure}[H]
\centering
\includegraphics[width=\textwidth]{33-Topological preliminaries-20250322.png}
% \caption{}
\label{}
\end{figure}

\begin{definition}[countably compact]
A topological space is said to be \textbf{countably compact} provided every countable open cover has a finite subcover.
\end{definition}
\subsection{Separate, Connected, Intermediate value property}

\begin{definition}[separate]
Two nonempty open subsets of a topological spaces $X$ are said to \textbf{separate} $X$ if they are disjoint and their union is $X$.
\end{definition}
\begin{definition}[connected]
A topological space which cannot be separated by such a pair is said to be \textbf{connected}.
\end{definition}
Connectness is preserved under continuous mapping.
\begin{figure}[H]
\centering
\includegraphics[width=\textwidth]{34-Topological preliminaries-20250322.png}
% \caption{}
\label{}
\end{figure}

For a set $C$ of real number, the following are equivalent:

\begin{itemize}
	\item $C$ is an interval.
	\item $C$ is convex.
	\item $C$ is connected.
\end{itemize}

\begin{figure}[H]
\centering
\includegraphics[width=\textwidth]{35-Topological preliminaries-20250322.png}
% \caption{}
\label{}
\end{figure}
\begin{figure}[H]
\centering
\includegraphics[width=\textwidth]{36-Topological preliminaries-20250322.png}
% \caption{}
\label{}
\end{figure}

\section{Three Fundamental Theorems}
