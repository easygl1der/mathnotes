\documentclass[12pt,a4paper]{amsart}
\usepackage[utf8]{inputenc}
\usepackage[T1]{fontenc}
\usepackage{amsmath,amsfonts,amssymb}
\usepackage{tikz}
\usepackage{algorithm}
\usepackage{algorithmic}
\usepackage{float}
\usepackage{listings}
\usepackage{xcolor}
\usepackage{geometry}

% Define custom colors
\definecolor{lightblue}{RGB}{173,216,230}
\definecolor{lightgreen}{RGB}{144,238,144}
\definecolor{lightyellow}{RGB}{255,255,224}
\definecolor{lightgray}{RGB}{211,211,211}
\usepackage{fancyhdr}
\usepackage{enumitem}
\usepackage{booktabs}
\usepackage{multirow}
\usepackage{mdframed}
\usepackage[hidelinks,bookmarksnumbered,bookmarksopen]{hyperref}
\usepackage{xeCJK}
\setCJKmainfont{LXGW WenKai}

\geometry{left=2.5cm,right=2.5cm,top=2.5cm,bottom=2.5cm}

% 代码高亮设置
\lstset{
    language=C++,
    basicstyle=\ttfamily\small,
    keywordstyle=\color{blue}\bfseries,
    commentstyle=\color{green!60!black},
    stringstyle=\color{red},
    numbers=left,
    numberstyle=\tiny\color{gray},
    stepnumber=1,
    numbersep=10pt,
    backgroundcolor=\color{gray!10},
    frame=single,
    tabsize=4,
    captionpos=b,
    breaklines=true,
    breakatwhitespace=false,
    showspaces=false,
    showstringspaces=false,
    showtabs=false
}

% 定理环境
\newtheorem{definition}{定义}[section]
\newtheorem{theorem}{定理}[section]
\newtheorem{example}{例题}[section]
\newtheorem{property}{性质}[section]

% TikZ库
\usetikzlibrary{trees,positioning,arrows,automata,calc,shapes}


\title{\textbf{数据结构 - 栈和队列复习资料}}
\author{详细版本}
\date{\today}

\begin{document}

% \maketitle

% % 目录设置
% \tableofcontents
% \newpage

\section{栈的基本概念}

\subsection{栈的定义}

\begin{definition}[栈]
栈(stack)是限定仅在表的一端进行插入和删除操作的线性表。允许插入和删除的一端称为栈顶(stack top),另一端称为栈底(stack bottom),不含任何数据元素的栈称为空栈。
\end{definition}

\subsection{栈的特性}

栈中元素具有以下特性:
\begin{enumerate}
\item \textbf{后进先出(LIFO)}:Last In First Out,最后进入栈的元素最先出栈
\item \textbf{线性关系}:栈中元素具有一对一的前驱后继关系
\item \textbf{操作受限}:只能在栈顶进行插入(入栈、压栈)和删除(出栈、弹栈)操作
\end{enumerate}

\begin{center}
\begin{tikzpicture}[scale=1]
% 栈的示意图
\draw[thick] (0,0) rectangle (2,4);
\node[rectangle,draw,fill=lightblue,minimum width=1.8cm,minimum height=0.6cm] at (1,0.4) {$a_1$};
\node[rectangle,draw,fill=lightblue,minimum width=1.8cm,minimum height=0.6cm] at (1,1.2) {$a_2$};
\node[rectangle,draw,fill=lightblue,minimum width=1.8cm,minimum height=0.6cm] at (1,2.0) {$a_3$};
\node[rectangle,draw,fill=lightblue,minimum width=1.8cm,minimum height=0.6cm] at (1,2.8) {$a_4$};

% 标注
\node[left] at (0,0.4) {栈底};
\node[left] at (0,2.8) {栈顶};
\node[right] at (2.3,2.8) {出栈};
\node[above] at (1,4.2) {入栈};

% 箭头
\draw[->,thick,red] (1,3.5) -- (1,3.2);
\draw[->,thick,blue] (2.5,2.8) -- (2.2,2.8);
\end{tikzpicture}
\end{center}

\subsection{栈的抽象数据类型定义}

\indent

\begin{lstlisting}[caption=栈的抽象数据类型定义]
ADT Stack {
    数据对象:D = {ai | ai ∈ ElemType, i = 1,2,...,n, n ≥ 0}
    数据关系:R1 = {<ai-1, ai> | ai-1, ai ∈ D, i = 2,...,n}
    基本操作:
        InitStack(S);           // 栈的初始化
        DestroyStack(S);        // 栈的销毁
        Push(S, x);             // 入栈操作
        Pop(S);                 // 出栈操作
        GetTop(S);              // 取栈顶元素
        Empty(S);               // 判断栈是否为空
} ADT Stack
\end{lstlisting}

\section{栈的存储结构及实现}

\subsection{顺序栈}

\subsubsection{顺序栈的存储结构}

顺序栈是栈的顺序存储结构,用一维数组存储栈中元素,并用一个整型变量top记录栈顶元素在数组中的位置。

\begin{center}
\begin{tikzpicture}[scale=1]
% 顺序栈示意图
\foreach \x in {0,1,2,3,4,5,6,7,8,9} {
    \draw (\x,0) rectangle (\x+1,1);
    \node at (\x+0.5,-0.3) {\x};
}

% 填充数据
\node at (0.5,0.5) {$a_1$};
\node at (1.5,0.5) {$a_2$};
\node at (2.5,0.5) {$a_3$};
\node at (3.5,0.5) {$a_4$};

% 标注
\node[below] at (0.5,-0.8) {栈底};
\node[above] at (3.5,1.3) {top = 3};
\node[above] at (3.5,1.7) {栈顶};
\draw[->,thick] (3.5,1.5) -- (3.5,1.1);

% 空闲区域标注
\node at (6.5,0.5) {空闲};

% 数组名
\node[left] at (0,0.5) {data};
\end{tikzpicture}
\end{center}

\subsubsection{顺序栈的实现}

\indent

\begin{lstlisting}[caption=顺序栈类定义]
const int StackSize = 100;
template <typename DataType>
class SeqStack {
public:
    SeqStack();                    // 构造函数
    ~SeqStack();                   // 析构函数
    void Push(DataType x);         // 入栈操作
    DataType Pop();                // 出栈操作
    DataType GetTop();             // 取栈顶元素
    int Empty();                   // 判断栈是否为空
private:
    DataType data[StackSize];      // 存放栈元素的数组
    int top;                       // 栈顶指针
};
\end{lstlisting}

\textbf{基本操作实现:}

\begin{lstlisting}[caption=顺序栈基本操作]
// 构造函数
template <typename DataType>
SeqStack<DataType>::SeqStack() {
    top = -1;    // 栈空时top = -1
}

// 入栈操作
template <typename DataType>
void SeqStack<DataType>::Push(DataType x) {
    if (top == StackSize - 1) 
        throw "栈上溢";
    data[++top] = x;
}

// 出栈操作
template <typename DataType>
DataType SeqStack<DataType>::Pop() {
    if (top == -1) 
        throw "栈下溢";
    return data[top--];
}

// 取栈顶元素
template <typename DataType>
DataType SeqStack<DataType>::GetTop() {
    if (top == -1) 
        throw "栈为空";
    return data[top];
}

// 判空操作
template <typename DataType>
int SeqStack<DataType>::Empty() {
    return (top == -1) ? 1 : 0;
}
\end{lstlisting}

\subsection{链栈}

\subsubsection{链栈的存储结构}

链栈是栈的链式存储结构,用单链表实现,通常以链表的头部作为栈顶。
\begin{center}
\begin{tikzpicture}[scale=1]
% 链栈示意图
\node[rectangle,draw,minimum width=1cm,minimum height=0.8cm] at (0,0) {top};
\draw[->] (0.5,0) -- (1.5,0);

\node[rectangle,draw,minimum width=1cm,minimum height=0.8cm,fill=lightblue] at (2,0) {$a_4$};
\node[rectangle,draw,minimum width=0.8cm,minimum height=0.8cm] at (3,0) {};
\draw[->] (3.4,0) -- (4.5,0);

\node[rectangle,draw,minimum width=1cm,minimum height=0.8cm,fill=lightblue] at (5,0) {$a_3$};
\node[rectangle,draw,minimum width=0.8cm,minimum height=0.8cm] at (6,0) {};
\draw[->] (6.4,0) -- (7.5,0);

\node[rectangle,draw,minimum width=1cm,minimum height=0.8cm,fill=lightblue] at (8,0) {$a_2$};
\node[rectangle,draw,minimum width=0.8cm,minimum height=0.8cm] at (9,0) {};
\draw[->] (9.4,0) -- (10.5,0);

\node[rectangle,draw,minimum width=1cm,minimum height=0.8cm,fill=lightblue] at (11,0) {$a_1$};
\node[rectangle,draw,minimum width=0.8cm,minimum height=0.8cm] at (12,0) {$\wedge$};

% 标注
\node[above] at (2,0.5) {栈顶};
\node[above] at (11,0.5) {栈底};
\end{tikzpicture}
\end{center}

\subsubsection{链栈的实现}

\indent

\begin{lstlisting}[caption=链栈类定义]
template <typename DataType>
struct Node {
    DataType data;
    Node<DataType>* next;
};

template <typename DataType>
class LinkStack {
public:
    LinkStack();                   // 构造函数
    ~LinkStack();                  // 析构函数
    void Push(DataType x);         // 入栈操作
    DataType Pop();                // 出栈操作
    DataType GetTop();             // 取栈顶元素
    int Empty();                   // 判断栈是否为空
private:
    Node<DataType>* top;           // 栈顶指针
};
\end{lstlisting}

\begin{lstlisting}[caption=链栈基本操作]
// 入栈操作
template <typename DataType>
void LinkStack<DataType>::Push(DataType x) {
    Node<DataType>* s = new Node<DataType>;
    s->data = x;
    s->next = top;
    top = s;
}

// 出栈操作
template <typename DataType>
DataType LinkStack<DataType>::Pop() {
    if (top == nullptr) 
        throw "栈下溢";
    DataType x = top->data;
    Node<DataType>* p = top;
    top = top->next;
    delete p;
    return x;
}
\end{lstlisting}

\subsection{顺序栈与链栈的比较}

\begin{center}
\begin{tabular}{|l|l|l|}
\hline
\textbf{比较项目} & \textbf{顺序栈} & \textbf{链栈} \\
\hline
存储空间 & 预分配固定大小 & 动态分配 \\
\hline
空间利用率 & 可能浪费空间 & 充分利用空间 \\
\hline
存储密度 & 高(无指针开销) & 低(有指针开销) \\
\hline
操作效率 & 高 & 相对较低 \\
\hline
栈满判断 & 容易判断 & 依赖内存状态 \\
\hline
适用场景 & 元素个数相对稳定 & 元素个数变化较大 \\
\hline
\end{tabular}
\end{center}

\section{队列的基本概念}

\subsection{队列的定义}

\begin{definition}[队列]
队列(queue)是只允许在一端进行插入操作,在另一端进行删除操作的线性表。允许插入(入队)的一端称为队尾(rear),允许删除(出队)的一端称为队头(front)。
\end{definition}

\subsection{队列的特性}

队列中元素具有以下特性:
\begin{enumerate}
\item \textbf{先进先出(FIFO)}:First In First Out,最先进入队列的元素最先出队
\item \textbf{线性关系}:队列中元素具有一对一的前驱后继关系
\item \textbf{操作受限}:只能在队尾插入,在队头删除
\end{enumerate}

\begin{center}
\begin{tikzpicture}[scale=1]
% 队列示意图
\foreach \x in {0,1,2,3,4,5} {
    \draw (\x,0) rectangle (\x+1,1);
}

% 填充数据
\node at (0.5,0.5) {$a_1$};
\node at (1.5,0.5) {$a_2$};
\node at (2.5,0.5) {$a_3$};
\node at (3.5,0.5) {$a_4$};

% 标注
\node[below] at (0.5,-0.5) {队头};
\node[below] at (3.5,-0.5) {队尾};

% 箭头
\draw[->,thick,blue] (-0.5,0.5) -- (-0.1,0.5);
\node[left] at (-0.5,0.5) {出队};

\draw[->,thick,red] (6.5,0.5) -- (6.1,0.5);
\node[right] at (6.5,0.5) {入队};

\draw[->] (0,-0.3) -- (0,-0.1);
\draw[->] (4,-0.3) -- (4,-0.1);
\end{tikzpicture}
\end{center}

\subsection{队列的抽象数据类型定义}

\indent

\begin{lstlisting}[caption=队列的抽象数据类型定义]
ADT Queue {
    数据对象:D = {ai | ai ∈ ElemType, i = 1,2,...,n, n ≥ 0}
    数据关系:R1 = {<ai-1, ai> | ai-1, ai ∈ D, i = 2,...,n}
    基本操作:
        InitQueue(Q);           // 队列的初始化
        DestroyQueue(Q);        // 队列的销毁
        EnQueue(Q, x);          // 入队操作
        DeQueue(Q);             // 出队操作
        GetHead(Q);             // 取队头元素
        Empty(Q);               // 判断队列是否为空
} ADT Queue
\end{lstlisting}

\section{队列的存储结构及实现}

\subsection{顺序队列的问题}

如果用顺序存储实现队列,会遇到"假溢出"问题:

\begin{center}
\begin{tikzpicture}[scale=1]
% 假溢出示意图
\foreach \x in {0,1,2,3,4,5,6,7} {
    \draw (\x,0) rectangle (\x+1,1);
    \node at (\x+0.5,-0.3) {\x};
}

% 状态1:正常队列
\node at (1.5,2.5) {\textbf{状态1:正常队列}};
\node at (1.5,0.5) {$a_1$};
\node at (2.5,0.5) {$a_2$};
\node at (3.5,0.5) {$a_3$};
\draw[->] (1,-0.1) -- (1,0.1);
\node[below] at (1,-0.4) {front};
\draw[->] (4,-0.1) -- (4,0.1);
\node[below] at (4,-0.4) {rear};

% 状态2:假溢出
\foreach \x in {0,1,2,3,4,5,6,7} {
    \draw (\x,3) rectangle (\x+1,4);
    \node at (\x+0.5,2.7) {\x};
}
\node at (1.5,5.5) {\textbf{状态2:假溢出}};
\node at (5.5,3.5) {$a_4$};
\node at (6.5,3.5) {$a_5$};
\node at (7.5,3.5) {$a_6$};
\draw[->] (5,2.9) -- (5,3.1);
\node[below] at (5,2.6) {front};
\draw[->] (8,2.9) -- (8,3.1);
\node[below] at (8,2.6) {rear};
\node[above] at (4,4.3) {空闲但无法使用};
\end{tikzpicture}
\end{center}

\subsection{循环队列}

为了解决假溢出问题,将存储队列的数组看成头尾相接的循环结构,称为循环队列。

\begin{center}
\begin{tikzpicture}[scale=1]
% 循环队列示意图
\draw[thick] (0,0) circle (2.5);

% 数组位置
\foreach \i in {0,1,2,3,4,5,6,7} {
    \pgfmathsetmacro{\angle}{90 - \i * 45}
    \coordinate (pos\i) at (\angle:2.5);
    \node[circle,draw,fill=white,minimum size=0.6cm] at (pos\i) {\i};
}

% 填充数据
\node[circle,draw,fill=lightblue,minimum size=0.6cm] at (pos4) {$a_1$};
\node[circle,draw,fill=lightblue,minimum size=0.6cm] at (pos5) {$a_2$};
\node[circle,draw,fill=lightblue,minimum size=0.6cm] at (pos6) {$a_3$};
\node[circle,draw,fill=lightblue,minimum size=0.6cm] at (pos7) {$a_4$};

% 指针标注
\draw[->,thick,red] ([yshift=15pt]pos3) -- (pos3);
\node[above] at ([yshift=15pt]pos3) {\textbf{front}};

\draw[->,thick,blue] ([yshift=15pt]pos7) -- (pos7);
\node[above] at ([yshift=15pt]pos7) {\textbf{rear}};

\node[below] at (0,-3) {\textbf{循环队列示意图}};
\end{tikzpicture}
\end{center}

\subsubsection{循环队列的队空和队满判断}

\textbf{问题:}循环队列中,队空和队满时都有front == rear,如何区分?

\textbf{解决方案:}牺牲一个存储单元,队满条件为:$(rear + 1) \% QueueSize == front$

\begin{center}
\begin{tabular}{|l|l|}
\hline
\textbf{状态} & \textbf{判断条件} \\
\hline
队空 & front == rear \\
\hline
队满 & (rear + 1) \% QueueSize == front \\
\hline
队列长度 & (rear - front + QueueSize) \% QueueSize \\
\hline
\end{tabular}
\end{center}

\subsubsection{循环队列的实现}

\indent

\begin{lstlisting}[caption=循环队列类定义]
const int QueueSize = 100;
template <typename DataType>
class CirQueue {
public:
    CirQueue();                    // 构造函数
    ~CirQueue();                   // 析构函数
    void EnQueue(DataType x);      // 入队操作
    DataType DeQueue();            // 出队操作
    DataType GetHead();            // 取队头元素
    int Empty();                   // 判断队列是否为空
private:
    DataType data[QueueSize];      // 存放队列元素的数组
    int front, rear;               // 队头和队尾指针
};
\end{lstlisting}

\begin{lstlisting}[caption=循环队列基本操作]
// 构造函数
template <typename DataType>
CirQueue<DataType>::CirQueue() {
    front = rear = 0;    // 队空时front = rear
}

// 入队操作
template <typename DataType>
void CirQueue<DataType>::EnQueue(DataType x) {
    if ((rear + 1) % QueueSize == front) 
        throw "队列上溢";
    rear = (rear + 1) % QueueSize;
    data[rear] = x;
}

// 出队操作
template <typename DataType>
DataType CirQueue<DataType>::DeQueue() {
    if (front == rear) 
        throw "队列下溢";
    front = (front + 1) % QueueSize;
    return data[front];
}

// 取队头元素
template <typename DataType>
DataType CirQueue<DataType>::GetHead() {
    if (front == rear) 
        throw "队列为空";
    return data[(front + 1) % QueueSize];
}
\end{lstlisting}

\subsection{链队列}

\subsubsection{链队列的存储结构}

链队列是队列的链式存储结构,用单链表实现,需要设置队头指针front和队尾指针rear。

\begin{center}
\begin{tikzpicture}[scale=1,
    node distance=0.5cm,
    box/.style={rectangle,draw,thick,minimum width=1.2cm,minimum height=0.8cm,rounded corners=2pt},
    pointer/.style={rectangle,draw,thick,minimum width=0.6cm,minimum height=0.4cm,fill=white},
    data/.style={box,fill=lightblue!80},
    header/.style={box,fill=lightgray!60},
    label/.style={rectangle,draw,thick,fill=orange!20,rounded corners=3pt,font=\small\bfseries}]

% 指针标签
\node[label] (front_label) at (0,1.5) {front};
\node[label] (rear_label) at (0,0) {rear};

% 头结点
\node[header] (head) at (3,0.75) {head};
\node[pointer] (head_ptr) at (4.2,0.75) {};

% 数据结点
\node[data] (node1) at (6,0.75) {$a_1$};
\node[pointer] (ptr1) at (7.2,0.75) {};

\node[data] (node2) at (9,0.75) {$a_2$};
\node[pointer] (ptr2) at (10.2,0.75) {};

\node[data] (node3) at (12,0.75) {$a_3$};
\node[pointer] (ptr3) at (13.2,0.75) {$\wedge$};

% 连接箭头 - 使用更美观的弯曲箭头
\draw[->,very thick,red] (front_label.east) to[bend left=10] (head.west);
\draw[->,very thick,blue] (rear_label.east) to[bend right=20] (node3.south west);

\draw[->,thick] (head_ptr.east) to[bend left=5] (node1.west);
\draw[->,thick] (ptr1.east) to[bend left=5] (node2.west);
\draw[->,thick] (ptr2.east) to[bend left=5] (node3.west);

% 操作方向指示
\draw[->,very thick,green!70!black] (5.5,1.8) -- (6.5,1.8);
\node[above] at (6,2.1) {\footnotesize \textcolor{green!70!black}{出队方向}};

\draw[->,very thick,purple!70!black] (11.5,1.8) -- (12.5,1.8);
\node[above] at (12,2.1) {\footnotesize \textcolor{purple!70!black}{入队方向}};

% 节点类型标注
\node[below] at (3,-0.4) {\footnotesize 头结点};
\node[below] at (6,-0.4) {\footnotesize 队头数据};
\node[below] at (12,-0.4) {\footnotesize 队尾数据};

% 整体标题和说明
\node[below] at (6.5,-1.2) {\textbf{链队列存储结构}};
\node[below] at (6.5,-1.6) {\footnotesize 蓝色为数据节点,灰色为头结点,$\wedge$表示空指针};
\end{tikzpicture}
\end{center}

\subsubsection{链队列的实现}

\indent

\begin{lstlisting}[caption=链队列类定义]
template <typename DataType>
class LinkQueue {
public:
    LinkQueue();                   // 构造函数
    ~LinkQueue();                  // 析构函数
    void EnQueue(DataType x);      // 入队操作
    DataType DeQueue();            // 出队操作
    DataType GetHead();            // 取队头元素
    int Empty();                   // 判断队列是否为空
private:
    Node<DataType>* front, * rear; // 队头和队尾指针
};
\end{lstlisting}

\begin{lstlisting}[caption=链队列基本操作]
// 构造函数
template <typename DataType>
LinkQueue<DataType>::LinkQueue() {
    Node<DataType>* s = new Node<DataType>;
    s->next = nullptr;
    front = rear = s;    // 头结点
}

// 入队操作
template <typename DataType>
void LinkQueue<DataType>::EnQueue(DataType x) {
    Node<DataType>* s = new Node<DataType>;
    s->data = x;
    s->next = nullptr;
    rear->next = s;
    rear = s;
}

// 出队操作
template <typename DataType>
DataType LinkQueue<DataType>::DeQueue() {
    if (front == rear) 
        throw "队列下溢";
    Node<DataType>* p = front->next;
    DataType x = p->data;
    front->next = p->next;
    if (rear == p) rear = front;  // 队列变空
    delete p;
    return x;
}
\end{lstlisting}

\section{栈和队列的应用}

\subsection{括号匹配问题}

\begin{example}[括号匹配算法]
设计算法判断表达式中括号是否正确配对。

\textbf{算法思想:}
\begin{enumerate}
\item 用栈保存未配对的左括号
\item 遇到左括号时入栈
\item 遇到右括号时出栈一个左括号与之配对
\item 最后栈空则配对成功
\end{enumerate}
\end{example}

\begin{algorithm}[H]
\caption{括号匹配算法}
\begin{algorithmic}[1]
\REQUIRE 表达式字符串$str$
\ENSURE 匹配结果:0表示匹配,1表示多左括号,-1表示多右括号
\STATE 栈$S$初始化
\FOR{$i = 0$ to $strlen(str) - 1$}
    \IF{$str[i] == '('$}
        \STATE $S.Push('(')$
    \ELSIF{$str[i] == ')'$}
        \IF{$S.Empty()$}
            \RETURN $-1$ \COMMENT{多右括号}
        \ELSE
            \STATE $S.Pop()$
        \ENDIF
    \ENDIF
\ENDFOR
\IF{$S.Empty()$}
    \RETURN $0$ \COMMENT{正确匹配}
\ELSE
    \RETURN $1$ \COMMENT{多左括号}
\ENDIF
\end{algorithmic}
\end{algorithm}

\subsection{表达式求值}

\begin{example}[表达式求值算法]
利用两个栈(操作数栈和运算符栈)实现中缀表达式求值。

\textbf{算法过程:}
\begin{enumerate}
\item 扫描表达式,操作数直接入操作数栈
\item 运算符与运算符栈顶比较优先级
\item 当前运算符优先级高,入运算符栈
\item 当前运算符优先级低,弹出栈顶运算符计算
\end{enumerate}
\end{example}

\begin{center}
\begin{tabular}{|c|c|c|c|}
\hline
\textbf{当前字符} & \textbf{操作数栈} & \textbf{运算符栈} & \textbf{说明} \\
\hline
$($ & & $\#,($ & $($ 入运算符栈 \\
\hline
$4$ & $4$ & $\#,($ & $4$ 入操作数栈 \\
\hline
$+$ & $4$ & $\#,(,+$ & $+$ 入运算符栈 \\
\hline
$2$ & $4,2$ & $\#,(,+$ & $2$ 入操作数栈 \\
\hline
$)$ & $6$ & $\#,($ & 计算 $4+2=6$ \\
\hline
$)$ & $6$ & $\#$ & 括号匹配,$($出栈 \\
\hline
$*$ & $6$ & $\#,*$ & $*$ 入运算符栈 \\
\hline
$3$ & $6,3$ & $\#,*$ & $3$ 入操作数栈 \\
\hline
$\#$ & $18$ & $\#$ & 计算 $6*3=18$ ,结束 \\
\hline
\end{tabular}
\end{center}

\subsection{函数调用栈}

函数的嵌套调用使用系统栈保存调用现场:

\begin{center}
\begin{tikzpicture}[scale=1]
% 函数调用栈示意图
% 栈的外框
\draw[thick] (-1.5,-0.5) rectangle (1.5,4);

% 栈中的函数调用现场
\node[rectangle,draw,thick,minimum width=2.5cm,minimum height=0.8cm,fill=lightblue,rounded corners=2pt] at (0,3.2) {\small 函数A调用现场};
\node[rectangle,draw,thick,minimum width=2.5cm,minimum height=0.8cm,fill=lightgreen,rounded corners=2pt] at (0,2.2) {\small 函数B调用现场};
\node[rectangle,draw,thick,minimum width=2.5cm,minimum height=0.8cm,fill=lightyellow,rounded corners=2pt] at (0,1.2) {\small 函数C调用现场};
\node[rectangle,draw,thick,minimum width=2.5cm,minimum height=0.8cm,fill=lightgray,rounded corners=2pt] at (0,0.2) {\small $\cdots$};

% 栈顶和栈底标记
\node[above,font=\bfseries] at (0,4.2) {栈顶};
\node[below,font=\bfseries] at (0,-0.7) {栈底};

% 状态指示箭头和标签
\draw[->,thick,red] (2,3.2) -- (1.3,3.2);
\node[right,font=\small] at (2,3.2) {\textcolor{red}{当前执行}};

\draw[->,thick,blue] (2,2.2) -- (1.3,2.2);
\node[right,font=\small] at (2,2.2) {\textcolor{blue}{被挂起}};

\draw[->,thick,blue] (2,1.2) -- (1.3,1.2);
\node[right,font=\small] at (2,1.2) {\textcolor{blue}{被挂起}};

% 说明文字
\node[below,font=\small] at (0,-1.2) {函数调用栈结构};
\end{tikzpicture}
\end{center}

\subsection{队列的应用场景}

\begin{enumerate}
\item \textbf{操作系统}:进程调度、作业队列
\item \textbf{打印缓冲}:打印任务排队
\item \textbf{键盘缓冲}:按键事件队列
\item \textbf{BFS遍历}:广度优先搜索算法
\item \textbf{银行排队}:客户服务系统
\end{enumerate}

\section{特殊栈和队列}

\subsection{共享栈}

利用一个数组实现两个栈,充分利用存储空间:

\begin{center}
\begin{tikzpicture}[scale=1]
% 共享栈示意图
% 绘制数组格子
\foreach \x in {0,1,2,3,4,5,6,7,8,9} {
    \draw[thick] (\x,0) rectangle (\x+1,1);
    \node[font=\small] at (\x+0.5,-0.3) {\x};
}

% 栈1数据 (从左侧开始)
\node[rectangle,draw,thick,minimum width=0.8cm,minimum height=0.8cm,fill=lightblue,rounded corners=1pt] at (0.5,0.5) {$a_1$};
\node[rectangle,draw,thick,minimum width=0.8cm,minimum height=0.8cm,fill=lightblue,rounded corners=1pt] at (1.5,0.5) {$a_2$};
\node[rectangle,draw,thick,minimum width=0.8cm,minimum height=0.8cm,fill=lightblue,rounded corners=1pt] at (2.5,0.5) {$a_3$};

% 栈2数据 (从右侧开始)
\node[rectangle,draw,thick,minimum width=0.8cm,minimum height=0.8cm,fill=lightgreen,rounded corners=1pt] at (9.5,0.5) {$b_1$};
\node[rectangle,draw,thick,minimum width=0.8cm,minimum height=0.8cm,fill=lightgreen,rounded corners=1pt] at (8.5,0.5) {$b_2$};
\node[rectangle,draw,thick,minimum width=0.8cm,minimum height=0.8cm,fill=lightgreen,rounded corners=1pt] at (7.5,0.5) {$b_3$};

% 栈顶指针
\draw[->,very thick,red] (3,1.3) -- (3,1.1);
\node[above,font=\bfseries,fill=red!20,rounded corners=2pt,inner sep=2pt] at (3,1.5) {\small top1};

\draw[->,very thick,blue] (7,1.3) -- (7,1.1);
\node[above,font=\bfseries,fill=blue!20,rounded corners=2pt,inner sep=2pt] at (7,1.5) {\small top2};

% 区域标注
\node[above,font=\bfseries,color=blue] at (1.5,2) {栈1};
\node[above,font=\bfseries,color=green!60!black] at (8.5,2) {栈2};
\node[above,font=\small,color=gray] at (5,2.2) {共享空间};

% 栈满条件
\node[below,font=\small] at (5,-0.8) {栈满条件:top1 + 1 == top2};

% 生长方向指示
\draw[->,thick,red,dashed] (3.5,0.5) -- (4.5,0.5);
\node[above,font=\tiny,color=red] at (4,0.8) {栈1增长};

\draw[->,thick,blue,dashed] (6.5,0.5) -- (5.5,0.5);
\node[above,font=\tiny,color=blue] at (6,0.8) {栈2增长};
\end{tikzpicture}
\end{center}

\subsection{双端队列}

允许在队列两端进行插入和删除操作:

\begin{center}
\begin{tikzpicture}[scale=1]
% 双端队列示意图
\foreach \x in {0,1,2,3,4,5} {
    \draw (\x,0) rectangle (\x+1,1);
}

\node at (0.5,0.5) {$a_1$};
\node at (1.5,0.5) {$a_2$};
\node at (2.5,0.5) {$a_3$};
\node at (3.5,0.5) {$a_4$};

% 双向箭头
\draw[<->,thick,red] (-0.5,0.5) -- (-0.1,0.5);
\node[left] at (-0.5,0.5) {出队};
\draw[<->,thick,red] (6.5,0.5) -- (6.1,0.5);
\node[right] at (6.5,0.5) {出队};

\draw[<->,thick,blue] (-0.5,1.5) -- (-0.1,1.5);
\node[left] at (-0.5,1.5) {入队};
\draw[<->,thick,blue] (6.5,1.5) -- (6.1,1.5);
\node[right] at (6.5,1.5) {入队};

\node[below] at (0,-0.5) {队头/队尾};
\node[below] at (5,-0.5) {队尾/队头};
\end{tikzpicture}
\end{center}

\section{算法复杂度分析}

\subsection{时间复杂度}

\begin{center}
\begin{tabular}{|l|c|c|c|}
\hline
\textbf{操作} & \textbf{顺序栈} & \textbf{链栈} & \textbf{说明} \\
\hline
Push & $O(1)$ & $O(1)$ & 入栈操作 \\
\hline
Pop & $O(1)$ & $O(1)$ & 出栈操作 \\
\hline
GetTop & $O(1)$ & $O(1)$ & 取栈顶元素 \\
\hline
Empty & $O(1)$ & $O(1)$ & 判空操作 \\
\hline
\end{tabular}
\end{center}

\begin{center}
\begin{tabular}{|l|c|c|c|}
\hline
\textbf{操作} & \textbf{循环队列} & \textbf{链队列} & \textbf{说明} \\
\hline
EnQueue & $O(1)$ & $O(1)$ & 入队操作 \\
\hline
DeQueue & $O(1)$ & $O(1)$ & 出队操作 \\
\hline
GetHead & $O(1)$ & $O(1)$ & 取队头元素 \\
\hline
Empty & $O(1)$ & $O(1)$ & 判空操作 \\
\hline
\end{tabular}
\end{center}

\subsection{空间复杂度}

\begin{itemize}
\item \textbf{顺序栈}:$O(n)$,需要预分配固定大小的数组
\item \textbf{链栈}:$O(n)$,动态分配,但有指针开销
\item \textbf{循环队列}:$O(n)$,需要预分配固定大小的数组
\item \textbf{链队列}:$O(n)$,动态分配,但有指针开销
\end{itemize}

\section{常见考点总结}

\subsection{重点掌握内容}

\begin{enumerate}
\item \textbf{栈和队列的基本概念}:LIFO和FIFO特性
\item \textbf{循环队列}:队空和队满的判断条件
\item \textbf{栈的应用}:括号匹配、表达式求值、函数调用
\item \textbf{队列的应用}:BFS算法、操作系统调度
\item \textbf{出入栈序列}:判断序列的合法性
\item \textbf{算法设计}:用栈和队列解决实际问题
\end{enumerate}

\subsection{常见题型}

\begin{enumerate}
\item 判断给定的出栈序列是否合法
\item 循环队列的队空、队满判断
\item 用栈实现括号匹配算法
\item 用栈实现中缀表达式求值
\item 用栈实现中缀转后缀表达式
\item 用队列实现BFS算法
\item 共享栈的设计与实现
\end{enumerate}

\subsection{解题技巧}

\begin{enumerate}
\item \textbf{出栈序列判断}:利用栈的LIFO特性,模拟入栈出栈过程
\item \textbf{循环队列}:注意取模运算和牺牲一个存储单元
\item \textbf{表达式求值}:掌握运算符优先级和两个栈的配合
\item \textbf{递归转迭代}:用栈模拟递归调用过程
\end{enumerate}

\section{典型例题解析}

\begin{example}[出栈序列判断]
栈的入栈序列为1,2,3,4,5,判断以下哪些是合法的出栈序列:
\begin{enumerate}
\item 5,4,3,2,1
\item 4,5,3,2,1  
\item 4,3,5,1,2
\end{enumerate}

\textbf{解:}
\begin{enumerate}
\item \textbf{5,4,3,2,1}:合法。全部入栈后依次出栈。
\item \textbf{4,5,3,2,1}:合法。1,2,3,4入栈,4出栈,5入栈,5出栈,3,2,1依次出栈。
\item \textbf{4,3,5,1,2}:不合法。当5出栈时,1,2还在栈中,不可能2比1先出栈。
\end{enumerate}
\end{example}

\begin{example}[循环队列容量计算]
循环队列的存储空间为Q[0..m-1],约定front指向队头元素前一个位置,rear指向队尾元素位置。当front=rear时队列为空,当(rear+1)%m=front时队列为满。求该循环队列最多可以存储多少个元素?

\textbf{解:}
设队列最多存储n个元素。当队列满时,有一个位置被浪费(用于区分队空和队满),因此:
$$n = m - 1$$
该循环队列最多可以存储$m-1$个元素。
\end{example}

\begin{example}[共享栈设计]
设计一个共享栈,用一个数组S[0..n-1]存储两个栈,栈1从数组低端开始,栈2从数组高端开始。写出入栈和出栈算法。

\textbf{解:}
\begin{lstlisting}
// 共享栈结构
struct SharedStack {
    int data[MAXSIZE];
    int top1, top2;  // 两个栈顶指针
};

// 初始化
void InitStack(SharedStack &S) {
    S.top1 = -1;
    S.top2 = MAXSIZE;
}

// 栈满判断
bool StackFull(SharedStack S) {
    return S.top1 + 1 == S.top2;
}

// 入栈操作
bool Push(SharedStack &S, int i, int x) {
    if (StackFull(S)) return false;
    if (i == 1) 
        S.data[++S.top1] = x;
    else
        S.data[--S.top2] = x;
    return true;
}

// 出栈操作
int Pop(SharedStack &S, int i) {
    if (i == 1) {
        if (S.top1 == -1) throw "栈1下溢";
        return S.data[S.top1--];
    } else {
        if (S.top2 == MAXSIZE) throw "栈2下溢";
        return S.data[S.top2++];
    }
}
\end{lstlisting}
\end{example}
\end{document} 