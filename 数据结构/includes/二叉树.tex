\section{二叉树}

\subsection{满二叉树、完全二叉树}

好的,我们来解释一下\textbf{完全二叉树(Complete Binary Tree)}。

在了解完全二叉树之前,先回顾一下\textbf{满二叉树(Full Binary Tree)\textbf{和}完美二叉树(Perfect Binary Tree)}:

\begin{itemize}
	\item \textbf{满二叉树:} 除了叶子节点外,所有节点都有两个子节点。
	\item \textbf{完美二叉树:} 是一种特殊的满二叉树,所有叶子节点都在同一层,并且所有内部节点都有两个子节点。完美二叉树的形状是一个完全对称的三角形。
\end{itemize}

现在来说\textbf{完全二叉树}:

完全二叉树是指这样一种二叉树:

\begin{enumerate}
	\item 除最后一层外,\textbf{所有层都是完全填满的}。
	\item 最后一层上的节点\textbf{从左到右依次排列},不能有空缺。也就是说,如果最后一层不满,那么节点必须连续地靠左排列。
\end{enumerate}

\textbf{关键特点:}

\begin{itemize}
	\item 它的节点分布非常紧凑,除了最后一层,每一层都达到最大节点数。
	\item 最后一层的节点尽可能靠左。
\end{itemize}

\textbf{举例说明:}

这是一个\textbf{完全二叉树}:

\begin{lstlisting}
      A
     / \
    B   C
   / \ /
  D   E F
\end{lstlisting}
解释:第0层 (A) 是满的,第1层 (B, C) 是满的,第2层 (D, E, F) 是最后一层,虽然不满,但节点从左到右是连续排列的 (D、E 在左子树,F 在右子树且靠左)。

这不是一个完全二叉树(因为最后一层有空缺,且不是最右边的空缺):

\begin{lstlisting}
      A
     / \
    B   C
   / \   \
  D   E   G
\end{lstlisting}
解释:节点 \lstinline{F} 缺失了,导致最后一层的节点 \lstinline{E} 和 \lstinline{G} 之间有空缺。

这不是一个完全二叉树(因为倒数第二层不满):

\begin{lstlisting}
      A
     /
    B
   / \
  C   D
\end{lstlisting}
解释:第1层只 \lstinline{B} 一个节点,没有右子节点 \lstinline{C} 的位置(如果是完全二叉树,\lstinline{C} 应该在 \lstinline{B} 的右边或者 \lstinline{B} 的左子树被填满后才到 \lstinline{C}),直接跳到了第2层。

\textbf{为什么完全二叉树很重要?}

完全二叉树的一个重要特性是它们可以非常高效地用\textbf{数组}来表示,而不需要使用指针。

\begin{itemize}
	\item 如果根节点存储在数组索引 0 (或 1) 的位置,那么它的左子节点将存储在索引 \lstinline{2 * i + 1} (或 \lstinline{2 * i}) 的位置,右子节点存储在索引 \lstinline{2 * i + 2} (或 \lstinline{2 * i + 1}) 的位置(取决于数组从 0 还是 1 开始索引)。
	\item 反过来,对于索引为 \lstinline{i} 的非根节点,其父节点存储在索引 \lstinline{(i - 1) / 2} (或 \lstinline{i / 2}) 的位置。
\end{itemize}

这种基于数组的表示使得完全二叉树成为实现**堆(Heap)**数据结构的最佳选择,因为堆需要快速访问父子节点,并且需要紧凑地存储数据。

简而言之,完全二叉树是一种结构紧凑、节点靠左排列的二叉树,它非常适合用数组存储,常用于实现堆。完美二叉树和满二叉树都是完全二叉树的特例。

\subsection{前序、中序、后序遍历}

对于二叉树:

\begin{lstlisting}
    A
   / \
  B   C
 / \
D   E
\end{lstlisting}
\begin{itemize}
	\item 前序遍历:根左右
	\begin{itemize}
		\item 对于上述二叉树,前序遍历的结果是:A, B, D, E, C
	\end{itemize}
	\item 中序遍历:左根右
	\begin{itemize}
		\item 对于上述二叉树,中序遍历的结果是:D, B, E, A, C
	\end{itemize}
	\item 后序遍历:左右根
	\begin{itemize}
		\item 对于上述二叉树,后序遍历的结果是:D, E, B, C, A
	\end{itemize}
\end{itemize}

\begin{note}
前序遍历加上中序遍历序列,可以唯一确定一颗二叉树.
\end{note}
\subsection{将完全二叉树存储到数组中}

好的,将一颗完全二叉树存储到数组(或 C++中的 \lstinline{std::vector})中,利用了完全二叉树的紧凑性,使得节点位置和数组索引之间存在固定的数学关系。

最常用的映射方式是:

\begin{itemize}
	\item 如果根节点存储在索引 \lstinline{i} (通常是 0 或 1) 的位置。
	\item 那么它的左子节点存储在索引 \lstinline{2 * i + 1} 的位置(如果从 0 开始索引)。
	\item 它的右子节点存储在索引 \lstinline{2 * i + 2} 的位置(如果从 0 开始索引)。
	\item 反过来,对于非根节点存储在索引 \lstinline{i} 的位置,它的父节点存储在索引 \lstinline{(i - 1) / 2} 的位置(如果从 0 开始索引,且 \lstinline{i > 0})。
\end{itemize}

由于完全二叉树除了最后一层外都是满的,且最后一层节点靠左排列,这意味着如果一棵完全二叉树有 N 个节点,那么它们可以恰好填满数组的前 N 个位置,而不会有空缺(前提是按照层序从左到右排列)。

因此,将完全二叉树存入数组最直接的方法就是进行\textbf{层序遍历(Level-Order Traversal)},并依次将遍历到的节点值存入数组。层序遍历通常使用队列(Queue)来实现。

下面是一个使用 C++ 语言,将一个给定根节点的完全二叉树存储到 \lstinline{std::vector} 中的程序示例:

\begin{lstlisting}[language=C++]
#include <iostream>
#include <vector>
#include <queue>
#include <cmath> // for calculating parent/child index if needed, but direct level-order filling is easier

// 定义二叉树的节点结构
struct TreeNode {
    int val;
    TreeNode *left;
    TreeNode *right;
    TreeNode(int x) : val(x), left(NULL), right(NULL) {}
};

// 函数:将完全二叉树存储到 vector 中
std::vector<int> storeCompleteBinaryTree(TreeNode* root) {
    std::vector<int> result; // 用于存储节点值的vector
    if (root == NULL) {
        return result; // 如果树为空,返回空vector
    }

    // 使用队列进行层序遍历
    std::queue<TreeNode*> q;
    q.push(root);

    while (!q.empty()) {
        TreeNode* current = q.front();
        q.pop();

        // 将当前节点的值存入 vector
        result.push_back(current->val);

        // 将当前节点的左孩子入队
        // 对于完全二叉树的数组表示,即使某个位置理论上是左孩子,
        // 但如果原始树中该节点不存在,我们也不会将其加入队列或vector。
        // 这里我们假设输入的TreeNode结构已经代表了一棵完全二叉树的结构。
        if (current->left != NULL) {
            q.push(current->left);
        }

        // 将当前节点的右孩子入队
        if (current->right != NULL) {
            q.push(current->right);
        }
    }

    return result;
}

// 辅助函数:创建一个示例完全二叉树
// 结构:
//       1
//      / \
//     2   3
//    / \ /
//   4   5 6
TreeNode* createExampleCompleteBinaryTree() {
    TreeNode* root = new TreeNode(1);
    root->left = new TreeNode(2);
    root->right = new TreeNode(3);
    root->left->left = new TreeNode(4);
    root->left->right = new TreeNode(5);
    root->right->left = new TreeNode(6);
    // 注意:这里特意不创建右子树的右孩子(7),使其成为非完美的完全二叉树
    return root;
}

// 辅助函数:释放树的内存
void deleteTree(TreeNode* root) {
    if (root == NULL) return;
    deleteTree(root->left);
    deleteTree(root->right);
    delete root;
}


int main() {
    // 创建一个示例完全二叉树
    TreeNode* root = createExampleCompleteBinaryTree();

    // 将完全二叉树存储到 vector
    std::vector<int> treeArray = storeCompleteBinaryTree(root);

    // 打印存储在 vector 中的节点值
    std::cout << "完全二叉树存储在数组中的顺序 (层序遍历): ";
    for (int i = 0; i < treeArray.size(); ++i) {
        std::cout << treeArray[i] << (i == treeArray.size() - 1 ? "" : " ");
    }
    std::cout << std::endl;

    // 验证数组索引和节点位置的关系 (例如,索引0的左孩子在索引1,右孩子在索引2)
    if (!treeArray.empty()) {
         std::cout << "--- 验证索引关系 (0-based) ---" << std::endl;
         std::cout << "节点 " << treeArray[0] << " (索引 0):" << std::endl;
         if (2*0 + 1 < treeArray.size()) {
             std::cout << "  左孩子 (索引 1): " << treeArray[2*0 + 1] << std::endl;
         }
         if (2*0 + 2 < treeArray.size()) {
             std::cout << "  右孩子 (索引 2): " << treeArray[2*0 + 2] << std::endl;
         }

         if (treeArray.size() > 1) {
             std::cout << "节点 " << treeArray[1] << " (索引 1):" << std::endl;
              std::cout << "  父节点 (索引 0): " << treeArray[(1 - 1) / 2] << std::endl;
              if (2*1 + 1 < treeArray.size()) {
                  std::cout << "  左孩子 (索引 3): " << treeArray[2*1 + 1] << std::endl;
              }
              if (2*1 + 2 < treeArray.size()) {
                  std::cout << "  右孩子 (索引 4): " << treeArray[2*1 + 2] << std::endl;
              }
         }
          if (treeArray.size() > 2) {
             std::cout << "节点 " << treeArray[2] << " (索引 2):" << std::endl;
              std::cout << "  父节点 (索引 0): " << treeArray[(2 - 1) / 2] << std::endl;
              if (2*2 + 1 < treeArray.size()) {
                  std::cout << "  左孩子 (索引 5): " << treeArray[2*2 + 1] << std::endl;
              }
              if (2*2 + 2 < treeArray.size()) {
                  std::cout << "  右孩子 (索引 6): " << treeArray[2*2 + 2] << std::endl; // 这里会输出 6
              }
         }
           if (treeArray.size() > 3) {
             std::cout << "节点 " << treeArray[3] << " (索引 3):" << std::endl;
              std::cout << "  父节点 (索引 1): " << treeArray[(3 - 1) / 2] << std::endl;
              // Left and Right children checks...
         }
    }


    // 释放内存
    deleteTree(root);

    return 0;
}
\end{lstlisting}
\textbf{代码解释:}

\begin{enumerate}
	\item \textbf{\lstinline{TreeNode} 结构体:} 定义了二叉树节点的基本结构,包含一个值 (\lstinline{val}) 和指向左右子节点的指针 (\lstinline{left}, \lstinline{right})。
	\item \textbf{\lstinline{storeCompleteBinaryTree} 函数:}
	\begin{itemize}
		\item 接收 \lstinline{TreeNode* root} 作为输入(树的根节点)。
		\item 创建一个 \lstinline{std::vector<int>} 叫做 \lstinline{result} 用来存放遍历结果。
		\item 如果输入的 \lstinline{root} 是 \lstinline{NULL} (空树),直接返回空的 \lstinline{result}。
		\item 创建一个 \lstinline{std::queue<TreeNode*>} 队列 \lstinline{q},用于实现层序遍历。
		\item 将根节点 \lstinline{root} 入队。
		\item 进入 \lstinline{while} 循环,当队列不为空时继续:
		\begin{itemize}
			\item 取出队列头部的节点 (\lstinline{current}) 并出队。
			\item 将 \lstinline{current} 节点的值 (\lstinline{current->val}) 添加到 \lstinline{result} 向量的末尾。
			\item 如果 \lstinline{current} 的左子节点 \lstinline{current->left} 不为空,将其入队。
			\item 如果 \lstinline{current} 的右子节点 \lstinline{current->right} 不为空,将其入队。
		\end{itemize}
		\item 循环结束后,\lstinline{result} 向量中就按层序从左到右存储了完全二叉树所有节点的值。
		\item 返回 \lstinline{result} 向量。
	\end{itemize}
	\item \textbf{\lstinline{createExampleCompleteBinaryTree} 函数:} 这是一个辅助函数,用于手动构建一个符合完全二叉树定义的示例树,方便测试。
	\item \textbf{\lstinline{deleteTree} 函数:} 辅助函数,用于递归地删除树节点,释放内存,防止内存泄漏。
	\item \textbf{\lstinline{main} 函数:}
	\begin{itemize}
		\item 调用 \lstinline{createExampleCompleteBinaryTree} 创建示例树。
		\item 调用 \lstinline{storeCompleteBinaryTree} 函数将树存储到 \lstinline{treeArray} 向量中。
		\item 打印 \lstinline{treeArray} 的内容,展示存储结果。
		\item 演示了如何根据数组索引计算父子节点的索引,并打印出对应的值,验证了完全二叉树的数组索引特性。
		\item 调用 \lstinline{deleteTree} 释放内存。
	\end{itemize}
\end{enumerate}

运行此程序,你将看到完全二叉树的节点值按照层序从左到右的顺序被打印出来,并验证了父子节点在数组中的索引关系。

\subsection{时间复杂度}

\subsubsection{二叉链表}

存储两个孩子

找双亲的时间复杂度为 $O(n)$,找孩子的时间复杂度为 $O(1)$.

\subsubsection{三叉链表}

存储两个孩子和一个双亲

\section{使用前序遍历创建二叉树}

例如创建下面的二叉树:

\begin{lstlisting}
    A
   / \
  B   C
   \
    D
\end{lstlisting}
使用前序遍历创建的过程:

\begin{enumerate}
	\item 输入前序序列:\lstinline{A B # D # # C # #},每次输入一个字符
	\begin{itemize}
		\item 输入 A:创建根节点 A
		\item 递归创建 A 的左子树:
		\begin{itemize}
			\item 输入 B:创建节点 B 作为 A 的左子节点
			\item 递归创建 B 的左子树:
			\begin{itemize}
				\item 输入\#:B 的左子节点为 NULL
			\end{itemize}
			\item 递归创建 B 的右子树:
			\begin{itemize}
				\item 输入 D:创建节点 D 作为 B 的右子节点
				\item 递归创建 D 的左子树:
				\begin{itemize}
					\item 输入\#:D 的左子节点为 NULL
				\end{itemize}
				\item 递归创建 D 的右子树:
				\begin{itemize}
					\item 输入\#:D 的右子节点为 NULL
				\end{itemize}
			\end{itemize}
		\end{itemize}
		\item 递归创建 A 的右子树:
		\begin{itemize}
			\item 输入 C:创建节点 C 作为 A 的右子节点
			\item 递归创建 C 的左子树:
			\begin{itemize}
				\item 输入\#:C 的左子节点为 NULL
			\end{itemize}
			\item 递归创建 C 的右子树:
			\begin{itemize}
				\item 输入\#:C 的右子节点为 NULL
			\end{itemize}
		\end{itemize}
	\end{itemize}
\end{enumerate}

这样,通过序列 \lstinline{A B # D # # C # #} 就完成了整棵二叉树的创建。

\subsection{Root 的作用}

举例说明 \lstinline{root} 的作用:

假设我们创建一棵如下的二叉树:

\begin{lstlisting}
    A
   / \
  B   C
\end{lstlisting}
\begin{enumerate}
	\item \textbf{创建过程}:
	\begin{itemize}
		\item 构造函数\lstinline{BiTree()}调用\lstinline{Creat(root)}
		\item 用户输入\lstinline{A B # # C # #}
		\item \lstinline{Creat}函数递归构建树并返回指向根节点A的指针
		\item 这个指针被赋值给\lstinline{root}
	\end{itemize}
	\item \textbf{访问和遍历}:
	\begin{itemize}
		\item 执行\lstinline{T.PreOrder()}时,内部会调用\lstinline{PreOrder(root)}
		\item \lstinline{root}作为起点,从A开始访问整棵树
		\item 如果没有\lstinline{root},就无法知道从哪里开始遍历
	\end{itemize}
	\item \textbf{树的判空}:
\begin{lstlisting}[language=C++]
if (root == NULL) return; //判断是否为空树
\end{lstlisting}	\item \textbf{内存释放}:
	\begin{itemize}
		\item 析构函数调用\lstinline{Release(root)}
		\item 从\lstinline{root}开始递归释放所有节点内存
	\end{itemize}
\end{enumerate}

\lstinline{root} 就像是树的"把手",有了它才能对整棵树进行操作。如果 \lstinline{root} 丢失,即使树的节点还存在内存中,也无法访问到它们,会导致内存泄漏

。

\subsection{通过中序+后序确定二叉树}

是的,通过后序遍历和中序遍历序列可以唯一地确定一棵二叉树。

\textbf{核心思想:}

\begin{enumerate}
	\item \textbf{后序遍历 (Left-Right-Root):} 序列的最后一个元素一定是当前(子)树的根节点。
	\item \textbf{中序遍历 (Left-Root-Right):} 一旦找到了根节点,那么在在中序遍历序列中,根节点左边的所有元素都属于左子树,右边的所有元素都属于右子树。
\end{enumerate}

\textbf{步骤:}

假设我们有后序遍历序列 \lstinline{postOrder} 和中序遍历序列 \lstinline{inOrder}。

\begin{enumerate}
	\item \textbf{确定根节点:}
	\begin{itemize}
		\item \lstinline{postOrder} 的最后一个元素就是整棵树的根节点。设这个根节点为 \lstinline{rootVal}。
	\end{itemize}
	\item \textbf{划分左右子树:}
	\begin{itemize}
		\item 在 \lstinline{inOrder} 序列中找到 \lstinline{rootVal} 的位置。
		\item \lstinline{inOrder} 中 \lstinline{rootVal} 左边的所有元素构成了左子树的中序遍历序列 (\lstinline{leftInOrder})。
		\item \lstinline{inOrder} 中 \lstinline{rootVal} 右边的所有元素构成了右子树的中序遍历序列 (\lstinline{rightInOrder})。
	\end{itemize}
	\item \textbf{确定左右子树的后序遍历序列:}
	\begin{itemize}
		\item 左子树的节点数量等于 \lstinline{leftInOrder} 的长度。设为 \lstinline{leftTreeSize}。
		\item 右子树的节点数量等于 \lstinline{rightInOrder} 的长度。设为 \lstinline{rightTreeSize}。
		\item 在 \lstinline{postOrder} 序列中(除去最后一个根节点 \lstinline{rootVal}):
		\begin{itemize}
			\item 前 \lstinline{leftTreeSize} 个元素构成了左子树的后序遍历序列 (\lstinline{leftPostOrder})。
			\item 接下来的 \lstinline{rightTreeSize} 个元素构成了右子树的后序遍历序列 (\lstinline{rightPostOrder})。
		\end{itemize}
	\end{itemize}
	\item \textbf{递归构建:}
	\begin{itemize}
		\item 现在我们有了左子树的 \lstinline{leftPostOrder} 和 \lstinline{leftInOrder},以及右子树的 \lstinline{rightPostOrder} 和 \lstinline{rightInOrder}。
		\item 递归地对左子树和右子树执行上述步骤,直到处理的序列为空(表示子树为空)。
		\item 将递归构建出的左子树连接到根节点的左孩子,右子树连接到根节点的右孩子。
	\end{itemize}
\end{enumerate}

\textbf{举例说明:}

假设:

\begin{itemize}
	\item \textbf{后序遍历 (postOrder):} \lstinline{[D, E, B, F, G, C, A]}
	\item \textbf{中序遍历 (inOrder):} \lstinline{[D, B, E, A, F, C, G]}
\end{itemize}

\textbf{构建过程:}

\begin{enumerate}
	\item \textbf{第一次递归 (构建整棵树):}
	\begin{itemize}
		\item \textbf{根节点:} \lstinline{postOrder} 的最后一个元素是 \lstinline{A}。所以 \lstinline{A} 是根。
		\item \textbf{划分中序:} 在 \lstinline{inOrder} \lstinline{[D, B, E, A, F, C, G]} 中,\lstinline{A} 的位置:
		\begin{itemize}
			\item \lstinline{leftInOrder} = \lstinline{[D, B, E]} (左子树的中序遍历)
			\item \lstinline{rightInOrder} = \lstinline{[F, C, G]} (右子树的中序遍历)
		\end{itemize}
		\item \textbf{划分后序:}
		\begin{itemize}
			\item 左子树节点数 \lstinline{leftTreeSize} = 3 (来自 \lstinline{leftInOrder} 的长度)。
			\item 右子树节点数 \lstinline{rightTreeSize} = 3 (来自 \lstinline{rightInOrder} 的长度)。
			\item \lstinline{postOrder} (除去 \lstinline{A}) 是 \lstinline{[D, E, B, F, G, C]}。
			\item \lstinline{leftPostOrder} = \lstinline{[D, E, B]} (前 3 个元素)
			\item \lstinline{rightPostOrder} = \lstinline{[F, G, C]} (后 3 个元素)
		\end{itemize}
		\item 现在,根节点 \lstinline{A} 的左孩子通过 \lstinline{leftPostOrder: [D, E, B]} 和 \lstinline{leftInOrder: [D, B, E]} 构建。
		\item 根节点 \lstinline{A} 的右孩子通过 \lstinline{rightPostOrder: [F, G, C]} 和 \lstinline{rightInOrder: [F, C, G]} 构建。
	\end{itemize}
	\item \textbf{第二次递归 (构建 A 的左子树):}
	\begin{itemize}
		\item 输入: \lstinline{postOrder_L = [D, E, B]}, \lstinline{inOrder_L = [D, B, E]}
		\item \textbf{根节点:} \lstinline{postOrder_L} 的最后一个元素是 \lstinline{B}。所以 \lstinline{B} 是该子树的根。
		\item \textbf{划分中序:} 在 \lstinline{inOrder_L} \lstinline{[D, B, E]} 中,\lstinline{B} 的位置:
		\begin{itemize}
			\item \lstinline{leftInOrder_L_sub} = \lstinline{[D]}
			\item \lstinline{rightInOrder_L_sub} = \lstinline{[E]}
		\end{itemize}
		\item \textbf{划分后序:}
		\begin{itemize}
			\item 左子树节点数 = 1。右子树节点数 = 1。
			\item \lstinline{postOrder_L} (除去 \lstinline{B}) 是 \lstinline{[D, E]}。
			\item \lstinline{leftPostOrder_L_sub} = \lstinline{[D]}
			\item \lstinline{rightPostOrder_L_sub} = \lstinline{[E]}
		\end{itemize}
		\item \lstinline{B} 的左孩子通过 \lstinline{post: [D]}, \lstinline{in: [D]} 构建。
		\item \lstinline{B} 的右孩子通过 \lstinline{post: [E]}, \lstinline{in: [E]} 构建。
	\end{itemize}
	\item \textbf{第三次递归 (构建 B 的左子树):}
	\begin{itemize}
		\item 输入: \lstinline{postOrder = [D]}, \lstinline{inOrder = [D]}
		\item \textbf{根节点:} \lstinline{D}。
		\item \textbf{划分中序:} \lstinline{leftInOrder} 为空,\lstinline{rightInOrder} 为空。
		\item \lstinline{D} 是叶子节点。返回 \lstinline{D}。
	\end{itemize}
	\item \textbf{第三次递归 (构建 B 的右子树):}
	\begin{itemize}
		\item 输入: \lstinline{postOrder = [E]}, \lstinline{inOrder = [E]}
		\item \textbf{根节点:} \lstinline{E}。
		\item \textbf{划分中序:} \lstinline{leftInOrder} 为空,\lstinline{rightInOrder} 为空。
		\item \lstinline{E} 是叶子节点。返回 \lstinline{E}。
		\item 此时,\lstinline{B} 的左右孩子都已确定:\lstinline{B} 的左孩子是 \lstinline{D},右孩子是 \lstinline{E}。
	\end{itemize}
	\item \textbf{第二次递归 (构建 A 的右子树):}
	\begin{itemize}
		\item 输入: \lstinline{postOrder_R = [F, G, C]}, \lstinline{inOrder_R = [F, C, G]}
		\item \textbf{根节点:} \lstinline{postOrder_R} 的最后一个元素是 \lstinline{C}。所以 \lstinline{C} 是该子树的根。
		\item \textbf{划分中序:} 在 \lstinline{inOrder_R} \lstinline{[F, C, G]} 中,\lstinline{C} 的位置:
		\begin{itemize}
			\item \lstinline{leftInOrder_R_sub} = \lstinline{[F]}
			\item \lstinline{rightInOrder_R_sub} = \lstinline{[G]}
		\end{itemize}
		\item \textbf{划分后序:}
		\begin{itemize}
			\item 左子树节点数 = 1。右子树节点数 = 1。
			\item \lstinline{postOrder_R} (除去 \lstinline{C}) 是 \lstinline{[F, G]}。
			\item \lstinline{leftPostOrder_R_sub} = \lstinline{[F]}
			\item \lstinline{rightPostOrder_R_sub} = \lstinline{[G]}
		\end{itemize}
		\item \lstinline{C} 的左孩子通过 \lstinline{post: [F]}, \lstinline{in: [F]} 构建 (得到叶子节点 \lstinline{F})。
		\item \lstinline{C} 的右孩子通过 \lstinline{post: [G]}, \lstinline{in: [G]} 构建 (得到叶子节点 \lstinline{G})。
		\item 此时,\lstinline{C} 的左右孩子都已确定:\lstinline{C} 的左孩子是 \lstinline{F},右孩子是 \lstinline{G}。
	\end{itemize}
\end{enumerate}

\textbf{最终结果:}

将所有构建好的子树连接起来,得到的二叉树结构如下:

\begin{lstlisting}
      A
     / \
    B   C
   / \ / \
  D  E F  G
\end{lstlisting}
这个过程通常用递归函数来实现,基本情况是当传入的遍历序列为空时,返回 \lstinline{null} (或表示空子树的标记)。

\subsection{通过前序+中序确定二叉树}

是的,通过前序遍历和中序遍历序列同样可以唯一地确定一棵二叉树。

\textbf{核心思想:}

\begin{enumerate}
	\item \textbf{前序遍历 (Root-Left-Right):} 序列的第一个元素一定是当前(子)树的根节点。
	\item \textbf{中序遍历 (Left-Root-Right):} 一旦找到了根节点,那么在中序遍历序列中,根节点左边的所有元素都属于左子树,右边的所有元素都属于右子树。
\end{enumerate}

\textbf{步骤:}

假设我们有前序遍历序列 \lstinline{preOrder} 和中序遍历序列 \lstinline{inOrder}。

\begin{enumerate}
	\item \textbf{确定根节点:}
	\begin{itemize}
		\item \lstinline{preOrder} 的第一个元素就是整棵树的根节点。设这个根节点为 \lstinline{rootVal}。
	\end{itemize}
	\item \textbf{划分左右子树(在中序遍历中):}
	\begin{itemize}
		\item 在 \lstinline{inOrder} 序列中找到 \lstinline{rootVal} 的位置。
		\item \lstinline{inOrder} 中 \lstinline{rootVal} 左边的所有元素构成了左子树的中序遍历序列 (\lstinline{leftInOrder})。
		\item \lstinline{inOrder} 中 \lstinline{rootVal} 右边的所有元素构成了右子树的中序遍历序列 (\lstinline{rightInOrder})。
	\end{itemize}
	\item \textbf{确定左右子树的前序遍历序列:}
	\begin{itemize}
		\item 左子树的节点数量等于 \lstinline{leftInOrder} 的长度。设为 \lstinline{leftTreeSize}。
		\item 右子树的节点数量等于 \lstinline{rightInOrder} 的长度。
		\item 在 \lstinline{preOrder} 序列中(除去第一个根节点 \lstinline{rootVal}):
		\begin{itemize}
			\item 接下来的 \lstinline{leftTreeSize} 个元素构成了左子树的前序遍历序列 (\lstinline{leftPreOrder})。
			\item 再往后的元素构成了右子树的前序遍历序列 (\lstinline{rightPreOrder})。
		\end{itemize}
	\end{itemize}
	\item \textbf{递归构建:}
	\begin{itemize}
		\item 现在我们有了左子树的 \lstinline{leftPreOrder} 和 \lstinline{leftInOrder},以及右子树的 \lstinline{rightPreOrder} 和 \lstinline{rightInOrder}。
		\item 递归地对左子树和右子树执行上述步骤,直到处理的序列为空(表示子树为空)。
		\item 将递归构建出的左子树连接到根节点的左孩子,右子树连接到根节点的右孩子。
	\end{itemize}
\end{enumerate}

\textbf{举例说明:}

假设:

\begin{itemize}
	\item \textbf{前序遍历 (preOrder):} \lstinline{[A, B, D, E, C, F, G]}
	\item \textbf{中序遍历 (inOrder):} \lstinline{[D, B, E, A, F, C, G]}
\end{itemize}

\textbf{构建过程:}

\begin{enumerate}
	\item \textbf{第一次递归 (构建整棵树):}
	\begin{itemize}
		\item \textbf{根节点:} \lstinline{preOrder} 的第一个元素是 \lstinline{A}。所以 \lstinline{A} 是根。
		\item \textbf{划分中序:} 在 \lstinline{inOrder} \lstinline{[D, B, E, A, F, C, G]} 中,\lstinline{A} 的位置:
		\begin{itemize}
			\item \lstinline{leftInOrder} = \lstinline{[D, B, E]} (左子树的中序遍历)
			\item \lstinline{rightInOrder} = \lstinline{[F, C, G]} (右子树的中序遍历)
		\end{itemize}
		\item \textbf{划分前序:}
		\begin{itemize}
			\item 左子树节点数 \lstinline{leftTreeSize} = 3 (来自 \lstinline{leftInOrder} 的长度)。
			\item \lstinline{preOrder} (除去 \lstinline{A}) 是 \lstinline{[B, D, E, C, F, G]}。
			\item \lstinline{leftPreOrder} = \lstinline{[B, D, E]} (接下来的 3 个元素)
			\item \lstinline{rightPreOrder} = \lstinline{[C, F, G]} (剩下的元素)
		\end{itemize}
		\item 现在,根节点 \lstinline{A} 的左孩子通过 \lstinline{leftPreOrder: [B, D, E]} 和 \lstinline{leftInOrder: [D, B, E]} 构建。
		\item 根节点 \lstinline{A} 的右孩子通过 \lstinline{rightPreOrder: [C, F, G]} 和 \lstinline{rightInOrder: [F, C, G]} 构建。
	\end{itemize}
	\item \textbf{第二次递归 (构建 A 的左子树):}
	\begin{itemize}
		\item 输入: \lstinline{preOrder_L = [B, D, E]}, \lstinline{inOrder_L = [D, B, E]}
		\item \textbf{根节点:} \lstinline{preOrder_L} 的第一个元素是 \lstinline{B}。所以 \lstinline{B} 是该子树的根。
		\item \textbf{划分中序:} 在 \lstinline{inOrder_L} \lstinline{[D, B, E]} 中,\lstinline{B} 的位置:
		\begin{itemize}
			\item \lstinline{leftInOrder_L_sub} = \lstinline{[D]}
			\item \lstinline{rightInOrder_L_sub} = \lstinline{[E]}
		\end{itemize}
		\item \textbf{划分前序:}
		\begin{itemize}
			\item 左子树节点数 = 1。
			\item \lstinline{preOrder_L} (除去 \lstinline{B}) 是 \lstinline{[D, E]}。
			\item \lstinline{leftPreOrder_L_sub} = \lstinline{[D]} (接下来的 1 个元素)
			\item \lstinline{rightPreOrder_L_sub} = \lstinline{[E]} (剩下的元素)
		\end{itemize}
		\item \lstinline{B} 的左孩子通过 \lstinline{pre: [D]}, \lstinline{in: [D]} 构建。
		\item \lstinline{B} 的右孩子通过 \lstinline{pre: [E]}, \lstinline{in: [E]} 构建。
	\end{itemize}
	\item \textbf{第三次递归 (构建 B 的左子树):}
	\begin{itemize}
		\item 输入: \lstinline{preOrder = [D]}, \lstinline{inOrder = [D]}
		\item \textbf{根节点:} \lstinline{D}。
		\item \textbf{划分中序:} \lstinline{leftInOrder} 为空,\lstinline{rightInOrder} 为空。
		\item \lstinline{D} 是叶子节点。返回 \lstinline{D}。
	\end{itemize}
	\item \textbf{第三次递归 (构建 B 的右子树):}
	\begin{itemize}
		\item 输入: \lstinline{preOrder = [E]}, \lstinline{inOrder = [E]}
		\item \textbf{根节点:} \lstinline{E}。
		\item \textbf{划分中序:} \lstinline{leftInOrder} 为空,\lstinline{rightInOrder} 为空。
		\item \lstinline{E} 是叶子节点。返回 \lstinline{E}。
		\item 此时,\lstinline{B} 的左右孩子都已确定:\lstinline{B} 的左孩子是 \lstinline{D},右孩子是 \lstinline{E}。
	\end{itemize}
	\item \textbf{第二次递归 (构建 A 的右子树):}
	\begin{itemize}
		\item 输入: \lstinline{preOrder_R = [C, F, G]}, \lstinline{inOrder_R = [F, C, G]}
		\item \textbf{根节点:} \lstinline{preOrder_R} 的第一个元素是 \lstinline{C}。所以 \lstinline{C} 是该子树的根。
		\item \textbf{划分中序:} 在 \lstinline{inOrder_R} \lstinline{[F, C, G]} 中,\lstinline{C} 的位置:
		\begin{itemize}
			\item \lstinline{leftInOrder_R_sub} = \lstinline{[F]}
			\item \lstinline{rightInOrder_R_sub} = \lstinline{[G]}
		\end{itemize}
		\item \textbf{划分前序:}
		\begin{itemize}
			\item 左子树节点数 = 1。
			\item \lstinline{preOrder_R} (除去 \lstinline{C}) 是 \lstinline{[F, G]}。
			\item \lstinline{leftPreOrder_R_sub} = \lstinline{[F]} (接下来的 1 个元素)
			\item \lstinline{rightPreOrder_R_sub} = \lstinline{[G]} (剩下的元素)
		\end{itemize}
		\item \lstinline{C} 的左孩子通过 \lstinline{pre: [F]}, \lstinline{in: [F]} 构建 (得到叶子节点 \lstinline{F})。
		\item \lstinline{C} 的右孩子通过 \lstinline{pre: [G]}, \lstinline{in: [G]} 构建 (得到叶子节点 \lstinline{G})。
		\item 此时,\lstinline{C} 的左右孩子都已确定:\lstinline{C} 的左孩子是 \lstinline{F},右孩子是 \lstinline{G}。
	\end{itemize}
\end{enumerate}

\textbf{最终结果:}

将所有构建好的子树连接起来,得到的二叉树结构如下:

\begin{lstlisting}
      A
     / \
    B   C
   / \ / \
  D  E F  G
\end{lstlisting}
这个过程与使用后序和中序遍历构建树非常相似,主要的区别在于从哪个遍历序列中取根节点(前序是第一个,后序是最后一个),以及如何根据根节点和中序遍历划分对应遍历序列中的子序列。

\subsection{前序+后序不能确定二叉树}

让我们来看一个例子:

\textbf{情况 1:} 树 A:

\begin{lstlisting}
    1
   /
  2
\end{lstlisting}
\begin{itemize}
	\item 前序遍历: \lstinline{[1, 2]}
	\item 后序遍历: \lstinline{[2, 1]}
\end{itemize}

\textbf{情况 2:} 树 B:

\begin{lstlisting}
    1
     \
      2
\end{lstlisting}
\begin{itemize}
	\item 前序遍历: \lstinline{[1, 2]}
	\item 后序遍历: \lstinline{[2, 1]}
\end{itemize}

对于这两棵不同的树,它们的前序遍历和后序遍历结果是完全相同的。因此,仅给出前序 \lstinline{[1, 2]} 和后序 \lstinline{[2, 1]},我们无法判断节点 \lstinline{2} 究竟是节点 \lstinline{1} 的左孩子还是右孩子。
