\section{邻接矩阵与邻接表}

好的,我们来介绍图论中两种常用的图存储结构:\textbf{邻接矩阵 (Adjacency Matrix)} 和 \textbf{邻接表 (Adjacency List)}。

这两种结构都是用来描述图中顶点与顶点之间的连接关系的。选择哪种结构取决于图的特性(比如边的数量)以及我们希望对图进行的操作的效率。

让我们用一个具体的例子来说明。假设我们有以下这个\textbf{无向图 G1}:

\begin{lstlisting}
      A --- B
      |   /
      |  /
      C --- D
\end{lstlisting}
顶点 (Vertices): A, B, C, D

边 (Edges): (A,B), (A,C), (B,C), (C,D)

再假设一个\textbf{有向图 G2}:

\begin{lstlisting}
      1 ---> 2
      ^      |
      |      |
      4 <--- 3 <--- 2 (重复边,简化为 2->3)
      |      ^
      |______|
\end{lstlisting}
顶点 (Vertices): 1, 2, 3, 4

边 (Edges): (1,2), (2,3), (3,4), (4,1), (4,3)


\subsection{邻接矩阵 (Adjacency Matrix)}

邻接矩阵是一个二维数组(通常是方阵),用 \lstinline{matrix[i][j]} 的值来表示顶点 \lstinline{i} 和顶点 \lstinline{j} 之间是否存在边。

\begin{itemize}
	\item \textbf{大小:} 如果图中有 \lstinline{V} 个顶点,邻接矩阵就是一个 \lstinline{V x V} 的方阵。
	\item \textbf{表示方法:}
	\begin{itemize}
		\item \textbf{无权图:} 如果顶点 \lstinline{i} 和顶点 \lstinline{j} 之间有边,则 \lstinline{matrix[i][j] = 1};否则为 \lstinline{0}。
		\item \textbf{加权图:} 如果顶点 \lstinline{i} 和顶点 \lstinline{j} 之间有边,则 \lstinline{matrix[i][j]} 等于该边的权重;如果不存在边,可以是 \lstinline{0}、无穷大 (∞) 或一个特殊值(取决于具体应用,0 可能也是一个有效的权重)。
		\item \textbf{无向图:} 邻接矩阵是对称的,即 \lstinline{matrix[i][j] = matrix[j][i]}。
		\item \textbf{有向图:} 如果存在从顶点 \lstinline{i} 到顶点 \lstinline{j} 的边,则 \lstinline{matrix[i][j] = 1} (或权重);\lstinline{matrix[j][i]} 则表示从 \lstinline{j} 到 \lstinline{i} 的边,两者不一定相同。
	\end{itemize}
\end{itemize}

\textbf{示例 G1 (无向无权图):}

为了方便表示,我们将顶点 A, B, C, D 映射为索引 0, 1, 2, 3。

(A=0, B=1, C=2, D=3)

\begin{lstlisting}
  A B C D
A[0,1,1,0]
B[1,0,1,0]
C[1,1,0,1]
D[0,0,1,0]
\end{lstlisting}
\begin{itemize}
	\item \lstinline{matrix[0][1] = 1} 表示 A 和 B 之间有边。由于是无向图,\lstinline{matrix[1][0]} 也为 \lstinline{1}。
	\item \lstinline{matrix[0][3] = 0} 表示 A 和 D 之间没有边。
\end{itemize}

\textbf{示例 G2 (有向无权图):}

(1=0, 2=1, 3=2, 4=3)

\begin{lstlisting}
  1 2 3 4
1[0,1,0,0]  (1->2)
2[0,0,1,0]  (2->3)
3[0,0,0,1]  (3->4)
4[1,0,1,0]  (4->1, 4->3)
\end{lstlisting}
\textbf{优点:}

\begin{enumerate}
	\item \textbf{判断边是否存在效率高:} 检查任意两个顶点 \lstinline{i} 和 \lstinline{j} 之间是否有边(或者获取边的权重)只需要 O(1) 的时间复杂度,即直接访问 \lstinline{matrix[i][j]}。
	\item \textbf{添加或删除边操作简单:} 对于无权图,只需修改 \lstinline{matrix[i][j]} 的值为 1 或 0。对于加权图,修改为对应的权重或表示无边的值。
	\item \textbf{结构简单直观:} 易于理解和实现。
\end{enumerate}

\textbf{缺点:}

\begin{enumerate}
	\item \textbf{空间复杂度高:} 无论图中有多少条边,邻接矩阵都需要 \lstinline{V*V} 的空间(\lstinline{V} 是顶点数)。对于稀疏图(边的数量远小于 \lstinline{V*V}),这会造成大量的空间浪费。例如,一个有10000个顶点但只有几千条边的图,邻接矩阵仍然需要 10000 * 10000 的空间。
	\item \textbf{遍历一个顶点的所有邻接点效率低:} 要找到顶点 \lstinline{i} 的所有邻接点,需要遍历矩阵的第 \lstinline{i} 行(或第 \lstinline{i} 列),时间复杂度为 O(V)。对于稀疏图来说,这很慢,因为大部分检查都是针对不存在的边。
	\item \textbf{添加或删除顶点操作复杂:} 因为需要重建整个矩阵,或者至少动态调整二维数组的大小,这通常很麻烦。
\end{enumerate}


\subsection{邻接表 (Adjacency List)}

邻接表是一种更节省空间的图存储结构,尤其适用于稀疏图。它由一个包含所有顶点的列表(或数组)组成,对于列表中的每个顶点,都有一个与之关联的链表(或其他动态数据结构,如动态数组),该链表存储了该顶点的所有邻接点。

\begin{itemize}
	\item \textbf{结构:} 通常是一个数组,数组的每个元素代表一个顶点。数组的第 \lstinline{i} 个元素指向一个链表,该链表包含了所有与顶点 \lstinline{i} 相邻的顶点。
	\item \textbf{表示方法:}
	\begin{itemize}
		\item \textbf{无权图:} 链表中只存储邻接顶点的编号。
		\item \textbf{加权图:} 链表中的每个节点不仅存储邻接顶点的编号,还存储边的权重。通常用一个 (邻接点, 权重) 的序对来表示。
		\item \textbf{无向图:} 如果顶点 \lstinline{u} 和 \lstinline{v} 之间有一条边,那么 \lstinline{v} 会出现在 \lstinline{u} 的邻接表中,同时 \lstinline{u} 也会出现在 \lstinline{v} 的邻接表中。
		\item \textbf{有向图:} 如果存在从顶点 \lstinline{u} 到顶点 \lstinline{v} 的边,那么 \lstinline{v} 会出现在 \lstinline{u} 的邻接表中,但 \lstinline{u} 不一定会出现在 \lstinline{v} 的邻接表中(除非也存在从 \lstinline{v} 到 \lstinline{u} 的边)。
	\end{itemize}
\end{itemize}

\textbf{示例 G1 (无向无权图):}

(A=0, B=1, C=2, D=3)

\begin{lstlisting}
A (0) -> [B(1), C(2)]
B (1) -> [A(0), C(2)]
C (2) -> [A(0), B(1), D(3)]
D (3) -> [C(2)]
\end{lstlisting}
或者更常见的表示形式是一个数组,每个索引对应一个顶点,其值为指向邻接点链表的指针/引用:

\begin{lstlisting}
0 (A): -> B -> C -> NULL
1 (B): -> A -> C -> NULL
2 (C): -> A -> B -> D -> NULL
3 (D): -> C -> NULL
\end{lstlisting}
\textbf{示例 G2 (有向无权图):}

(1=0, 2=1, 3=2, 4=3)

\begin{lstlisting}
0 (1): -> 2 -> NULL
1 (2): -> 3 -> NULL
2 (3): -> 4 -> NULL
3 (4): -> 1 -> 3 -> NULL
\end{lstlisting}
如果是有向加权图,例如从 4 到 3 的权重是 5,那么 \lstinline{3 (4): -> (1, w_41) -> (3, 5) -> NULL}。

\textbf{优点:}

\begin{enumerate}
	\item \textbf{空间效率高:} 特别是对于稀疏图。邻接表的空间复杂度为 O(V + E),其中 \lstinline{V} 是顶点数,\lstinline{E} 是边数。因为它只存储实际存在的边。
	\item \textbf{遍历一个顶点的所有邻接点效率高:} 找到顶点 \lstinline{i} 的所有邻接点,只需要遍历其对应的链表即可。对于度数(邻接点数量)为 \lstinline{k} 的顶点,时间复杂度为 O(k)。
	\item \textbf{添加顶点相对容易:} 只需在主列表中添加一个新条目。
	\item \textbf{添加边也相对高效:} 对于无向图,在两个顶点的链表中各添加一个节点;对于有向图,在一个顶点的链表中添加一个节点。
\end{enumerate}

\textbf{缺点:}

\begin{enumerate}
	\item \textbf{判断边是否存在效率较低:} 检查顶点 \lstinline{i} 和 \lstinline{j} 之间是否存在边,需要遍历顶点 \lstinline{i}(或 \lstinline{j})的邻接链表,最坏情况下时间复杂度为 O(V) (在一个完全图中,一个顶点的度可以是 V-1)。
	\item \textbf{删除边操作可能稍复杂:} 需要在对应的链表中找到并删除节点。
	\item \textbf{实现上比邻接矩阵稍复杂一些:} 因为涉及到链表或其他动态数据结构的操作。
\end{enumerate}


\textbf{总结对比:}

\begin{table}[h]
	\centering
	\begin{tabular}{|c|c|c|}
		\hline
		 &  &  \\
		\hline
		\textbf{特性} & \textbf{邻接矩阵} & \textbf{邻接表} \\
		\hline
		\textbf{空间复杂度} & O(V²) & O(V + E) \\
		\hline
		\textbf{判断边 (u,v) 是否存在} & O(1) & O(degree(u)) 或 O(V) 最坏情况 \\
		\hline
		\textbf{列出顶点的所有邻居} & O(V) & O(degree(u)) \\
		\hline
		\textbf{添加边} & O(1) & O(1) (如果链表头插) 或 O(degree(u)) 防止重复 \\
		\hline
		\textbf{删除边} & O(1) & O(degree(u)) \\
		\hline
		\textbf{添加顶点} & O(V²) (重建或扩展) & O(1) (主列表) \\
		\hline
		\textbf{删除顶点} & O(V²) (重建或标记) & O(V+E) (需要更新所有相关邻接表) \\
		\hline
		\textbf{适用于} & 稠密图 (边数接近 V²) & 稀疏图 (边数远小于 V²) \\
		\hline
	\end{tabular}
\end{table}
选择哪种表示方法通常取决于图的密度和需要执行的主要操作类型。如果需要频繁检查边的存在性,且图比较小或者是稠密图,邻接矩阵可能更好。如果图是稀疏的,并且经常需要遍历顶点的邻居,邻接表通常是更优的选择。
