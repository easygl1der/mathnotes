\section{数据结构:栈 (Stack) - C++ 快速复习指南}

\subsection{1. 什么是栈?}

栈是一种特殊的线性数据结构,遵循\textbf{后进先出 (Last-In, First-Out, LIFO)} 的原则。可以将其想象成一叠盘子:你最后放上去的盘子,会是第一个被取走的。

\begin{itemize}
	\item \textbf{栈顶 (Top):} 允许进行插入和删除操作的一端。
	\item \textbf{栈底 (Bottom):} 栈的另一端,通常固定不动。
	\item \textbf{压栈/入栈 (Push):} 向栈顶添加元素的过程。
	\item \textbf{弹栈/出栈 (Pop):} 从栈顶移除元素的过程。
\end{itemize}

\textbf{核心特性总结:}

\begin{itemize}
	\item \textbf{LIFO原则:} 最后进入栈的元素最先离开。
	\item \textbf{受限的线性表:} 只能在栈顶进行操作。
\end{itemize}

\subsection{2. 栈的基本操作}

栈主要支持以下几种基本操作:

\begin{itemize}
	\item \lstinline{push(element)}: 将元素 \lstinline{element} 压入栈顶。
	\begin{itemize}
		\item \textbf{前置条件 (对于有界实现):} 栈未满。
		\item \textbf{后置条件:} \lstinline{element} 成为新的栈顶元素,栈的大小增加1。
	\end{itemize}
	\item \lstinline{pop()}: 移除栈顶元素。\textbf{注意:} 在 C++ 的 \lstinline{std::stack} 中,\lstinline{pop()} 只移除元素,不返回值。获取栈顶元素需要使用 \lstinline{top()}。
	\begin{itemize}
		\item \textbf{前置条件:} 栈不能为空。
		\item \textbf{后置条件:} 原栈顶元素被移除,其下的元素(如果存在)成为新的栈顶,栈的大小减少1。
	\end{itemize}
	\item \lstinline{top()} (或 \lstinline{peek()}): 返回对栈顶元素的引用(或值),但不移除它。
	\begin{itemize}
		\item \textbf{前置条件:} 栈不能为空。
		\item \textbf{后置条件:} 栈的状态不改变。
	\end{itemize}
	\item \lstinline{empty()}: 检查栈是否为空。
	\begin{itemize}
		\item \textbf{后置条件:} 如果栈中没有元素,返回 \lstinline{true};否则返回 \lstinline{false}。
	\end{itemize}
	\item \lstinline{size()}: 返回栈中元素的数量。
\end{itemize}

\subsection{3. 栈的常见应用场景}

栈在计算机科学中有广泛的应用,例如:

\begin{itemize}
	\item \textbf{函数调用栈 (Call Stack):} 管理函数调用的顺序,存储局部变量、参数和返回地址。
	\item \textbf{表达式求值:}
	\begin{itemize}
		\item 中缀表达式转后缀表达式 (逆波兰表达式)。
		\item 后缀表达式求值。
	\end{itemize}
	\item \textbf{括号匹配:} 检查代码或表达式中的括号对(圆括号、花括号、方括号)是否成对出现且正确嵌套。
	\item \textbf{深度优先搜索 (DFS - Depth First Search):} 在图或树的遍历中,栈可以用来存储待访问的节点。
	\item \textbf{浏览器的前进/后退功能:} 后退历史可以看作一个栈。
	\item \textbf{撤销/重做 (Undo/Redo) 操作:} 用户操作可以被存储在一个栈中。
	\item \textbf{递归的非递归实现:} 许多递归算法可以用栈来模拟实现。
\end{itemize}

\subsection{4. 栈的 C++ 实现方式}

在 C++ 中,栈可以通过多种方式实现:

\begin{itemize}
	\item \textbf{基于数组的实现 (静态或动态)}
	\item \textbf{基于链表的实现}
	\item \textbf{使用 C++ 标准库的 \lstinline{std::stack} (适配器类)}
\end{itemize}

\subsubsection{4.1 基于数组的实现 (静态数组)}

使用一个固定大小的数组来存储栈中的元素。需要一个额外的变量(通常称为 \lstinline{topIndex} 或 \lstinline{count})来指示栈顶元素在数组中的下一个可用位置或当前栈顶元素的索引。

\textbf{优点:}

\begin{itemize}
	\item 实现相对简单。
	\item 内存分配在编译时或对象创建时完成,访问速度快(缓存友好)。
\end{itemize}

\textbf{缺点:}

\begin{itemize}
	\item \textbf{大小固定:} 栈的容量在创建时就确定了。如果元素数量超过数组容量,会导致栈溢出 (Stack Overflow)。
	\item \textbf{空间浪费:} 如果分配的数组空间远大于实际存储的元素数量,会造成空间浪费。
\end{itemize}

\textbf{C++ 代码示例 (静态数组):}

\begin{lstlisting}[language=C++]
#include <iostream>
#include <stdexcept> // 用于 std::out_of_range 和 std::runtime_error

const int MAX_SIZE = 100; // 预设栈的最大容量

template <typename T>
class ArrayStack {
 private:
  T arr[MAX_SIZE];
  int topIndex;  // 指向栈顶元素的索引,-1 表示空栈

 public:
  ArrayStack() : topIndex(-1) {}

  // 压栈
  void push(const T& element) {
    if (isFull()) {
      throw std::overflow_error("Stack overflow: Cannot push to a full stack.");
    }
    arr[++topIndex] = element;
  }

  // 弹栈 (只移除,不返回)
  void pop() {
    if (isEmpty()) {
      throw std::underflow_error("Stack underflow: Cannot pop from an empty stack.");
    }
    topIndex--;
  }

  // 查看栈顶元素
  T& top() {
    if (isEmpty()) {
      throw std::underflow_error("Stack is empty: Cannot get top element.");
    }
    return arr[topIndex];
  }

  // 查看栈顶元素 (const 版本)
  const T& top() const {
    if (isEmpty()) {
      throw std::underflow_error("Stack is empty: Cannot get top element.");
    }
    return arr[topIndex];
  }

  // 检查栈是否为空
  bool isEmpty() const {
    return topIndex == -1;
  }

  // 检查栈是否已满
  bool isFull() const {
    return topIndex == MAX_SIZE - 1;
  }

  // 获取栈中元素数量
  int size() const {
    return topIndex + 1;
  }

  // 打印栈内容 (用于调试)
  void print() const {
    if (isEmpty()) {
      std::cout << "Stack is empty." << std::endl;
      return;
    }
    std::cout << "Stack (top to bottom): ";
    for (int i = topIndex; i >= 0; --i) {
      std::cout << arr[i] << " ";
    }
    std::cout << std::endl;
  }
};

// --- ArrayStack 示例用法 ---
void testArrayStack() {
  std::cout << "--- ArrayStack (Static) 示例 ---" << std::endl;
  ArrayStack<int> s;
  std::cout << "Is empty: " << s.isEmpty() << std::endl;  // 1 (true)

  s.push(10);
  s.push(20);
  s.push(30);
  s.print();  // Stack (top to bottom): 30 20 10

  std::cout << "Top element: " << s.top() << std::endl;  // 30
  std::cout << "Size: " << s.size() << std::endl;         // 3
  std::cout << "Is empty: " << s.isEmpty() << std::endl;  // 0 (false)

  s.pop();                                                      // 移除 30
  s.print();                                                    // Stack (top to bottom): 20 10
  std::cout << "Top element after pop: " << s.top() << std::endl;  // 20

  s.pop();    // 移除 20
  s.pop();    // 移除 10
  s.print();  // Stack is empty.
  std::cout << "Is empty: " << s.isEmpty() << std::endl;  // 1 (true)

  try {
    s.pop();  // 尝试从空栈 pop
  } catch (const std::underflow_error& e) {
    std::cerr << "Caught exception: " << e.what() << std::endl;
  }

  // 填满栈并尝试溢出
  ArrayStack<char> charStack;
  for (int i = 0; i < MAX_SIZE; ++i) {
    charStack.push('A' + (i % 26));
  }
  std::cout << "Char stack is full: " << charStack.isFull() << std::endl;
  try {
    charStack.push('Z');
  } catch (const std::overflow_error& e) {
    std::cerr << "Caught exception: " << e.what() << std::endl;
  }
  std::cout << std::endl;
}
\end{lstlisting}
\textbf{时间复杂度 (静态数组实现):}

\begin{itemize}
	\item \lstinline{push(element)}: $O(1)$
	\item \lstinline{pop()}: $O(1)$
	\item \lstinline{top()}: $O(1)$
	\item \lstinline{isEmpty()}: $O(1)$
	\item \lstinline{isFull()}: $O(1)$
	\item \lstinline{size()}: $O(1)$
\end{itemize}

\textbf{空间复杂度 (静态数组实现):} $O(N)$,其中 $N$ 是数组的最大容量 (MAX\_SIZE)。

\begin{itemize}
	\item \textbf{注意:} 对于动态数组实现 (使用 \lstinline{new T[capacity]} 和 \lstinline{delete[]}),\lstinline{push} 操作在需要扩容时的时间复杂度会是 $O(N)$ (其中 N 为当前元素数量),但均摊时间复杂度仍为 $O(1)$。
\end{itemize}

\subsubsection{4.2 基于链表的实现}

使用链表(通常是单向链表)来存储栈中的元素。链表的头节点通常作为栈顶。

\textbf{优点:}

\begin{itemize}
	\item \textbf{动态大小:} 栈的大小可以根据需要动态增长或缩小,没有预设的容量限制(受限于系统可用内存)。
	\item \textbf{空间高效:} 只为实际存储的元素分配内存。
\end{itemize}

\textbf{缺点:}

\begin{itemize}
	\item \textbf{额外的内存开销:} 每个节点都需要额外的空间来存储指向下一个节点的指针。
	\item \textbf{内存不连续:} 相比数组,可能在缓存利用率上稍差。
\end{itemize}

\textbf{C++ 代码示例 (链表):}

\begin{lstlisting}[language=C++]
#include <iostream>
#include <stdexcept>  // 用于 std::out_of_range

template <typename T>
class LinkedStack {
 private:
  struct Node {
    T data;
    Node* next;
    Node(const T& val) : data(val), next(nullptr) {}
  };

  Node* topNode;  // 指向栈顶节点
  int count;      // 栈中元素数量

 public:
  LinkedStack() : topNode(nullptr), count(0) {}

  ~LinkedStack() {  // 析构函数,释放所有节点内存
    while (!isEmpty()) {
      pop();
    }
  }

  // 压栈
  void push(const T& element) {
    Node* newNode = new Node(element);
    newNode->next = topNode;
    topNode = newNode;
    count++;
  }

  // 弹栈 (只移除,不返回)
  void pop() {
    if (isEmpty()) {
      throw std::underflow_error("Stack underflow: Cannot pop from an empty stack.");
    }
    Node* temp = topNode;
    topNode = topNode->next;
    delete temp;
    count--;
  }

  // 查看栈顶元素
  T& top() {
    if (isEmpty()) {
      throw std::underflow_error("Stack is empty: Cannot get top element.");
    }
    return topNode->data;
  }

  // 查看栈顶元素 (const 版本)
  const T& top() const {
    if (isEmpty()) {
      throw std::underflow_error("Stack is empty: Cannot get top element.");
    }
    return topNode->data;
  }

  // 检查栈是否为空
  bool isEmpty() const {
    return topNode == nullptr;  // 或者 count == 0
  }

  // 获取栈中元素数量
  int size() const {
    return count;
  }

  // 打印栈内容 (用于调试)
  void print() const {
    if (isEmpty()) {
      std::cout << "LinkedStack is empty." << std::endl;
      return;
    }
    std::cout << "LinkedStack (top to bottom): ";
    Node* current = topNode;
    while (current != nullptr) {
      std::cout << current->data << " ";
      current = current->next;
    }
    std::cout << std::endl;
  }
};

// --- LinkedStack 示例用法 ---
void testLinkedStack() {
  std::cout << "--- LinkedStack 示例 ---" << std::endl;
  LinkedStack<std::string> s;
  std::cout << "Is empty: " << s.isEmpty() << std::endl;  // 1 (true)

  s.push("Hello");
  s.push("World");
  s.push("C++");
  s.print();  // LinkedStack (top to bottom): C++ World Hello

  std::cout << "Top element: " << s.top() << std::endl;  // C++
  std::cout << "Size: " << s.size() << std::endl;         // 3

  s.pop();                                                      // 移除 "C++"
  s.print();                                                    // LinkedStack (top to bottom): World Hello
  std::cout << "Top element after pop: " << s.top() << std::endl;  // World

  s.pop();    // 移除 "World"
  s.pop();    // 移除 "Hello"
  s.print();  // LinkedStack is empty.
  std::cout << "Is empty: " << s.isEmpty() << std::endl;  // 1 (true)

  try {
    std::cout << s.top() << std::endl;  // 尝试从空栈获取 top
  } catch (const std::underflow_error& e) {
    std::cerr << "Caught exception: " << e.what() << std::endl;
  }
  std::cout << std::endl;
}
\end{lstlisting}
\textbf{时间复杂度 (链表实现):}

\begin{itemize}
	\item \lstinline{push(element)}: $O(1)$ (在链表头部插入)
	\item \lstinline{pop()}: $O(1)$ (删除链表头部节点)
	\item \lstinline{top()}: $O(1)$ (访问链表头部节点)
	\item \lstinline{isEmpty()}: $O(1)$
	\item \lstinline{size()}: $O(1)$ (如果维护了一个 \lstinline{count} 变量)
\end{itemize}

\textbf{空间复杂度 (链表实现):} $O(N)$,其中 $N$ 是栈中元素的数量。每个元素都需要一个节点的空间。

\subsubsection{4.3 使用 C++ 标准库的 \lstinline{std::stack}}

C++ 标准库提供了一个容器适配器 \lstinline{std::stack},它默认使用 \lstinline{std::deque} (双端队列) 作为其底层容器,但也可以配置为使用 \lstinline{std::vector} 或 \lstinline{std::list}。\lstinline{std::stack} 封装了底层容器,使其表现出栈的行为。

\textbf{优点:}

\begin{itemize}
	\item \textbf{标准、可靠、高效:} 由标准库提供,经过良好测试和优化。
	\item \textbf{易于使用:} 无需自己实现,直接包含 \lstinline{<stack>} 头文件即可。
	\item \textbf{灵活性:} 可以选择底层容器。
\end{itemize}

\textbf{缺点:}

\begin{itemize}
	\item \textbf{功能受限:} 只提供标准的栈操作,不能直接访问底层容器的非栈操作(比如直接访问非栈顶元素)。
\end{itemize}

\textbf{C++ 代码示例 (\lstinline{std::stack}):}

\begin{lstlisting}[language=C++]
#include <iostream>
#include <stack>   // 包含 stack 头文件
#include <vector>  // 可以用作 std::stack 的底层容器
#include <list>    // 也可以用作 std::stack 的底层容器
#include <deque>   // std::stack 默认的底层容器

// --- std::stack 示例用法 ---
void testStdStack() {
  std::cout << "--- std::stack (default: deque) 示例 ---" << std::endl;
  std::stack<int> s_deque;  // 默认底层容器是 std::deque<int>

  std::cout << "Is empty: " << s_deque.empty() << std::endl;  // 1 (true)

  s_deque.push(100);
  s_deque.push(200);
  s_deque.push(300);
  // std::stack 没有直接的 print 方法,需要逐个取出打印 (会改变栈)
  // 或者访问底层容器(但不推荐,破坏封装)

  std::cout << "Top element: " << s_deque.top() << std::endl;  // 300
  std::cout << "Size: " << s_deque.size() << std::endl;         // 3

  s_deque.pop();  // 移除 300
  std::cout << "Top element after pop: " << s_deque.top() << std::endl;  // 200
  std::cout << "Size after pop: " << s_deque.size() << std::endl;         // 2

  std::cout << "Popping elements: ";
  while (!s_deque.empty()) {
    std::cout << s_deque.top() << " ";
    s_deque.pop();
  }
  std::cout << std::endl;
  std::cout << "Is empty after all pops: " << s_deque.empty() << std::endl;  // 1 (true)

  // 使用 std::vector 作为底层容器
  std::cout << "\n--- std::stack (with std::vector) 示例 ---" << std::endl;
  std::stack<double, std::vector<double>> s_vector;
  s_vector.push(1.1);
  s_vector.push(2.2);
  std::cout << "Top element (vector-based): " << s_vector.top() << std::endl;  // 2.2
  s_vector.pop();
  std::cout << "Top element after pop (vector-based): " << s_vector.top() << std::endl;  // 1.1

  // 使用 std::list 作为底层容器
  std::cout << "\n--- std::stack (with std::list) 示例 ---" << std::endl;
  std::stack<char, std::list<char>> s_list;
  s_list.push('a');
  s_list.push('b');
  std::cout << "Top element (list-based): " << s_list.top() << std::endl;  // 'b'
  s_list.pop();
  std::cout << "Top element after pop (list-based): " << s_list.top() << std::endl;  // 'a'
  std::cout << std::endl;
}
\end{lstlisting}
\textbf{\lstinline{std::stack} 的操作及其复杂度 (通常):}

\begin{itemize}
	\item \lstinline{push(element)}: 通常 $O(1)$ (均摊,取决于底层容器的 \lstinline{push_back} 或 \lstinline{push_front})
	\item \lstinline{pop()}: 通常 $O(1)$ (取决于底层容器的 \lstinline{pop_back} 或 \lstinline{pop_front})
	\item \lstinline{top()}: $O(1)$ (取决于底层容器的 \lstinline{back} 或 \lstinline{front})
	\item \lstinline{empty()}: $O(1)$
	\item \lstinline{size()}: $O(1)$
\end{itemize}

\subsection{5. 选择哪种实现?}

\begin{itemize}
	\item \textbf{\lstinline{std::stack}:} \textbf{强烈推荐}在大多数情况下使用。它简单、安全,并且是标准的一部分。
	\item \textbf{基于数组的实现:}
	\begin{itemize}
		\item \textbf{优点:} 对于固定且较小的最大容量,内存是连续的,可能具有较好的缓存性能。
		\item \textbf{缺点:} 容量固定(静态数组)或动态扩容有开销。
		\item \textbf{适用场景:} 教学目的,或者在嵌入式系统等对内存控制有严格要求的环境中,且最大容量已知。
	\end{itemize}
	\item \textbf{基于链表的实现:}
	\begin{itemize}
		\item \textbf{优点:} 真正的动态大小,插入删除操作稳定 $O(1)$。
		\item \textbf{缺点:} 每个元素有额外的指针开销,内存不连续。
		\item \textbf{适用场景:} 教学目的,或者当栈的大小非常动态且难以预估时。
	\end{itemize}
\end{itemize}

\subsection{6. 备考复习要点}

\begin{itemize}
	\item \textbf{理解 LIFO 原则。}
	\item \textbf{掌握基本操作:} \lstinline{push}, \lstinline{pop}, \lstinline{top}, \lstinline{empty}, \lstinline{size} 的含义和 C++ 中的对应。
	\item \textbf{熟悉 C++ 中的实现方式:} 数组(静态/动态)、链表、\lstinline{std::stack}。
	\item \textbf{时间与空间复杂度:} 能够分析各种操作在不同实现下的复杂度。
	\item \textbf{应用场景:} 能够举例说明栈在实际问题中的应用,并理解其工作原理。
	\item \textbf{边界条件处理:}
	\begin{itemize}
		\item 对空栈进行 \lstinline{pop} 或 \lstinline{top} 操作(如何处理,如抛出异常)。
		\item 对满栈(数组实现)进行 \lstinline{push} 操作。
	\end{itemize}
	\item \textbf{\lstinline{std::stack} 的用法:} 知道它是容器适配器,默认底层容器,以及如何指定其他底层容器。
	\item \textbf{资源管理:} 在链表或动态数组实现中,注意内存的分配 (\lstinline{new}) 和释放 (\lstinline{delete}/\lstinline{delete[]}),防止内存泄漏(特别是在析构函数和 \lstinline{pop} 中)。
\end{itemize}

\subsection{7. 练习题方向}

\begin{enumerate}
	\item 手动实现栈:用静态数组、动态数组(带扩容)和链表分别实现一个栈类。
	\item 栈的应用编程:
	\begin{itemize}
		\item 用栈实现有效的括号匹配检查器。
		\item 用栈实现后缀表达式(逆波兰表达式)求值。
		\item 用栈实现中缀表达式到后缀表达式的转换。
		\item 使用两个栈模拟一个队列。
		\item 设计一个栈,除了基本操作外,还支持一个 \lstinline{getMin()} 方法,该方法可以在 $O(1)$ 时间内返回栈中的最小元素。
		\item 用栈实现递归函数的非递归版本(例如斐波那契数列或树的遍历)。
	\end{itemize}
	\item 分析问题:判断某个问题是否适合用栈来解决,并说明理由。
\end{enumerate}

\begin{lstlisting}[language=C++]
// Main function to run test cases
int main() {
  testArrayStack();
  testLinkedStack();
  testStdStack();
  return 0;
}
\end{lstlisting}
(请将上述 \lstinline{main} 函数与之前的类定义和测试函数放在同一个 \lstinline{.C++} 文件中编译运行,或者分别编译链接。)

祝你复习顺利,考试取得好成绩!
