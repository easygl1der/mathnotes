\documentclass[
	leqno, % Add leqno option here
	a4paper, % Page size
	fontsize=12pt, % Base font size
	twoside=false, % Use different layouts for even and odd pages (in particular, if twoside=true, the margin column will be always on the outside)
	open=any, % If twoside=true, uncomment this to force new chapters to start on any page, not only on right (odd) pages
	chapterentrydots=true, % Uncomment to output dots from the chapter name to the page number in the table of contents
	numbers=noenddot, % Comment to output dots after chapter numbers; the most common values for this option are: enddot, noenddot and auto (see the KOMAScript documentation for an in-depth explanation)
    UTF-8, % Add this fix lots of style compile issue when using ctex and XeLaTex
	toc=nottotoc,
	BCOR=12mm,
	DIV=calc,
]{styles/kaobook}
%----------------------------------------------------------------------------------------
%	加载包以及文档配置,新项目可以直接复制粘贴
%----------------------------------------------------------------------------------------
\renewcommand{\marginlayout}{%
	\newgeometry{
		top=27.4mm, % 页面顶部到正文内容的距离
		bottom=27.4mm, % 页面底部到正文内容的距离
		inner=17.8mm, % 页面内侧(装订侧)到正文内容的距离
		textwidth=137mm, % width of the text
		marginparsep=5.2mm, % width between text and margin
		marginparwidth=39.4mm, % width of the margin
	}%
}

%----------------------------------------------------------------------------------------
%	包以及文档配置,新项目可以直接复制粘贴
%----------------------------------------------------------------------------------------

% 加载排版包(不支持 chinese)
\usepackage{polyglossia}
\setmainlanguage{english}
\SetLanguageKeys{english}{indentfirst=true}
% 加载引号风格包(不支持 chinese)
\usepackage[style=english]{csquotes}
% 加载参考文献相关设置
\usepackage[style=gb7714-2015ms,backref=false]{styles/kaobiblioCJKsc} % 默认使用 GB 标准
\addbibresource{main.bib} % 参考文献文本
% 加载数学相关设置
\usepackage[framed]{styles/kaotheoremsCJKsc}
% 加载超链接相关设置
\usepackage{styles/kaorefsCJKsc}
% 加载平台中文字体及相关设置
\usepackage[testMathBox]{styles/kaoWinCJKsc}
% 加载测试包
\usepackage{blindtext}
\usepackage{zhlipsum} % 中文随机文本

%----------------------------------------------------------------------------------------
%	以下用来在首页中加入封面图的宏包(可直接复制粘贴)
%----------------------------------------------------------------------------------------
\usepackage{eso-pic}
\usepackage{graphicx} % 绘制图片所需宏包
\usepackage{adjustbox} % 图片缩放所需宏包
\usepackage{changepage} % 调整页面宽度所需宏包,用来调整\tableofcontents的minipage宽度


\DeclareMathAlphabet{\mathcal}{OMS}{cmsy}{m}{n}
\usetikzlibrary{patterns}
% Load packages for testing
\usepackage{blindtext}
% 从mathtools包中引入\coloneqq和\eqqcolon命令
\makeatletter
\providecommand{\coloneqq}{\mathrel{\mathop{:}=}}
\providecommand{\eqqcolon}{\mathrel{\mathop{=}:}}
\makeatother

\addbibresource{main.bib} % Bibliography file
% Load mathematical packages for theorems and related environments
% \usepackage[framed=true]{styles/kaotheorems}
% Load the package for hyperreferences
% \usepackage{styles/kaorefs}
\graphicspath{{examples/documentation/images/}{images/}} 
% 定义note环境
\usepackage{mdframed} % 用于创建带框的环境
% \usepackage{amsthm}   % 用于定理类环境



\let\boldsymbol\symbfit
\let\mathbf\symbf

\usepackage{adjustbox}
\usepackage[utf8]{inputenc}
% \usepackage{tikz}
\usepackage{float}
\usepackage{adjustbox}
\usepackage{eso-pic}
\usepackage{blindtext}
\usepackage{enumitem}
\usepackage{mathrsfs}
\usepackage{centernot}
% hyperref应放在大多数包之后
\usepackage{hyperref}
\usepackage{tikz-cd}
\usetikzlibrary{decorations.pathmorphing}
\hypersetup{
    unicode=true,
    pdfencoding=auto,
    bookmarksnumbered=true,
    bookmarksopen=true,
    CJKbookmarks=true
}
\setlength{\intextsep}{1pt plus 1pt minus 1pt}
% bookmark包必须在hyperref之后加载
\usepackage{cleveref}

\graphicspath{{../figures/}}
% 重定义proof环境为prf环境
% \let\proof\relax
% \let\endproof\relax
% \let\solution\relax
% \let\endsolution\relax
\renewenvironment{proof}{\begin{prf}}{\end{prf}}
\newenvironment{solution}{\begin{soln}}{\end{soln}}
\renewenvironment{exercise}{\begin{exr}}{\end{exr}}
\renewenvironment{definition}{\begin{defi}}{\end{defi}}
\renewenvironment{corollary}{\begin{cor}}{\end{cor}}
\renewenvironment{theorem}{\begin{thm}}{\end{thm}}
\renewenvironment{lemma}{\begin{lem}}{\end{lem}}
\renewenvironment{proposition}{\begin{prop}}{\end{prop}}
\renewenvironment{property}{\begin{ppt}}{\end{ppt}}
\renewenvironment{example}{\begin{expl}}{\end{expl}}
\renewenvironment{remark}{\begin{rmk}}{\end{rmk}}

\usepackage{amsthm}
\newtheorem{note}{Note}[chapter]
\begin{document}

\subject{使用此文档作为模板}

\title[示例及说明文档 {\normalfont\texttt{kaobook}} 类]{现代数学笔记\\Modern Mathematics Notes}
\subtitle{中文 Kaobook 模板制作}

\author[乐绎华]{乐绎华·中大数学系2023级本科生}

\date{\zhtoday}

\publishers{Sun Yat-sen University}

%----------------------------------------------------------------------------------------

\frontmatter % Denotes the start of the pre-document content, uses roman numerals


\definecolor{titlecolorcoverpage}{RGB}{64, 62, 91}

\dedication{
	世界的和谐体现在形式和数量上,自然哲学的心和灵魂以及一切诗歌都体现在数学美的概念上。\\
	\flushright -- D'Arcy Wentworth Thompson
}

%----------------------------------------------------------------------------------------
%	OUTPUT TITLE PAGE AND PREVIOUS
%----------------------------------------------------------------------------------------

% Note that \maketitle outputs the pages before here
% \maketitle
\begin{titlepage} % 创建一个新的页面
%用来将图片从左下角开始平铺整个封面
	\AddToShipoutPicture*{%
	\AtPageLowerLeft{%
		\adjustbox{width=1.1\paperwidth, height=1.5\paperheight, keepaspectratio}{% 强制图片至纸张尺寸,但可能裁切
			\includegraphics{images/Elias_Maur.jpg}
		}
	}
}
\begin{flushleft} % 左对齐环境
	\setlength{\leftskip}{1cm} % 左侧缩进
	\makeatletter % 允许访问带有@字符的内部命令
	\vspace*{4cm} % 标题距离页面顶端的空白
	{\color{titlecolorcoverpage}\Huge \textbf{\@title} \par} % 使用前文定义的title作为标题
	\vspace{1cm} % 标题和子标题的间距
	{\color{titlecolorcoverpage}\Large\@subtitle \par} % 使用前文定义的subtitle作为子标题
	\vfill % 作者信息前的垂直填充
	{\color{white}\large{\@author}\par} % 作者名
	{\color{white}\large \@publishers \par} % 出版者
	{\color{white}\large \zhtoday\par} % 日期,默认为当天
	\makeatother % 将@重置为非字母字符
	\vspace{0cm} % 标题和子标题的间距
	% 结束左对齐环境
\end{flushleft}
\ClearShipoutPicture % 清除背景图片
\end{titlepage}

%----------------------------------------------------------------------------------------
%	PREFACE
%----------------------------------------------------------------------------------------

% \input{chapters/preface.tex}
% \index{preface}

%----------------------------------------------------------------------------------------
%	TABLE OF CONTENTS & LIST OF FIGURES/TABLES
%----------------------------------------------------------------------------------------

\begingroup % 以下命令的局部范围

% 定义目录、插图列表和表格列表的样式
%\setstretch{1} % 取消注释以修改目录中的行距
%\hypersetup{linkcolor=blue} % 取消注释以设置目录中链接的颜色
\setlength{\textheight}{230\hscale} % 手动调整目录页面的高度

% 打开etoc包的兼容模式
%\etocstandarddisplaystyle % "目录显示"仿佛etoc未加载
%\etocstandarddisplaystyle 已弃用,使用\etocclasstocstyle
% \etocclasstocstyle % "目录显示"仿佛etoc未加载
% \etocstandardlines % "目录行"仿佛etoc未加载

\tableofcontents % 输出目录
\listoffigures % 输出插图列表
% 注释掉以下两行以在不同页面上显示插图列表和表格列表
\let\cleardoublepage\bigskip
\let\clearpage\bigskip
\listoftables % 输出表格列表

\endgroup


%----------------------------------------------------------------------------------------
%	MAIN BODY
%----------------------------------------------------------------------------------------
\mainmatter % 表示正文内容的开始,重置页码并使用阿拉伯数字
\chapter{测试}	
\section{测试}
\subsection{测试}
\subsubsection{测试}


\begin{figure}[H]
    \centering
    \includegraphics[width=0.5\textwidth]{example-image}
    % \caption{第一张图片}
\end{figure}
dsahkdajlajd
\begin{figure}[H]
    \centering
    \includegraphics[width=0.5\textwidth]{example-image}
    % \caption{第一张图片}
\end{figure}
\[
	l, \mathscr{A}, \boldsymbol{A}, \mathbf{A}
\]

\begin{solution}
	fahkfa
	
	fahkfja


	fafjla\sidenote{这是一段随机文字,用于测试文档排版效果。中文排版在学术文档中需要特别注意字间距和行间距的调整。这里我们可以看到边注功能的使用方法。}
\end{solution}

\begin{equation}\label{eq:test}
	\begin{aligned}
		a &= b \\
		c &= d
	\end{aligned}
\end{equation}

\marginnote{\blindtext}

\cref{eq:test}
\begin{note}
	this is a note.
\end{note}

\begin{remark}
	this is a remark.
\end{remark}	

\begin{proof}[fjjjjjj]
	 djlfjlafj
\end{proof}

\begin{solution}[fjjjjjj]
	 djlfjlafj
\end{solution}

\begin{exr}
	dad
\end{exr}	

\begin{exr}\label{exr:test}
	dad
\end{exr}	

\cref{exr:test}


\begin{defi}[jflasj]\label{defi:test}
djalda
\end{defi}

\cref{defi:test}

\begin{cor}
	dad
\end{cor}

\begin{lem}[fkaj]
	hfas
\end{lem}

\begin{prop}
	fafjla
\end{prop}

\begin{thm}
	fafjla
\end{thm}

\begin{ppt}
	fafjla
\end{ppt}

\begin{expl}
	fafjla
\end{expl}

\begin{rmk}
	fafjla
\end{rmk}

%----------------------------------------------------------------------------------------

\backmatter % Denotes the end of the main document content
% \setchapterstyle{plain} % Output plain chapters from this point onwards


\end{document}
