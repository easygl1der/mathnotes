\documentclass[
	leqno,
	a4paper, % Page size
	fontsize=12pt, % Base font size
	twoside=false, % Use different layouts for even and odd pages (in particular, if twoside=true, the margin column will be always on the outside)
	open=any, % If twoside=true, uncomment this to force new chapters to start on any page, not only on right (odd) pages
	chapterentrydots=true, % Uncomment to output dots from the chapter name to the page number in the table of contents
	numbers=noenddot, % Comment to output dots after chapter numbers; the most common values for this option are: enddot, noenddot and auto (see the KOMAScript documentation for an in-depth explanation)
    UTF-8, % Add this fix lots of style compile issue when using ctex and XeLaTex
	toc=nottotoc,
	BCOR=12mm,
	DIV=calc,
]{styles/kaobook}
%----------------------------------------------------------------------------------------
%	加载包以及文档配置,新项目可以直接复制粘贴
%----------------------------------------------------------------------------------------
%----------------------------------------------------------------------------------------
%	包以及文档配置,新项目可以直接复制粘贴
%----------------------------------------------------------------------------------------

% 加载排版包(不支持 chinese)
\usepackage{polyglossia}
\setmainlanguage{english}
\SetLanguageKeys{english}{indentfirst=true}
% 加载引号风格包(不支持 chinese)
\usepackage[style=english]{csquotes}
% 加载参考文献相关设置
\usepackage[style=gb7714-2015ms,backref=false]{styles/kaobiblioCJKsc} % 默认使用 GB 标准
\addbibresource{main.bib} % 参考文献文本
% 加载数学相关设置
\usepackage[framed]{styles/kaotheoremsCJKsc}
% 加载超链接相关设置
\usepackage{styles/kaorefsCJKsc}
% 加载平台中文字体及相关设置
\usepackage[testMathBox]{styles/kaoWinCJKsc}
% 加载测试包
\usepackage{blindtext}
\usepackage{zhlipsum} % 中文随机文本

%----------------------------------------------------------------------------------------
%	以下用来在首页中加入封面图的宏包(可直接复制粘贴)
%----------------------------------------------------------------------------------------
\usepackage{eso-pic}
\usepackage{graphicx} % 绘制图片所需宏包
\usepackage{adjustbox} % 图片缩放所需宏包
\usepackage{changepage} % 调整页面宽度所需宏包,用来调整\tableofcontents的minipage宽度

% % Choose the language
% \ifxetexorluatex
% 	\usepackage{polyglossia}
% 	\setmainlanguage{english}
% \else
% 	\usepackage[english]{babel} % Load characters and hyphenation
% \fi
% \usepackage[english=british]{csquotes}	% English quotes
\DeclareMathAlphabet{\mathcal}{OMS}{cmsy}{m}{n}
% \usepackage{ctex}
% \newcommand{\coloneqq}{\mathrel{\vcenter{\baselineskip0.5ex \lineskiplimit0pt \hbox{\scriptsize.}\hbox{\scriptsize.}}}=}
% \usepackage{tikz}
\usetikzlibrary{patterns}
% Load packages for testing
\usepackage{blindtext}
% 从mathtools包中引入\coloneqq和\eqqcolon命令
\makeatletter
\providecommand{\coloneqq}{\mathrel{\mathop{:}=}}
\providecommand{\eqqcolon}{\mathrel{\mathop{=}:}}
\makeatother
% \usepackage{amssymb}
% \usepackage{ulem}
% \usepackage{amsmath,amsfonts,amssymb}
%\usepackage{showframe} % Uncomment to show boxes around the text area, margin, header and footer
%\usepackage{showlabels} % Uncomment to output the content of \label commands to the document where they are used
% Load the bibliography package
% \usepackage{styles/kaobiblio}
\addbibresource{main.bib} % Bibliography file
% Load mathematical packages for theorems and related environments
% \usepackage[framed=true]{styles/kaotheorems}
% Load the package for hyperreferences
% \usepackage{styles/kaorefs}
% \graphicspath{{examples/documentation/images/}{images/}} 
% 定义note环境
\usepackage{mdframed} % 用于创建带框的环境
% \usepackage{amsthm}   % 用于定理类环境

% \setcounter{tocdepth}{4}
% \setcounter{secnumdepth}{4}
% 替换设置粗斜体数学符号
% 重新定义\boldsymbol命令,使其使用\symbfit命令
% 这样可以改变数学公式中粗体符号的显示方式
% \renewcommand{\boldsymbol}{\symbfit}
% 基本的note环境定义
\newmdtheoremenv{note}{注记}[chapter]
\mdfdefinestyle{solutionstyle}{
    linecolor=gray,
    backgroundcolor=gray!10,
    linewidth=1pt,
    innertopmargin=10pt,
    innerbottommargin=10pt,
    leftmargin=0pt,
    rightmargin=0pt,
}

\newtheorem*{solution}{解答}
\mdfdefinestyle{solutionstyle}{
    linecolor=gray,
    backgroundcolor=gray!10,
    linewidth=1pt,
    innertopmargin=10pt,
    innerbottommargin=10pt,
    leftmargin=0pt,
    rightmargin=0pt,
    font=\small % 设置字体大小为small
}
\surroundwithmdframed[style=solutionstyle]{solution} % 使用mdframed包将solution环境包围起来,并应用solutionstyle样式
% Paths in which to look for images
\makeindex[columns=3, title=Alphabetical Index, intoc] % 取消注释以生成按字母顺序排列的索引
\makeindex[columns=3, title=按字母排序的索引, intoc] % 创建一个按字母排序的索引,并将其添加到目录中
\makeglossaries % 生成术语表所需的文件
% 定义术语“computer”
\newglossaryentry{computer}{
	name=computer,
	description={is a programmable machine that receives input, stores and manipulates data, and provides output in a useful format}
}

% 定义缩略语(在文本中使用,例如 \acrfull{fpsLabel} 或 \acrshort{fpsLabel})
% 定义缩略语“FPS”,全称为“Frame per Second”,复数形式为“Frames per Second”
\newacronym[longplural={Frames per Second}]{fpsLabel}{FPS}{Frame per Second}
% 定义缩略语“TOC”,全称为“Table of Contents”,复数形式为“Tables of Contents”
\newacronym[longplural={Tables of Contents}]{tocLabel}{TOC}{Table of Contents}
 % 包含术语表定义
\makenomenclature % 生成术语表所需的文件
% Reset sidenote counter at chapters
%\counterwithin*{sidenote}{chapter} % 取消注释以在每章重置旁注计数器
\renewcommand\proofname{证明}

\usepackage{adjustbox}
% % 使用utf8编码,允许文档中包含UTF-8编码的字符
% % 在使用XeLaTeX或LuaLaTeX时通常不需要此包,因为它们原生支持UTF-8
% \usepackage[utf8]{inputenc}
% \usepackage{tikz}
\usepackage{float}
\usepackage{adjustbox}
\usepackage{eso-pic}
\usepackage{enumitem}
\usepackage{mathrsfs}
\usepackage{centernot}
\newcommand{\degree}{^\circ}
% hyperref应放在大多数包之后
\usepackage{hyperref}

\hypersetup{
    unicode=true,
    pdfencoding=auto,
    bookmarksnumbered=true,
    bookmarksopen=true,
    CJKbookmarks=true
}

% bookmark包必须在hyperref之后加载
% \usepackage{cleveref}

\graphicspath{{../figures/}}

\begin{document}

\subject{使用此文档作为模板}

\title[示例及说明文档 {\normalfont\texttt{kaobook}} 类]{现代数学笔记\\Modern Mathematics Notes}
\subtitle{中文 Kaobook 模板制作}

\author[乐绎华]{乐绎华·中大数学系2023级本科生}

\date{\zhtoday}

\publishers{Sun Yat-sen University}

%----------------------------------------------------------------------------------------

\frontmatter % Denotes the start of the pre-document content, uses roman numerals


\definecolor{titlecolorcoverpage}{RGB}{64, 62, 91}

\dedication{
	世界的和谐体现在形式和数量上,自然哲学的心和灵魂以及一切诗歌都体现在数学美的概念上。\\
	\flushright -- D'Arcy Wentworth Thompson
}

%----------------------------------------------------------------------------------------
%	OUTPUT TITLE PAGE AND PREVIOUS
%----------------------------------------------------------------------------------------

% Note that \maketitle outputs the pages before here
% \maketitle
\begin{titlepage} % 创建一个新的页面
%用来将图片从左下角开始平铺整个封面
	\AddToShipoutPicture*{%
	\AtPageLowerLeft{%
		\adjustbox{width=1.1\paperwidth, height=1.5\paperheight, keepaspectratio}{% 强制图片至纸张尺寸,但可能裁切
			\includegraphics{images/Elias_Maur.jpg}
		}
	}
}
\begin{flushleft} % 左对齐环境
	\setlength{\leftskip}{1cm} % 左侧缩进
	\makeatletter % 允许访问带有@字符的内部命令
	\vspace*{4cm} % 标题距离页面顶端的空白
	{\color{titlecolorcoverpage}\Huge \textbf{\@title} \par} % 使用前文定义的title作为标题
	\vspace{1cm} % 标题和子标题的间距
	{\color{titlecolorcoverpage}\Large\@subtitle \par} % 使用前文定义的subtitle作为子标题
	\vfill % 作者信息前的垂直填充
	{\color{white}\large{\@author}\par} % 作者名
	{\color{white}\large \@publishers \par} % 出版者
	{\color{white}\large \zhtoday\par} % 日期,默认为当天
	\makeatother % 将@重置为非字母字符
	\vspace{0cm} % 标题和子标题的间距
	% 结束左对齐环境
\end{flushleft}
\ClearShipoutPicture % 清除背景图片
\end{titlepage}

%----------------------------------------------------------------------------------------
%	PREFACE
%----------------------------------------------------------------------------------------

% \input{chapters/preface.tex}
% \index{preface}

%----------------------------------------------------------------------------------------
%	TABLE OF CONTENTS & LIST OF FIGURES/TABLES
%----------------------------------------------------------------------------------------

\begingroup % 以下命令的局部范围

% 定义目录、插图列表和表格列表的样式
%\setstretch{1} % 取消注释以修改目录中的行距
%\hypersetup{linkcolor=blue} % 取消注释以设置目录中链接的颜色
\setlength{\textheight}{230\hscale} % 手动调整目录页面的高度

% 打开etoc包的兼容模式
%\etocstandarddisplaystyle % "目录显示"仿佛etoc未加载
%\etocstandarddisplaystyle 已弃用,使用\etocclasstocstyle
% \etocclasstocstyle % "目录显示"仿佛etoc未加载
% \etocstandardlines % "目录行"仿佛etoc未加载

\tableofcontents % 输出目录
\listoffigures % 输出插图列表
% 注释掉以下两行以在不同页面上显示插图列表和表格列表
\let\cleardoublepage\bigskip
\let\clearpage\bigskip
\listoftables % 输出表格列表

\endgroup


%----------------------------------------------------------------------------------------
%	MAIN BODY
%----------------------------------------------------------------------------------------
\mainmatter % 表示正文内容的开始,重置页码并使用阿拉伯数字


\setchapterstyle{kao} % 选择默认的章节标题样式
\pagelayout{wide}
\addpartwithbg{课内笔记}{elias-maurer-gLo2UCpH5i8-unsplash}
\pagelayout{margin}

% \input{chapters/ode.tex}
\input{chapters/probability.tex}
% \input{chapters/mathematical_analysis_3.tex}
\input{chapters/numerical_analysis.tex}




%----------------------------------------------------------------------------------------

\backmatter % Denotes the end of the main document content
% \setchapterstyle{plain} % Output plain chapters from this point onwards

% %----------------------------------------------------------------------------------------
% %	BIBLIOGRAPHY
% %----------------------------------------------------------------------------------------

% % The bibliography needs to be compiled with biber using your LaTeX editor, or on the command line with 'biber main' from the template directory

% \defbibnote{bibnote}{Here are the references in citation order.\par\bigskip} % Prepend this text to the bibliography
% %\printbibliography[heading=bibintoc, title=Bibliography, prenote=bibnote] % Add the bibliography heading to the ToC, set the title of the bibliography and output the bibliography note
% \printbibliography[heading=bibintoc, title=参考文献, prenote=bibnote] % Add the bibliography heading to the ToC and set the title of the bibliography

% %----------------------------------------------------------------------------------------
% %	NOMENCLATURE
% %----------------------------------------------------------------------------------------

% % The nomenclature needs to be compiled on the command line with 'makeindex main.nlo -s nomencl.ist -o main.nls' from the template directory

% \nomenclature{$c$}{Speed of light in a vacuum inertial frame}
% \nomenclature{$h$}{Planck constant}

% \renewcommand{\nomname}{Notation} % Rename the default 'Nomenclature'
% \renewcommand{\nompreamble}{The next list describes several symbols that will be later used within the body of the document.} % Prepend this text to the nomenclature

% \printnomenclature % Output the nomenclature

% %----------------------------------------------------------------------------------------
% %	GREEK ALPHABET
% % 	Originally from https://gitlab.com/jim.hefferon/linear-algebra
% %----------------------------------------------------------------------------------------

% % \vspace{1cm}

% % {\usekomafont{chapter}Greek Letters with Pronunciations} \\[2ex]
% % \begin{center}
% % 	\newcommand{\pronounced}[1]{\hspace*{.2em}\small\textit{#1}}
% % 	\begin{tabular}{l l @{\hspace*{3em}} l l}
% % 		\toprule
% % 		Character & Name & Character & Name \\ 
% % 		\midrule
% % 		$\alpha$ & alpha \pronounced{AL-fuh} & $\nu$ & nu \pronounced{NEW} \\
% % 		$\beta$ & beta \pronounced{BAY-tuh} & $\xi$, $\Xi$ & xi \pronounced{KSIGH} \\ 
% % 		$\gamma$, $\Gamma$ & gamma \pronounced{GAM-muh} & o & omicron \pronounced{OM-uh-CRON} \\
% % 		$\delta$, $\Delta$ & delta \pronounced{DEL-tuh} & $\pi$, $\Pi$ & pi \pronounced{PIE} \\
% % 		$\epsilon$ & epsilon \pronounced{EP-suh-lon} & $\rho$ & rho \pronounced{ROW} \\
% % 		$\zeta$ & zeta \pronounced{ZAY-tuh} & $\sigma$, $\Sigma$ & sigma \pronounced{SIG-muh} \\
% % 		$\eta$ & eta \pronounced{AY-tuh} & $\tau$ & tau \pronounced{TOW (as in cow)} \\
% % 		$\theta$, $\Theta$ & theta \pronounced{THAY-tuh} & $\upsilon$, $\Upsilon$ & upsilon \pronounced{OOP-suh-LON} \\
% % 		$\iota$ & iota \pronounced{eye-OH-tuh} & $\phi$, $\Phi$ & phi \pronounced{FEE, or FI (as in hi)} \\
% % 		$\kappa$ & kappa \pronounced{KAP-uh} & $\chi$ & chi \pronounced{KI (as in hi)} \\
% % 		$\lambda$, $\Lambda$ & lambda \pronounced{LAM-duh} & $\psi$, $\Psi$ & psi \pronounced{SIGH, or PSIGH} \\
% % 		$\mu$ & mu \pronounced{MEW} & $\omega$, $\Omega$ & omega \pronounced{oh-MAY-guh} \\
% % 		\bottomrule
% % 	\end{tabular} \\[1.5ex]
% % 	Capitals shown are the ones that differ from Roman capitals.
% % \end{center}

% %----------------------------------------------------------------------------------------
% %	GLOSSARY
% %----------------------------------------------------------------------------------------

% % The glossary needs to be compiled on the command line with 'makeglossaries main' from the template directory

% \setglossarystyle{listgroup} % Set the style of the glossary (see https://en.wikibooks.org/wiki/LaTeX/Glossary for a reference)
% \printglossary[title=Special Terms, toctitle=List of Terms] % Output the glossary, 'title' is the chapter heading for the glossary, toctitle is the table of contents heading

% %----------------------------------------------------------------------------------------
% %	INDEX
% %----------------------------------------------------------------------------------------

% % The index needs to be compiled on the command line with 'makeindex main' from the template directory

% \printindex % Output the index

%----------------------------------------------------------------------------------------
%	BACK COVER
%----------------------------------------------------------------------------------------

% If you have a PDF/image file that you want to use as a back cover, uncomment the following lines

%\clearpage
%\thispagestyle{empty}
%\null%
%\clearpage
%\includepdf{cover-back.pdf}

%----------------------------------------------------------------------------------------

\end{document}
