\section{Group algebra}

The group algebras are the source of all you need to know about representatio theory. In particular, the ultimate goal of representation theory -- that of understanding all the representations of finite groups -- would be achieved if group algebras could be fully analysed.

\begin{definition}[Definition]
Let $G$ be a finite group and $F$ be $\mathbb{R}$ or $\mathbb{C}$. The vector space $F G$, with the natural multiplication $v g(v \in F G, g \in G)$, is called the \textbf{regular $F G$ -module}.
The representation $g \rightarrow[g]_{\mathscr{B}}$ obtained by taking $\mathscr{B}$ to be the natural basis of $F G$ is called the \textbf{regular representation of $G$ over $F$}.
\end{definition}
\begin{definition}[group algebra]
The vector space $FG$, with multiplication defined by
\[
\left( \sum_{g\in G}\lambda_{g}g \right)\left( \sum_{h\in G}\mu_{h}h \right)=\sum_{g,h\in G}\lambda_{g}\mu_{h}(gh)\qquad (\lambda_{g},\mu_{h}\in F)
\]is called the \textbf{group algebra} of $G$ over $F$.
\end{definition}
Recall that an $FG$ -module is a vector space over $F$, together with a multiplication $vg$ for $v\in V$ and $g\in G$. Now we can expend the definition of the multiplication so that we have an element $vr$ of $V$ for all elements $r$ in the group algebra $FG$.

Suppose that $V$ is an $FG$ -module, and that $v\in V$ and $r\in FG$; say
\[
r=\sum_{g\in G}\mu_{g}g\qquad (\mu_{g}\in F)
\]
Define $vr$ by
\[
vr=\sum_{g\in G}\mu_{g}(vg)
\]
\subsection{Exercises}

Omitted...
