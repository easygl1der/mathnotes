\section{\texorpdfstring{$FG$}{FG} -modules}

\begin{note}
See Representations and Characters of Groups (Gordon James, Martin Liebeck)
\end{note}
We will show that there is a connection between $FG$ -modules and representations of $G$ over $F$.

Let $G$ be a group and let $F$ be $\mathbb{R}$ or $\mathbb{C}$.

Suppose that $\rho:G\to \mathrm{GL}(n,F)$ is a representation of $G$. Write $V=F^{n}$, as the vector space. For all $v\in V$ and $g\in G$, the matrix product\footnote{$g\rho=\rho_{g}=\rho (g)\in \mathrm{GL}(n,F)$}
\[
v\cdot\rho _{g}
\]
is a row vector in $V$.

Now we list some basic properties
\[
v(\rho_{gh})=v(\rho_{g})(\rho_{h})
\]
\[
v(\rho_{1})=v
\]
\[
(\lambda v)(g\rho_{g})=\lambda(v(\rho_{g})),\qquad \forall \lambda\in F
\]
\[
(u+v)(\rho_{g})=u\rho_{g}+v\rho_{g}
\]
\begin{figure}[H]
\centering
\includegraphics[width=\textwidth]{FG-modules-2025050119.png}
% \caption{}
\label{}
\end{figure}
Motivatied by the above observations on the product $v(\rho_{g})$, we now define an $FG$ -modules.

\begin{definition}[$FG$-module]
Let $V$ be a vector space over $F$ and let $G$ be a group. Then $V$ is an \textbf{$FG$-module} if a multiplication $v g(v \in V, g \in G)$ is defined, satisfying the following conditions for all $u, v \in V, \lambda \in F$ and $g, h \in G$ :
	\begin{enumerate}
		\item $v g \in V$;
		\item $v(g h)=(v g) h$;
		\item $v 1=v$;
		\item $(\lambda v) g=\lambda(v g)$;
		\item $(u+v) g=u g+v g$\label{4f2b8a}
	\end{enumerate}
\end{definition}
\begin{definition}[Matrix representation]
Let $V$ be an $FG$-module, and let $\mathscr{B}$ be a basis of $V$. For each $g \in G$, let
\[
[g]_{\mathscr{B}}
\]denote the \textbf{matrix of the endomorphism} $v \rightarrow v g$ of $V$, relative to the basis $\mathscr{B}$.
\end{definition}
If $\rho :G\to \mathrm{GL}(n,F)$ is a representation of $G$ over $F$, and $V=F^{n}$, then $V$ becomes an $FG$ -module if we define the multiplication $vg$ by
\[
vg\coloneqq v(\rho_{g})\qquad (v\in V,g\in G)
\]
Moreover, there is a basis $\mathscr{B}$ of $V$ such that
\[
\rho_{g}=[g]_{\mathscr{B}}\qquad \forall g\in G
\]
On the other hand, assume that $V$ is an $FG$ -module and let $\mathscr{B}$ $e$ a basis of $V$. Then the function
\[
g\to[g]_{\mathscr{B}}\qquad (g\in G)
\]
is a representation of $G$ over $F$.

\begin{figure}[H]
\centering
\includegraphics[width=\textwidth]{FG-modules-2025050120.png}
% \caption{}
\label{}
\end{figure}
\begin{figure}[H]
\centering
\includegraphics[width=\textwidth]{1-FG-modules-2025050120.png}
% \caption{}
\label{}
\end{figure}

\begin{definition}[faithful]
An $FG$-module $V$ is \textbf{faithful} if the identity element of $G$ is the only element $g$ for which
\[
v g=v \quad \text { for all } v \in V
\]
\end{definition}
For instance, the $F D_8$ -module which appears in Example $4.5(1)$ is faithful.

Our next aim is to construct faithful $FG$ -modules by the basis for all subgroups of symmetric groups.

\begin{definition}[Permutation module]
Let $G$ be a subgroup of $S_n$. The $F G$-module $V$ with basis $v_1, \ldots, v_n$ such that
\[
v_i g=v_{i g} \quad \text { for all } i, \text { and all } g \in G,
\]is called the \textbf{permutation module} for $G$ over $F$. We call $v_1, \ldots, v_n$ the natural basis of $V$.
\end{definition}
\begin{figure}[H]
\centering
\includegraphics[width=\textwidth]{2-FG-modules-2025050120.png}
% \caption{}
\label{}
\end{figure}

\begin{figure}[H]
\centering
\includegraphics[width=\textwidth]{3-FG-modules-2025050120.png}
% \caption{}
\label{}
\end{figure}

\subsection{Exercise}

\begin{exercise}
\begin{figure}[H]
\centering
\includegraphics[width=\textwidth]{4-FG-modules-2025050120.png}
% \caption{}
\label{}
\end{figure}
\end{exercise}
When $g=e$, then $[g]_{\mathscr{B}_{1}}=I_3$. When $g=(1\ 2)$, then
\[
[g]_{\mathscr{B}_{1}}=\begin{pmatrix}
0 & 1 & 0 \\
1 & 0 & 0 \\
0 & 0 & 1
\end{pmatrix}
\]
\[
[(1\ 3)]_{\mathscr{B}_{1}}=\begin{pmatrix}
 &  & 1 \\
 & 1  &  \\
1 &  & 
\end{pmatrix}
\]
\[
[(2\ 3)]_{\mathscr{B}_{1}}=\begin{pmatrix}
1 &  &  \\
 &  & 1 \\
 & 1 & 
\end{pmatrix}
\]
\[
[(1\ 2\ 3)]_{\mathscr{B}_{1}}=\begin{pmatrix}
 & 1 &  \\
 &  & 1 \\
1 &  & 
\end{pmatrix}
\]
\[
[(1\ 3\ 2)]_{\mathscr{B}_{1}}=\begin{pmatrix}
 &  & 1 \\
1 &  &  \\
 & 1 &  
\end{pmatrix}
\]
Now let $u_1=v_1+v_2+v_3$, $u_2=v_1-v_2$, $u_3=v_1-v_3$. Then
\[
u_1e=u_1\qquad u_2e=u_2\qquad u_3e=u_3
\]
\[
u_1(1\ 2)=u_1\qquad u_2(1\ 2)=-u_2\qquad u_3(1\ 2)=v_2-v_3=-u_2+u_3
\]
Then
\[
[e]_{\mathscr{B}_{2}}=\begin{pmatrix}
1 &  &  \\
 & 1 &  \\
 &  & 1
\end{pmatrix}
\]
\[
[(1\ 2)]_{\mathscr{B}_{2}}=\begin{pmatrix}
1 &  &  \\
 & -1 &  \\
 & -1 & 1 
\end{pmatrix}
\]
Omitted....

\begin{exercise}
\begin{figure}[H]
\centering
\includegraphics[width=\textwidth]{5-FG-modules-2025050120.png}
% \caption{}
\label{}
\end{figure}
\end{exercise}
Let $\mathscr{B}=(v_1,v_2,v_3,v_4)$. Then
\[
[a]_{\mathscr{B}}=\begin{pmatrix}
 & 1 &  &  \\
-1 &  &  &  \\
 &  &  & -1  \\
 &  & 1 & 
\end{pmatrix},\qquad [b]_{\mathscr{B}}=\begin{pmatrix}
 &  & 1 &  \\
 &  &  & 1  \\
-1 &  &  &  \\
 & -1 &  & 
\end{pmatrix}
\]
Check that $V$ is a $\mathbb{R}Q_8$ -module. NTS: $g\mapsto[g]_{\mathscr{B}}$ is a representation of $Q_8$, i.e.
\[
[a]_{\mathscr{B}}^{4}=I_4,\qquad [a]_{\mathscr{B}}^2=[b]_{\mathscr{B}}^2,\qquad [b]_{\mathscr{B}}^{-1}[a]_{\mathscr{B}}[b]_{\mathscr{B}}=[a]_{\mathscr{B}}^{-1}
\]
which is routine.


\section{\texorpdfstring{$FG$}{FG} -submodules and reducibility}

\begin{definition}[$FG$ -submodule]
Let $V$ be an $F G$ -module. A subset $W$ of $V$ is said to be an $F G$ -submodule of $V$ if $W$ is a subspace and $w g \in W$ for all $w \in W$ and all $g \in G$.
\end{definition}
\begin{figure}[H]
\centering
\includegraphics[width=\textwidth]{FG-modules-2025050121.png}
% \caption{}
\label{}
\end{figure}
\begin{figure}[H]
\centering
\includegraphics[width=\textwidth]{1-FG-modules-2025050121.png}
% \caption{}
\label{}
\end{figure}

\begin{definition}[Irreducible FG-module]
An \textbf{irreducible FG-module} $V$ is said to be irreducible if it is non-zero and it has no $F G$-submodules apart from $\{0\}$ and $V$.
\end{definition}
\begin{figure}[H]
\centering
\includegraphics[width=\textwidth]{FG-modules-2025050122.png}
% \caption{}
\label{}
\end{figure}

\begin{figure}[H]
\centering
\includegraphics[width=\textwidth]{1-FG-modules-2025050122.png}
% \caption{}
\label{}
\end{figure}
\begin{figure}[H]
\centering
\includegraphics[width=\textwidth]{2-FG-modules-2025050122.png}
% \caption{}
\label{}
\end{figure}
\begin{figure}[H]
\centering
\includegraphics[width=\textwidth]{3-FG-modules-2025050122.png}
% \caption{}
\label{}
\end{figure}

\begin{figure}[H]
\centering
\includegraphics[width=\textwidth]{4-FG-modules-2025050122.png}
% \caption{}
\label{}
\end{figure}

\subsection{Exercises}

Omitted....
