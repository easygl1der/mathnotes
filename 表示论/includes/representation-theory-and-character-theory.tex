\section{Representation Theory and Character Theory}

\subsection{Linear Actions and modules over group rings}

$F$ is a ring, $G$ is a group then the \textbf{group ring} $FG$ is defined by
\[
\sum_{g\in G}\alpha_{g}g,\qquad \alpha_{g}\in F.
\]
with the following properties:
\[
\sum_{g\in G}\alpha_{g}g+\sum_{g\in G}\beta_{g}g=\sum_{g\in G}(\alpha_{g}+\beta_{g})g
\]
\[
\left( \sum_{g\in G}\alpha_{g}g \right)\left( \sum_{g\in G}\beta_{g}g \right)=\sum_{g\in G}\left( \sum_{\substack{h,k\in G\\hk=g}}\alpha_{h}\beta_{k} \right)g
\]
$FG$ is a commutative ring iff $G$ is an abelian group.

$FG$ is a vector space over $F$ with the elements of $G$ as a basis. ($\dim FG=\lvert G \rvert$)

\begin{figure}[H]
\centering
\includegraphics[width=\textwidth]{Representation Theory and Character Theory-2025040114.png}
% \caption{}
\label{}
\end{figure}

Formally, a representation $\rho: G \rightarrow \mathrm{GL}(V)$ is \textbf{faithful} if the only element of $G$ that is mapped to the identity transformation is the identity element of $G$, i.e.,
\[
\operatorname{ker}(\rho)=\{e\}
\]
where $e$ is the identity element of $G$.

\subsubsection{Definitions of irreducible (simple), indecomposable, completely reducible, constituent}

\begin{figure}[H]
\centering
\includegraphics[width=\textwidth]{Representation Theory and Character Theory-2025040115.png}
% \caption{}
\label{}
\end{figure}

\begin{figure}[H]
\centering
\includegraphics[width=\textwidth]{1-Representation Theory and Character Theory-2025040115.png}
% \caption{}
\label{}
\end{figure}

An $FG$ -module is finitely generated iff it is finite dimensional.

\subsection{Wedderburn's theorem and some consequences}

\begin{figure}[H]
\centering
\includegraphics[width=\textwidth]{2-Representation Theory and Character Theory-2025040115.png}
% \caption{}
\label{}
\end{figure}
\begin{figure}[H]
\centering
\includegraphics[width=\textwidth]{3-Representation Theory and Character Theory-2025040115.png}
% \caption{}
\label{}
\end{figure}

\subsubsection{Definitions of semisimple, idempotent, orthogonal, primitive, primitive central idempotent}

\begin{figure}[H]
\centering
\includegraphics[width=\textwidth]{4-Representation Theory and Character Theory-2025040115.png}
% \caption{}
\label{}
\end{figure}
\begin{figure}[H]
\centering
\includegraphics[width=\textwidth]{5-Representation Theory and Character Theory-2025040115.png}
% \caption{}
\label{}
\end{figure}

\subsubsection{Propositions}

\begin{figure}[H]
\centering
\includegraphics[width=\textwidth]{6-Representation Theory and Character Theory-2025040115.png}
% \caption{}
\label{}
\end{figure}
