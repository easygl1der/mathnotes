\chapter{根系统 Root Systems - 第三章要点总结}

\section{第9节:公理化理论 Axiomatics}

\subsection{9.1 欧几里得空间中的反射 Reflections in Euclidean Space}

\subsubsection{定义 9.1.1 (反射 Reflection)}

设 $E$ 是有限维实向量空间,带有正定对称双线性形式 $(\alpha, \beta)$。\textbf{反射 (Reflection)} 是可逆线性变换,满足:

\begin{itemize}
	\item 逐点固定某个超平面(余维数1的子空间)
	\item 将该超平面的正交向量映为其负向量
\end{itemize}

\subsubsection{定义 9.1.2 (反射 \texorpdfstring{$\sigma_\alpha$}{sigma_alpha})}

对任意非零向量 $\alpha \in E$,反射 $\sigma_\alpha$ 定义为:
\[
\sigma_\alpha(\beta) = \beta - \frac{2(\beta, \alpha)}{(\alpha, \alpha)} \alpha
\]

反射超平面为 $P_\alpha = \{\beta \in E \mid (\beta, \alpha) = 0\}$

\subsubsection{记号}

定义 $\langle\beta, \alpha\rangle = \frac{2(\beta, \alpha)}{(\alpha, \alpha)}$

\textbf{重要}:$\langle\beta, \alpha\rangle$ 仅对第一个变量线性

\subsubsection{引理 9.1.3 (反射的唯一性)}

设 $\Phi$ 是有限集合且张成 $E$,所有反射 $\sigma_\alpha$ ($\alpha \in \Phi$) 保持 $\Phi$ 不变。若 $\sigma \in GL(E)$ 保持 $\Phi$ 不变,逐点固定超平面 $P$,并将某个非零 $\alpha \in \Phi$ 映为其负向量,则 $\sigma = \sigma_\alpha$。

\subsection{9.2 根系统 Root Systems}

\subsubsection{定义 9.2.1 (根系统 Root System)}

欧几里得空间 $E$ 的子集 $\Phi$ 称为 \textbf{根系统 (Root System)},若满足以下公理:

\textbf{(R1)} $\Phi$ 有限,张成 $E$,且不含 0
\textbf{(R2)} 若 $\alpha \in \Phi$,则 $\Phi$ 中 $\alpha$ 的倍数只有 $\pm\alpha$\textbf{(R3)} 若 $\alpha \in \Phi$,则反射 $\sigma_\alpha$ 保持 $\Phi$ 不变
\textbf{(R4)} 若 $\alpha, \beta \in \Phi$,则 $\langle\beta, \alpha\rangle \in \mathbb{Z}$

\textbf{注记}:(R2) 和 (R3) 均蕴含 $\Phi = -\Phi$

\subsubsection{定义 9.2.2 (魏尔群 Weyl Group)}

$\Phi$ 的 \textbf{魏尔群 (Weyl Group)} $\mathcal{W}$ 是由所有反射 $\sigma_\alpha$ ($\alpha \in \Phi$) 生成的 $GL(E)$ 的子群。

由 (R3),$\mathcal{W}$ 置换集合 $\Phi$;由 (R1),$\mathcal{W}$ 有限。

\subsubsection{引理 9.2.3 (自同构的性质)}

设 $\Phi$ 是根系统,$\mathcal{W}$ 为其魏尔群。若 $\sigma \in GL(E)$ 保持 $\Phi$ 不变,则:

\begin{enumerate}
	\item $\sigma \sigma_\alpha \sigma^{-1} = \sigma_{\sigma(\alpha)}$ 对所有 $\alpha \in \Phi$
	\item $\langle\beta, \alpha\rangle = \langle\sigma(\beta), \sigma(\alpha)\rangle$ 对所有 $\alpha, \beta \in \Phi$
\end{enumerate}

\subsubsection{定义 9.2.4 (对偶根系统 Dual Root System)}

$\Phi$ 的 \textbf{对偶 (Dual)} 或 \textbf{逆 (Inverse)} 为 $\Phi^\vee = \{\alpha^\vee \mid \alpha \in \Phi\}$,其中 $\alpha^\vee = \frac{2\alpha}{(\alpha, \alpha)}$。

\subsection{9.3 例子 Examples}

\subsubsection{秩1:型 \texorpdfstring{$A_1$}{A_1}}

$\Phi = \{\alpha, -\alpha\}$ 是唯一的秩1根系统

\subsubsection{秩2:四种类型}

\begin{enumerate}
	\item \textbf{型 $A_1 \times A_1$}:两个正交的 $A_1$ 副本
	\item \textbf{型 $A_2$}:$\Phi = \{\pm(\varepsilon_1 - \varepsilon_2), \pm(\varepsilon_2 - \varepsilon_3), \pm(\varepsilon_1 - \varepsilon_3)\}$
	\item \textbf{型 $B_2$}:包含两种不同长度的根
	\item \textbf{型 $G_2$}:包含两种长度的根,比例为 $\sqrt{3}$
\end{enumerate}

\subsection{9.4 根对 Pairs of Roots}

\subsubsection{引理 9.4.1 (角度限制)}

对不成比例的根 $\alpha, \beta$,$\langle\alpha, \beta\rangle\langle\beta, \alpha\rangle$ 的可能值为 0, 1, 2, 3,对应特定的角度和长度比:

\begin{table}[h]
	\centering
	\begin{tabular}{|c|c|c|c|}
		\hline
		$\langle\alpha, \beta\rangle$ & $\langle\beta, \alpha\rangle$ & $\|\beta\|^2/\|\alpha\|^2$ & $\theta$ \\
		\hline
		0 & 0 & 不定 & $\pi/2$ \\
		\hline
		1 & 1 & 1 & $\pi/3$ \\
		\hline
		-1 & -1 & 1 & $2\pi/3$ \\
		\hline
		1 & 2 & 2 & $\pi/4$ \\
		\hline
		-1 & -2 & 2 & $3\pi/4$ \\
		\hline
		1 & 3 & 3 & $\pi/6$ \\
		\hline
		-1 & -3 & 3 & $5\pi/6$ \\
		\hline
	\end{tabular}
\end{table}
\subsubsection{引理 9.4.2 (根的加法 Root Addition)}

设 $\alpha, \beta$ 是不成比例的根:

\begin{itemize}
	\item 若 $(\alpha, \beta) > 0$(即角度严格锐角),则 $\alpha - \beta$ 是根
	\item 若 $(\alpha, \beta) < 0$,则 $\alpha + \beta$ 是根
\end{itemize}

\subsubsection{根串 Root Strings}

对不成比例的根 $\alpha, \beta$,通过 $\beta$ 的 \textbf{$\alpha$-串 ($\alpha$-string)} 由所有形如 $\beta + i\alpha$ ($i \in \mathbb{Z}$) 的根组成。设 $r, q \in \mathbb{Z}^+$ 是最大整数使得 $\beta - r\alpha \in \Phi$,$\beta + q\alpha \in \Phi$,则:

\begin{enumerate}
	\item $\alpha$-串是不间断的:$\beta - r\alpha, \beta - (r-1)\alpha, \ldots, \beta + q\alpha$
	\item $r - q = \langle\beta, \alpha\rangle$
	\item 根串长度至多为4
\end{enumerate}


\section{第10节:简单根与魏尔群 Simple Roots and Weyl Group}

\subsection{10.1 基与魏尔腔 Bases and Weyl Chambers}

\subsubsection{定义 10.1.1 (基 Base / 简单根 Simple Roots)}

$\Phi$ 的子集 $\Delta$ 称为 \textbf{基 (Base)},若满足:

\textbf{(B1)} $\Delta$ 是 $E$ 的基\textbf{(B2)} 每个根 $\beta$ 可写成 $\beta = \sum k_\alpha \alpha$ ($\alpha \in \Delta$),整系数 $k_\alpha$ 全非负或全非正

$\Delta$ 中的根称为 \textbf{简单根 (Simple Roots)}

\subsubsection{定义 10.1.2 (正根/负根,高度)}

相对于基 $\Delta$:

\begin{itemize}
	\item 根 $\beta$ 是 \textbf{正的 (Positive)}(记作 $\beta \succ 0$)若所有 $k_\alpha \geq 0$
	\item 根 $\beta$ 是 \textbf{负的 (Negative)}(记作 $\beta \prec 0$)若所有 $k_\alpha \leq 0$
	\item $\beta$ 的 \textbf{高度 (Height)} 为 $\text{ht}(\beta) = \sum k_\alpha$
\end{itemize}

记 $\Phi^+ = \{\beta \in \Phi \mid \beta \succ 0\}$,$\Phi^- = \{\beta \in \Phi \mid \beta \prec 0\}$

\subsubsection{引理 10.1.3 (简单根的角度性质)}

若 $\Delta$ 是 $\Phi$ 的基,则对 $\Delta$ 中不同元素 $\alpha \neq \beta$:

\begin{enumerate}
	\item $(\alpha, \beta) \leq 0$
	\item $\alpha - \beta$ 不是根
\end{enumerate}

\subsubsection{定义 10.1.4 (魏尔腔 Weyl Chambers)}

超平面 $P_\alpha$ ($\alpha \in \Phi$) 将 $E$ 分割成有限多个区域。$E - \bigcup_\alpha P_\alpha$ 的连通分支称为 \textbf{魏尔腔 (Weyl Chambers)}。

对正则向量 $\gamma \in E$(即不在任何 $P_\alpha$ 上),定义:

\begin{itemize}
	\item $\Phi^+(\gamma) = \{\alpha \in \Phi \mid (\gamma, \alpha) > 0\}$
	\item $\Delta(\gamma)$ = $\Phi^+(\gamma)$ 中不可分解根的集合
\end{itemize}

\subsubsection{定理 10.1.5 (基的存在性)}

\textbf{定理}:$\Phi$ 有基。

\textbf{更精确地}:对任意正则 $\gamma \in E$,集合 $\Delta(\gamma)$ 是 $\Phi$ 的基,且每个基都以此方式得到。

\subsection{10.2 关于简单根的引理 Lemmas on Simple Roots}

\subsubsection{引理 10.2.1 (引理A)}

若 $\alpha$ 是正根但非简单根,则对某个 $\beta \in \Delta$,$\alpha - \beta$ 是根(必为正根)。

\subsubsection{引理 10.2.2 (引理B)}

设 $\alpha$ 是简单根,则 $\sigma_\alpha$ 置换除 $\alpha$ 外的正根。

\textbf{推论}:设 $\delta = \frac{1}{2}\sum_{\beta>0} \beta$,则对所有 $\alpha \in \Delta$:$\sigma_\alpha(\delta) = \delta - \alpha$

\subsubsection{引理 10.2.3 (引理C)}

设 $\alpha_1, \ldots, \alpha_t \in \Delta$(不必不同),记 $\sigma_i = \sigma_{\alpha_i}$。若 $\sigma_1 \cdots \sigma_{t-1}(\alpha_t)$ 是负根,则存在某个指标 $1 \leq s < t$ 使得:
\[
\sigma_1 \cdots \sigma_t = \sigma_1 \cdots \sigma_{s-1} \sigma_{s+1} \cdots \sigma_{t-1}
\]

\textbf{推论}:若 $\sigma = \sigma_1 \cdots \sigma_t$ 是 $\sigma \in \mathcal{W}$ 的最小表达式,则 $\sigma(\alpha_t) \prec 0$

\subsection{10.3 魏尔群 The Weyl Group}

\subsubsection{定理 10.3.1 (魏尔群的主要性质)}

设 $\Delta$ 是 $\Phi$ 的基,则:

\textbf{(a)} $\mathcal{W}$ 在魏尔腔上传递作用\textbf{(b)} $\mathcal{W}$ 在基上传递作用\textbf{(c)} 若 $\alpha$ 是任意根,存在 $\sigma \in \mathcal{W}$ 使得 $\sigma(\alpha) \in \Delta$\textbf{(d)} $\mathcal{W}$ 由 $\sigma_\alpha$ ($\alpha \in \Delta$) 生成\textbf{(e)} $\mathcal{W}$ 在基上单传递作用

\subsubsection{定义 10.3.2 (长度函数 Length Function)}

对 $\sigma \in \mathcal{W}$,\textbf{长度 (Length)} $\ell(\sigma)$ 是表示 $\sigma$ 所需简单反射的最少个数。

\subsubsection{引理 10.3.3 (长度公式 Length Formula)}

对所有 $\sigma \in \mathcal{W}$:$\ell(\sigma) = n(\sigma)$,其中 $n(\sigma)$ 是被 $\sigma$ 映为负根的正根个数。

\subsubsection{引理 10.3.4 (基本域)}

设 $\lambda, \mu \in \overline{\mathfrak{C}(\Delta)}$(基本魏尔腔的闭包)。若 $\sigma \lambda = \mu$ 对某个 $\sigma \in \mathcal{W}$,则 $\sigma$ 是固定 $\lambda$ 的简单反射的乘积;特别地,$\lambda = \mu$。

\subsection{10.4 不可约根系统 Irreducible Root Systems}

\subsubsection{定义 10.4.1 (不可约根系统)}

$\Phi$ 是 \textbf{不可约的 (Irreducible)},若不能分解为两个非平凡正交子集的并。

\subsubsection{引理 10.4.2}

$\Phi$ 不可约当且仅当 $\Delta$ 不能分解为正交子集。

\subsubsection{引理 10.4.3 (唯一最高根)}

设 $\Phi$ 不可约,则存在唯一最高根 $\beta$(相对于偏序 $\prec$),且若 $\beta = \sum k_\alpha \alpha$,则所有 $k_\alpha > 0$。

\subsubsection{引理 10.4.4 (不可约作用)}

设 $\Phi$ 不可约,则 $\mathcal{W}$ 在 $E$ 上不可约作用。特别地,根的 $\mathcal{W}$-轨道张成 $E$。

\subsubsection{引理 10.4.5 (最多两种根长)}

设 $\Phi$ 不可约,则 $\Phi$ 中最多出现两种根长,且给定长度的所有根在 $\mathcal{W}$ 下共轭。

\textbf{注记}:若有两种根长,称为长根和短根。习惯上,若所有根等长,称为长根。


\section{第11节:分类 Classification}

\subsection{11.1 嘉当矩阵 Cartan Matrix}

\subsubsection{定义 11.1.1 (嘉当矩阵)}

对基 $\Delta = \{\alpha_1, \ldots, \alpha_\ell\}$,\textbf{嘉当矩阵 (Cartan Matrix)} 为 $(\langle\alpha_i, \alpha_j\rangle)$。

其元素称为 \textbf{嘉当整数 (Cartan Integers)}。

\subsubsection{命题 11.1.2 (嘉当矩阵确定同构类)}

嘉当矩阵完全确定根系统的同构类。

\textbf{证明思路}:相同嘉当矩阵的两个根系统间存在唯一向量空间同构,保持所有嘉当整数,从而扩展为根系统同构。

\subsection{11.2 考克斯特图与戴金图}

\subsubsection{定义 11.2.1 (考克斯特图 Coxeter Graph)}

$\Phi$ 的 \textbf{考克斯特图 (Coxeter Graph)} 有 $\ell$ 个顶点,顶点 $i$ 和 $j$ ($i \neq j$) 之间有 $\langle\alpha_i, \alpha_j\rangle\langle\alpha_j, \alpha_i\rangle$ 条边。

\subsubsection{定义 11.2.2 (戴金图 Dynkin Diagram)}

当出现多重边时,加箭头指向较短根。这个 \textbf{戴金图 (Dynkin Diagram)} 完全确定嘉当矩阵。

\subsection{11.3 不可约分解}

\subsubsection{命题 11.3.1 (不可约分解)}

$\Phi$ 唯一分解为不可约根系统 $\Phi_i$ 在正交子空间 $E_i$ 中的并,使得 $E = E_1 \oplus \cdots \oplus E_t$(正交直和)。

\subsection{11.4 分类定理}

\subsubsection{定理 11.4.1 (完全分类)}

若 $\Phi$ 是秩 $\ell$ 的不可约根系统,其戴金图恰好是以下之一:

\begin{itemize}
	\item \textbf{$A_\ell$ 型} ($\ell \geq 1$):$\circ - \circ - \cdots - \circ$ (直链)
	\item \textbf{$B_\ell$ 型} ($\ell \geq 2$):$\circ - \circ - \cdots - \circ \Rightarrow \circ$ (末端双边指向最后)
	\item \textbf{$C_\ell$ 型} ($\ell \geq 3$):$\circ - \circ - \cdots - \circ \Leftarrow \circ$ (末端双边指向倒数第二)
	\item \textbf{$D_\ell$ 型} ($\ell \geq 4$):分叉图
	\item \textbf{$E_6, E_7, E_8$ 型}:例外情形
	\item \textbf{$F_4$ 型}:$\circ - \circ \Rightarrow \circ - \circ$ (中间双边)
	\item \textbf{$G_2$ 型}:$\circ \equiv \circ$ (三重边)
\end{itemize}

\textbf{证明策略}:

\begin{enumerate}
	\item 用几何约束分类可能的考克斯特图
	\item 证明这些是"可容许"向量集的唯一连通图
	\item 确定每个考克斯特图产生的戴金图
\end{enumerate}


\section{第12节:构造与自同构 Construction and Automorphisms}

\subsection{12.1 A-G型的构造}

\subsubsection{\texorpdfstring{$A_\ell$}{A_ell} 型 (\texorpdfstring{$\ell \geq 1$}{ell geq 1})}

\begin{itemize}
	\item $E = \mathbb{R}^{\ell+1}$ 中正交于 $\varepsilon_1 + \cdots + \varepsilon_{\ell+1}$ 的 $\ell$ 维子空间
	\item $\Phi = \{\varepsilon_i - \varepsilon_j \mid i \neq j\}$
	\item 基:$\{\varepsilon_1 - \varepsilon_2, \varepsilon_2 - \varepsilon_3, \ldots, \varepsilon_\ell - \varepsilon_{\ell+1}\}$
	\item 魏尔群:$\mathcal{W} \cong S_{\ell+1}$(对称群)
\end{itemize}

\subsubsection{\texorpdfstring{$B_\ell$}{B_ell} 型 (\texorpdfstring{$\ell \geq 2$}{ell geq 2})}

\begin{itemize}
	\item $E = \mathbb{R}^\ell$
	\item $\Phi = \{\pm\varepsilon_i, \pm(\varepsilon_i \pm \varepsilon_j) \mid i \neq j\}$
	\item 基:$\{\varepsilon_1 - \varepsilon_2, \ldots, \varepsilon_{\ell-1} - \varepsilon_\ell, \varepsilon_\ell\}$
	\item 魏尔群:$\mathcal{W} \cong (\mathbb{Z}/2\mathbb{Z})^\ell \rtimes S_\ell$
\end{itemize}

\subsubsection{\texorpdfstring{$C_\ell$}{C_ell} 型 (\texorpdfstring{$\ell \geq 3$}{ell geq 3})}

可视为 $B_\ell$ 的对偶根系统,或:

\begin{itemize}
	\item $E = \mathbb{R}^\ell$
	\item $\Phi = \{\pm 2\varepsilon_i, \pm(\varepsilon_i \pm \varepsilon_j) \mid i \neq j\}$
	\item 基:$\{\varepsilon_1 - \varepsilon_2, \ldots, \varepsilon_{\ell-1} - \varepsilon_\ell, 2\varepsilon_\ell\}$
\end{itemize}

\subsubsection{\texorpdfstring{$D_\ell$}{D_ell} 型 (\texorpdfstring{$\ell \geq 4$}{ell geq 4})}

\begin{itemize}
	\item $E = \mathbb{R}^\ell$
	\item $\Phi = \{\pm(\varepsilon_i \pm \varepsilon_j) \mid i \neq j\}$
	\item 基:$\{\varepsilon_1 - \varepsilon_2, \ldots, \varepsilon_{\ell-1} - \varepsilon_\ell, \varepsilon_{\ell-1} + \varepsilon_\ell\}$
	\item 魏尔群:$\mathcal{W} \cong (\mathbb{Z}/2\mathbb{Z})^{\ell-1} \rtimes S_\ell$
\end{itemize}

\subsubsection{例外型 (\texorpdfstring{$E_6, E_7, E_8, F_4, G_2$}{E_6, E_7, E_8, F_4, G_2})}

通过在适当格中选取特定长度的向量构造。

\subsection{12.2 自同构群}

\subsubsection{定理 12.2.1 (自同构群结构)}

$\text{Aut}(\Phi) = \mathcal{W} \rtimes \Gamma$,其中 $\Gamma$ 是图自同构群。

对不可约 $\Phi$:

\begin{itemize}
	\item $\Gamma \cong \mathbb{Z}/2\mathbb{Z}$:$A_\ell$ ($\ell \geq 2$), $D_\ell$ ($\ell > 4$), $E_6$
	\item $\Gamma \cong S_3$:$D_4$
	\item $\Gamma = 1$:其他类型
\end{itemize}


\section{第13节:权的抽象理论 Abstract Theory of Weights}

\subsection{13.1 权 Weights}

\subsubsection{定义 13.1.1 (权格 Weight Lattice)}

\textbf{权格 (Weight Lattice)} 为:
\[
\Lambda = \{\lambda \in E \mid \langle\lambda, \alpha\rangle \in \mathbb{Z} \text{ 对所有 } \alpha \in \Phi\}
\]

\subsubsection{定义 13.1.2 (基本权 Fundamental Weights)}

\textbf{基本支配权 (Fundamental Dominant Weights)} $\lambda_1, \ldots, \lambda_\ell$ 是相对于内积对 $\{2\alpha_i/(\alpha_i, \alpha_i)\}$ 的对偶基:
\[
\frac{2(\lambda_i, \alpha_j)}{(\alpha_j, \alpha_j)} = \delta_{ij}
\]

\subsubsection{命题 13.1.3 (权格的结构)}

$\Lambda$ 由基本权生成,且 $\Lambda/\Lambda_r$ 有限(称为 \textbf{基本群 (Fundamental Group)}),其中 $\Lambda_r$ 是 \textbf{根格 (Root Lattice)}。

基本群的阶数等于嘉当矩阵的行列式。

\subsection{13.2 支配权 Dominant Weights}

\subsubsection{定义 13.2.1 (支配权)}

$\lambda \in \Lambda$ 是 \textbf{支配的 (Dominant)},若对所有 $\alpha \in \Delta$:$\langle\lambda, \alpha\rangle \geq 0$。

$\lambda$ 是 \textbf{强支配的 (Strongly Dominant)},若所有不等式严格成立。

记 $\Lambda^+$ 为所有支配权的集合。

\subsubsection{引理 13.2.2 (支配权的唯一性)}

每个权都与唯一的支配权在 $\mathcal{W}$ 下共轭。若 $\lambda$ 支配,则对所有 $\sigma \in \mathcal{W}$:$\sigma \lambda \preceq \lambda$,且若 $\lambda$ 强支配,则 $\sigma \lambda = \lambda$ 仅当 $\sigma = 1$。

\subsubsection{引理 13.2.3 (支配权的有界性)}

设 $\lambda \in \Lambda^+$,则支配权 $\mu \prec \lambda$ 的个数有限。

\subsection{13.3 权 \texorpdfstring{$\delta$}{delta}}

\subsubsection{定义与性质}

\textbf{权 $\delta$} 定义为:
\[
\delta = \frac{1}{2}\sum_{\alpha>0} \alpha = \sum_{i=1}^\ell \lambda_i
\]

\subsubsection{引理 13.3.1}

$\delta$ 是强支配权,且对所有 $\alpha_i \in \Delta$:$\sigma_i(\delta) = \delta - \alpha_i$。

\subsubsection{引理 13.3.2 (重要不等式)}

设 $\mu \in \Lambda^+$,$\nu = \sigma^{-1}\mu$ ($\sigma \in \mathcal{W}$),则:
\[
(\nu + \delta, \nu + \delta) \leq (\mu + \delta, \mu + \delta)
\]
等号成立当且仅当 $\nu = \mu$。

\subsection{13.4 饱和权集 Saturated Sets of Weights}

\subsubsection{定义 13.4.1 (饱和集)}

$\Lambda$ 的子集 $\Pi$ 是 \textbf{饱和的 (Saturated)},若对所有 $\lambda \in \Pi$,$\alpha \in \Phi$,$0 \leq i \leq \langle\lambda, \alpha\rangle$,都有 $\lambda - i\alpha \in \Pi$。

\subsubsection{性质}

\begin{itemize}
	\item 任何饱和集自动在 $\mathcal{W}$ 下稳定
	\item 饱和集有最高权 $\lambda$ 若 $\lambda \in \Pi$ 且对所有 $\mu \in \Pi$:$\mu \preceq \lambda$
\end{itemize}

\subsubsection{引理 13.4.1}

有最高权 $\lambda$ 的饱和集必定有限。

\subsubsection{引理 13.4.2 (饱和集的刻画)}

设 $\Pi$ 饱和,有最高权 $\lambda$。若 $\mu \in \Lambda^+$ 且 $\mu \prec \lambda$,则 $\mu \in \Pi$。

\subsubsection{定理 13.4.3 (饱和集的存在唯一性)}

对每个 $\lambda \in \Lambda^+$,存在唯一有限饱和集以 $\lambda$ 为最高权,由所有支配权 $\mu \preceq \lambda$ 及其 $\mathcal{W}$-共轭组成。

\subsubsection{引理 13.4.4 (弗罗伊登塔尔不等式)}

设 $\Pi$ 饱和,有最高权 $\lambda$。若 $\mu \in \Pi$,则:
\[
(\mu + \delta, \mu + \delta) \leq (\lambda + \delta, \lambda + \delta)
\]
等号成立当且仅当 $\mu = \lambda$。


\section{核心理论脉络与关键洞察}

\subsection{逻辑发展}

\begin{enumerate}
	\item \textbf{公理 (R1)-(R4)} → \textbf{反射几何学} → \textbf{魏尔群的基本性质}
	\item \textbf{基的存在性 (10.1.5)} → \textbf{正负根分解} → \textbf{简单根理论 (10.2)}
	\item \textbf{魏尔群生成 (10.3.1d)} → \textbf{长度函数 (10.3.3)} → \textbf{约化表达式理论}
	\item \textbf{不可约性概念 (10.4)} → \textbf{几何约束} → \textbf{图论分析} → \textbf{完全分类 (11.4.1)}
	\item \textbf{嘉当矩阵理论 (11.1)} → \textbf{戴金图} → \textbf{同构类的完全刻画}
	\item \textbf{显式构造 (12.1)} → \textbf{自同构群结构 (12.2.1)}
	\item \textbf{权格理论 (13.1)} → \textbf{支配权 (13.2)} → \textbf{饱和集 (13.4)} → \textbf{表示论基础}
\end{enumerate}

\subsection{关键洞察}

\textbf{根系统通过有限几何对象(反射、超平面配置、魏尔群)完全刻画了半单李代数的组合-几何结构,建立了代数结构与欧几里得几何的深刻联系,为李群表示论提供了坚实的组合基础。}

\textbf{核心思想}:复杂的无穷代数结构可以通过有限的组合数据(戴金图)完全分类和理解。
