\section{复分析知识简介-于品}
假设$\Omega\subset \mathbb{C}$是开集,$K\subset \Omega$是有界带边区域
(特别地,$K$是紧的),其边界$\gamma=\partial K$是$C^1$曲线(可以有多个连通分支)。
\begin{figure}[H]
    \centering
    \includegraphics[width=0.5\textwidth]{L0701.png}
    \caption{}
\end{figure}

我们已经对复解析函数$F(z)$证明了Cauchy积分公式:如果$F(z)$在$\Omega$上是复解析的,
那么,对于$z_0\in \mathring{K}$($K$的内部),我们有
\begin{equation*}
F(z_0)=\frac{1}{2\pi i}\int_\gamma \frac{F(z)}{z-z_0}dz.
\end{equation*}


利用Cauchy公式,我们证明,复解析函数是``解析''的,也就是说可以在如下意义下写成幂级数的形式:
\begin{theorem}
假设$F$为开集$\Omega\subset \mathbb{C}$上的复解析函数,
并且以$z_0$为圆心以$R$为半径的开球$B_R(z_0)\subset \Omega$。
那么,$F(z)$在$z_0$处的\textbf{解析半径}至少是$R$,也就是说,在$B_R(z_0)$上,我们有
\begin{equation*}
F(z)=a_0+a_1(z-z_0)+a_2(z-z_0)^2+\cdots+a_n(z-z_0)^n+\cdots,
\end{equation*}
其中
\[a_k=\frac{1}{2\pi i}\int_{|\xi-z_0|=r}\frac{F(\xi)}{(\xi-z_0)^{k+1}}d\xi, \ \ r<R.\]
这里,等式右边的幂级数对任意的$z\in B_R(z_0)$都是收敛的。
上面系数定义中的$r$可以是$(0,R)$中的任意一个数。
\end{theorem}
证明的想法很简单:我们把Cauchy积分公式
\begin{equation*}
F(z)=\frac{1}{2\pi i}\int_\gamma \frac{F(\xi)}{\xi-z}d\xi
\end{equation*}
的右边的积分项强行展开即可。
\begin{proof}
我们选取一个正实数$r$,使得$|z-z_0|<r<R$。我们任意给定$\xi$,使得$|\xi-z_0|=r$。
\begin{figure}[H]
    \centering
    \includegraphics[width=0.5\textwidth]{L0702.png}
    \caption{}
\end{figure}
此时,我们有
\begin{align*}
\frac{1}{\xi-z}&=\frac{1}{(\xi-z_0)-(z-z_0)}=\frac{1}{\xi-z_0}\frac{1}{1-\frac{z-z_0}{\xi-z_0}}\\
&=\sum_{k=0}^\infty \frac{(z-z_0)^k}{(\xi-z_0)^{k+1}}.
\end{align*}
在上面的式子中,由于$\left|\dfrac{z-z_0}{\xi-z_0}\right|<1$,所以,我们有(级数收敛):
\[\frac{1}{1-\frac{z-z_0}{\xi-z_0}}=\sum_{k=0}^\infty \left(\frac{z-z_0}{\xi-z_0}\right)^k.\]
这些级数显然是绝对收敛的,从而与积分可交换(也可以利用Lebesgue控制收敛定理)。
所以,代入Cauchy积分公式,我们就得到
\begin{align*}
F(z)&=\frac{1}{2\pi i}\int_{|\xi-z_0|=r} \sum_{k=0}^\infty  F(\xi)\frac{(z-z_0)^k}{(\xi-z_0)^{k+1}}d\xi\\
&= \sum_{k=0}^\infty   \left(\frac{1}{2\pi i}\int_{|\xi-z_0|=r}\frac{F(\xi)}{(\xi-z_0)^{k+1}}d\xi \right)(z-z_0)^k
\end{align*}
比较系数,这就给出了定理的证明。
\end{proof}
\begin{corollary}[零点的离散性]
这里的假设与定理中是一致的。我们进一步假设$\Omega$是道路连通的,
即对任意的$z_1,z_2\in \Omega$,存在连续映射
\[\gamma\colon[0,1]\rightarrow \Omega, \ \ t\mapsto \gamma(t),\]
使得$\gamma(0)=z_1,\gamma(1)=z_2$。那么,复解析函数$F$在$\Omega$中的零点是离散的
(即如果$z_0$是$F$的一个零点,那么,存在$\varepsilon>0$,
使得对任意的$z$,$|z-z_0|<\varepsilon$,$F(z)\neq 0$),除非$F\equiv 0$。

特别地,给定$\Omega$上的两个复解析函数$F$和$G$,如果$F$和$G$在$Z\subset \Omega$上
取值相同,并且$Z$在$\Omega$中有聚点,那么,$F\equiv G$。
\end{corollary}
\begin{proof}
假设$z_0$是$F$的一个零点,即$F(z_0)=0$。根据$F$的解析表达式,
在$B_{R}(z_0)\subset \Omega$上,我们有
\begin{equation*} 
F(z)=a_0 + a_{1}(z-z_0)+a_{2}(z-z_0)^2+\cdots+a_n(z-z_0)^n+\cdots.
\end{equation*}
如果这些系数$a_i$全部为$0$, 那么,$F$在$B_{R}(z_0)$上恒为$0$;
否则,假设$a_m$是第一个不是$0$的系数,那么,我们有
\begin{equation*}
F(z)=(z-z_0)^m\big(a_m + a_{m+1}(z-z_0)+a_{m+2}(z-z_0)^2+\cdots\big)=(z-z_0)^mG(z).
\end{equation*}
根据定理的证明,级数
\[G(z)=a_m + a_{m+1}(z-z_0)+a_{m+2}(z-z_0)^2+\cdots\]
也是绝对收敛的,特别地,这是连续的,所以,当$z=z_0$时,$G(z_0)=a_m\neq 0$。
从而,存在$\delta$,使得$G$在$B_{\delta}(z_0)$上不会等于$0$,
此时,我们知道,$F$在$z_0$的附近($B_{\delta}(z_0)$上)恰好有一个零点。\sidenote{
    这就是说,$F$要么在$z_0$的附近恒为$0$,要么在$z_0$的附近恰好只有一个零点。
}

我们现在证明,如果$F$在$z_0$的一个邻域$B_{\delta}(z_0)$上恒为$0$,
那么,$F$在$\Omega$上恒为$0$:任意选取$z_1 \in \Omega$和曲线
\[\gamma\colon[0,1]\rightarrow \Omega, \ \ t\mapsto \gamma(t),\]
使得$\gamma(0)=z_0,\gamma(1)=z_1$。令
\[t_*=\sup\big\{t\in [0,1]\bigm| F(\gamma(t)\big)\big|_{[0,t]}\equiv 0\big\}.\]
由于$F$在$z_0$的一个邻域$B_{\delta}(z_0)$上恒为$0$,所以,$t_*>0$。我们现在证明$t_*=1$:
如果假设$t_*<1$,根据连续性,我们知道$F(\gamma(t_*))=0$。根据前面的构造,
由于$F$在$\gamma(t_*)$的任意一个小邻域中都有零点(因为和$\gamma^{-1}([0,t_*])$相交),
根据之前的推导,存在$\gamma(t_*)$一个邻域$B_{\delta_1}\big(\gamma(t_*)\big)$,
使得$F$在$B_{\delta_1}\big(\gamma(t_*)\big)$上恒为$0$,根据连续性,那么,
存在$\varepsilon>0$,使得$F(\gamma(t)\big)\big|_{[0,t_*+\varepsilon]}\equiv 0$,
这和$t_*$的最大性矛盾。

由于$t_*=1$,所以,$F(z_1)=0$,这就证明了在$\Omega$上,$F\equiv 0$。

定理中的第二个结论考虑$F-G$即可。
\end{proof}
\begin{corollary}
这里的假设与定理中是一致的。那么,$F$的$n$次导数$F^{(n)}$仍然是解析函数。我们进一步有如下的公式:
\begin{equation*}
F^{(n)}(z)=\frac{n!}{2\pi i}\int_{|\xi-z|=r}\frac{F(\xi)}{(\xi-z)^{n+1}}d\xi.
\end{equation*}
特别地,我们有如下的导数估计:
\begin{equation*}
|F^{(n)}(z)|\leqslant \frac{n!}{r^n}\sup_{|\xi-z|=r}|F(\xi)|.
\end{equation*}
\end{corollary}
\begin{proof}
我们将$F$在$z_0=0$(不妨)处展开为级数
\[F(z)=a_0+a_1z+a_2z^2+\cdots,\]
其中,我们假设上面的级数的收敛半径至少是$R>0$,即对于$|z|<R$都是绝对收敛的。根据定理中系数的计算,我们知道
\begin{equation*}
|a_k|\leqslant \frac{1}{r^k}\sup_{|\xi|=r}|F(\xi)|\leqslant \frac{1}{r^k}M, \ \ r<R.
\end{equation*}
其中,$M$为$|F|$在半径为$r$的圆圈上的最大值。从而,对$|z|\leqslant r'<r$,对任意的$k\geqslant 0$,我们有
\[|ka_kz^{k-1}|\leqslant M\underbrace{\frac{k}{r}\left|\frac{r'}{r}\right|^{k-1}}_{b_k}.\]
当$k$足够大的时候,比如说
\[k>k_0=\lfloor\left(1-\frac{r'}{r}\right)^{-1}\rfloor\]
时,我们有
\[\frac{b_{k+1}}{b_k}= \left(1+\frac{1}{k}\right)\frac{r'}{r}<1,\]
所以,$\{b_k\}_{k\geqslant k_0}$可以被一个公比小于$1$几何级数来控制,从而,级数
\[a_1+2a_2z^{1}+3a_3z^2+\cdots\]
在$|z|<r'$时是绝对收敛,这表明可以逐项求微分(根据Lebesgue控制收敛定理的推论)。这说明
\begin{equation*}
F'(z)=a_1+2a_2z^{1}+3a_3z^2+\cdots
\end{equation*}
对于$|z|<r'<r<R$成立。由于$r'$和$r$是任意选取的,所以,上面的的式子对于$|z|<R$都成立。
特别地,我们证明了
\[F'(z_0)=a_1.\]
由归纳法,对任意的$k$,我们就有
\[F^{(k)}(z_0)=k!a_k.\]
再根据定理中的计算,这就证明了这个推论叙述中的公式。导数估计是显然的。
\end{proof}
\begin{theorem}[Liouville]
假设$F(z)$是在整个$\mathbb{C}$上定义的复解析函数
\footnote{这样的函数称作是\textbf{整函数}。}。如果$F$是有界函数,那么$F$是常值函数。
\end{theorem}
\begin{proof}
我们只要证明$F'(z)\equiv 0$即可:根据$F$在一点处的展开,$F'(z)\equiv 0$,意味着定理中的系数
\[a_1=2a_2=3a_3=\cdots =0.\]
所以,$F$在一点附近恒为$a_0$,从而$F$为常数(一点附近的邻域有聚点)。

我们利用导数估计:
\begin{equation*}
|F'(z)|\leqslant \frac{1}{r}\sup_{|\xi-z|=r}|F(\xi)|\leqslant  \frac{\|F\|_\infty}{r}.
\end{equation*}
由于$F$在整个$\mathbb{C}$上定义,从而可以将$r$取得任意大,
这表明对任意的$z\in \mathbb{C}$,$F'(z)=0$。
\end{proof}
\begin{corollary}[代数基本定理]
对任意的次数非零的复系数多项式
\[P(z)=z^n+a_{n-1}z^{n-1}+\cdots+a_1z+a_0,\]
它在$\mathbb{C}$上必有一个根。
\end{corollary}
\begin{proof}
我们观察到$P$在整个$\mathbb{C}$上是复解析的,并且当$|z|\rightarrow \infty$时,我们有
\[|P(z)|\rightarrow \infty.\]
我们用反证法:如若不然,
\[F(z)=P(z)^{-1}\]
是在全平面$\mathbb{C}$上良好定义的函数。另外,我们有
\[\overline{\partial}(F)=-P(z)^{-2}\overline{\partial}P=0.\]
所以,$F$是复解析的。另外,当$|z|\rightarrow \infty$时,我们还有
\[|F(z)|\rightarrow 0.\]
这说明,$F$是有界的。根据Liouville定理,$F$为常数,从而$P$也是,那么它的次数是$0$,矛盾。
\end{proof}

Cauchy积分公式是复分析中最重要的公式,除了用来证明解析性,
它还有其他众多重要的推论,比如说关于复解析函数的极大模原理:
\begin{theorem}[极大模原理]假设$F(z)$是区域$\Omega\subset \mathbb{C}$上的复解析函数,
    那么$|F(z)|$的最大值,如果能取到的话,一定在$\Omega$的边界$\partial \Omega$上取到。进一步,如果$|F|$在$\Omega$的内部有最大值点,那么$F$一定是常值函数。\sidenote{
        按照同样的证明思路可知极小模也在边界上取到。
    }
\end{theorem}
\begin{proof}
假设$z_0\in \mathring{\Omega}$是$|F|$的最大值点,即
\[|F(z_0)|=\sup_{z\in \Omega}|F(z)|.\]
根据Cauchy积分公式,(选取较小的$\varepsilon$,使得$B_\varepsilon(z_0)\subset \Omega$,
下面的的$r$只要满足$r<\varepsilon$即可)我们有
\begin{align*}
|F(z_0)|&=\Big|\frac{1}{2\pi}\int_0^{2\pi}F(z_0+re^{i\theta})d\theta\Big|\\
&\leqslant \frac{1}{2\pi}\int_0^{2\pi}\|F\|_{L^\infty}d\theta\\
&= \frac{1}{2\pi}\int_0^{2\pi} |F(z_0)| d\theta\\
&=|F(z_0)|.
\end{align*}
这表明,上述不等式必须处处取等号,这表明在$z_0$的整个邻域$B_{\varepsilon}(z_0)$上,$|F(z)|$为常数。

我们现在证明$F$在$B_{\varepsilon}(z_0)$上为常数,不妨假设$z_0=0$。
如若不然,那么,我们当$\varepsilon$较小时,我们有解析表达式
\[F(z)=a_0+ z^m(a_m+a_{m+1}z+a_{m+2}z^2+\cdots),\]
其中$a_m\neq 0$。通过对$F(z)$乘以一个常数,我们还可以假设$a_0>0$
(如果$a_0=0$,那么$|F|$在$z_0$附近恒为零)。
选取$\xi$,使得$\xi^m=-\overline{a_m}$,所以,对于比较小的$\delta>0$,我们有
\[F(\delta\xi)=a_0-\delta^m |a_m|^2 +O(\delta^{m+1}).\]
令$\delta\rightarrow 0$,这就和$|F(z_0)|$最大矛盾。
所以,$F$在$B_{\varepsilon}(z_0)$上为常数,所以$F$为常数。
\end{proof}
\begin{figure}[H]
    \centering
    \includegraphics[width=0.5\textwidth]{L0803.png}
    \caption{}
\end{figure}
\begin{theorem}[Laurent展开]$F$是环面$\big\{z\in \mathbb{C}\bigm|R_1<|z|<R_2\big\}$ 
    (我们经常取$R_1=0$)上的复解析函数。
    对每个整数$k\in \mathbb{Z}$,对任意的$R_1<r<R_2$,我们定义
\begin{equation*}
a_k=\frac{1}{2\pi i}\int_{|z|=r}\frac{F(z)}{z^{k+1}}dz
=\frac{1}{2\pi r^k}\int_{0}^{2\pi}e^{-ik\theta}F(re^{i\theta})d\theta.
\end{equation*}
那么,对任意满足$R_1<|z|<R_2$的$z$,我们有(对每个固定的$z$,以下的级数绝对收敛)
\begin{equation*}
F(z)=\sum_{k=-\infty}^\infty a_kz^k.
\end{equation*}
\end{theorem}
\begin{proof}
 根据Cauchy积分公式,$a_k$与$r$在$(R_1,R_2)$中的选择无关。我们选取
 \[R_1<r_1'<r_1<|z|<r_2<r'_2<R_2\]
 并令$C'_{i}$为半径为$r'_i$并且中心在原点的圆,其中$i=1,2$。根据~Cauchy~积分公式,我们有
\begin{equation*}
F(z)=\frac{1}{2\pi i}\int_{C'_2}\frac{F(\xi)}{\xi-z}d\xi-\frac{1}{2\pi i}\int_{C'_1}\frac{F(\xi)}{\xi-z}d\xi.
\end{equation*}
我们重复之前证明复解析函数能做解析展开的做法。

对于第一项,由于对任意的$\xi \in C'_2$,我们有$|z|<|\xi|$,我们有
\begin{equation*}
\frac{1}{\xi-z}=\frac{1}{\xi}\frac{1}{1-\frac{z}{\xi}}=\frac{1}{\xi}\sum_{k=0}^\infty\left(\frac{z}{\xi}\right)^k.
\end{equation*}
第二项之中,由于$|z|>|\xi|$,其中$\xi \in C'_1$,我们有
\begin{equation*}
\frac{1}{\xi-z}=-\frac{1}{z}\frac{1}{1-\frac{\xi}{z}}=-\frac{1}{z}\sum_{k=0}^\infty\left(\frac{\xi}{z}\right)^k.
\end{equation*}
将上面的两个展开代入$F(z)$的表达式,我们就有
\begin{align*}
F(z)=\frac{1}{2\pi i}\int_{C'_2}\frac{F(\xi)}{\xi}\sum_{k=0}^\infty\left(\frac{z}{\xi}\right)^kd\xi+\frac{1}{2\pi i}\int_{C'_1}\frac{F(\xi)}{z}\sum_{k=0}^\infty\left(\frac{\xi}{z}\right)^kd\xi.
\end{align*}
将求和与积分交换就得到了要证明的结论。
\end{proof}



\begin{definition}假定$F(z)$在区域$\{z\bigm|0<|z-z_0|<r\}$上是复解析的,其中$r>0$,
    它的Laurent展开为
\[F(z)=\sum_{k=-\infty}^\infty a_kz^k.\]
我们称其中的$z^{-1}$的系数$a_{-1}$为$F$在$z_0$处的\textbf{留数},并记作$\textrm{Res}(F;z_0)$。
\end{definition}
\begin{remark}
按照定义,我们有
\begin{equation*}
\textrm{Res}(F;z_0)=\frac{1}{2\pi i}\int_{|z-z_0|=r} {F(z)}dz.
\end{equation*}
这因为Laurent展开中其它幂次的积分都是$0$。
\end{remark}
\begin{theorem}[留数定理]
假定$\Omega\subset \mathbb{C}$是一个紧区域,$\Omega$边界为分段光滑的$C^1$-曲线。
除去点$z_1,\cdots,z_N\in \mathring{\Omega}$之外,
$F$为$\mathbb{C}-\{z_1,\cdots,z_N\}$上的复解析函数,那么,
\begin{equation*}
\sum_{k=1}^N\textrm{Res}(F;z_k)=\frac{1}{2\pi i}\int_{\partial \Omega} {F(z)}dz.
\end{equation*}
\end{theorem}

\begin{proof}
对每个$k\leqslant N$,先将每个$z_k$附近的小圆盘$B_{\varepsilon}(z_k)$从$\Omega$上抠掉,
使得$z\in \Omega-\bigcup_{k\leqslant N}B_\varepsilon(z_k)$。
我们在区域$\Omega-\bigcup_{k\leqslant N}B_\varepsilon(z_k)$上用Cauchy积分公式。
\begin{figure}[H]
    \centering
    \includegraphics[width=0.5\textwidth]{L0804.png}
    \caption{}
\end{figure}
考虑到曲线的定向,我们有
\[\frac{1}{2\pi i}\int_{\partial \Omega} {F(z)}dz-\sum_{k=1}^N\frac{1}{2\pi i}\int_{\partial B_{\varepsilon}(z_k)} {F(z)}dz=0.
\]
所以,
\[\frac{1}{2\pi i}\int_{\partial \Omega} {F(z)}dz=\sum_{k=1}^N\frac{1}{2\pi i}\int_{\partial B_{\varepsilon}(z_k)} {F(z)}dz=\sum_{k=1}^N\textrm{Res}(F;z_k).
\]
这就证明了留数定理。
\end{proof}

我们试举一个有趣的应用,其它在计算上的应用我们将在Fourier变换的一部分再做演示。
\begin{example}
我们在第一学期已经定义了三角函数
\[\cos(z)=\frac{1}{2}\big(e^{iz}+e^{-iz}\big),\ \ \sin(z)=\frac{1}{2i}\big(e^{iz}-e^{-iz}\big).\]
我们研究$\sin(z)$的零点,即找到$z_0$,使得
\[e^{iz_0}-e^{-iz_0}=0 \ \ \Leftrightarrow \ \ e^{2iz_0}=1 \ \ \Leftrightarrow \ z_0\in \pi\mathbb{Z}.\]
对任意的$a\notin \mathbb{Z}$,我们考虑
\[F(z)=\frac{\pi\cot(\pi z)}{(z-a)^2}=\frac{\pi\cos(\pi z)}{\sin(\pi z)(z-a)^2}.\]
那么,$F(z)$不解析的地方只能是$a$和$n\in \mathbb{Z}$。

在$z=a$处,要想有非平凡的留数,$\cot(\pi z)$需要贡献一个$z-a$的因子,所以
\[\textrm{Res}(F;a)=(\pi \cot(\pi z))'|_{z=a}=-\frac{\pi^2}{\sin^2(\pi a)}.\]

在$z=n$处,由于$\sin(\pi z)'\big|_{z=n}\neq 0$,
所以,$\sin(\pi z)$的零点是$1$阶的,据此,我们知道
\[\textrm{Res}(F;n)=\frac{1}{(n-a)^2}.\]

\begin{figure}[H]
    \centering
    \includegraphics[width=0.5\textwidth]{L0805.png}
    \caption{}
\end{figure}
我们对于顶点在~$(\pm(n+\frac{1}{2}),\pm(n+\frac{1}{2}))$的正方形$Q_n$上用留数定理,
其中,$n$足够大使得$a\in Q_n$。所以,
\begin{equation*}
\sum_{|k|\leqslant n}\frac{1}{(k-a)^2}-\frac{\pi^2}{\sin^2(\pi a)}=\frac{1}{2\pi i}\int_{\partial Q_n} \frac{\pi\cos(\pi z)}{\sin(\pi z)(z-a)^2}dz.
\end{equation*}
利用$z=x+iy$,我们很容易看出
\[ \frac{\pi\cos(\pi z)}{\sin(\pi z)(z-a)^2}=O(\frac{1}{n^2}).\]
所以上面的积分项的贡献为
\[\frac{1}{2\pi i}\int_{\partial Q_n} \frac{\pi\cos(\pi z)}{\sin(\pi z)(z-a)^2}dz= O(\frac{1}{n}).\]
令$n\rightarrow \infty$,我们就得到了
\begin{equation*}
\sum_{n\in \mathbb{Z}}\frac{1}{(n-a)^2}=\frac{\pi^2}{\sin^2(\pi a)}.
\end{equation*}

\end{example}

利用上面例子中的分析,我们还可以证明Euler的著名公式。
首先,在$\mathbb{C}-\mathbb{Z}$上的任意一个紧集上,级数
\[f(z)=\displaystyle \sum_{n\in \mathbb{Z}} \frac{1}{(z-n)^2}\]
是一致收敛的,所以,$f(z)$是$\mathbb{C}-\mathbb{Z}$上的复解析函数(可逐项求导数)。
Euler观察到,函数
\[g(z)= \frac{\pi^2}{\sin^2(\pi z)}\]
也是$\mathbb{C}-\mathbb{Z}$上的复解析函数。进一步,由于$\sin(\pi z)$的零点是单零点,
所以,$g(z)$在每个$n\in \mathbb{Z}$处的Laurent展开的负幂和$f(z)$的是一致的。所以,
\begin{equation*}
F(z)=g(z)-f(z)=\frac{\pi^2}{\sin^2(\pi z)}-\sum_{n\in \mathbb{Z}} \frac{1}{(z-n)^2}
\end{equation*}
是全平面上定义的复解析函数并且具有周期性$F(z+1)=F(z)$。

对于任意的$x\in[0,1]$,我们很容易证明下面的极限:
\[\lim_{|y|\rightarrow \infty} F(x+iy)=0.\]
利用周期性,我们就知道$F$是有界的,从而根据Liouville定理,我们得到$F\equiv 0$。这表明
\begin{equation*}
\frac{\pi^2}{\sin^2(\pi z)}=\sum_{n\in \mathbb{Z}} \frac{1}{(z-n)^2}.
\end{equation*}
稍加变形,我们有
\begin{equation*}
\frac{\pi^2}{\sin^2(\pi z)}-\frac{1}{z^2}=\sum_{n\neq 0} \frac{1}{(z-n)^2}.
\end{equation*}
左右在$z=0$处取极限,我们就证明了著名的Euler公式:
\begin{equation*}
\frac{1}{1^2}+\frac{1}{2^2}+\frac{1}{3^2}+\frac{1}{4^2}+\cdots =\frac{1}{6}\pi^2.
\end{equation*}

