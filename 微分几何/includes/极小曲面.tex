\section{极小曲面}

参见彭家贵《微分几何教材》

\begin{definition}[极小曲面]
$\mathbb{E}^3$ 中平均曲率处处恒为零的曲面称为\textbf{极小曲面}.
\end{definition}
\subsection{判定极小曲面的准则}

第一基本形式系数
\[
\begin{aligned}
& E=x_u \cdot x_u=\cosh ^2 u, \quad F=x_u \cdot x_v=0 \\
& G=x_v \cdot x_v=\cosh ^2 u \\
& L=x_{u u} \cdot n=-1, \quad M=x_{u v} \cdot n=0, \quad N=x_{v v} \cdot n=1
\end{aligned}
\]
中的曲率函数
\[
H=\frac{1}{2} \cdot \frac{L G-2 M F+N E}{E G-F^2}=0
\]
即曲面为极小曲面.
