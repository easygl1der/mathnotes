\documentclass[12pt,a4paper]{article}
\usepackage[UTF8]{ctex}
\usepackage{amsmath,amssymb,amsthm}
\usepackage{geometry}
\usepackage{graphicx}
\usepackage{xcolor}
\usepackage{hyperref}
\usepackage{fancyhdr}
\usepackage{enumitem}
\usepackage{booktabs}

\geometry{left=2.5cm,right=2.5cm,top=2.5cm,bottom=2.5cm}

\title{\textbf{丘成桐大学生数学竞赛\\统计方向复习资料}}
\author{数理统计复习指南}
\date{\today}

\newtheorem{theorem}{定理}[section]
\newtheorem{definition}{定义}[section]
\newtheorem{example}{例题}[section]
\newtheorem{corollary}{推论}[section]
\newtheorem{lemma}{引理}[section]

\theoremstyle{remark}
\newtheorem{remark}{注记}[section]

\pagestyle{fancy}
\fancyhf{}
\fancyhead[L]{\leftmark}
\fancyfoot[C]{\thepage}

\begin{document}

\maketitle
\tableofcontents
\newpage

\section{分布理论与基础统计}

\subsection{连续分布族}

\subsubsection{正态分布 (Normal Distribution)}

\begin{definition}[正态分布]
随机变量$X$服从参数为$\mu$和$\sigma^2$的正态分布,记作$X \sim N(\mu, \sigma^2)$,其概率密度函数为:
$$f(x) = \frac{1}{\sqrt{2\pi\sigma^2}} e^{-\frac{(x-\mu)^2}{2\sigma^2}}, \quad x \in \mathbb{R}$$
\end{definition}

\textbf{重要性质:}
\begin{itemize}
\item $E[X] = \mu$,$\text{Var}(X) = \sigma^2$
\item 标准化:$Z = \frac{X-\mu}{\sigma} \sim N(0,1)$
\item 线性组合:若$X_i \sim N(\mu_i, \sigma_i^2)$独立,则$\sum a_i X_i \sim N(\sum a_i \mu_i, \sum a_i^2 \sigma_i^2)$
\end{itemize}

\begin{example}[正态分布计算]
设$X \sim N(100, 16)$,求$P(96 < X < 108)$。

\textbf{解:}
$$Z = \frac{X-100}{4} \sim N(0,1)$$
$$P(96 < X < 108) = P\left(\frac{96-100}{4} < Z < \frac{108-100}{4}\right) = P(-1 < Z < 2)$$
$$= \Phi(2) - \Phi(-1) = 0.9772 - 0.1587 = 0.8185$$
\end{example}

\subsubsection{卡方分布 ($\chi^2$ Distribution)}

\begin{definition}[卡方分布]
若$Z_1, Z_2, \ldots, Z_n$独立同分布于$N(0,1)$,则$\chi^2 = \sum_{i=1}^n Z_i^2$服从自由度为$n$的卡方分布,记作$\chi^2 \sim \chi^2(n)$。
\end{definition}

\textbf{性质:}
\begin{itemize}
\item $E[\chi^2] = n$,$\text{Var}(\chi^2) = 2n$
\item 可加性:若$\chi_1^2 \sim \chi^2(n_1)$,$\chi_2^2 \sim \chi^2(n_2)$独立,则$\chi_1^2 + \chi_2^2 \sim \chi^2(n_1 + n_2)$
\item 渐近正态性:当$n$很大时,$\frac{\chi^2 - n}{\sqrt{2n}} \stackrel{d}{\to} N(0,1)$
\end{itemize}

\subsubsection{t分布 (t-Distribution)}

\begin{definition}[t分布]
若$Z \sim N(0,1)$,$V \sim \chi^2(n)$且相互独立,则$T = \frac{Z}{\sqrt{V/n}}$服从自由度为$n$的t分布,记作$T \sim t(n)$。
\end{definition}

\textbf{性质:}
\begin{itemize}
\item 当$n > 1$时,$E[T] = 0$
\item 当$n > 2$时,$\text{Var}(T) = \frac{n}{n-2}$
\item 当$n \to \infty$时,$t(n) \to N(0,1)$
\end{itemize}

\subsubsection{F分布 (F-Distribution)}

\begin{definition}[F分布]
若$U \sim \chi^2(m)$,$V \sim \chi^2(n)$且相互独立,则$F = \frac{U/m}{V/n}$服从自由度为$(m,n)$的F分布,记作$F \sim F(m,n)$。
\end{definition}

\textbf{性质:}
\begin{itemize}
\item $E[F] = \frac{n}{n-2}$(当$n > 2$时)
\item $\text{Var}(F) = \frac{2n^2(m+n-2)}{m(n-2)^2(n-4)}$(当$n > 4$时)
\item $\frac{1}{F} \sim F(n,m)$
\end{itemize}

\subsubsection{Gamma分布}

\begin{definition}[Gamma分布]
随机变量$X$服从参数为$\alpha > 0$和$\beta > 0$的Gamma分布,记作$X \sim \text{Gamma}(\alpha, \beta)$,其概率密度函数为:
$$f(x) = \frac{\beta^\alpha}{\Gamma(\alpha)} x^{\alpha-1} e^{-\beta x}, \quad x > 0$$
其中$\Gamma(\alpha) = \int_0^\infty t^{\alpha-1} e^{-t} dt$是Gamma函数。
\end{definition}

\textbf{性质:}
\begin{itemize}
\item $E[X] = \frac{\alpha}{\beta}$,$\text{Var}(X) = \frac{\alpha}{\beta^2}$
\item 可加性:若$X_1 \sim \text{Gamma}(\alpha_1, \beta)$,$X_2 \sim \text{Gamma}(\alpha_2, \beta)$独立,则$X_1 + X_2 \sim \text{Gamma}(\alpha_1 + \alpha_2, \beta)$
\item 特殊情况:$\chi^2(n) = \text{Gamma}(n/2, 1/2)$,指数分布$\text{Exp}(\lambda) = \text{Gamma}(1, \lambda)$
\end{itemize}

\subsubsection{Beta分布}

\begin{definition}[Beta分布]
随机变量$X$服从参数为$\alpha > 0$和$\beta > 0$的Beta分布,记作$X \sim \text{Beta}(\alpha, \beta)$,其概率密度函数为:
$$f(x) = \frac{\Gamma(\alpha + \beta)}{\Gamma(\alpha)\Gamma(\beta)} x^{\alpha-1} (1-x)^{\beta-1}, \quad 0 < x < 1$$
\end{definition}

\textbf{性质:}
\begin{itemize}
\item $E[X] = \frac{\alpha}{\alpha + \beta}$,$\text{Var}(X) = \frac{\alpha\beta}{(\alpha + \beta)^2(\alpha + \beta + 1)}$
\item 与Gamma分布的关系:若$Y_1 \sim \text{Gamma}(\alpha, \theta)$,$Y_2 \sim \text{Gamma}(\beta, \theta)$独立,则$\frac{Y_1}{Y_1 + Y_2} \sim \text{Beta}(\alpha, \beta)$
\end{itemize}

\subsection{离散分布族}

\subsubsection{多项分布 (Multinomial Distribution)}

\begin{definition}[多项分布]
进行$n$次独立试验,每次试验有$k$种可能结果,结果$i$出现的概率为$p_i$($\sum_{i=1}^k p_i = 1$)。设$X_i$为结果$i$出现的次数,则$(X_1, \ldots, X_k)$服从多项分布:
$$P(X_1 = n_1, \ldots, X_k = n_k) = \frac{n!}{n_1! \cdots n_k!} p_1^{n_1} \cdots p_k^{n_k}$$
其中$\sum_{i=1}^k n_i = n$。
\end{definition}

\textbf{性质:}
\begin{itemize}
\item $E[X_i] = np_i$,$\text{Var}(X_i) = np_i(1-p_i)$
\item $\text{Cov}(X_i, X_j) = -np_ip_j$($i \neq j$)
\item 边际分布:$X_i \sim \text{Binomial}(n, p_i)$
\end{itemize}

\subsubsection{Poisson分布}

\begin{definition}[Poisson分布]
随机变量$X$服从参数为$\lambda > 0$的Poisson分布,记作$X \sim \text{Poisson}(\lambda)$,其概率质量函数为:
$$P(X = k) = \frac{\lambda^k e^{-\lambda}}{k!}, \quad k = 0, 1, 2, \ldots$$
\end{definition}

\textbf{性质:}
\begin{itemize}
\item $E[X] = \text{Var}(X) = \lambda$
\item 可加性:若$X_1 \sim \text{Poisson}(\lambda_1)$,$X_2 \sim \text{Poisson}(\lambda_2)$独立,则$X_1 + X_2 \sim \text{Poisson}(\lambda_1 + \lambda_2)$
\item Poisson逼近:当$n$很大,$p$很小,$np = \lambda$时,$\text{Binomial}(n,p) \approx \text{Poisson}(\lambda)$
\end{itemize}

\subsubsection{负二项分布}

\begin{definition}[负二项分布]
进行独立的Bernoulli试验,成功概率为$p$,设$X$为第$r$次成功前的失败次数,则$X$服从负二项分布:
$$P(X = k) = \binom{k+r-1}{k} p^r (1-p)^k, \quad k = 0, 1, 2, \ldots$$
\end{definition}

\textbf{性质:}
\begin{itemize}
\item $E[X] = \frac{r(1-p)}{p}$,$\text{Var}(X) = \frac{r(1-p)}{p^2}$
\item 特殊情况:当$r = 1$时为几何分布
\end{itemize}

\subsection{基本统计量}

\begin{definition}[样本均值和方差]
设$X_1, X_2, \ldots, X_n$为来自总体$X$的简单随机样本,则:
\begin{itemize}
\item 样本均值:$\bar{X} = \frac{1}{n}\sum_{i=1}^n X_i$
\item 样本方差:$S^2 = \frac{1}{n-1}\sum_{i=1}^n (X_i - \bar{X})^2$
\item 样本标准差:$S = \sqrt{S^2}$
\end{itemize}
\end{definition}

\begin{theorem}[样本均值和方差的性质]
设$X_1, \ldots, X_n$为来自$N(\mu, \sigma^2)$的样本,则:
\begin{enumerate}
\item $\bar{X} \sim N\left(\mu, \frac{\sigma^2}{n}\right)$
\item $\frac{(n-1)S^2}{\sigma^2} \sim \chi^2(n-1)$
\item $\bar{X}$与$S^2$相互独立
\item $\frac{\bar{X} - \mu}{S/\sqrt{n}} \sim t(n-1)$
\end{enumerate}
\end{theorem}

\begin{definition}[样本分位数]
设$X_{(1)} \leq X_{(2)} \leq \cdots \leq X_{(n)}$为顺序统计量,则:
\begin{itemize}
\item 样本中位数:$\text{Med} = \begin{cases} X_{((n+1)/2)} & \text{若}n\text{为奇数} \\ \frac{X_{(n/2)} + X_{(n/2+1)}}{2} & \text{若}n\text{为偶数} \end{cases}$
\item 样本$p$分位数:$Q_p = X_{(\lceil np \rceil)}$
\end{itemize}
\end{definition}

\section{假设检验}

\subsection{Neyman-Pearson理论框架}

\begin{definition}[假设检验的基本概念]
\begin{itemize}
\item \textbf{原假设}(零假设):$H_0$,我们希望检验的假设
\item \textbf{备择假设}:$H_1$或$H_a$,与原假设对立的假设
\item \textbf{简单假设}:完全确定参数值的假设
\item \textbf{复合假设}:不完全确定参数值的假设
\end{itemize}
\end{definition}

\begin{definition}[检验的错误类型]
\begin{itemize}
\item \textbf{第一类错误}(Type I Error):$H_0$为真时拒绝$H_0$,概率记为$\alpha$
\item \textbf{第二类错误}(Type II Error):$H_0$为假时接受$H_0$,概率记为$\beta$
\item \textbf{检验的势}(Power):$H_0$为假时正确拒绝$H_0$的概率,为$1-\beta$
\end{itemize}
\end{definition}

\begin{theorem}[Neyman-Pearson引理]
考虑简单假设$H_0: \theta = \theta_0$对$H_1: \theta = \theta_1$。对于给定的显著性水平$\alpha$,最优检验是似然比检验:
$$\Lambda(x) = \frac{L(\theta_1|x)}{L(\theta_0|x)} \gtrless k$$
其中$k$由$P_{H_0}(\Lambda(X) > k) = \alpha$确定。
\end{theorem}

\subsection{似然比检验}

\begin{definition}[似然比统计量]
设$\Theta_0 \subset \Theta$,似然比统计量定义为:
$$\Lambda(x) = \frac{\sup_{\theta \in \Theta_0} L(\theta|x)}{\sup_{\theta \in \Theta} L(\theta|x)}$$
\end{definition}

\begin{theorem}[广义似然比检验的渐近分布]
在正则条件下,当样本量$n \to \infty$时:
$$-2\log\Lambda(X) \stackrel{d}{\to} \chi^2(r)$$
其中$r = \dim(\Theta) - \dim(\Theta_0)$是自由度。
\end{theorem}

\begin{example}[正态总体均值的检验]
设$X_1, \ldots, X_n \sim N(\mu, \sigma^2)$,$\sigma^2$已知。检验$H_0: \mu = \mu_0$对$H_1: \mu \neq \mu_0$。

\textbf{解:}
检验统计量:$Z = \frac{\bar{X} - \mu_0}{\sigma/\sqrt{n}} \sim N(0,1)$

拒绝域:$|Z| > z_{\alpha/2}$,其中$z_{\alpha/2}$是标准正态分布的$\alpha/2$上分位数。

$p$值:$p = 2P(Z > |z|)$,其中$z$是观测值。
\end{example}

\subsection{常用检验方法}

\subsubsection{单样本t检验}

检验$H_0: \mu = \mu_0$($\sigma^2$未知):
$$t = \frac{\bar{X} - \mu_0}{S/\sqrt{n}} \sim t(n-1)$$

\subsubsection{两样本t检验}

\textbf{等方差情况:}
$$t = \frac{\bar{X}_1 - \bar{X}_2}{S_p\sqrt{1/n_1 + 1/n_2}} \sim t(n_1 + n_2 - 2)$$
其中$S_p^2 = \frac{(n_1-1)S_1^2 + (n_2-1)S_2^2}{n_1 + n_2 - 2}$

\textbf{不等方差情况(Welch检验):}
$$t = \frac{\bar{X}_1 - \bar{X}_2}{\sqrt{S_1^2/n_1 + S_2^2/n_2}}$$
自由度用Satterthwaite近似。

\subsubsection{卡方检验}

\textbf{拟合优度检验:}
$$\chi^2 = \sum_{i=1}^k \frac{(O_i - E_i)^2}{E_i} \sim \chi^2(k-1-p)$$
其中$O_i$是观测频数,$E_i$是期望频数,$p$是估计参数个数。

\textbf{独立性检验:}
$$\chi^2 = \sum_{i=1}^r \sum_{j=1}^c \frac{(O_{ij} - E_{ij})^2}{E_{ij}} \sim \chi^2((r-1)(c-1))$$

\section{参数估计}

\subsection{参数估计的基本概念}

\begin{definition}[估计量的性质]
设$\hat{\theta}_n$是参数$\theta$的估计量:
\begin{itemize}
\item \textbf{无偏性}:$E[\hat{\theta}_n] = \theta$
\item \textbf{一致性}:$\hat{\theta}_n \stackrel{P}{\to} \theta$
\item \textbf{渐近正态性}:$\sqrt{n}(\hat{\theta}_n - \theta) \stackrel{d}{\to} N(0, \sigma^2)$
\item \textbf{有效性}:在所有无偏估计量中方差最小
\end{itemize}
\end{definition}

\subsection{矩估计法}

\begin{definition}[矩估计法]
设总体$X$的$k$阶原点矩为$\mu_k = E[X^k]$,样本$k$阶原点矩为$M_k = \frac{1}{n}\sum_{i=1}^n X_i^k$。

矩估计法是令样本矩等于总体矩:
$$M_k = \mu_k(\theta_1, \ldots, \theta_p), \quad k = 1, 2, \ldots, p$$
解这个方程组得到参数的矩估计量。
\end{definition}

\begin{example}[正态分布的矩估计]
设$X_1, \ldots, X_n \sim N(\mu, \sigma^2)$。

一阶矩:$E[X] = \mu$,$M_1 = \bar{X}$,所以$\hat{\mu} = \bar{X}$

二阶中心矩:$E[(X-\mu)^2] = \sigma^2$,$\frac{1}{n}\sum_{i=1}^n (X_i - \bar{X})^2$,所以$\hat{\sigma}^2 = \frac{1}{n}\sum_{i=1}^n (X_i - \bar{X})^2$
\end{example}

\subsection{最大似然估计}

\begin{definition}[最大似然估计]
设样本$X_1, \ldots, X_n$的似然函数为$L(\theta) = \prod_{i=1}^n f(x_i|\theta)$,最大似然估计量$\hat{\theta}_{MLE}$是使似然函数最大的$\theta$值:
$$\hat{\theta}_{MLE} = \arg\max_\theta L(\theta) = \arg\max_\theta \log L(\theta)$$
\end{definition}

\textbf{求解步骤:}
\begin{enumerate}
\item 写出似然函数$L(\theta)$
\item 取对数得$\ell(\theta) = \log L(\theta)$
\item 求导:$\frac{d\ell(\theta)}{d\theta} = 0$
\item 解方程得到$\hat{\theta}_{MLE}$
\item 验证二阶导数条件确保是最大值
\end{enumerate}

\begin{example}[指数分布的最大似然估计]
设$X_1, \ldots, X_n \sim \text{Exp}(\lambda)$,$f(x|\lambda) = \lambda e^{-\lambda x}$。

似然函数:$L(\lambda) = \prod_{i=1}^n \lambda e^{-\lambda x_i} = \lambda^n e^{-\lambda \sum x_i}$

对数似然:$\ell(\lambda) = n\log\lambda - \lambda\sum_{i=1}^n x_i$

求导:$\frac{d\ell}{d\lambda} = \frac{n}{\lambda} - \sum_{i=1}^n x_i = 0$

解得:$\hat{\lambda}_{MLE} = \frac{n}{\sum_{i=1}^n x_i} = \frac{1}{\bar{x}}$
\end{example}

\subsection{估计量的评价准则}

\begin{definition}[Cramér-Rao下界]
对于满足正则性条件的估计问题,任何无偏估计量$\hat{\theta}$的方差都满足:
$$\text{Var}(\hat{\theta}) \geq \frac{1}{I(\theta)}$$
其中$I(\theta) = E\left[\left(\frac{\partial \log f(X|\theta)}{\partial \theta}\right)^2\right]$是Fisher信息量。
\end{definition}

\begin{theorem}[最大似然估计的渐近性质]
在正则条件下,最大似然估计量$\hat{\theta}_{MLE}$具有以下性质:
\begin{enumerate}
\item 一致性:$\hat{\theta}_{MLE} \stackrel{P}{\to} \theta$
\item 渐近正态性:$\sqrt{n}(\hat{\theta}_{MLE} - \theta) \stackrel{d}{\to} N(0, I^{-1}(\theta))$
\item 渐近有效性:达到Cramér-Rao下界
\end{enumerate}
\end{theorem}

\subsection{Fisher信息量}

\begin{definition}[Fisher信息量]
对于参数$\theta$,Fisher信息量定义为:
$$I(\theta) = E\left[\left(\frac{\partial \log f(X|\theta)}{\partial \theta}\right)^2\right] = -E\left[\frac{\partial^2 \log f(X|\theta)}{\partial \theta^2}\right]$$

对于样本$(X_1, \ldots, X_n)$,总信息量为:$I_n(\theta) = nI(\theta)$
\end{definition}

\begin{example}[正态分布的Fisher信息量]
设$X \sim N(\mu, \sigma^2)$,$\sigma^2$已知。

$\log f(x|\mu) = -\frac{1}{2}\log(2\pi\sigma^2) - \frac{(x-\mu)^2}{2\sigma^2}$

$\frac{\partial \log f}{\partial \mu} = \frac{x-\mu}{\sigma^2}$

$I(\mu) = E\left[\left(\frac{X-\mu}{\sigma^2}\right)^2\right] = \frac{1}{\sigma^2}$

所以样本均值$\bar{X}$的方差下界为$\frac{\sigma^2}{n}$,恰好等于$\text{Var}(\bar{X})$。
\end{example}

\subsection{置信区间}

\begin{definition}[置信区间]
参数$\theta$的$100(1-\alpha)\%$置信区间是一个随机区间$[L(X), U(X)]$,满足:
$$P(L(X) \leq \theta \leq U(X)) = 1-\alpha$$
\end{definition}

\textbf{构造方法:}

\textbf{1. 枢轴量法:}
找到一个枢轴量$Q(X, \theta)$,其分布不依赖于未知参数。

\textbf{2. 大样本方法:}
利用渐近正态性:$\hat{\theta} \pm z_{\alpha/2} \cdot SE(\hat{\theta})$

\textbf{3. 似然比方法:}
基于$-2\log\Lambda(\theta)$的渐近分布。

\begin{example}[正态均值的置信区间]
设$X_1, \ldots, X_n \sim N(\mu, \sigma^2)$,$\sigma^2$未知。

枢轴量:$T = \frac{\bar{X} - \mu}{S/\sqrt{n}} \sim t(n-1)$

$95\%$置信区间:$\bar{X} \pm t_{0.025}(n-1) \cdot \frac{S}{\sqrt{n}}$
\end{example}

\section{贝叶斯统计}

\subsection{贝叶斯定理与先验后验}

\begin{theorem}[贝叶斯定理]
设$\theta$是参数,$x$是观测数据,则:
$$\pi(\theta|x) = \frac{f(x|\theta)\pi(\theta)}{m(x)}$$
其中:
\begin{itemize}
\item $\pi(\theta)$是先验分布
\item $f(x|\theta)$是似然函数
\item $\pi(\theta|x)$是后验分布
\item $m(x) = \int f(x|\theta)\pi(\theta)d\theta$是边际分布
\end{itemize}
\end{theorem}

\begin{definition}[共轭先验]
如果先验分布$\pi(\theta)$和后验分布$\pi(\theta|x)$属于同一分布族,则称先验分布为共轭先验。
\end{definition}

\textbf{常见的共轭分布对:}
\begin{enumerate}
\item \textbf{二项分布-Beta分布:}
   \begin{itemize}
   \item 似然:$X|\theta \sim \text{Binomial}(n, \theta)$
   \item 先验:$\theta \sim \text{Beta}(\alpha, \beta)$
   \item 后验:$\theta|x \sim \text{Beta}(\alpha + x, \beta + n - x)$
   \end{itemize}

\item \textbf{正态分布-正态分布(已知方差):}
   \begin{itemize}
   \item 似然:$X|\mu \sim N(\mu, \sigma^2)$($\sigma^2$已知)
   \item 先验:$\mu \sim N(\mu_0, \tau^2)$
   \item 后验:$\mu|x \sim N\left(\frac{\tau^2 x + \sigma^2\mu_0}{\tau^2 + \sigma^2}, \frac{\sigma^2\tau^2}{\sigma^2 + \tau^2}\right)$
   \end{itemize}

\item \textbf{Poisson分布-Gamma分布:}
   \begin{itemize}
   \item 似然:$X|\lambda \sim \text{Poisson}(\lambda)$
   \item 先验:$\lambda \sim \text{Gamma}(\alpha, \beta)$
   \item 后验:$\lambda|x \sim \text{Gamma}(\alpha + x, \beta + 1)$
   \end{itemize}
\end{enumerate}

\subsection{贝叶斯估计}

\begin{definition}[贝叶斯估计量]
常用的贝叶斯估计量包括:
\begin{itemize}
\item \textbf{后验均值:}$\hat{\theta}_{Bayes} = E[\theta|x]$
\item \textbf{后验中位数:}$\text{Med}(\theta|x)$
\item \textbf{后验众数(MAP):}$\arg\max_\theta \pi(\theta|x)$
\end{itemize}
\end{definition}

\begin{example}[Beta-二项模型]
设观测到$n$次试验中有$x$次成功,成功概率$\theta$的先验为$\text{Beta}(\alpha, \beta)$。

后验分布:$\theta|x \sim \text{Beta}(\alpha + x, \beta + n - x)$

贝叶斯估计(后验均值):
$$\hat{\theta}_{Bayes} = \frac{\alpha + x}{\alpha + \beta + n}$$

这可以写成:
$$\hat{\theta}_{Bayes} = \frac{\alpha + \beta}{\alpha + \beta + n} \cdot \frac{\alpha}{\alpha + \beta} + \frac{n}{\alpha + \beta + n} \cdot \frac{x}{n}$$

即先验均值和样本比例的加权平均。
\end{example}

\begin{theorem}[贝叶斯估计的性质]
贝叶斯估计量(后验均值)具有以下性质:
\begin{enumerate}
\item 在平方损失下是最优的:$\min_{\hat{\theta}} E[(\theta - \hat{\theta})^2|x]$
\item 当样本量趋于无穷时,贝叶斯估计趋向于最大似然估计
\item 自动包含了参数的不确定性
\end{enumerate}
\end{theorem}

\section{大样本性质}

\subsection{一致性}

\begin{definition}[一致性的类型]
设$\{\hat{\theta}_n\}$是参数$\theta$的估计量序列:
\begin{itemize}
\item \textbf{弱一致性:}$\hat{\theta}_n \stackrel{P}{\to} \theta$
\item \textbf{强一致性:}$\hat{\theta}_n \stackrel{a.s.}{\to} \theta$
\item \textbf{均方一致性:}$E[(\hat{\theta}_n - \theta)^2] \to 0$
\end{itemize}
\end{definition}

\begin{theorem}[大数定律的应用]
\begin{enumerate}
\item \textbf{弱大数定律:}$\bar{X}_n \stackrel{P}{\to} \mu$
\item \textbf{强大数定律:}$\bar{X}_n \stackrel{a.s.}{\to} \mu$
\item \textbf{Glivenko-Cantelli定理:}经验分布函数一致收敛到真实分布函数
\end{enumerate}
\end{theorem}

\subsection{渐近正态性}

\begin{theorem}[中心极限定理]
设$X_1, X_2, \ldots$独立同分布,$E[X_i] = \mu$,$\text{Var}(X_i) = \sigma^2 < \infty$,则:
$$\frac{\sqrt{n}(\bar{X}_n - \mu)}{\sigma} \stackrel{d}{\to} N(0, 1)$$
\end{theorem}

\begin{theorem}[Delta方法]
设$\sqrt{n}(X_n - \theta) \stackrel{d}{\to} N(0, \sigma^2)$,$g$是在$\theta$处可微的函数,$g'(\theta) \neq 0$,则:
$$\sqrt{n}(g(X_n) - g(\theta)) \stackrel{d}{\to} N(0, [g'(\theta)]^2 \sigma^2)$$
\end{theorem}

\begin{example}[Delta方法的应用]
设$X_n \sim \text{Binomial}(n, p)$,考虑$\hat{p} = X_n/n$。

已知:$\sqrt{n}(\hat{p} - p) \stackrel{d}{\to} N(0, p(1-p))$

设$g(p) = \log\left(\frac{p}{1-p}\right)$(logit变换),$g'(p) = \frac{1}{p(1-p)}$

由Delta方法:
$$\sqrt{n}(g(\hat{p}) - g(p)) \stackrel{d}{\to} N\left(0, \frac{1}{p(1-p)}\right)$$
\end{example}

\subsection{似然比统计量的渐近分布}

\begin{theorem}[Wilks定理]
在正则条件下,当$H_0$为真时:
$$-2\log\Lambda_n \stackrel{d}{\to} \chi^2(r)$$
其中$r$是约束的个数。
\end{theorem}

\textbf{应用:}
\begin{enumerate}
\item 假设检验:构造似然比检验
\item 置信区间:基于似然比的置信域
\item 模型选择:通过AIC、BIC等准则
\end{enumerate}

\subsection{渐近有效性}

\begin{definition}[渐近有效性]
估计量$\hat{\theta}_n$是渐近有效的,如果:
$$\sqrt{n}(\hat{\theta}_n - \theta) \stackrel{d}{\to} N(0, I^{-1}(\theta))$$
其中$I(\theta)$是Fisher信息量。
\end{definition}

\begin{theorem}[最大似然估计的渐近性质]
在正则条件下,最大似然估计量$\hat{\theta}_{MLE}$是:
\begin{enumerate}
\item 一致的
\item 渐近正态的
\item 渐近有效的
\end{enumerate}
\end{theorem}

\section{总结与应试策略}

\subsection{重点知识梳理}

\textbf{1. 分布理论(25\%)}
\begin{itemize}
\item 掌握各种分布的性质、参数、期望方差
\item 重点:正态分布族($\chi^2$, $t$, $F$分布)的关系
\item 常考:分布的可加性、变换性质
\end{itemize}

\textbf{2. 假设检验(30\%)}
\begin{itemize}
\item 理解Type I/II错误,检验的势
\item 掌握Neyman-Pearson引理和似然比检验
\item 常用检验:$t$检验、$\chi^2$检验、$F$检验
\end{itemize}

\textbf{3. 参数估计(30\%)}
\begin{itemize}
\item 矩估计法和最大似然估计的计算
\item Fisher信息量和Cramér-Rao下界
\item 置信区间的构造方法
\end{itemize}

\textbf{4. 贝叶斯统计(10\%)}
\begin{itemize}
\item 共轭先验的计算
\item 贝叶斯估计量的性质
\end{itemize}

\textbf{5. 大样本理论(15\%)}
\begin{itemize}
\item 一致性、渐近正态性
\item Delta方法的应用
\item 似然比统计量的渐近分布
\end{itemize}

\subsection{解题技巧}

\textbf{1. 分布识别}
\begin{itemize}
\item 看到独立正态随机变量的平方和 → $\chi^2$分布
\item 看到标准正态与$\chi^2$的比 → $t$分布
\item 看到两个$\chi^2$分布的比 → $F$分布
\end{itemize}

\textbf{2. 最大似然估计}
\begin{itemize}
\item 写似然函数时注意取值范围
\item 对数变换简化计算
\item 检查边界值和驻点
\end{itemize}

\textbf{3. 假设检验}
\begin{itemize}
\item 明确$H_0$和$H_1$
\item 选择合适的检验统计量
\item 确定拒绝域或计算$p$值
\end{itemize}

\textbf{4. 置信区间}
\begin{itemize}
\item 找到合适的枢轴量
\item 利用对称性简化计算
\item 大样本时使用渐近正态性
\end{itemize}

\subsection{常见陷阱}

\begin{itemize}
\item 混淆样本方差$S^2$和总体方差$\sigma^2$
\item 忘记自由度的计算(特别是$t$分布和$\chi^2$分布)
\item 贝叶斯估计中先验和后验的参数更新
\item Delta方法中函数的可微性条件
\item 似然比检验中约束条件的个数
\end{itemize}

通过系统掌握这些理论和方法,并配合大量练习,可以有效提高在丘成桐数学竞赛统计方向的表现。重点要理解概念之间的联系,熟练掌握计算技巧,并能灵活应用到具体问题中。

\end{document}
