\section{Markov Chain}

\subsection{基本性质}

好的,马尔可夫链具有一些非常重要的基本性质,这些性质是我们理解和应用它的基础。以下是一些核心的性质:

\begin{enumerate}
	\item \textbf{马尔可夫性 (Markov Property) - 无记忆性}
	\begin{itemize}
		\item \textbf{定义:} 这是马尔可夫链最根本的性质。它指系统的下一个状态只依赖于其当前状态,而与它过去的状态序列无关。
		\item \textbf{数学表达:} $P(X_{n+1} = j | X_n = i, X_{n-1} = i_{n-1}, ..., X_0 = i_0) = P(X_{n+1} = j | X_n = i)$。
		\item \textbf{意义:} 大大简化了对随机过程的分析,因为我们不需要追踪系统的全部历史。
	\end{itemize}
	\item \textbf{状态空间 (State Space)}
	\begin{itemize}
		\item \textbf{定义:} 系统所有可能状态的集合。
		\item \textbf{分类:}
		\begin{itemize}
			\item \textbf{离散状态空间 (Discrete State Space):} 状态是可数的(有限个或无限可数个),例如天气(晴、阴、雨)、骰子点数(1, 2, 3, 4, 5, 6)。我们通常讨论的马尔可夫链都是离散状态的。
			\item \textbf{连续状态空间 (Continuous State Space):} 状态是连续的,例如某个物理量的精确值。这通常涉及到更复杂的马尔可夫过程。
			\item \textbf{有限状态空间 (Finite State Space):} 状态的个数是有限的。
			\item \textbf{无限状态空间 (Infinite State Space):} 状态的个数是无限的。
		\end{itemize}
	\end{itemize}
	\item \textbf{转移概率 (Transition Probabilities)}
	\begin{itemize}
		\item \textbf{定义:} 从一个状态 $i$ 转移到另一个状态 $j$ 的概率,记为 $P_{ij}$ 或 $P(X_{n+1}=j | X_n=i)$。
		\item \textbf{性质:}
		\begin{itemize}
			\item $0 \le P_{ij} \le 1$ (概率非负且不大于1)
			\item $\sum_{j \in S} P_{ij} = 1$ (对于任意状态 $i$,从 $i$ 出发到所有可能状态 $j$ 的转移概率之和为1,其中 $S$ 是状态空间)
		\end{itemize}
	\end{itemize}
	\item \textbf{时齐性 (Time Homogeneity) / 稳态转移概率 (Stationary Transition Probabilities)}
	\begin{itemize}
		\item \textbf{定义:} 转移概率 $P_{ij}$ 不随时间 $n$ 的变化而变化。也就是说,在任何时刻从状态 $i$ 转移到状态 $j$ 的概率都是相同的。
		\item \textbf{数学表达:} $P(X_{n+1}=j | X_n=i) = P(X_{m+1}=j | X_m=i)$ 对所有 $n, m$ 成立。
		\item \textbf{意义:} 大多数我们初学的马尔可夫链都假设是时齐的,这使得分析更加简单。如果转移概率随时间变化,则称为非时齐马尔可夫链。
	\end{itemize}
	\item \textbf{n步转移概率 (n-step Transition Probabilities)}
	\begin{itemize}
		\item \textbf{定义:} 从状态 $i$ 经过 $n$ 步转移到状态 $j$ 的概率,记为 $P_{ij}^{(n)}$。
		\item \textbf{计算:} 可以通过 Chapman-Kolmogorov 方程计算:$P_{ij}^{(n+m)} = \sum_{k \in S} P_{ik}^{(n)} P_{kj}^{(m)}$。特别地,状态转移矩阵的 n 次幂 $\mathbf{P}^n$ 的 $(i,j)$ 元就是 $P_{ij}^{(n)}$。
	\end{itemize}
\end{enumerate}

以下是一些更深入的、描述马尔可夫链长期行为的性质:

\begin{enumerate}
	\item \textbf{可达性 (Accessibility)}
	\begin{itemize}
		\item \textbf{定义:} 如果从状态 $i$ 出发,经过有限步(可能为0步)能够到达状态 $j$ (即 $P_{ij}^{(n)} > 0$ 对某个 $n \ge 0$ 成立),则称状态 $j$ 从状态 $i$ 可达,记为 $i \to j$。
	\end{itemize}
	\item \textbf{互通性 (Communication)}
	\begin{itemize}
		\item \textbf{定义:} 如果状态 $i$ 和状态 $j$ 相互可达 (即 $i \to j$ 且 $j \to i$),则称状态 $i$ 和状态 $j$ 是互通的,记为 $i \leftrightarrow j$。
		\item \textbf{性质:} 互通是一种等价关系,可以将状态空间划分为若干个互通类。
	\end{itemize}
	\item \textbf{不可约性 (Irreducibility)}
	\begin{itemize}
		\item \textbf{定义:} 如果一个马尔可夫链中所有状态都属于同一个互通类(即从任意状态都可以到达任意其他状态),则称该马尔可夫链是不可约的。
		\item \textbf{意义:} 不可约马尔可夫链的行为更为整体和一致。
	\end{itemize}
	\item \textbf{常返性 (Recurrence) 与瞬逝性 (Transience)}
	\begin{itemize}
		\item \textbf{常返状态:} 如果从状态 $i$ 出发,最终能以概率1返回到状态 $i$,则称状态 $i$ 是常返的。
		\item \textbf{瞬逝状态:} 如果从状态 $i$ 出发,最终返回到状态 $i$ 的概率小于1(意味着有可能永远不再返回),则称状态 $i$ 是瞬逝的。
		\item \textbf{性质:}
		\begin{itemize}
			\item 在不可约马尔可夫链中,所有状态要么都是常返的,要么都是瞬逝的。
			\item 对于有限状态空间的不可约马尔可夫链,所有状态都是常返的。
		\end{itemize}
	\end{itemize}
	\item \textbf{周期性 (Periodicity)}
	\begin{itemize}
		\item \textbf{定义:} 状态 $i$ 的周期 $d(i)$ 是使得 $P_{ii}^{(n)} > 0$ 的所有步数 $n \ge 1$ 的最大公约数。如果 $P_{ii}^{(n)} = 0$ 对所有 $n \ge 1$ 成立,则定义 $d(i) = 0$ 或 $\infty$ (不同教材定义可能略有差异,但核心是不能返回自身)。
		\item \textbf{非周期性:} 如果一个状态的周期 $d(i) = 1$,则称该状态是非周期的。
		\item \textbf{性质:} 同一个互通类中的所有状态具有相同的周期。如果一个不可约马尔可夫链的所有状态都是非周期的,则称该链是非周期的。
	\end{itemize}
	\item \textbf{平稳分布 (Stationary Distribution) / 极限分布 (Limiting Distribution)}
	\begin{itemize}
		\item \textbf{平稳分布 ($\pi$):} 一个概率分布向量 $\pi = (\pi_1, \pi_2, ...)$,其中 $\pi_j$ 是系统处于状态 $j$ 的概率,并且满足 $\sum \pi_j = 1$ 以及 $\pi \mathbf{P} = \pi$(即 $\pi_j = \sum_i \pi_i P_{ij}$)。如果系统当前的概率分布是 $\pi$,那么经过一步转移后,其概率分布仍然是 $\pi$。
		\item \textbf{极限分布:} 当时间 $n \to \infty$ 时, $P_{ij}^{(n)}$ 的极限(如果存在)。
		\item \textbf{重要定理:} 对于一个不可约、非周期、正常返(对于有限状态空间,不可约即正常返)的马尔可夫链,存在唯一的平稳分布 $\pi$,并且该平稳分布也是极限分布,即 $\lim_{n \to \infty} P_{ij}^{(n)} = \pi_j$ (与初始状态 $i$ 无关)。
		\item \textbf{意义:} 描述了马尔可夫链在长时间运行后的稳定状态。
	\end{itemize}
\end{enumerate}

这些是马尔可夫链的一些最基本的性质,理解它们对于进一步学习和应用马尔可夫链至关重要。根据研究的深入程度,还会有更多细致的性质和分类。

\subsection{验证为 Markov chain}

要证明一个随机过程是一个马尔可夫链,核心在于验证它是否满足\textbf{马尔可夫性质 (Markov Property)},并且明确其\textbf{状态空间 (State Space)} 和 \textbf{时间参数 (Time Parameter)}。

\begin{definition}[Markov Property]
A stochastic process $\{X_t: t \in T\}$ has the \textbf{Markov property} if, for any $t_1 < t_2 < \cdots < t_n$ in $T$,
\[
P\left(X_{t_n} \in A \mid X_{t_{n-1}}=x_{n-1}, \ldots, X_{t_1}=x_1\right)=P\left(X_{t_n} \in A \mid X_{t_{n-1}}=x_{n-1}\right)
\]for all $A \in \mathcal{B}$ and $x_i \in \mathbb{R}$. In other words, the future is independent of the past given the present.
\end{definition}
以下是证明一个过程是马尔可夫链的一般步骤和关键点:

\begin{enumerate}
	\item \textbf{定义状态空间 (Define the State Space} $S$\textbf{):}
	\begin{itemize}
		\item 首先,你需要清晰地定义系统所有可能的状态。状态空间 $S$ 可以是有限的,也可以是无限可数的。
		\item 确保每个状态都是互斥的,即系统在任何时刻只能处于一个状态。
		\item \textbf{例子:}
		\begin{itemize}
			\item 天气模型:$S= \{ \text{晴天, 阴天, 雨天} \}$
			\item 硬币投掷:$S=\{ \text{正面朝上, 反面朝上} \}$
			\item 排队系统中的顾客数量:$S=\{ 0,1,2,3,\dots \}$
		\end{itemize}
	\end{itemize}
	\item \textbf{定义时间参数 (Define the Time Parameter} $T$\textbf{):}
	\begin{itemize}
		\item 确定过程是离散时间的还是连续时间的。对于初学者,我们通常关注离散时间马尔可夫链,时间参数 $T=\{0,1,2,\dots\}$ 或 \{$t_0, t_1, t_2, ...$\}.
		\item $X_n$ 或 $X(t_n)$ 表示在时间步 $n$ (或时刻 $t_n$) 系统所处的状态。
	\end{itemize}
	\item \textbf{验证马尔可夫性质 (Verify the Markov Property):}
这是最关键的一步。马尔可夫性质指出,给定现在,未来与过去独立。更正式地说,对于任意时间步 $n$ 和任意状态序列 $i_0, i_1, ..., i_{n-1}, i, j \in S$:
\[
P(X_{n+1}=j \mid X_n=i, X_{n-1}=i_{n-1},...,X_0=i_0) = P(X_{n+1}=j \mid X_n=i)
\]要证明这一点,你需要:
	\begin{enumerate}
		\item 明确条件概率的含义: 理解 $P(A \mid B)$ 是在事件 $B$ 发生的条件下事件 $A$ 发生的概率。
		\item 利用过程的定义和规则: 根据你所研究的随机过程的具体规则和描述,推导其一步转移的条件概率。
		\item 展示“无记忆性”:
		\begin{itemize}
			\item 考虑在已知当前状态 $X_n=i$ 以及过去所有状态 $X_{n-1}=i_{n-1},...,X_0=i_0$ 的情况下,系统在下一个时刻 $X_{n+1}$ 转移到状态 $j$ 的概率。
			\item 然后,证明这个概率仅仅依赖于当前状态 $X_n=i$,而与过去的状态 $X_{n-1},...,X_0$ 无关。也就是说,那些过去的状态信息对于预测下一个状态是多余的。
		\end{itemize}
\textbf{如何操作?}
		\begin{itemize}
			\item \textbf{直接从定义出发:} 有些过程的定义本身就隐含了这种无记忆性。例如,如果规则明确说明“下一步只看当前”,那么马尔可夫性质就比较容易验证。
			\item \textbf{通过计算条件概率:}
			\begin{enumerate}
				\item 写出 $P(X_{n+1}=j \mid X_n=i, X_{n-1}=i_{n-1},...,X_0=i_0)$ 的表达式。
				\item 利用过程的内在机制,看看这个表达式是否可以简化为只依赖于 $i$ 和 $j$ 的形式,即 $P(X_{n+1}=j \mid X_n=i)$。
				\item 如果可以,那么马尔可夫性质就得到了证明。
			\end{enumerate}
		\end{itemize}
	\end{enumerate}
	\item \textbf{(可选,但常伴随)确定转移概率 (Determine Transition Probabilities):}
	\begin{itemize}
		\item 一旦证明了马尔可夫性质,通常下一步是确定(或者说明如何确定)一步转移概率 $P_{ij}=P(X_{n+1}=j \mid X_n=i)$。
		\item 如果这些转移概率不依赖于时间 $n$,那么这个马尔可夫链是\textbf{时齐的 (Time-Homogeneous)}。大多数基础讨论中的马尔可夫链都是时齐的。如果依赖于时间 $n$,即 $P_{ij}(n)=P(X_{n+1}=j \mid X_n=i)$,则它是\textbf{非时齐的 (Non-Time-Homogeneous)}。证明过程本身是马尔可夫链时,并不强制要求时齐性,但通常会关注这一点。
	\end{itemize}
\end{enumerate}

\textbf{一个简单的例子来说明思路:简化的天气模型}

假设我们有一个天气模型,规则如下:

\begin{itemize}
	\item 如果今天下雨,明天有 $70\%$ 的概率下雨, $30\%$ 的概率晴天。
	\item 如果今天晴天,明天有 $40\%$ 的概率下雨, $60\%$ 的概率晴天。
	\item 明天的天气只取决于今天的天气。
\end{itemize}

\textbf{证明步骤:}

\begin{enumerate}
	\item \textbf{状态空间:} $S=\{ \text{雨天 (R), 晴天 (S)} \}$
	\item \textbf{时间参数:} 离散时间,$T=\{ 0,1,2,\dots \}$ (天数)
	\item 验证马尔可夫性质:
我们要证明:$P(X_{n+1} \mid X_n, X_{n-1},...,X_0) = P(X_{n+1} \mid X_n)$
\end{enumerate}

\begin{itemize}
	\item 考虑 $P(X_{n+1}=\text{雨天} \mid X_n=\text{雨天},X_{n-1},...,X_0)$: 根据模型规则,“明天的天气只取决于今天的天气”。所以,即使我们知道 $X_{n-1},...,X_0$ 的具体天气情况,只要 $X_n=\text{雨天}$,下一天是雨天的概率就是 0.7。因此,$P(X_{n+1}=\text{雨天} \mid X_n=\text{雨天},X_{n-1},...,X_0)=0.7$。同时,$P(X_{n+1}=\text{雨天} \mid X_n=\text{雨天})=0.7$。两者相等。
	\item 类似地,可以对所有可能的当前状态和下一个状态进行验证:
	\begin{enumerate}
		\item $P(X_{n+1}=\text{晴天} \mid X_n=\text{雨天},...) = P(X_{n+1}=\text{晴天} \mid X_n=\text{雨天})=0.3$
		\item $P(X_{n+1}=\text{雨天} \mid X_n=\text{晴天},...) = P(X_{n+1}=\text{雨天} \mid X_n=\text{晴天})=0.4$
		\item $P(X_{n+1}=\text{晴天} \mid X_n=\text{晴天},...) = P(X_{n+1}=\text{晴天} \mid X_n=\text{晴天})=0.6$
	\end{enumerate}
\end{itemize}

由于对于所有情况,未来状态的概率只依赖于当前状态,所以该过程满足马尔可夫性质。

\begin{enumerate}
	\item \textbf{转移概率(时齐):}
	\begin{itemize}
		\item $P(雨天 \mid 雨天)=P_{RR}=0.7$
		\item $P(晴天 \mid 雨天)=P_{RS}=0.3$
		\item $P(雨天 \mid 晴天)=P_{SR}=0.4$
		\item $P(晴天 \mid 晴天)=P_{SS}=0.6$
这些概率不随天数 $n$ 变化,所以这是一个时齐马尔可夫链。
	\end{itemize}
\end{enumerate}

\textbf{总结关键:}

证明的核心是严格地、形式化地展示出“给定现在,未来与过去无关”。这通常需要仔细分析随机过程的规则和定义,并运用条件概率的知识。对于一些复杂的过程,这可能涉及到更多的数学推导。
