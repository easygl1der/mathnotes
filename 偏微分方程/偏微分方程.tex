\documentclass{mynote}


\begin{document}

\pagenumbering{Roman} % 设置页码格式为罗马数字
\setcounter{page}{1} % 将页码计数器设置为1
\tableofcontents % 生成目录
\newpage % 插入新页
\setcounter{page}{1} % 将页码计数器重置为1
\pagenumbering{arabic} % 设置页码格式为阿拉伯数字


\chapter{偏微分方程}

\section{记号说明}
对于$u(x)=u(x_1,x_2,\cdots,x_n)$,我们定义:

\textbf{梯度}:
\[
    Du=\nabla u=\left(\partial_{1} u, \partial_{2} u, \cdots, \partial_{n} u\right)
\]
\textbf{Hessian矩阵}:
\[
    D^2u=\begin{pmatrix}
        \partial_{11} u & \partial_{12} u & \cdots & \partial_{1n} u \\
        \partial_{21} u & \partial_{22} u & \cdots & \partial_{2n} u \\
        \vdots & \vdots & \ddots & \vdots \\
        \partial_{n1} u & \partial_{n2} u & \cdots & \partial_{nn} u
    \end{pmatrix}
\]
设$\boldsymbol{F}=(F_1,F_2,\cdots,F_n):\Omega\to \mathbb{R}^n$,则$\boldsymbol{F}$的\textbf{散度}为
\[
    \operatorname{div} \boldsymbol{F}=\sum_{i=1}^n \partial_{i} F_i
\]
于是
\[
    \Delta u=\operatorname{div} Du=\mathrm{tr}(D^2u)
\]

\section{基础公式}
\begin{theorem}{Gauss-Green公式}
    假设 $\Omega$ 是 $\mathbf{R}^n$ 中的一个有界开集,且 $\partial \Omega \in C^1$ .如果 $\boldsymbol{F}=\left(F_1, F_2, \cdots, F_n\right): \bar{\Omega} \rightarrow \mathbf{R}^n$ 属于 $C^1(\Omega) \cap C(\bar{\Omega})$ ,则
    \begin{equation}\label{Gauss-Green}
        \int_{\Omega} \operatorname{div} \boldsymbol{F} \mathrm{d} x=\int_{\partial \Omega} \boldsymbol{F} \cdot \boldsymbol{n} \mathrm{d} S(x),
    \end{equation}
    其中 $\boldsymbol{n}$ 为 $\partial \Omega$ 的单位外法向量.
\end{theorem}

\begin{corollary}
    \begin{enumerate}
        \item 如果 $u, v \in C^1(\Omega) \cap C(\bar{\Omega})$ ,则
        \begin{equation}
        \int_{\Omega} \partial_i u \cdot v \mathrm{~d} x=-\int_{\Omega} u\cdot \partial_i v \mathrm{~d} x+\int_{\partial \Omega} u v n_i \mathrm{~d} S(x),
        \label{eq:integration_by_parts_1}
        \end{equation}
        其中 $n_i$ 是 $\partial \Omega$ 的单位外法向量 $\boldsymbol{n}$ 的第 $i$ 个分量;
        \item 如果 $u \in C^2(\Omega) \cap C^1(\bar{\Omega})$ ,则
        \begin{equation}
        \int_{\Omega} \Delta u \mathrm{~d} x=\int_{\partial \Omega} \frac{\partial u}{\partial \boldsymbol{n}} \mathrm{~d} S(x) ;
        \label{eq:integration_by_parts_2}
        \end{equation}
        \item 如果 $u \in C^1(\Omega) \cap C(\bar{\Omega}), v \in C^2(\Omega) \cap C^1(\bar{\Omega})$ ,则
        \begin{equation}
        \int_{\Omega} D u \cdot D v \mathrm{~d} x=-\int_{\Omega} u \Delta v \mathrm{~d} x+\int_{\partial \Omega} u \frac{\partial v}{\partial \boldsymbol{n}} \mathrm{~d} S(x)
        \label{eq:integration_by_parts_3}
        \end{equation}
        \item 如果 $u, v \in C^2(\Omega) \cap C^1(\bar{\Omega})$ ,则
        如果 $u, v \in C^2(\Omega) \cap C^1(\bar{\Omega})$ ,则
        \begin{equation}
        \int_{\Omega}(u \Delta v-v \Delta u) \mathrm{d} x=\int_{\partial \Omega}\left(u \frac{\partial v}{\partial \boldsymbol{n}}-v \frac{\partial u}{\partial \boldsymbol{n}}\right) \mathrm{d} S(x)
        \label{eq:integration_by_parts_4}
        \end{equation}
    \end{enumerate}
\end{corollary}

\begin{proof}
    \begin{enumerate}
        \item 取$\boldsymbol{F}=(0,\cdots,0,uv,0,\cdots,0)$,代入公式\ref{Gauss-Green}即可
        \item 取$\boldsymbol{F}=Du$,代入公式\ref{Gauss-Green}即可
        \item 取$\boldsymbol{F}=u\cdot Dv$,注意到
        \[
            \operatorname{div}(u\cdot Dv)=\operatorname{div}(\cdots,u\cdot \partial_i v,\cdots)=\sum_{i=1}^n \partial_i (u\cdot \partial_i v)=\sum_{i=1}^n (\partial_i u\partial_i v+u\cdot \partial_i^2 v)=Du\cdot Dv+u\Delta v
        \]
        代入公式\ref{Gauss-Green}即可。
        \item 取$\boldsymbol{F}=(\cdots,u\cdot \partial_i v-v\cdot \partial_i u,\cdots)$,代入公式\ref{Gauss-Green}即可。
    \end{enumerate}
\end{proof}

利用分部积分的思想:
\begin{equation}\label{eq:integration_by_parts_5}
    \int_{\Omega}u \Delta u \mathrm{d} x=\sum_{i=1}^n\int_{\Omega} u\cdot \partial_i^2 u \mathrm{d} x=-\sum_{i=1}^n \int_{\Omega}(\partial_i u)^2\mathrm{d} x=-\int_{\Omega}|Du|^2\mathrm{d} x
\end{equation}
\section{调和函数}

\begin{theorem}{平均值公式}\label{average_value_formula}
    假设 $u \in C^2(\Omega)$ 是 $\Omega$ 上的调和函数,则对于任意的球 $B(x, r) \subset \Omega$ ,有

$$
u(x)=\dashint_{\partial B(x, r)} u(y) \mathrm{d} S(y)=\dashint_{B(x, r)} u(y) \mathrm{d} y,
$$
其中积分符号 $\dashint$ 表示求平均值,即
$$
\begin{aligned}
\dashint_{\partial B(x, r)} u(y) \mathrm{d} S(y) & =\frac{1}{n \alpha(n) r^{n-1}} \int_{\partial B(x, r)} u(y) \mathrm{d} S(y), \\
\dashint_{B(x, r)} u(y) \mathrm{d} y & =\frac{1}{\alpha(n) r^n} \int_{B(x, r)} u(y) \mathrm{d} y,
\end{aligned}
$$
这里$\alpha(n)=\frac{2\pi^{n/2}}{n \Gamma(n/2)}$是$n$维球的体积。
\end{theorem}
\begin{proof}
    注意到
    \[
        u(x)=\lim_{r\to 0}\dashint_{\partial B(x, r)} u(y) \mathrm{d} S(y)\overset{\text{why?}}{=}\dashint_{\partial B(x, r)} u(y) \mathrm{d} S(y)
    \]
    以及
    \[
        \int_{B(x,r)}u(y)dy=\int_0^r\int_{\partial B(x,t)}u(y)dS(y)dt
    \]
\end{proof}
上述证明中的一个核心技巧就是,记
\[
    \phi(r)=\dashint_{\partial B(x, r)} u(y) \mathrm{d} S(y)=\dashint_{\partial B(0, 1)} u(x+rz) \mathrm{d} S(z)
\]
其中任意给定$x\in\Omega,B(x,r)\subset\Omega$,于是
\begin{align*}
    \phi'(r)&=\partial_r\dashint_{\partial B(0, 1)} u(x+rz) \mathrm{d} S(z)\\
    &=\dashint_{\partial B(0, 1)} \partial_r u(x+rz) \mathrm{d} S(z)\qquad r\text{存在于每个分量上}\\
    &=\dashint_{\partial B(0, 1)}  Du(x+rz) \cdot z \mathrm{d} S(z)\\
    &=\dashint_{\partial B(x, r)}  Du(y) \cdot \frac{y-x}{r} \mathrm{d} S(y)\\
    &=\dashint_{\partial B(x, r)}  Du(y) \cdot \vec{\boldsymbol{n}} \mathrm{d} S(y)\\
    &=\dashint_{\partial B(x, r)}  \frac{\partial u}{\partial \boldsymbol{n}} \mathrm{d} S(y)\\
    &=\frac{r}{n}\dashint_{B(x,r)} \Delta u(y) \mathrm{d} y=0
\end{align*}
由$u(x)$的连续性,$\phi(r)$在$r=0$处连续,因此$\phi(r)=\lim_{t\to 0}\phi(t)=u(x)$.

\begin{theorem}\label{harmonic_function_theorem}
    假设 $u \in C^2(\Omega)$ 满足,对于任意 $B(x, r) \subset \Omega$ ,
$$
u(x)=\dashint_{\partial B(x, r)} u(y) \mathrm{d} S(y)
$$
则 $u$ 是调和函数.
\end{theorem}

\begin{theorem}{Harnack 不等式}\label{Harnack_inequality}
    假设 $u$ 是半径为 $R$ 的球 $B_R$ 上的任意非负调和函数,则对任意 $r \in(0, R)$ ,成立不等式
$$
\sup _{B_r} u \leq\left(\frac{R+r}{R-r}\right)^n \inf _{B_r} u,
$$
其中 $B_r$ 是以 $r$ 为半径,与 $B_R$ 同心的球.
\end{theorem}

\begin{theorem}{弱极值原理}\label{weak_maximum_principle}
    若调和函数 $u$ 在 $\bar{\Omega}$ 上连续,则 $\max _{\bar{\Omega}} u=\max _{\partial \Omega} u$ 。
\end{theorem}
\begin{proof}
    令 $w=u+\varepsilon e^{x_1}$ ,易见
    \begin{enumerate}
        \item $w>u$ ,因而有 $\max _{\bar{\Omega}} u \leq \max _{\bar{\Omega}} w$ 。
        \item $\Delta w=\varepsilon e^{x_1}>0, ~ w$ 的最大值只可能在边界上达到,因而有 $\max _{\bar{\Omega}} w=\max _{\partial \Omega} w$ 。
        \item $\max _{\partial \Omega} w \leq \max _{\partial \Omega} u+\max _{\partial \Omega} \varepsilon e^{x_1}=\max _{\partial \Omega} u+\varepsilon \max _{\partial \Omega} e^{x_1}$
    \end{enumerate}
    综合上面三点观察,就得到不等式 $\max _{\bar{\Omega}} u \leq \max _{\partial \Omega} u+\varepsilon \max _{\partial \Omega} e^{x_1}$ 。

    令 $\varepsilon \rightarrow 0$ 就有 $\max _{\bar{\Omega}} u \leq \max _{\partial \Omega} u$ ,但是 $\partial \Omega \subset \bar{\Omega}$ ,又应该有 $\max _{\partial \Omega} u \leq \max _{\bar{\Omega}} u$ ,所以必须有 $\max _{\bar{\Omega}} u=\max _{\partial \Omega} u$ ,这就证明了结论。
\end{proof}
\begin{theorem}{强极值原理}\label{strong_maximum_principle}
    假设 $\Omega$ 是 $\mathbf{R}^n$ 上的有界开集,$u \in$ $C^2(\Omega) \cap C(\bar{\Omega})$ 是 $\Omega$ 上的调和函数,则
    \begin{enumerate}
        \item\label{strong_maximum_principle_1} $u(x)$ 在 $\bar{\Omega}$ 上的最大(小)值一定在边界 $\partial \Omega$ 上达到,即
            $$
            \max _{\bar{\Omega}} u=\max _{\partial \Omega} u
            $$
        \item\label{strong_maximum_principle_2} 如果 $\Omega$ 是连通的,且存在 $x_0 \in \Omega$ 使得调和函数 $u(x)$ 在 $x_0$点达到 $u(x)$ 在 $\bar{\Omega}$ 上的最大(小)值,则 $u$ 在 $\bar{\Omega}$ 上是常数.
    \end{enumerate}
\end{theorem}

强极值原理\ref{strong_maximum_principle}的一个应用是:

对于Dirichlet问题
\begin{equation}\label{Dirichlet_problem}
    \begin{cases}
        -\Delta u=f, & x\in \Omega\\
        u|_{\partial \Omega}=g, & x\in \partial \Omega
    \end{cases}
\end{equation}
其中$\Omega$是$\mathbb{R}^n$中的有界开集,$g\in C(\partial \Omega),f\in C(\Omega)$,则问题\ref{Dirichlet_problem}的解是唯一的。

这是因为如果存在两个解$u_1,u_2$,则$w=u_1-u_2$是调和函数,满足
\[
    \begin{cases}
        -\Delta w=0, & x\in \Omega\\
        w|_{\partial \Omega}=0, & x\in \partial \Omega
    \end{cases}
\]  
根据强极值原理,对于$x\in \Omega$,$w(x)\le 0$. 注意到$-w$满足同样的初值问题,于是对于$x\in \Omega$,$w(x)\ge 0$. 因此$w\equiv 0$,即$u_1\equiv u_2$.

\begin{theorem}\label{Cauchy_estimate}
    假设 $u$ 是 $\Omega$ 上的调和函数,则对于任意球 $B(x, r) \subset$ $\Omega$ ,任意阶数为 $k$ 的多重指标 $\alpha$ ,估计\footnote{类似复变函数中的Cauchy估计}
$$
\left|D^\alpha u(x)\right| \leq \frac{C_k}{r^{n+k}} \int_{B(x, r)}|u(y)| \mathrm{d} y
$$
成立,其中
$$
C_0=\frac{1}{\alpha(n)}, \quad C_k=\frac{(n+k)^{n+k}(n+1)^k}{\alpha(n)(n+1)^{n+1}} \quad(k=1,2, \cdots) .
$$
\end{theorem}
作为推论,我们有(这跟复变函数中定理发展络如出一辙)
\begin{theorem}{Liouville定理}
    假设 $u$ 是 $\mathbb{R}^n$ 上的调和函数,则 $u$ 必为常数函数。
\end{theorem}
\begin{proof}
    设 $|u| \leq M$ .固定 $x \in \mathbf{R}^n$ .对任意 $r>0$ ,在球 $B(x, r)$ 上利用定理\ref{Cauchy_estimate}当 $k=1$ 时的结论,则
$$
|D u(x)| \leq \frac{n C_1}{r^{n+1}} \int_{B(x, r)}|u(y)| \mathrm{d} y \leq \frac{n C_1 \alpha(n)}{r} M .
$$
令 $r \rightarrow+\infty$ ,我们有
$$
D u(x) \equiv 0, \quad x \in \mathbf{R}^n .
$$
因此 $u$ 是常数.
\end{proof}
更强的,使用Harnack不等式,定理\ref{Cauchy_estimate}只需要函数$u$有上界或下界即可.
\begin{theorem}
    假设 $u$ 是 $\Omega$ 上的调和函数,则 $u$ 是 $\Omega$ 上的解析函数.
\end{theorem}
\begin{theorem}
    假设 $u$ 是 $\mathbf{R}^n$ 的单位球 $B_1$ 上的调和函数,则
$$
H(r)=\int_{\partial B_r} u^2 \mathrm{~d} S, \quad D(r)=r^2 \int_{B r}|\nabla u|^2 \mathrm{~d} y
$$
是 $r$ 的单调递增函数,其中 $B_r$ 是与 $B_1$ 同心,以 $r$ 为半径的球.
\end{theorem}
\begin{proof}
    直接暴力求导!(过于暴力)
\end{proof}

\section{基本解与Green函数}

\begin{proposition}
    如果一个在$\mathbb{R}^n$上的调和函数是径向对称的,则它必为常数函数。
\end{proposition}
假设$u(x)=v(r)$,其中$r=|x|$,则
\[
    0=\Delta u =v''(r)+\frac{n-1}{r}v'(r)
\]
求解这个微分方程得到
\[
    v(r)=\left\{\begin{array}{cl}
        b \ln r+c, & n=2 \\
        \frac{b}{r^{n-2}}+c, & n \geq 3
        \end{array}\right.
\]
但当$b\ne 0,r=0$时,$v(r)$在原点无定义。也就是Laplace方程在\textbf{全空间}没有径向对称的解。

尽管如此,由于$v(r)$仅仅在$r=0$时不满足调和性,我们称$v(r)$为\textbf{基本解}。
\begin{definition}
    对于$x\in \mathbb{R}^n$,如果$u(x)$满足
    \[
        \Gamma(x)= \begin{cases}-\frac{1}{2 \pi} \ln |x|, & n=2, \\ \frac{1}{n(n-2) \alpha(n)|x|^{n-2}}, & n \geq 3\end{cases}
    \]
    则称$\Gamma(x)$为\textbf{基本解}。
\end{definition}
关于$\Gamma(x)$的系数选取,我们注意到:对于$x\ne 0$
\[
    |D \Gamma(x)| \leq \frac{C}{|x|^{n-1}}, \quad\left|D^2 \Gamma(x)\right| \leq \frac{C}{|x|^n}
\]
其中$C=C(n)$是仅依赖于$n$的常数。
\begin{theorem}
    假设 $f(x) \in C_0^2\left(\mathbf{R}^n\right)$ ,则
$$
u(x)=(\Gamma*f)(x)+C=\int_{\mathbf{R}^n} \Gamma(x-y) f(y) \mathrm{d} y+C
$$
是位势方程$-\Delta u=f$在全空间 $\mathbf{R}^n$ 上所有的有界解.
\end{theorem}

我还是没有理解Green函数的定义...

\section{极值原理和最大模估计}
我们讨论更一般的方程:
\begin{equation}\label{eq:general_equation}
    \cal{L}u=-\Delta u+c(x)u=f(x),\quad x\in \Omega
\end{equation}
$\Omega$都是$\mathbb{R}^n$中的有界开集. 假定$c(x)\ge 0$且$f(x)< 0$.

方程\ref{eq:general_equation}的极值原理只需要把Laplace方程的极值原理中的\textit{最大值}换成\textit{非负最大值}即可.

关于最大模估计,我们考虑位势方程的Dirichlet问题:
\begin{equation}\label{eq:potential_equation}
    \begin{cases}
        \Delta u=f(x), & x\in \Omega\\
        u|_{\partial \Omega}=g, & x\in \partial \Omega
    \end{cases}
\end{equation}
若$u\in C^2(\Omega)\cap C(\bar{\Omega})$,则
\[
    \max_{\bar{\Omega}}|u|\le \max_{\partial \Omega}|g|+C\cdot \sup_{\Omega}|f|
\]
其中$C=C(n,\sup_{x,u\in \Omega}|x-y|)$.

\section{能量模估计}

考虑$n$维位势方程的Dirichlet问题:
\begin{equation}\label{eq:potential_equation_2}
    \begin{cases}
        -\Delta u+c(x)u=f(x), & x\in \Omega\\
        u|_{\partial \Omega}=0, & x\in \partial \Omega
    \end{cases}
\end{equation}
利用Gauss-Green公式\ref{Gauss-Green},我们有
\begin{theorem}
    假设 $c(x) \geq c_0>0, u \in C^2(\Omega) \cap C^1(\bar{\Omega})$ 是 Dirichlet问题\ref{eq:potential_equation_2}的解,则
$$
\int_{\Omega}|D u(x)|^2 \mathrm{~d} x+\frac{c_0}{2} \int_{\Omega}|u(x)|^2 \mathrm{~d} x \leq M \int_{\Omega}|f(x)|^2 \mathrm{~d} x,
$$
其中 $M$ 是仅依赖于 $c_0$ 的常数.
\end{theorem}
\begin{proof}
    这是显然的,因为
    \[
        \underbrace{\int_{\Omega}|Du(x)|^2\mathrm{d}x}_{=\displaystyle \int_{\Omega}|Du|^2\mathrm{d}x}+\underbrace{\int_{\Omega}c(x)u^2\mathrm{d}x}_{\displaystyle \ge c_0\int_{\Omega}u^2\mathrm{d}x}=\int_{\Omega}fu\mathrm{d}x\overset{Cauchy\text{ 不等式}}\le \frac{c_0}{2} \int_{\Omega} u^2 \mathrm{~d} x+\frac{1}{2 c_0} \int_{\Omega} f^2 \mathrm{~d} x .
    \]
\end{proof}
\begin{lemma}{Friedrichs不等式}\label{Friedrichs_inequality}
    假设$u\in C^1_0(\Omega)$,则
    \[
        \int_{\Omega}|u(x)|^2\mathrm{d}x\le 4d^2\int_\Omega|Du(x)|^2\mathrm{d}x
    \]
    其中$d$是$\Omega$的直径。
\end{lemma}
该证明也利用了经典的思想.
\begin{proof}
    做一个边长为$2d$且平行于坐标轴的$n$维正方体含住$\Omega$,不妨设为
    \[
        Q=\{x(x_1,\cdots,_n)|0\le x_i\le 2d,i=1,\cdots,n\}
    \]
    令$\tilde{u}=u \mathbb{1}_{Q}\in C_0^1(Q)$,且
    \[
        \tilde{u}(x_1,...,x_n)=\int_0^{x_1}\partial_{1}\tilde{u}(\xi,x_2,...,x_n)\mathrm{d}\xi
    \]
    利用Schwarz不等式,我们有
    \[
        \tilde{u}^2(x) \leq x_1 \int_0^{x_1}\left|\partial_\xi \tilde{u}\left(\xi, x_2, \cdots, x_n\right)\right|^2 \mathrm{~d} \xi\leq 2 d \int_0^{2 d}\left|\partial_1 \tilde{u}\left(x_1, x_2, \cdots, x_n\right)\right|^2 \mathrm{~d} \xi .
    \]
    两端在$Q$上积分,得到
    \[
        \int_Q \tilde{u}^2(x) \mathrm{d} x \leq 4 d^2 \int_Q\left|\tilde{u}_{x_1}\right|^2 \mathrm{~d} x \leq 4 d^2 \int_Q\underbrace{\sum_{i=1}^n\left|\partial_i \tilde{u}\right|^2}_{=|D \tilde{u}|^2} \mathrm{~d} x
    \]
\end{proof}
利用上述不等式,我们可以证明下面结论.
\begin{theorem}
    假设 $c(x) \geq 0$ .如果 $u \in C^2(\Omega) \cap C^1(\bar{\Omega})$ 是问题\ref{eq:potential_equation_2}的解,则
$$
\int_{\Omega}|D u(x)|^2 \mathrm{~d} x+\int_{\Omega}|u(x)|^2 \mathrm{~d} x \leq M \int_{\Omega}|f(x)|^2 \mathrm{~d} x
$$
其中 $M$ 是仅依赖于 $\Omega$ 的直径的常数.
\end{theorem}

\section{练习}

\begin{exercise}
    利用推导Laplace方程的思想推导极小曲面方程.
\end{exercise}
极小曲面方程为
\[
    \operatorname{div}\left(\frac{D u}{\left(1+|D u|^2\right)^{1/2}}\right)=0
\]
给定$M_g=\{v\in C(\bar{\Omega}),v|_{\partial \Omega}=g\}$,求$u\in M_g$,使得$J(u)=\min_{v\in M_g}J(v)$,其中
\[
    J(v)=\iint_\Omega \sqrt{1+|Dv|^2}\mathrm{d}x
\]
对于任意的$\varphi\in C_0^1(\Omega)$,设$f(\varepsilon)=J(v+\varepsilon\varphi)$,我们想要$f(\varepsilon)\ge f(0)$,这意味着$f'(0)=0$.

直接展开:
\[
    f'(\varepsilon)=\iint_\Omega \frac{1}{\sqrt{1+|D(v+\varepsilon\varphi)|^2}}(Dv\cdot D\varphi+2\varepsilon |D\varphi|^2)\mathrm{d}x
\]
于是
\[
    f'(0)=\iint_\Omega \frac{1}{\sqrt{1+|Dv|^2}}(Dv\cdot D\varphi)\mathrm{d}x=0
\]
由于$\varphi\in C_0^1(\Omega)$,所以$D\varphi|_{\partial \Omega}=0$,于是分部积分得到
\[
    \iint_\Omega \left(-\operatorname{div}\frac{Dv}{\sqrt{1+|Dv|^2}}\right)\varphi\mathrm{d}x=0
\]
由$\varphi$的任意性,我们得到
\[
    \operatorname{div}\frac{Dv}{\sqrt{1+|Dv|^2}}=0
\]
即$v$是极小曲面. 得到极小曲面方程.

\begin{exercise}
    构造 $\mathbb{R}^n$ 上所有二次调和多项式组成的线性空间.
\end{exercise}
对于一个$n$维二次多项式
\[
    P(x)=x^\top A x +b^\top x +c
\]
我们要求$P$是调和的,即$\Delta P=0$,这意味着
\[
    \mathrm{tr}(A)=0
\]
于是该线性空间为$n$维二次多项式空间:
\[
    \{P(x)=x^\top A x +b^\top x +c|\mathrm{tr}(A)=0\}
\]
\begin{exercise}
    证明定理2.1
\end{exercise}
显然。
\begin{exercise}\label{exercise:average_value_formula}
    仿照平均值公式的的推导证明:当 $n \geq 3$ 时,对于第一边值问题
$$
\begin{cases}-\triangle u=f, & x \in B(0, r), \\ u=g, & x \in \partial B(0, r)\end{cases}
$$
在 $C^2(B(0, r)) \cap C^1(\bar{B}(0, r))$ 上的解 $u(x)$ ,成立
$$
\begin{aligned}
u(0)= & \frac{1}{n \alpha(n) r^{n-1}} \int_{\partial B(0, r)} g(x) \mathrm{d} S(x) \\
& +\frac{1}{n(n-2) \alpha(n)} \int_{B(0, r)}\left(\frac{1}{|x|^{n-2}}-\frac{1}{r^{n-2}}\right) f(x) \mathrm{d} x .
\end{aligned}
$$
\end{exercise}
记$\phi(x)=\dashint_{\partial B(0,r)}u(y)\mathrm{d}S(y)$,则
\[
    \phi'(r)=\frac{r}{n}\dashint_{B(0,r)}\Delta u(y)\mathrm{d}y=-\frac{1}{n\alpha(n)r^{n-1}}\int_{B(0,r)}f(y)\mathrm{d}y
\]
于是
\begin{align*}
    \phi(0) &= u(0)=\phi(r)-\int_0^r\phi'(t)\mathrm{d}t\\
    &= \dashint_{\partial B(0,r)}u(y)\mathrm{d}S(y)+\int_0^r\frac{1}{n\alpha(n)t^{n-1}}\int_{B(0,t)}f(y)\mathrm{d}y\mathrm{d}t\\
    & = \dashint_{\partial B(0,r)}g(y)\mathrm{d}S(y)+\frac{1}{\alpha(n)}\int_0^r\frac{1}{t^{n-1}}\int_{B(0,t)}f(y)\mathrm{d}y\mathrm{d}t\\
    & = \dashint_{\partial B(0,r)}g(y)\mathrm{d}S(y)+\frac{1}{n\alpha(n)}\int_0^r\frac{1}{t^{n-1}}\int_0^t\int_{\partial B(0,\rho)}f(y)\mathrm{d}y\mathrm{d}\rho\mathrm{d}t\\
    & \overset{\text{Fubini}}{=} \dashint_{\partial B(0,r)}g(y)\mathrm{d}S(y)+\frac{1}{n\alpha(n)}\int_0^r\int_\rho^r\frac{1}{t^{n-1}}\int_{\partial B(0,\rho)}f(y)\mathrm{d}y\mathrm{d}t\mathrm{d}\rho\\
    & = \dashint_{\partial B(0,r)}g(y)\mathrm{d}S(y)+\frac{1}{n\alpha(n)}\int_0^r\frac{1}{n-2}\left(  \frac{1}{\rho^{n-2}}-\frac{1}{r^{n-2}} \right)\int_{\partial B(0,\rho)}f(y)\mathrm{d}y\mathrm{d}\rho\\
    & = \dashint_{\partial B(0,r)}g(y)\mathrm{d}S(y)+\frac{1}{n(n-2)\alpha(n)}\int_{B(0,r)}\left(  \frac{1}{|x|^{n-2}}-\frac{1}{r^{n-2}} \right)f(x)\mathrm{d}x
\end{align*}


\begin{exercise}
    仿照平均值公式的的推导证明:当 $n=2$ 时,对于第一边值问题
$$
\begin{cases}-\triangle u=f, & x \in B(0, r), \\ u=g, & x \in \partial B(0, r)=C(0, r)\end{cases}
$$
在 $C^2(B(0, r)) \cap C^1(\bar{B}(0, r))$ 的解 $u(x)$ ,成立
$$
\begin{aligned}
u(0)= & \frac{1}{2 \pi r} \int_{C(0, r)} g(x) \mathrm{d} l \\
& +\frac{1}{2 \pi} \int_{B(0, r)}(\ln r-\ln |x|) f(x) \mathrm{d} x,
\end{aligned}
$$
其中 $\mathrm{d} l$ 为弧长微分.此时 $C(0, r)$ 是环绕圆盘 $B(0, r)$ 的圆周.
\end{exercise}

证明与习题\ref{exercise:average_value_formula}如出一辙,这里略去。

\begin{exercise}
    若 $v \in C^2(\Omega)$ 满足
$$
-\Delta v \leq 0, \quad x \in \Omega
$$
则称 $v$ 在 $\Omega$ 上是下调和的.
\begin{enumerate}
    \item 证明:对于任意球 $B(x, r) \subset \Omega$ ,成立
$$
v(x) \leq \int_{B(x, r)} v(y) \mathrm{d} y
$$
    \item 证明: $\max _{\bar{\Omega}} v(x)=\max _{\partial \Omega} v(x)$ .
    \item 设 $\phi: \mathbf{R} \rightarrow \mathbf{R}$ 是光滑凸函数,且 $u$ 是 $\Omega$ 上的调和函数.证明:$v=\phi(u)$ 是 $\Omega$ 上的下调和函数.
    \item 设 $u$ 是 $\Omega$ 上的调和函数.证明:$v=|D u|^2$ 是 $\Omega$ 上的下调和函数.
\end{enumerate}
\end{exercise}

\begin{enumerate}
    \item 参考定理\ref{average_value_formula}的证明,记
    \[
        \phi(r)=\dashint_{B(x,r)}v(y)\mathrm{d}y,\quad \phi'(r)=\dashint_{B(x,r)}\Delta v(y)\mathrm{d}y\ge 0
    \]
    则
    \[
        v(x)=\phi(0)\le \phi(r)=\dashint_{B(x,r)}v(y)\mathrm{d}y
    \]
    \item 仿照定理\ref{weak_maximum_principle}的证明,令$w=v+\varepsilon e^{x_1}$,其中$\varepsilon>0$,则
    \[
        \Delta w=\Delta v+\varepsilon e^{x_1}\ge 0
    \]
    于是$w$的最大值只可能在边界上达到,即$\max_{\bar{\Omega}}w=\max_{\partial \Omega}w$,再由于$u\in C^2(\Omega)$,令$\varepsilon\to 0$,则有$\max_{\bar{\Omega}}v=\max_{\partial \Omega}v$。
    \item 直接计算
    \[
        \Delta v(x)=\sum_{i=1}^n\partial_{i}^2v(x)=\sum_{i=1}^n\partial_{i}^2\phi(u(x))=\sum_{i=1}^n\phi''(u(x))(\partial_{i}u(x))^2+\phi'(u(x))\underbrace{\Delta u(x)}_{=0}\ge 0
    \]
    \item 直接计算
    \[
        v(x)=|Du|^2=\sum_{i=1}^n\left(\partial_{i}u(x)\right)^2
    \]
    于是
    \[
        \Delta v(x)=\sum_{i=1}^n\partial_{i}^2v(x)=2\sum_{i,j=1}^n(\partial_{ij}u)^2+2\sum_{i=1}^n\partial_i\underbrace{\sum_{j=1}^n\partial_j^2u}_{=\operatorname{div}(\Delta u)=0}=2\sum_{i,j=1}^n(\partial_{ij}u)^2\ge 0
    \]
\end{enumerate}
\begin{exercise}{Harnack 定理}
    假设 $\left\{u_n\right\} \subset C(\bar{\Omega}) \cap C^2(\Omega)$ 是 $\Omega$ 上的调和函数列.如果 $\left\{u_n\right\}$ 在 $\partial \Omega$ 上一致收敛,则 $\left\{u_n\right\}$ 在 $\bar{\Omega}$ 上一致收敛,且收敛于一个调和函数.(提示:利用极值原理和平均值公式)
\end{exercise}
已知$\forall n,\Delta u_n=0,\lim_{m,n\to \infty}\sup_{x\in\partial \Omega}|u_n(x)-u_m(x)|=0$,由于$u_n-u_m$也是调和函数,所以
\[
    \lim_{m,n\to \infty}\sup_{x\in\bar{\Omega}}|u_n(x)-u_m(x)|\le \lim_{m,n\to \infty}\sup_{x\in\partial \Omega}|u_n(x)-u_m(x)|=0
\]
即$\left\{u_n\right\}$在$\bar{\Omega}$上一致收敛.

由平均值公式\ref{average_value_formula},对于任意$B(x,r)\subset \Omega$,我们有
\[
    u_n(x)=\dashint_{\partial B(x,r)}u_n(y)\mathrm{d}S(y)
\]
由一致收敛性,两边令$n\to \infty$,得到
\[
    u(x)=\dashint_{\partial B(x,r)}u(y)\mathrm{d}S(y)
\]
根据定理\ref{harmonic_function_theorem}可知,$u$是调和函数.

\begin{exercise}{Schwarz 反射定理}
    记上半球
$$
B^{+}=\left\{x=\left(x_1, x_2, \cdots, x_n\right) \in B(0,1) \mid x_n>0\right\},
$$
假设 $u$ 是上半球 $B^{+}$上的调和函数且在边界 $\left\{x \in \partial B^{+} \mid x_n=0\right\}$ 上满足 $u=0$ .令
$$
v(x)= \begin{cases}u\left(x_1, x_2, \cdots, x_n\right), & x_n \geq 0, \\ -u\left(x_1, x_2, \cdots,-x_n\right), & x_n<0 .\end{cases}
$$
证明:$v$ 是球 $B(0,1)$ 上的调和函数.(提示:利用平均值公式的推导)
\end{exercise}

显然.

\begin{exercise}
    设 $\Omega$ 是 $\mathbf{R}^n$ 的一个开集.如果原点 $0 \notin \Omega$ ,我们记 $x^*=\frac{x}{|x|^2}$ , $\Omega^*=\left\{x^* \mid x \in \Omega\right\}$ .假设 $u$ 是在 $\Omega$ 上的一个函数,并定义 $\Omega^*$ 上的函数 $K[u](x)=|x|^{2-n} u\left(x^*\right), x \in \Omega^*$ 为函数 $u$ 的 Kelvin 变换.证明: $u(x)$ 是 $\Omega$ 上的调和函数当且仅当 $K[u]$ 是 $\Omega^*$ 上的调和函数.
\end{exercise}

爆算$\Delta K[u]$.

\begin{exercise}
    设 $u(x)$ 是球 $B(0, R)$ 上的调和函数,且在 $\bar{B}(0, R)$ 上连续.又设
$$
M=\int_{B(0, R)} u^2(x) \mathrm{d} x
$$
试证:
\begin{enumerate}   
    \item $|u(0)| \leq\left[\frac{M}{\alpha(n) R^n}\right]^{\frac{1}{2}}$ ;
    \item $|u(x)| \leq\left[\frac{M}{\alpha(n)(R-|x|)^n}\right]^{\frac{1}{2}}$ .
\end{enumerate}
\end{exercise}
\begin{enumerate}
    \item 由平均值公式\ref{average_value_formula},我们有
    \begin{equation}\label{eq:u0}
    u(0)=\dashint_{ B(0,R)}u(x)\mathrm{d}x=\frac{1}{\alpha(n)R^{n}}\int_{\partial B(0,R)}u(x)\mathrm{d}S(x)
\end{equation}
再利用Schwarz不等式,我们有
\begin{equation}\label{eq:u0_2} 
    \left( \int_{B(0,R)}u(x)\mathrm{d}x\right)^2\le \left( \int_{B(0,R)}1 \mathrm{d}x\right)\cdot \left( \int_{B(0,R)}u^2(x)\mathrm{d}x\right)=\frac{1}{\alpha(n)R^{n}}\left( \int_{B(0,R)}u^2(x)\mathrm{d}x\right)
\end{equation}
结合\ref{eq:u0}和\ref{eq:u0_2},即可得证.
    \item 类似的,平移再多一步放缩即可.
\end{enumerate}

\begin{exercise}
    设 $u(x)$ 是单位球 $B=B(0,1)$ 上的有界调和函数.证明:
$$
\sup _{x \in B}(1-|x|)|D u(x)|<+\infty .
$$
(提示:利用周蜀林书上定理 2.7 的第一步证明)
\end{exercise}

已知
\[
    |\partial_i u(x)|\le \frac{n+1}{\alpha(n)r^{n+1}}\int_{B(x,r)}|u(y)|\mathrm{d}y\le \frac{n+1}{\alpha(n)r^{n+1}}\sup_{y\in B(x,r)}|u(y)|\cdot \alpha(n)r^{n}=\frac{n+1}{r}\sup_{y\in B(x,r)}|u(y)|
\]
对于任意给定的$x\in \mathring{B}$,令$r=1-|x|>0$,则
\[
    (1-|x|)|Du(x)|\le \sqrt{n}\cdot (n+1)\sup_{y\in B(x,r)}|u(y)|\le \sqrt{n}\cdot (n+1)\sup_{y\in B}|u(y)|
\]
由于$u$在$\bar{B}$上有界,所以
\[
    \sup_{x\in B}(1-|x|)|Du(x)|<+\infty
\]
\begin{exercise}
    设 $u(x)$ 是球 $B\left(0, R_0\right)$ 上的调和函数,对于 $R \in (0, R_0]$ 记
$$
\omega(R)=\sup _{B(0, R)} u-\inf _{B(0, R)} u .
$$
\begin{enumerate}
    \item 利用 Harnack 不等式证明:存在 $\eta \in(0,1)$ ,使得
$$
\omega\left(\frac{R}{2}\right) \leq \eta \omega(R)
$$
(提示:对调和函数 $w(x)=u(x)-\inf _{B(0, R)} u$ 在球 $B\left(0, \frac{R}{2}\right)$ 上利用 Harnack 不等式)
\item 如果 $\sup _{B\left(0, R_0\right)}|u(x)| \leq M_0$ ,则存在常数 $\alpha \in(0,1), C>0$ ,使得
$$
\omega(R) \leq C\left(M_0+1\right)\left(\frac{R}{R_0}\right)^\alpha, \quad R \in(0, R_0]
$$
\end{enumerate}
(提示:对任意 $R \in\left(0, R_0\right)$ ,一定存在一个正整数 $i \geq 1$ ,使得 $\frac{R_0}{2^i} \leq$ $\left.R<\frac{R_0}{2^{i-1}}\right)$

\end{exercise}


\begin{exercise}{推广的 Liouville 定理}
    假设 $u$ 是 $\mathbf{R}^n$ 上的调和函数,且
$$
|u(x)| \leq C_1|x|^m+C_2, \quad x \in \mathbf{R}^n,
$$
其中 $m$ 是非负整数,$C_1, C_2$ 是非负常数,则 $u$ 必为一个次数至多为 $m$ 的调和多项式.(提示:只要证明 $D^{m+1} u(x) \equiv 0$ )
\end{exercise}

\begin{exercise}
    假设 $u \in C^2\left(\mathbf{R}^n\right)$ .对于 $r>0$ ,定义   
$$
u_r(x)=\frac{1}{N \omega_n r^{n-1}} \int_{\partial B_{r(x)}} u(y) \mathrm{d} S(y)
$$
证明:
$$
\Delta u_r=(\Delta u)_r
$$
\end{exercise}

\begin{exercise}
    假设 $u$ 是 $B(0, R)$ 上的非负调和函数.
    \begin{enumerate}
        \item 利用 Poisson 公式(2.37)证明:
$$
R^{n-2} \frac{R-|x|}{(R+|x|)^{n-1}} u(0) \leq u(x) \leq R^{n-2} \frac{R+|x|}{(R-|x|)^{n-1}} u(0)
$$
        \item 证明定理 $2.4^{\prime}$ .
    \end{enumerate}
\end{exercise}

\begin{exercise}    
    利用上述不等式证明定理 $2.8^{\prime}$ .
\end{exercise}

\begin{exercise}
    证明:对于任意 $x=\left(x_1, x_2, \cdots, x_n\right) \in \mathbf{R}_{+}^n$ ,Poisson 核满足
$$
\int_{\mathbf{R}^{n-1}} K(x, y) \mathrm{d} y=1,
$$
其中
$$
\begin{gathered}
y=\left(y_1, y_2, \cdots, y_{n-1}, 0\right) \in \partial \mathbf{R}_{+}^n=\mathbf{R}^{n-1} \\
\mathrm{~d} y=\mathrm{d} y_1 \mathrm{~d} y_2 \cdots \mathrm{~d} y_{n-1}, \quad K(x, y)=\frac{2 x_n}{n \alpha(n)|y-x|^n}
\end{gathered}
$$
\end{exercise}

\begin{exercise}
    求边值问题
$$
\left\{\begin{array}{l}
-\triangle u=f(x, y), \quad(x, y) \in \Omega \\
\left.u\right|_{\partial \Omega}=g(x, y)
\end{array}\right.
$$
的 Green 函数,其中
\begin{enumerate}
    \item $\Omega$ 是上半平面;
    \item $\Omega$ 是第一象限;
    \item $\Omega$ 是带形区域 $\left\{(x, y) \in \mathbf{R}^2 \mid x \in \mathbf{R}, 0<y<l\right\}$ ,其中 $l$ 为正常数.
\end{enumerate}
\end{exercise}


\begin{exercise}
    记 $B^{+}(R)=\left\{x=\left(x_1, x_2, \cdots, x_n\right) \in \mathbf{R}^n\left|x_n>0,|x|<R\right\}\right.$ $(n \geq 2)$ .求边值问题
$$
\left\{\begin{array}{l}
-\triangle u=f(x), x \in B^{+}(R), \\
\left.u\right|_{\partial B^{+}(R)}=g(x)
\end{array}\right.
$$
的 Green 函数.(提示:分别考虑 $n=2$ 和 $n \geq 3$ 两种情形)
\end{exercise}

\begin{exercise}
    证明:第二边值问题
$$
\begin{cases}u_{x x}+u_{y y}=0, & (x, y) \in B(0, R), \\ \left.\frac{\partial u}{\partial r}\right|_{r=R}=g(\theta), & \quad \theta \in[0,2 \pi]\end{cases}
$$
的解在边值 $g(\theta)$ 满足条件 $\int_0^{2 \pi} g(\theta) \mathrm{d} \theta=0$ 时可以表示成
$$
u(r, \theta)=-\frac{1}{2 \pi} \int_0^{2 \pi} g(\tau) \ln \left[R^2+r^2-2 R r \cos (\tau-\theta)\right] \mathrm{d} \tau+C,
$$
其中 $C$ 为任意常数.这里 $(r, \theta)$ 是点 $(x, y)$ 的极坐标.
\end{exercise}


\begin{exercise}
    假设 $u(x) \in C(\bar{\Omega}) \cap C^2(\Omega)$ 是定解问题
$$
\left\{\begin{array}{l}
-\Delta u+c(x) u=f(x), x \in \Omega, \\
\left.u\right|_{\partial \Omega}=0
\end{array}\right.
$$
的一个解.
\begin{enumerate}
    \item 如果 $c(x) \geq c_0>0$ ,则
$$
\max _{\bar{\Omega}}|u(x)| \leq \frac{1}{c_0} \sup _{\Omega}|f(x)| ;
$$
\item 如果 $c(x) \geq 0$ ,则
$$
\max _{\bar{\Omega}}|u(x)| \leq M \sup _{\Omega}|f(x)|,
$$
其中 $M$ 依赖于 $\Omega$ 的直径 $d$ ;(提示:不妨设原点 $0 \in \Omega$ ,并令 $u(x)=$ $w(x) v(x)=\left(d^2-x|^2+1\right) v(x)$ ,然后考虑 $v(x)$ 满足的方程,利用(1)的证明方法可得 $v(x)$ 的最大模估计,从而得到 $u(x)$ 的最大模估计)
\item 如果 $c(x)<0$ ,试举反例说明上述最大模估计一般不成立.
\end{enumerate}
\end{exercise}


\begin{exercise}
    假设 $u(x) \in C(\bar{\Omega}) \cap C^2(\Omega)$ 是定解问题
$$
\left\{\begin{array}{l}
-\Delta u=1, x \in \Omega, \\
\left.u\right|_{\partial \Omega}=0
\end{array}\right.
$$
的一个解.试证明:对于任意 $x_0 \in \Omega$ ,估计
$$
\frac{1}{2 n} \min _{x \in \partial \Omega}\left|x-x_0\right|^2 \leq u\left(x_0\right) \leq \frac{1}{2 n} \max _{x \in \partial \Omega}\left|x-x_0\right|^2
$$
成立,这里 $n$ 是空间的维数.
\end{exercise}

\begin{exercise}
    假设 $u(x) \in C^1(\bar{\Omega}) \cap C^2(\Omega)$ 是定解问题
$$
\begin{cases}-\Delta u+c(x) u=f(x), & x \in \Omega \\ {\left.\left[\frac{\partial u}{\partial \boldsymbol{n}}+\alpha(x) u\right]\right|_{\Gamma_1}=g_1,} & \left.u\right|_{\Gamma_2}=g_2\end{cases}
$$
的一个解,其中 $\Gamma_1 \cup \Gamma_2=\partial \Omega, \Gamma_1 \cap \Gamma_2=\phi, \Gamma_2 \neq \phi$ .
\begin{enumerate}
    \item 如果 $c(x) \geq c_0>0, \alpha(x) \geq \alpha_0>0$ ,则有估计
$$
\max _{\bar{\Omega}}|u(x)| \leq \max \left\{\frac{1}{c_0} \sup _{\Omega}|f(x)|, \frac{1}{\alpha_0} \max _{\Gamma_1}\left|g_1\right|, \max _{\Gamma_2}\left|g_2\right|\right\} .
$$
    \item 如果 $c(x) \geq 0, \alpha(x) \geq 0$ ,且 $\Gamma_1$ 满足内球条件,则上述问题的解是惟一的.这里 $\Gamma_1$ 满足内球条件是指对于任意 $x_0 \in \Gamma_1$ ,存在一个球 $B$ 使得 $B \subset \Omega, \Gamma_1 \cap \partial B=\left\{x_0\right\}$ .(提示:令 $u(x)=w(x) v(x)$ ,其中 $w(x)$ 是待定的辅助函数.在球 $B$ 上我们考虑 $v(x)$ 满足的问题)
\end{enumerate}
\end{exercise}

\begin{exercise}
    试用辅助函数
$$
w(x)=\mathrm{e}^{-a|x|^2}-\mathrm{e}^{-a R^2}
$$
证明 Hopf 引理,这里 $a>0, R$ 是球 $B_R$ 的半径.
\end{exercise}

\begin{exercise}
    假设 $\Omega \subset \mathbf{R}^n$ 是一个有界开集,$u_i(x) \in C^2(\Omega) \cap C(\bar{\Omega})$ $(i=1,2)$ 满足定解问题
$$
\begin{cases}-\triangle u_i+c_i(x) u=0, & x \in \Omega, \\ u=g_i, & x \in \partial \Omega .\end{cases}
$$
如果 $c_2(x) \geq c_1(x) \geq 0, g_1(x) \geq g_2(x) \geq 0$ ,则
$$
u_1(x) \geq u_2(x)
$$
(提示:先证明 $u_i(x) \geq 0$ ,然后考虑 $w(x)=u_1(x)-u_2(x)$ 满足的定解问题)
\end{exercise}


\begin{exercise}
    假设 $\Omega_0 \subset \mathbf{R}^n$ 是一个有界区域,$\Omega=\mathbf{R}^n \backslash \bar{\Omega}_0$ .如果 $u(x) \in C^2(\Omega) \cap C(\bar{\Omega})$ 是外部问题
$$
\begin{cases}-\triangle u+c(x) u=0, & x \in \Omega, \\ \left.u\right|_{\partial \Omega=g(x),} & x \in \partial \Omega, \\ \lim _{|x| \rightarrow \infty} u(x)=l & \end{cases}
$$
的一个解,其中 $c(x) \geq 0$ 且在 $\bar{\Omega}$ 上局部有界,则
$$
\sup _{\Omega}|u(x)| \leq \max \left\{|l|, \max _{\partial \Omega}|g(x)|\right\} .
$$
(提示:先在区域 $B(0, R) \backslash \Omega_0$ 上证明关于 $u(x)$ 的估计,然后令 $R \rightarrow +\infty$)
\end{exercise}






\end{document}