\begin{figure}[H]
\centering
\includegraphics[width=\textwidth]{Evans Chap5-20250224.png}
% \caption{}
\label{}
\end{figure}
The irreversibility of its semiflow and the need to impose a growth condition as $\displaystyle \lvert x \rvert\to \infty$ in order to pick out a unique solution.

\section{Schwartz solutions}

The finite-dimensional phase sapce of the ODE is replaced by the infinite-dimensional function space $\displaystyle \mathcal{S}$; then we think of a solution of the heat equation as a parametrized curve in the vector space $\displaystyle \mathcal{S}$.

\subsection{Notiations}

\begin{figure}[H]
\centering
\includegraphics[width=\textwidth]{1-Evans Chap5-20250224.png}
% \caption{}
\label{}
\end{figure}
\begin{figure}[H]
\centering
\includegraphics[width=\textwidth]{Evans Chap5-20250225.png}
% \caption{}
\label{}
\end{figure}

\subsection{Spatial Fourier transform}

\begin{figure}[H]
\centering
\includegraphics[width=\textwidth]{1-Evans Chap5-20250225.png}
% \caption{}
\label{}
\end{figure}
\begin{figure}[H]
\centering
\includegraphics[width=\textwidth]{2-Evans Chap5-20250225.png}
% \caption{}
\label{}
\end{figure}
\begin{figure}[H]
\centering
\includegraphics[width=\textwidth]{3-Evans Chap5-20250225.png}
% \caption{}
\label{}
\end{figure}
\begin{figure}[H]
\centering
\includegraphics[width=\textwidth]{4-Evans Chap5-20250225.png}
% \caption{}
\label{}
\end{figure}

\subsection{Green's function}

\begin{figure}[H]
\centering
\includegraphics[width=\textwidth]{6-Evans Chap5-20250225.png}
% \caption{}
\label{}
\end{figure}
\begin{figure}[H]
\centering
\includegraphics[width=\textwidth]{5-Evans Chap5-20250225.png}
% \caption{}
\label{}
\end{figure}

\subsection{Smoothing}

\begin{figure}[H]
\centering
\includegraphics[width=\textwidth]{7-Evans Chap5-20250225.png}
% \caption{}
\label{}
\end{figure}

\subsection{Irreversibility}

\begin{figure}[H]
\centering
\includegraphics[width=\textwidth]{8-Evans Chap5-20250225.png}
% \caption{}
\label{}
\end{figure}
\begin{figure}[H]
\centering
\includegraphics[width=\textwidth]{9-Evans Chap5-20250225.png}
% \caption{}
\label{}
\end{figure}

\subsection{Nonuniqueness}

\begin{figure}[H]
\centering
\includegraphics[width=\textwidth]{10-Evans Chap5-20250225.png}
% \caption{}
\label{}
\end{figure}

\subsection{The Schrodinger equation}

\begin{figure}[H]
\centering
\includegraphics[width=\textwidth]{11-Evans Chap5-20250225.png}
% \caption{}
\label{}
\end{figure}
\begin{figure}[H]
\centering
\includegraphics[width=\textwidth]{13-Evans Chap5-20250225.png}
% \caption{}
\label{}
\end{figure}
\begin{figure}[H]
\centering
\includegraphics[width=\textwidth]{12-Evans Chap5-20250225.png}
% \caption{}
\label{}
\end{figure}

\begin{note}
Formally, the Schrodinger equation is obtained by the transformation $\displaystyle t\mapsto-it$ of the heat equation to 'imaginary time.' The analytical properties of the heat and Schr¨odinger equations are, however, completely different and it is interesting to compare them.
\end{note}
\subsection{Semigroups and groups}

\begin{figure}[H]
\centering
\includegraphics[width=\textwidth]{14-Evans Chap5-20250225.png}
% \caption{}
\label{}
\end{figure}
\begin{figure}[H]
\centering
\includegraphics[width=\textwidth]{15-Evans Chap5-20250225.png}
% \caption{}
\label{}
\end{figure}

\subsubsection{Convolution operator}

\begin{figure}[H]
\centering
\includegraphics[width=\textwidth]{16-Evans Chap5-20250225.png}
% \caption{}
\label{}
\end{figure}
