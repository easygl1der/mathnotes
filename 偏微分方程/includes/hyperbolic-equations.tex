The prototypical example of a hyperbolic PDE is the wave equation
\begin{equation}
u_{tt}=\Delta u
\label{4f0e56}
\end{equation}

To begin with, consider the one-dimensional wave equation on $\mathbb{R}$
\[
u_{tt}=u_{xx}
\]
The general solution is the \textbf{d'Alembert solution}
\[
u(x,t)=f(x-t)+g(x+t)
\]
where $f$, $g$ are arbitrary functions. This solution describes a superposition of two traveling waves with arbitrary profiles, one propagating with speed one to the right, the other with speed one to the left.
Let us compare this solution with the general solution of the one-dimensional heat equation
\[
u_{t}=u_{x x}
\]
which is given for $t>0$ by
\[
u(x,t)=\frac{1}{\sqrt{ 4\pi t }}\int_{\mathbb{R}}^{} e^{ -(x-y)^{2}/4t }f(y) \, dy
\]
Some of the qualitative properties of the wave equation that differ from those of the heat equaion, which are evident from these solutions.

\begin{figure}[H]
\centering
\includegraphics[width=\textwidth]{Hyperbolic equations-20250312-005414.png}
% \caption{}
\label{}
\end{figure}

To obtain the basic energy estimatie for the wave equation, we multiple \cref{4f0e56} ny $u_{t}$ and write
\[
\begin{aligned}
u_{t}u_{tt} & =\left( \frac{1}{2}u_{t}^{2} \right)_{t} \\
u_{t}\Delta u & =\mathrm{div}(u_{t}Du)-Du\cdot Du_{t}=\mathrm{div}(u_{t}Du)-\left( \frac{1}{2}\lvert Du \rvert ^{2} \right)_{t}
\end{aligned}
\]
to get
\[
\left( \frac{1}{2}u_{t}^{2}+\frac{1}{2}\lvert Du \rvert ^{2} \right)-\mathrm{div}(u_{t}Du)=0
\]
This is the differential form of conservation of energy. The quantity $\frac{1}{2}u_{t}^{2}+\frac{1}{2}\lvert Du \rvert ^{2}$ is the energy density and $-u_{t}Du$ is the energy flux.

\begin{figure}[H]
\centering
\includegraphics[width=\textwidth]{Hyperbolic equations-20250312.png}
% \caption{}
\label{}
\end{figure}
