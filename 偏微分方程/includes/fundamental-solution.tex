\section{Fundamental Solution: Laplace equation in \texorpdfstring{$\mathbb{R}^{3}$}{mathbbR^3}}

Okay, let's deduce the fundamental solution of the Laplacian more intuitively. The \textbf{fundamental solution} $\Phi(x)$ is defined as the solution to the equation:
\[
\Delta \Phi(x) = \delta(x)
\]
where $\Delta$ is the Laplacian operator and $\delta(x)$ is the Dirac delta function, representing a point source at the origin. The previous proof concerning the harmonic function $u(x)$ in $\mathbb{R}^3$ used the representation:
\[
u(x) = -\frac{1}{4\pi} \int_{\mathbb{R}^3} \frac{\rho(y)}{|x-y|} dy \quad \text{for} \quad \Delta u = \rho
\]
This implies that the fundamental solution satisfying $\Delta \Phi(x) = \delta(x)$ is $\Phi(x) = -\frac{1}{4\pi|x|}$ in $\mathbb{R}^3$. Let's see how we can arrive at this.

\subsection{1. Intuitive Deduction using Radial Symmetry and Gauss's Law (Divergence Theorem)}

This approach is very physical and geometric, especially for an operator like the Laplacian.

\textbf{a. Radial Symmetry:}

The Dirac delta function $\delta(x)$ is a point source at the origin. It's spherically symmetric (its value depends only on whether $x=0$). We can expect the potential $\Phi(x)$ generated by this point source to also be spherically symmetric. Thus, $\Phi(x)$ should depend only on the distance $r = |x|$ from the origin:
\[
\Phi(x) = \phi(r)
\]
\textbf{b. Laplacian in Spherical Coordinates:}

For a radially symmetric function $\phi(r)$ in $\mathbb{R}^n$, the Laplacian is:
\[
\Delta \phi(r) = \frac{1}{r^{n-1}} \frac{d}{dr} \left( r^{n-1} \frac{d\phi}{dr} \right)
\]
In $\mathbb{R}^3$ (which is the context of the original problem), $n=3$, so:
\[
\Delta \phi(r) = \frac{1}{r^2} \frac{d}{dr} \left( r^2 \frac{d\phi}{dr} \right)
\]
\textbf{c. Solution Away from the Origin ($r > 0$):}

Away from the origin ($x \neq 0$, so $r > 0$), the Dirac delta is zero, $\delta(x) = 0$. So, we must have $\Delta \Phi(x) = 0$.
\[
\frac{1}{r^2} \frac{d}{dr} \left( r^2 \frac{d\phi}{dr} \right) = 0 \quad \text{for } r > 0
\]
This implies that $r^2 \frac{d\phi}{dr}$ must be a constant, let's call it $A$:
\[
r^2 \frac{d\phi}{dr} = A
\]
So,
\[
\frac{d\phi}{dr} = \frac{A}{r^2}
\]
Integrating with respect to $r$ gives:
\[
\phi(r) = -\frac{A}{r} + B
\]
where $B$ is another constant. For potentials that vanish at infinity (which is a common physical condition, and matches $u(x) \to 0$ in the problem), we typically set $B=0$. So, $\phi(r) = -A/r$.

\textbf{d. Determining the Constant A (using the source):}

To find $A$, we use the defining equation $\Delta \Phi(x) = \delta(x)$. Integrate both sides over a small ball $B(0, \epsilon)$ of radius $\epsilon$ centered at the origin:
\[
\int_{B(0,\epsilon)} \Delta \Phi(x) dV = \int_{B(0,\epsilon)} \delta(x) dV
\]
The right side, by definition of the Dirac delta, is $1$.
For the left side, we use the Divergence Theorem (Gauss's Law):
\[
\int_{B(0,\epsilon)} \Delta \Phi(x) dV = \int_{\partial B(0,\epsilon)} \nabla \Phi(x) \cdot \mathbf{n} dS
\]
where $\partial B(0,\epsilon)$ is the sphere of radius $\epsilon$, and $\mathbf{n}$ is the outward unit normal vector.
Since $\Phi(x) = \phi(r)$, its gradient is $\nabla \Phi(x) = \frac{d\phi}{dr} \frac{x}{|x|} = \frac{d\phi}{dr} \mathbf{e}_r$, where $\mathbf{e}_r$ is the radial unit vector. The outward normal $\mathbf{n}$ on the sphere is also $\mathbf{e}_r$.
So, $\nabla \Phi(x) \cdot \mathbf{n} = \frac{d\phi}{dr}$. At $r=\epsilon$, this is $\left.\frac{d\phi}{dr}\right|_{r=\epsilon} = \frac{A}{\epsilon^2}$.
The surface area of the sphere $\partial B(0,\epsilon)$ in $\mathbb{R}^3$ is $4\pi\epsilon^2$.
So the surface integral becomes:
\[
\int_{\partial B(0,\epsilon)} \frac{A}{\epsilon^2} dS = \frac{A}{\epsilon^2} (4\pi\epsilon^2) = 4\pi A
\]
Equating this to the integral of the Dirac delta:
\[
4\pi A = 1 \implies A = \frac{1}{4\pi}
\]
Substituting this back into $\phi(r) = -A/r$:
\[
\phi(r) = -\frac{1}{4\pi r}
\]
Thus, the fundamental solution in $\mathbb{R}^3$ is:
\[
\Phi(x) = -\frac{1}{4\pi|x|}
\]
This is precisely the kernel that appeared in the integral representation for $u(x)$ in the problem you referenced.

\textbf{Intuition from Flux Spreading:}

Imagine a "flux" emanating from the point source at the origin.

\begin{itemize}
	\item The term $r^{n-1} \frac{d\phi}{dr}$ (which is $r^2 \frac{d\phi}{dr}$ in $\mathbb{R}^3$) can be thought of as proportional to the total flux of $\nabla \phi$ through a sphere of radius $r$.
	\item The condition $\Delta \phi = 0$ for $r>0$ means this flux is constant for any $r>0$.
	\item The strength of the field (gradient) $\frac{d\phi}{dr}$ must therefore decrease as $1/r^{n-1}$ to keep the flux constant, because the surface area of the sphere is proportional to $r^{n-1}$.
	\begin{itemize}
		\item In $\mathbb{R}^3$, surface area $\sim r^2$, so field $\sim 1/r^2$. Potential $\phi(r) \sim 1/r$.
		\item In $\mathbb{R}^2$, surface area (circumference) $\sim r$, so field $\sim 1/r$. Potential $\phi(r) \sim \ln r$.
	\end{itemize}
	\item The constant value of this flux is determined by the strength of the source $\delta(x)$, which is $1$ when integrated. This directly gives the constant $A$.
\end{itemize}

\subsection{2. Deduction using Fourier Transform}

This method is more algebraic and general for constant-coefficient linear PDEs.

Let $\widehat{f}(k) = \mathcal{F}\{f(x)\}(k) = \int_{\mathbb{R}^n} f(x) e^{-i k \cdot x} dx$ be the Fourier transform.
The equation is $\Delta \Phi(x) = \delta(x)$.
Taking the Fourier transform of both sides:
\[
\mathcal{F}\{\Delta \Phi(x)\}(k) = \mathcal{F}\{\delta(x)\}(k)
\]
\begin{itemize}
	\item \textbf{FT of Dirac Delta:} $\mathcal{F}\{\delta(x)\}(k) = 1$.
	\item \textbf{FT of Laplacian:} The Fourier transform of $\frac{\partial^2}{\partial x_j^2} \Phi(x)$ is $(i k_j)^2 \widehat{\Phi}(k) = -k_j^2 \widehat{\Phi}(k)$.
So, $\mathcal{F}\{\Delta \Phi(x)\}(k) = \mathcal{F}\left\{\sum_{j=1}^n \frac{\partial^2 \Phi}{\partial x_j^2}\right\}(k) = \sum_{j=1}^n (-k_j^2) \widehat{\Phi}(k) = -|k|^2 \widehat{\Phi}(k)$.
\end{itemize}

Putting these together, the equation in Fourier space becomes:
\[
-|k|^2 \widehat{\Phi}(k) = 1
\]
So, the Fourier transform of the fundamental solution is:
\[
\widehat{\Phi}(k) = -\frac{1}{|k|^2}
\]
Now, we need to find the inverse Fourier transform $\Phi(x) = \mathcal{F}^{-1}\{-\frac{1}{|k|^2}\}(x)$:
\[
\Phi(x) = -\int_{\mathbb{R}^n} \frac{1}{|k|^2} e^{i k \cdot x} \frac{dk}{(2\pi)^n}
\]
This integral needs to be computed.

\textbf{For $n=3$ (as in the problem):}
\[
\Phi(x) = -\frac{1}{(2\pi)^3} \int_{\mathbb{R}^3} \frac{1}{|k|^2} e^{i k \cdot x} dk
\]
To evaluate this, we can use spherical coordinates for $k=(k_1, k_2, k_3)$. Let $s = |k|$. By rotating coordinates, we can align $x$ with the $z$-axis, so $x=(0,0,|x|)$, and $k \cdot x = s |x| \cos\psi$, where $\psi$ is the angle between $k$ and $x$. The volume element is $dk = s^2 \sin\psi ds d\psi d\theta_k$ (where $\theta_k$ is the azimuthal angle for $k$).
\[
\begin{aligned}
\Phi(x) &= -\frac{1}{(2\pi)^3} \int_0^\infty \int_0^\pi \int_0^{2\pi} \frac{1}{s^2} e^{i s |x| \cos\psi} s^2 \sin\psi d\theta_k d\psi ds \\
&= -\frac{2\pi}{(2\pi)^3} \int_0^\infty \left( \int_0^\pi e^{i s |x| \cos\psi} \sin\psi d\psi \right) ds
\end{aligned}
\]
The inner integral is
\[
\int_0^\pi e^{i s |x| \cos\psi} \sin\psi d\psi = \left[ -\frac{e^{i s |x| \cos\psi}}{i s |x|} \right]_0^\pi = -\frac{e^{-i s |x|} - e^{i s |x|}}{i s |x|} = \frac{2\sin(s|x|)}{s|x|}
\]
So,
\[
\begin{aligned}
\Phi(x) &= -\frac{1}{4\pi^2} \int_0^\infty \frac{2\sin(s|x|)}{s|x|} ds \\
&= -\frac{1}{2\pi^2 |x|} \int_0^\infty \frac{\sin(s|x|)}{s} ds
\end{aligned}
\]
Let $t = s|x|$, then $ds = dt/|x|$. The integral becomes
\[
\int_0^\infty \frac{\sin t}{t/|x|} \frac{dt}{|x|} = \int_0^\infty \frac{\sin t}{t} dt
\]
This is the Dirichlet integral, which evaluates to $\pi/2$.
\[
\Phi(x) = -\frac{1}{2\pi^2 |x|} \cdot \frac{\pi}{2} = -\frac{1}{4\pi|x|}
\]
This matches the result from the radial symmetry method.

\textbf{Intuition from Fourier Transform:}

\begin{itemize}
	\item The Laplacian $\Delta$ acts as a multiplication by $-|k|^2$ in Fourier space. It heavily attenuates high frequencies.
	\item The Dirac delta $\delta(x)$ has a flat spectrum (all frequencies equally present: $\widehat{\delta}(k)=1$).
	\item $\widehat{\Phi}(k) = -1/|k|^2$ means the fundamental solution has a spectrum that decays fairly slowly for small $|k|$ (it's singular at $k=0$) and faster for large $|k|$. This $1/|k|^2$ behavior in Fourier space is characteristic of $1/|x|$ type potentials in real space for 3D. The singularity at $k=0$ in $\widehat{\Phi}(k)$ hints at the long-range nature of the potential $\Phi(x)$.
\end{itemize}

Both methods yield the same result. The radial symmetry/Gauss's law approach is often more physically intuitive for the Laplacian, while the Fourier transform method is a very powerful and general algebraic tool applicable to a wider range of constant-coefficient linear differential operators.

\section{Fundamental Solution: Laplace, heat, wave equation in \texorpdfstring{$\mathbb{R}^{n}$}{mathbbR^n}}

\subsection{1. Laplace Equation}

The fundamental solution $G(x)$ for the Laplace equation satisfies
\[
\Delta_x G(x)=\delta(x) \quad \text { in } \mathbb{R}^n
\]
where $\Delta_x=\sum_{j=1}^n \frac{\partial^2}{\partial x_j^2}$ is the Laplacian operator.

\begin{enumerate}
	\item \textbf{Fourier Transform:}
\end{enumerate}

Applying the Fourier transform with respect to $x$ to both sides:
\[
\mathcal{F}_x\{\Delta_x G(x)\}(k)=\mathcal{F}_x\{\delta(x)\}(k)
\]
We know that $\mathcal{F}_x\{\delta(x)\}(k)=1$. For the left side, the derivative property of Fourier transforms gives $\mathcal{F}_x\{\frac{\partial}{\partial x_j} f(x)\}(k)=i k_j \widehat{f}(k)$. Thus, $\mathcal{F}_x\{\frac{\partial^2}{\partial x_j^2} f(x)\}(k)=\left(i k_j\right)^2 \widehat{f}(k)=-k_j^2 \widehat{f}(k)$.
So, $\mathcal{F}_x\{\Delta_x G(x)\}(k)=\sum_{j=1}^n\left(-k_j^2\right) \widehat{G}(k)=-|k|^2 \widehat{G}(k)$.
The transformed equation is:
\[
-|k|^2 \widehat{G}(k)=1
\]
Solving for $\widehat{G}(k)$:
\[
\widehat{G}(k)=-\frac{1}{|k|^2}
\]
\begin{enumerate}
	\item \textbf{Inverse Fourier Transform:}
\end{enumerate}

Now we need to compute the inverse Fourier transform:
\[
G(x)=-\frac{1}{(2 \pi)^n} \int_{\mathbb{R}^n} \frac{1}{|k|^2} e^{i k \cdot x} d^n k
\]
\begin{itemize}
	\item \textbf{Case $n>2$:}
\end{itemize}

We use the Schwinger parametrization $\frac{1}{A}=\int_0^\infty e^{-s A} d s$. Here $A=|k|^2$:
\[
\frac{1}{|k|^2}=\int_0^\infty e^{-s|k|^2} d s
\]
Substituting this into the integral for $G(x)$:
\[
G(x)=-\frac{1}{(2 \pi)^n} \int_{\mathbb{R}^n}\left(\int_0^\infty e^{-s|k|^2} d s\right) e^{i k \cdot x} d^n k
\]
Assuming we can interchange the order of integration:
\[
G(x)=-\frac{1}{(2 \pi)^n} \int_0^\infty\left(\int_{\mathbb{R}^n} e^{-s|k|^2+i k \cdot x} d^n k\right) d s
\]
The inner integral is an $n$-dimensional Gaussian integral. The exponent is $-s|k|^2+i k \cdot x=-s \sum_j k_j^2+i \sum_j k_j x_j$.
Completing the square for each component $k_j$ : $-s k_j^2+i k_j x_j=-s\left(k_j-\frac{i x_j}{2 s}\right)^2-\frac{x_j^2}{4 s}$.
\[
\int_{\mathbb{R}^n} e^{-s|k|^2+i k \cdot x} d^n k=\prod_{j=1}^n \int_{-\infty}^\infty e^{-s\left(k_j-\frac{i x_j}{2 s}\right)^2-\frac{x_j^2}{4 s}} d k_j=e^{-\frac{|x|^2}{4 s}} \prod_{j=1}^n \int_{-\infty}^\infty e^{-s K_j^2} d K_j
\]
Since $\int_{-\infty}^\infty e^{-s K^2} d K=\sqrt{\frac{\pi}{s}}$, the $n$-dimensional integral is $e^{-\frac{|x|^2}{4 s}}\left(\frac{\pi}{s}\right)^{n / 2}$.
Substituting this back:
\[
G(x)=-\frac{1}{(2 \pi)^n}(\pi)^{n / 2} \int_0^\infty s^{-n / 2} e^{-\frac{|x|^2}{4 s}} d s=-\frac{1}{2^n \pi^{n / 2}} \int_0^\infty s^{-n / 2} e^{-\frac{|x|^2}{4 s}} d s
\]
Let $t=\frac{|x|^2}{4 s}$. Then $s=\frac{|x|^2}{4 t}$, so $d s=-\frac{|x|^2}{4 t^2} d t$.
The integral becomes (for $n>2$, so $n / 2-1>0$ for convergence of $\Gamma(n / 2-1)$ ):
\[
\int_0^\infty s^{-n / 2} e^{-\frac{|x|^2}{4 s}} d s=\int_\infty^0\left(\frac{4 t}{|x|^2}\right)^{n / 2} e^{-t}\left(-\frac{|x|^2}{4 t^2}\right) d t=\frac{2^{n-2}}{|x|^{n-2}} \int_0^\infty t^{n / 2-2} e^{-t} d t=\frac{2^{n-2}}{|x|^{n-2}} \Gamma(n / 2-1)
\]
So, for $n>2$ :
\[
G(x)=-\frac{1}{2^n \pi^{n / 2}} \frac{2^{n-2}}{|x|^{n-2}} \Gamma(n / 2-1)=-\frac{\Gamma(n / 2-1)}{4 \pi^{n / 2}|x|^{n-2}}
\]
For example, if $n=3$, $G(x)=-\frac{\Gamma(1 / 2)}{4 \pi^{3 / 2}|x|}=-\frac{\sqrt{\pi}}{4 \pi \sqrt{\pi}|x|}=-\frac{1}{4 \pi|x|}$.

\begin{itemize}
	\item \textbf{Case $n=2$:}
\end{itemize}

The above derivation for the $s$-integral requires $n / 2-1>0$, so it doesn't directly apply to $n=2$. The direct Fourier inversion of $-1 /|k|^2$ in 2D is subtle. The known result is:
\[
G(x)=\frac{1}{2 \pi} \ln |x|
\]
This can be verified by showing $\Delta G(x)=\delta(x)$ in the distributional sense.

\begin{itemize}
	\item \textbf{Case $n=1$:}
\end{itemize}

Similarly, for $n=1, \widehat{G}(k)=-1 / k^2$. Direct inversion is subtle. The known result is:
\[
G(x)=\frac{1}{2}|x|
\]
This satisfies $\frac{d^2}{d x^2} G(x)=\delta(x)$.


\subsection{2. Heat Equation}

The fundamental solution (or heat kernel) $G(x, t)$ satisfies the homogeneous heat equation with a Dirac delta initial condition:
\[
\left(\partial_t-\alpha \Delta_x\right) G(x, t)=0 \quad \text { for } t>0, x \in \mathbb{R}^n
\]
\[
G(x, 0)=\delta(x)
\]
where $\alpha>0$ is the thermal diffusivity.

\begin{enumerate}
	\item \textbf{Spatial Fourier Transform:}
\end{enumerate}

Let $\widehat{G}(k, t)=\mathcal{F}_x\{G(x, t)\}(k)$. Applying the Fourier transform to the PDE:
\[
\partial_t \widehat{G}(k, t)-\alpha\left(-|k|^2\right) \widehat{G}(k, t)=0
\]
\[
\partial_t \widehat{G}(k, t)=-\alpha|k|^2 \widehat{G}(k, t)
\]
The initial condition transforms to $\widehat{G}(k, 0)=\mathcal{F}_x\{\delta(x)\}(k)=1$.

\begin{enumerate}
	\item \textbf{Solve the ODE in Time:}
\end{enumerate}

This is a first-order ODE for $\widehat{G}(k, t)$ with respect to $t$:
\[
\widehat{G}(k, t)=C(k) e^{-\alpha|k|^2 t}
\]
Using the initial condition $\widehat{G}(k, 0)=1$, we find $C(k)=1$. So, for $t>0$ :
\[
\widehat{G}(k, t)=e^{-\alpha|k|^2 t}
\]
\begin{enumerate}
	\item \textbf{Inverse Spatial Fourier Transform:}
\end{enumerate}
\[
G(x, t)=\mathcal{F}_k^{-1}\left\{e^{-\alpha|k|^2 t}\right\}(x)=\frac{1}{(2 \pi)^n} \int_{\mathbb{R}^n} e^{-\alpha|k|^2 t} e^{i k \cdot x} d^n k
\]
The exponent is $-\alpha t|k|^2+i k \cdot x=-\alpha t \sum_j k_j^2+i \sum_j k_j x_j$.
We complete the square for each component $k_j$ :
\[
-\alpha t k_j^2+i k_j x_j=-\alpha t\left(k_j^2-\frac{i x_j}{\alpha t} k_j\right)=-\alpha t\left(\left(k_j-\frac{i x_j}{2 \alpha t}\right)^2+\frac{x_j^2}{4 \alpha^2 t^2}\right)=-\alpha t\left(k_j-\frac{i x_j}{2 \alpha t}\right)^2-\frac{x_j^2}{4 \alpha t}
\]
The integral becomes:
\[
G(x, t)=\frac{1}{(2 \pi)^n} e^{-\frac{|x|^2}{4 \alpha t}} \prod_{j=1}^n \int_{-\infty}^\infty e^{-\alpha t\left(k_j-\frac{i x_j}{2 \alpha t}\right)^2} d k_j
\]
Each 1D Gaussian integral is $\int_{-\infty}^\infty e^{-\alpha t K_j^2} d K_j=\sqrt{\frac{\pi}{\alpha t}}$.
Therefore:
\[
G(x, t)=\frac{1}{(2 \pi)^n} e^{-\frac{|x|^2}{4 \alpha t}}\left(\frac{\pi}{\alpha t}\right)^{n / 2}=\frac{\pi^{n / 2}}{(2 \pi)^n(\alpha t)^{n / 2}} e^{-\frac{|x|^2}{4 \alpha t}}
\]
\[
G(x, t)=\frac{1}{(4 \pi \alpha t)^{n / 2}} e^{-\frac{|x|^2}{4 \alpha t}} \quad \text { for } t>0
\]

\subsection{3. Wave Equation}

The fundamental solution $G(x, t)$ for the wave equation satisfies:
\[
\left(\partial_t^2-c^2 \Delta_x\right) G(x, t)=\delta(x) \delta(t) \quad \text { in } \mathbb{R}^n \times \mathbb{R}
\]
where $c>0$ is the wave speed. We seek the retarded Green's function, satisfying $G(x, t)=0$ for $t<0$.

\begin{enumerate}
	\item \textbf{Space-Time Fourier Transform:}
\end{enumerate}

Let $\widetilde{G}(k, \omega)=\mathcal{F}_{x, t}\{G(x, t)\}(k, \omega)$.
\[
\mathcal{F}\left\{\partial_t^2 G\right\}=(i \omega)^2 \widetilde{G}(k, \omega)=-\omega^2 \widetilde{G}(k, \omega).
\]
\[
\mathcal{F}\left\{-c^2 \Delta_x G\right\}=-c^2\left(-|k|^2\right) \widetilde{G}(k, \omega)=c^2|k|^2 \widetilde{G}(k, \omega).
\]
\[
\mathcal{F}\{\delta(x) \delta(t)\}=1.
\]
The transformed equation is:
\[
\left(-\omega^2+c^2|k|^2\right) \widetilde{G}(k, \omega)=1
\]
\[
\widetilde{G}(k, \omega)=\frac{1}{c^2|k|^2-\omega^2}
\]
\begin{enumerate}
	\item \textbf{Inverse Fourier Transform with respect to $\omega$ (Time):}
\end{enumerate}
\[
\widehat{G}(k, t)=\mathcal{F}_\omega^{-1}\left\{\frac{1}{c^2|k|^2-\omega^2}\right\}(t)=\frac{1}{2 \pi} \int_{-\infty}^\infty \frac{e^{i \omega t}}{c^2|k|^2-\omega^2} d \omega
\]
Let $\omega_0=c|k|$. The integral is $\frac{1}{2 \pi} \int_{-\infty}^\infty \frac{e^{i \omega t}}{\omega_0^2-\omega^2} d \omega$.
The poles are at $\omega=\pm \omega_0$. For the retarded Green's function, this integral evaluates to:
\[
\widehat{G}(k, t)=\frac{\sin \left(\omega_0 t\right)}{\omega_0} H(t)=\frac{\sin (c|k| t)}{c|k|} H(t)
\]
where $H(t)$ is the Heaviside step function. This ensures $G(x, t)=0$ for $t<0$.

\begin{definition}[Heaviside step function]
The Heaviside step function is defined by
\[
H(x)= \begin{cases}0 & \text { if } x<0 \\ 1 & \text { if } x \geq 0\end{cases}
\]
\end{definition}
\begin{enumerate}
	\item \textbf{Inverse Fourier Transform with respect to $k$ (Space):}
\end{enumerate}
\[
G(x, t)=H(t) \mathcal{F}_k^{-1}\left\{\frac{\sin (c|k| t)}{c|k|}\right\}(x)=\frac{H(t)}{(2 \pi)^n c} \int_{\mathbb{R}^n} \frac{\sin (c|k| t)}{|k|} e^{i k \cdot x} d^n k
\]
The evaluation of this integral depends heavily on the dimension $n$.

\begin{itemize}
	\item \textbf{For $n=1$:}
\end{itemize}
\[
G(x, t)=\frac{H(t)}{(2 \pi) c} \int_{-\infty}^\infty \frac{\sin \left(c\left|k_x\right| t\right)}{\left|k_x\right|} e^{i k_x x} d k_x
\]
The integral $\int_{-\infty}^\infty \frac{\sin \left(a\left|k_x\right|\right)}{\left|k_x\right|} e^{i k_x x} d k_x=\pi$ if $|x|<a$, and $0$ if $|x|>a$. Here $a=c t$.
\[
G(x, t)=\frac{H(t)}{(2 \pi) c} \cdot \pi H(c t-|x|)=\frac{H(t) H(c t-|x|)}{2 c}
\]
This means $G(x, t)=\frac{1}{2 c}$ for $t>0$ and $|x|<c t$, and $0$ otherwise.

\begin{itemize}
	\item \textbf{For $n=2$:}
\end{itemize}
\[
G(x, t)=\frac{H(t)}{(2 \pi)^2 c} \int_{\mathbb{R}^2} \frac{\sin (c|k| t)}{|k|} e^{i k \cdot x} d^2 k
\]
The integral evaluates to $\frac{2 \pi H(c t-|x|)}{\sqrt{c^2 t^2-|x|^2}}$.
\[
G(x, t)=\frac{H(t)}{(2 \pi)^2 c} \frac{2 \pi H(c t-|x|)}{\sqrt{c^2 t^2-|x|^2}}=\frac{H(t) H(c t-|x|)}{2 \pi c \sqrt{c^2 t^2-|x|^2}}
\]
\begin{itemize}
	\item \textbf{For $n=3$:}
\end{itemize}
\[
G(x, t)=\frac{H(t)}{(2 \pi)^3 c} \int_{\mathbb{R}^3} \frac{\sin (c|k| t)}{|k|} e^{i k \cdot x} d^3 k
\]
The integral can be evaluated using spherical coordinates. Let $r=|x|$.
\[
\int_{\mathbb{R}^3} \frac{\sin (c t|k|)}{|k|} e^{i k \cdot x} d^3 k=\frac{2 \pi^2}{r} \delta(c t-r)
\]
\[
G(x, t)=\frac{H(t)}{(2 \pi)^3 c} \frac{2 \pi^2}{|x|} \delta(c t-|x|)=\frac{H(t)}{4 \pi c|x|} \delta(c t-|x|)
\]
This can also be written as $\frac{H(t)}{4 \pi c^2|x|} \delta(t-|x| / c)$.

These derivations show how Fourier transforms can be systematically used to find fundamental solutions. The main challenge often lies in evaluating the resulting inverse Fourier transform integrals, which can be quite involved depending on the dimension and the complexity of the function in Fourier space.
