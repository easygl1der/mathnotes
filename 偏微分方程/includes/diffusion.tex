\section{Diffusion}

See Partial Differential Equations in Action, Complements and Exercises.

The \textbf{diffusion equation} (heat equation) is
\[
u_{t}-\underbrace{ D }_{ \text{const.} }\Delta u=f
\]
defined on a cylindrical domain $Q_{T}=\Omega \times(0,T)$, where $\Omega$ is a domain\footnote{connected, open subset} of $\mathbb{R}^{n}$, and the Laplacian $\Delta$ is taken w.r.t. the spatial variables $\mathbf{x}$ only.

\subsection{Cauchy-Dirichlet}

\begin{figure}[H]
\centering
\includegraphics[width=\textwidth]{Diffusion-2025050311.png}
% \caption{}
\label{}
\end{figure}
\begin{figure}[H]
\centering
\includegraphics[width=\textwidth]{1-Diffusion-2025050311.png}
% \caption{}
\label{}
\end{figure}
\begin{figure}[H]
\centering
\includegraphics[width=\textwidth]{2-Diffusion-2025050311.png}
% \caption{}
\label{}
\end{figure}
\begin{figure}[H]
\centering
\includegraphics[width=\textwidth]{3-Diffusion-2025050311.png}
% \caption{}
\label{}
\end{figure}

为了保证一个函数的傅里叶级数收敛,该函数需要满足一定的条件。以下是一些常见的条件:

\begin{enumerate}
	\item \textbf{狄利克雷条件(Dirichlet Conditions):}
一个周期为 $2\pi$ 的函数 $f(x)$,如果满足以下条件,则其傅里叶级数收敛:
	\begin{itemize}
		\item 在任何有限区间内,$f(x)$ 只有有限个不连续点。
		\item 在任何有限区间内,$f(x)$ 只有有限个极大值和极小值。
		\item $f(x)$ 在任何有限区间内是绝对可积的,即 $\int_{-\pi}^{\pi} |f(x)| dx < \infty$。
	\end{itemize}
如果 $f(x)$ 在 $x$ 处连续,则傅里叶级数收敛到 $f(x)$;如果 $f(x)$ 在 $x$ 处有跳跃不连续点,则傅里叶级数收敛到 $\frac{f(x^+) + f(x^-)}{2}$,其中 $f(x^+)$ 和 $f(x^-)$ 分别表示 $f(x)$ 在 $x$ 处的右极限和左极限。
	\item \textbf{更强的条件:}
	\begin{itemize}
		\item 如果 $f(x)$ 是连续的,并且 $f'(x)$ 存在且连续,那么 $f(x)$ 的傅里叶级数一致收敛到 $f(x)$。
		\item 如果 $f(x)$ 是平方可积的,即 $\int_{-\pi}^{\pi} |f(x)|^2 dx < \infty$,那么 $f(x)$ 的傅里叶级数在均方意义下收敛到 $f(x)$。
	\end{itemize}
\end{enumerate}

总的来说,狄利克雷条件是最经典和常用的判断傅里叶级数收敛的条件。

\subsection{Cauchy-Neumann}

\begin{figure}[H]
\centering
\includegraphics[width=\textwidth]{5-Diffusion-2025050311.png}
% \caption{}
\label{}
\end{figure}
\begin{figure}[H]
\centering
\includegraphics[width=\textwidth]{6-Diffusion-2025050311.png}
% \caption{}
\label{}
\end{figure}
\begin{figure}[H]
\centering
\includegraphics[width=\textwidth]{7-Diffusion-2025050311.png}
% \caption{}
\label{}
\end{figure}
\begin{figure}[H]
\centering
\includegraphics[width=\textwidth]{8-Diffusion-2025050311.png}
% \caption{}
\label{}
\end{figure}

\begin{definition}[Bessel's equality]
令$H$是一个希尔伯特空间,并设$\left\{e_{k}\right\}_{k=1}^{\infty}$是$H$的一个正交基。那么,对于$H$中的任何$x$,\textbf{Bessel 不等式}成立
\[
\|x\|^{2}=\sum_{k=1}^{\infty}\left|\left\langle x, e_{k}\right\rangle\right|^{2}
\]
\end{definition}
\begin{definition}[Bessel's equality]
Let $f$ be a function on $[-\pi, \pi]$ such that $\int_{-\pi}^\pi |f(x)|^2 dx < \infty$. Then
\[
\frac{1}{2 \pi} \int_{-\pi}^\pi |f(x)|^2 dx = |a_0|^2 + \sum_{n=1}^\infty (|a_n|^2 + |b_n|^2)
\]where
\[
a_n = \frac{1}{\pi} \int_{-\pi}^\pi f(x) \cos(nx) dx, \quad b_n = \frac{1}{\pi} \int_{-\pi}^\pi f(x) \sin(nx) dx
\]
\end{definition}
\begin{theorem}[Poincaré's inequality]
设 $\Omega \subset \mathbb{R}^n$ 是一个有界开集。存在一个常数 $C$,使得对于所有 $u \in H_0^1(\Omega)$,有
\[
\|u\|_{L^2(\Omega)} \leq C\|\nabla u\|_{L^2(\Omega)}.
\]
\end{theorem}
\begin{proof}
设 $\Omega \subset \mathbb{R}^n$ 是一个有界开集。那么存在一个 $R > 0$,使得 $\Omega \subset B(0, R)$。设 $u \in H_0^1(\Omega)$。我们可以将 $u$ 延拓到 $B(0, R)$ 上,使得 $u \in H_0^1(B(0, R))$。那么 $u$ 在 $\partial B(0, R)$ 上为 0。因此,我们可以将 $u$ 延拓到 $\mathbb{R}^n$ 上,使得 $u \in H^1(\mathbb{R}^n)$。

对于 $x \in \mathbb{R}^n$,有
\[
u(x) = \int_{-\infty}^{x_1} \frac{\partial u}{\partial x_1}(y_1, x_2, \ldots, x_n) dy_1.
\]
因此,
\[
|u(x)| \leq \int_{-\infty}^{\infty} \left|\frac{\partial u}{\partial x_1}(y_1, x_2, \ldots, x_n)\right| dy_1.
\]
利用 Cauchy-Schwarz 不等式,我们有
\[
|u(x)|^2 \leq \left(\int_{-\infty}^{\infty} \left|\frac{\partial u}{\partial x_1}(y_1, x_2, \ldots, x_n)\right|^2 dy_1\right) \left(\int_{-\infty}^{\infty} 1 dy_1\right).
\]
由于 $\Omega$ 是有界的,所以 $\int_{-\infty}^{\infty} 1 dy_1 < \infty$。因此,
\[
|u(x)|^2 \leq C \int_{-\infty}^{\infty} \left|\frac{\partial u}{\partial x_1}(y_1, x_2, \ldots, x_n)\right|^2 dy_1.
\]
在 $\Omega$ 上积分,我们有
\[
\int_{\Omega} |u(x)|^2 dx \leq C \int_{\Omega} \int_{-\infty}^{\infty} \left|\frac{\partial u}{\partial x_1}(y_1, x_2, \ldots, x_n)\right|^2 dy_1 dx.
\]
因此,
\[
\|u\|_{L^2(\Omega)}^2 \leq C \int_{\Omega} |\nabla u(x)|^2 dx = C \|\nabla u\|_{L^2(\Omega)}^2.
\]
取平方根,我们有
\[
\|u\|_{L^2(\Omega)} \leq C \|\nabla u\|_{L^2(\Omega)}.
\]
\end{proof}

\begin{theorem}[Poincaré不等式(n=1)]
设$u \in H_{0}^{1}(a, b)$,其中$-\infty<a<b<\infty$。那么
\[
\|u\|_{L^{2}(a, b)} \leq \frac{b-a}{\pi}\left\|u^{\prime}\right\|_{L^{2}(a, b)}
\]
\end{theorem}
\begin{proof}
通过延拓$u$为$[-\ell, \ell]$上的奇函数,其中$\ell = (b-a)/2$,我们有$u(-\ell)=u(\ell)=0$。因此,$u$的傅里叶级数为
\[
u(x)=\sum_{n=1}^{\infty} a_{n} \sin \left(\frac{n \pi x}{\ell}\right)
\]
其中
\[
a_{n}=\frac{2}{\ell} \int_{0}^{\ell} u(x) \sin \left(\frac{n \pi x}{\ell}\right) d x .
\]
Parseval恒等式给出
\[
\int_{-\ell}^{\ell}|u(x)|^{2} d x=\frac{\ell}{2} \sum_{n=1}^{\infty}\left|a_{n}\right|^{2} .
\]
此外,形式上的导数为
\[
u^{\prime}(x)=\sum_{n=1}^{\infty} a_{n} \frac{n \pi}{\ell} \cos \left(\frac{n \pi x}{\ell}\right)
\]
因此
\[
\int_{-\ell}^{\ell}\left|u^{\prime}(x)\right|^{2} d x=\frac{\ell}{2} \sum_{n=1}^{\infty}\left|a_{n}\right|^{2} \frac{n^{2} \pi^{2}}{\ell^{2}} .
\]
由于$n \geq 1$,我们有
\[
\int_{-\ell}^{\ell}|u(x)|^{2} d x \leq \frac{\ell^{2}}{\pi^{2}} \int_{-\ell}^{\ell}\left|u^{\prime}(x)\right|^{2} d x
\]
或者
\[
\|u\|_{L^{2}(-\ell, \ell)} \leq \frac{\ell}{\pi}\left\|u^{\prime}\right\|_{L^{2}(-\ell, \ell)} .
\]
由于$u$在$(a,b)$外为零,且$\ell = (b-a)/2$,我们得到
\[
\|u\|_{L^{2}(a, b)} \leq \frac{b-a}{2 \pi}\left\|u^{\prime}\right\|_{L^{2}(a, b)} .
\]
\end{proof}

\subsection{Cauchy-Neumann; non-homogeneous equation}

\begin{figure}[H]
\centering
\includegraphics[width=\textwidth]{3-Diffusion-2025050312.png}
% \caption{}
\label{}
\end{figure}
\begin{figure}[H]
\centering
\includegraphics[width=\textwidth]{4-Diffusion-2025050312.png}
% \caption{}
\label{}
\end{figure}

\begin{figure}[H]
\centering
\includegraphics[width=\textwidth]{5-Diffusion-2025050312.png}
% \caption{}
\label{}
\end{figure}

\subsection{Non-homogeneous Neumann}

\begin{figure}[H]
\centering
\includegraphics[width=\textwidth]{Diffusion-2025050312.png}
% \caption{}
\label{}
\end{figure}
\begin{figure}[H]
\centering
\includegraphics[width=\textwidth]{1-Diffusion-2025050312.png}
% \caption{}
\label{}
\end{figure}
\begin{figure}[H]
\centering
\includegraphics[width=\textwidth]{2-Diffusion-2025050312.png}
% \caption{}
\label{}
\end{figure}

\subsection{Maximum principle}

\begin{figure}[H]
\centering
\includegraphics[width=\textwidth]{Diffusion-2025050315.png}
% \caption{}
\label{}
\end{figure}

If $u_{t}-D\Delta u=0$ on $\Omega$, then $u$ reaches its maximum and minimum on the boundary $\partial_{p}\Omega$.

\begin{figure}[H]
\centering
\includegraphics[width=\textwidth]{1-Diffusion-2025050315.png}
% \caption{}
\label{}
\end{figure}
\begin{figure}[H]
\centering
\includegraphics[width=\textwidth]{2-Diffusion-2025050315.png}
% \caption{}
\label{}
\end{figure}

\subsection{Asumptotic behaviour}

\begin{figure}[H]
\centering
\includegraphics[width=\textwidth]{3-Diffusion-2025050315.png}
% \caption{}
\label{}
\end{figure}
\begin{figure}[H]
\centering
\includegraphics[width=\textwidth]{4-Diffusion-2025050315.png}
% \caption{}
\label{}
\end{figure}

\subsection{Applying the notion of fundamental solution}

\begin{figure}[H]
\centering
\includegraphics[width=\textwidth]{6-Diffusion-2025050315.png}
% \caption{}
\label{}
\end{figure}
\begin{figure}[H]
\centering
\includegraphics[width=\textwidth]{7-Diffusion-2025050315.png}
% \caption{}
\label{}
\end{figure}

\subsection{Problems on the half-line; reflection method}

\begin{figure}[H]
\centering
\includegraphics[width=\textwidth]{6-Diffusion-2025050316.png}
% \caption{}
\label{}
\end{figure}
\begin{figure}[H]
\centering
\includegraphics[width=\textwidth]{7-Diffusion-2025050316.png}
% \caption{}
\label{}
\end{figure}
\begin{figure}[H]
\centering
\includegraphics[width=\textwidth]{8-Diffusion-2025050316.png}
% \caption{}
\label{}
\end{figure}
\begin{figure}[H]
\centering
\includegraphics[width=\textwidth]{9-Diffusion-2025050316.png}
% \caption{}
\label{}
\end{figure}

\subsection{Use of Fourier and Laplace transforms}

\begin{figure}[H]
\centering
\includegraphics[width=\textwidth]{5-Diffusion-2025050315.png}
% \caption{}
\label{}
\end{figure}

\subsection{Fourier sine transform}

\begin{figure}[H]
\centering
\includegraphics[width=\textwidth]{8-Diffusion-2025050315.png}
% \caption{}
\label{}
\end{figure}
\begin{figure}[H]
\centering
\includegraphics[width=\textwidth]{10-Diffusion-2025050315.png}
% \caption{}
\label{}
\end{figure}

\subsection{Problems in dimension higher than one}

We separate the variables twice.

\begin{figure}[H]
\centering
\includegraphics[width=\textwidth]{1-Diffusion-2025050316.png}
% \caption{}
\label{}
\end{figure}
\begin{figure}[H]
\centering
\includegraphics[width=\textwidth]{2-Diffusion-2025050316.png}
% \caption{}
\label{}
\end{figure}
\begin{figure}[H]
\centering
\includegraphics[width=\textwidth]{3-Diffusion-2025050316.png}
% \caption{}
\label{}
\end{figure}

\subsection{Fourier transform on the half-plane}

\begin{figure}[H]
\centering
\includegraphics[width=\textwidth]{4-Diffusion-2025050316.png}
% \caption{}
\label{}
\end{figure}
\begin{figure}[H]
\centering
\includegraphics[width=\textwidth]{5-Diffusion-2025050316.png}
% \caption{}
\label{}
\end{figure}
