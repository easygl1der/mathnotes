\section{Poincare Inequality}

庞加莱不等式(Poincaré Inequality)通常叙述于有界区域上。对于整个 $\mathbb{R}^n$ 空间,标准形式的庞加莱不等式(即 $\|u\|_{L^p} \le C \|\nabla u\|_{L^p}$)通常不成立,除非对函数 $u$ 施加特定条件,例如具有紧支撑(compact support)或者属于特定的加权 Sobolev 空间。

一个在 $\mathbb{R}^n$ 上下文中最直接理解的庞加莱不等式版本是针对具有紧支撑的函数。我们将证明这个版本。

\begin{theorem}[$\mathbb{R}^n$ 上的庞加莱不等式 - 紧支撑函数版本]
设 $u \in C_c^1(\mathbb{R}^n)$,即 $u$ 是 $\mathbb{R}^n$ 上具有紧支撑的连续可微函数(也可以推广到 Sobolev 空间 $W_c^{1,p}(\mathbb{R}^n)$)。令 $\text{supp}(u)$ 表示 $u$ 的支撑集。则存在一个常数 $C > 0$,使得
\[
\|u\|_{L^p(\mathbb{R}^n)} \le C \|\nabla u\|_{L^p(\mathbb{R}^n)}
\]
其中常数 $C$ 依赖于 $u$ 的支撑集的大小(例如,其直径或包含该支撑集的某个立方体的边长)。
\end{theorem}
为了简便,我们考虑 $p \ge 1$ 的情况。
因为 $u$ 具有紧支撑,所以存在一个足够大的开立方体 $Q = (-M, M)^n \subset \mathbb{R}^n$,使得 $\text{supp}(u) \subset Q$。这意味着 $u$ 在 $Q$ 的边界 $\partial Q$ 上为零。立方体的边长为 $L=2M$。

对于任意 $x = (x_1, x_2, \dots, x_n) \in Q$,由于 $u(-M, x_2, \dots, x_n) = 0$(因为点 $(-M, x_2, \dots, x_n)$ 在 $Q$ 的边界上,而 $u$ 在边界外为零),我们可以沿 $x_1$ 方向使用微积分基本定理:
\[
u(x_1, x_2, \dots, x_n) = \int_{-M}^{x_1} \frac{\partial u}{\partial \xi_1}(\xi_1, x_2, \dots, x_n) \, d\xi_1
\]
取绝对值:
\[
|u(x_1, x_2, \dots, x_n)| \le \int_{-M}^{x_1} \left|\frac{\partial u}{\partial \xi_1}(\xi_1, x_2, \dots, x_n)\right| \, d\xi_1
\]
由于积分区间 $[-M, x_1]$ 包含于 $[-M, M]$,并且被积函数非负,我们可以扩大积分区间:
\[
|u(x)| \le \int_{-M}^{M} \left|\frac{\partial u}{\partial \xi_1}(\xi_1, x_2, \dots, x_n)\right| \, d\xi_1
\]
现在,我们对 $|u(x)|^p$ 在立方体 $Q$ 上积分。首先固定 $x_2, \dots, x_n$,对 $x_1$ 积分:
\[
\int_{-M}^{M} |u(x_1, x_2, \dots, x_n)|^p \, dx_1 \le \int_{-M}^{M} \left( \int_{-M}^{M} \left|\frac{\partial u}{\partial \xi_1}(\xi_1, x_2, \dots, x_n)\right| \, d\xi_1 \right)^p \, dx_1
\]
令 $K(x_2, \dots, x_n) = \int_{-M}^{M} \left|\frac{\partial u}{\partial \xi_1}(\xi_1, x_2, \dots, x_n)\right| \, d\xi_1$。这是一个关于 $x_1$ 的常数。
所以,
\[
\int_{-M}^{M} |u(x_1, x_2, \dots, x_n)|^p \, dx_1 \le K(x_2, \dots, x_n)^p \int_{-M}^{M} \, dx_1 = (2M) \cdot K(x_2, \dots, x_n)^p
\]
现在应用 Jensen 不等式(或 Hölder 不等式:$\int f \cdot 1 \le (\int f^p)^{1/p} (\int 1^q)^{1/q}$,这里 $f = |\frac{\partial u}{\partial \xi_1}|$,$1/p + 1/q = 1$):
对于 $K(x_2, \dots, x_n) = \int_{-M}^{M} \left|\frac{\partial u}{\partial \xi_1}(\xi_1, x_2, \dots, x_n)\right| \, d\xi_1$,我们有
\[
K(x_2, \dots, x_n)^p = \left( \int_{-M}^{M} \left|\frac{\partial u}{\partial \xi_1}(\xi_1, x_2, \dots, x_n)\right| \cdot 1 \, d\xi_1 \right)^p
\]
\[
\le \left( \left( \int_{-M}^{M} \left|\frac{\partial u}{\partial \xi_1}(\xi_1, x_2, \dots, x_n)\right|^p \, d\xi_1 \right)^{1/p} \left( \int_{-M}^{M} 1^q \, d\xi_1 \right)^{1/q} \right)^p
\]
\[
= \left( \int_{-M}^{M} \left|\frac{\partial u}{\partial \xi_1}(\xi_1, x_2, \dots, x_n)\right|^p \, d\xi_1 \right) \cdot (2M)^{p/q}
\]
由于 $1/p + 1/q = 1$, $p/q = p(1-1/p) = p-1$。
所以,
\[
K(x_2, \dots, x_n)^p \le (2M)^{p-1} \int_{-M}^{M} \left|\frac{\partial u}{\partial \xi_1}(\xi_1, x_2, \dots, x_n)\right|^p \, d\xi_1
\]
代回到关于 $x_1$ 的积分不等式:
\[
\int_{-M}^{M} |u(x_1, x_2, \dots, x_n)|^p \, dx_1 \le (2M) \cdot (2M)^{p-1} \int_{-M}^{M} \left|\frac{\partial u}{\partial \xi_1}(\xi_1, x_2, \dots, x_n)\right|^p \, d\xi_1
\]
\[
= (2M)^p \int_{-M}^{M} \left|\frac{\partial u}{\partial \xi_1}(\xi_1, x_2, \dots, x_n)\right|^p \, d\xi_1
\]
现在,我们在 $Q$ 的其余维度上积分 (即对 $dx_2 \dots dx_n$ 积分):
\[
\int_{(-M,M)^{n-1}} \left( \int_{-M}^{M} |u(x_1, \dots, x_n)|^p \, dx_1 \right) \, dx_2 \dots dx_n
\]
\[
\le \int_{(-M,M)^{n-1}} (2M)^p \left( \int_{-M}^{M} \left|\frac{\partial u}{\partial \xi_1}(\xi_1, x_2, \dots, x_n)\right|^p \, d\xi_1 \right) \, dx_2 \dots dx_n
\]
这给出:
\[
\int_Q |u(x)|^p \, dx \le (2M)^p \int_Q \left|\frac{\partial u}{\partial x_1}(x)\right|^p \, dx
\]
由于 $\left|\frac{\partial u}{\partial x_1}(x)\right|^p \le |\nabla u(x)|^p = \sum_{i=1}^n \left|\frac{\partial u}{\partial x_i}(x)\right|^p$,我们可以得到:
\[
\int_Q |u(x)|^p \, dx \le (2M)^p \int_Q |\nabla u(x)|^p \, dx
\]
因为 $\text{supp}(u) \subset Q$,所以 $u$ 在 $Q$ 之外为零,$\nabla u$ 也在 $Q$ 之外(几乎处处)为零。因此,积分可以扩展到整个 $\mathbb{R}^n$:
\[
\|u\|_{L^p(\mathbb{R}^n)}^p \le (2M)^p \|\nabla u\|_{L^p(\mathbb{R}^n)}^p
\]
取 $p$ 次根:
\[
\|u\|_{L^p(\mathbb{R}^n)} \le (2M) \|\nabla u\|_{L^p(\mathbb{R}^n)}
\]
这里的常数 $C = 2M = L$ 是包含 $u$ 支撑集的立方体的边长。更一般地,常数 $C$ 可以取为 $u$ 支撑集在某个方向上的最大长度,或者其直径。

重要说明:

\begin{enumerate}
	\item \textbf{常数的依赖性:} 这个版本的庞加莱不等式的常数 $C$ 依赖于函数 $u$ 支撑集的大小。对于支撑集越来越大的函数序列,这个常数会趋于无穷,这与有界区域上的庞加莱不等式常数仅依赖于区域的几何性质不同。
	\item \textbf{一般 $\mathbb{R}^n$ 情况:} 如果不对 $u$ 的行为(如紧支撑或在无穷远处衰减)做任何假设,标准形式的庞加莱不等式在 $\mathbb{R}^n$ 上是不成立的。例如,一个非零常数函数 $u(x)=c \neq 0$ 的梯度为零,但其 $L^p(\mathbb{R}^n)$ 范数为无穷大。
	\item \textbf{相关不等式:} 在 $\mathbb{R}^n$ 上,与函数及其梯度相关且不依赖于支撑集大小(或具有不同类型依赖)的更典型不等式是 Sobolev 不等式(例如 Gagliardo-Nirenberg-Sobolev 不等式),它通常将 $W^{1,p}(\mathbb{R}^n)$ 空间嵌入到某个 $L^q(\mathbb{R}^n)$ 空间中,其中 $q$ 可能不等于 $p$。
\end{enumerate}

上面证明的是在 $\mathbb{R}^n$ 上针对紧支撑函数最直接和常见的“庞加莱不等式”形式。
