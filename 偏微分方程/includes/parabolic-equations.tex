\section{Appendix}

Some results about the integration and differentiation of Banach-space valued functions of a single variable.
Vector-valued integrals of integrable functions have similar properties, often with similar proofs, to scalar-valued $\displaystyle L^{1}$ -integrals.
The existence of different topologies (such as the weak and strong topologies) in the range space of integrals that take values in an \textbf{infinite-dimensional} Banach space introduces significant new issues that do not arise in the scalar-valued case.

Suppose $X$ a real Banach space with $\lVert \cdot \rVert$ and dual space $X'$.

\subsection{Vector-Valued functions}

\subsubsection{Measurability}

\begin{definition}
\begin{figure}[H]
\centering
\includegraphics[width=\textwidth]{Parabolic equations-20250306.png}
% \caption{}
\label{}
\end{figure}
\end{definition}
\begin{definition}[strongly measurable]
\begin{figure}[H]
\centering
\includegraphics[width=\textwidth]{1-Parabolic equations-20250306.png}
% \caption{}
\label{}
\end{figure}
\end{definition}
Measurability is preserved under natural operations on functions.

\begin{figure}[H]
\centering
\includegraphics[width=\textwidth]{2-Parabolic equations-20250306.png}
% \caption{}
\label{}
\end{figure}

\begin{definition}[weakly measurable]
A function $\displaystyle f:(0,T)\to X$ is said to be \textbf{weakly measurable} if the real-valued function $\displaystyle \left< \omega,f \right>:(0,T)\to \mathbb{R}$ is measurable for every $\displaystyle \omega\in X'$.
\end{definition}
This amounts to a 'coordinatewise' defintion of measurablility, in which we represent a vector-valued function by its ral-valued coordinate functions.
For finite-dimensional, or separable, Banach spaces these definitions coincide, but for non-separable sapces a weakly measurable function need not be strongly measurable.
The relationship between weak and strong measurability is given by the following Pettis theorem.

\begin{definition}[almost separably valued]
\begin{figure}[H]
\centering
\includegraphics[width=\textwidth]{4-Parabolic equations-20250306.png}
% \caption{}
\label{}
\end{figure}
\end{definition}
\begin{theorem}[Pettis theorem]
A function $\displaystyle f:(0,T)\to X$ is strongly measurable iff it is weakly measurable and almost separably valued.
\end{theorem}
\begin{definition}[weak continuous]
\begin{figure}[H]
\centering
\includegraphics[width=\textwidth]{5-Parabolic equations-20250306.png}
% \caption{}
\label{}
\end{figure}
\end{definition}
Since a continuous function is measurable, every almost separably valued, weakly continuous function is strongly measurable.

\begin{example}[weak but not strong measurable]
\begin{figure}[H]
\centering
\includegraphics[width=\textwidth]{Parabolic equations-20250306-154640.png}
% \caption{}
\label{}
\end{figure}
\end{example}
\begin{example}[$L^{2}$ separable but $L^{\infty}$ not]
\begin{figure}[H]
\centering
\includegraphics[width=\textwidth]{Parabolic equations-20250306-154653.png}
% \caption{}
\label{}
\end{figure}
\end{example}
\subsubsection{Integration}

The definition of the Lebesgue integral as a supremum of integrals of simple functions does not extend directly to vector-valued integrals because it uses the ordering properties of $\mathbb{R}$ in and essential way.

One can use duality to define $X$ -valued integrals $\int f\,dt$ in terms of the corresponding real-valued integrals $\int\left< \omega,f \right>\, dt$ where $\omega\in X'$, but we will not consider such weak definitions of an integral here.

Instead, we define the integral of vector-valued functions by completing the space of simple functinos with respect to the $L^{1}(0,T;X)$ -norm. The resulting integral is called the Bochner integral, and its properties are similar to those of the Lebesgue integral of integrable real-valued functions.

\begin{definition}[simple function]
\begin{figure}[H]
\centering
\includegraphics[width=\textwidth]{Parabolic equations-20250306-155025.png}
% \caption{}
\label{}
\end{figure}
\end{definition}
\begin{definition}[Bochner integral]
\begin{figure}[H]
\centering
\includegraphics[width=\textwidth]{Parabolic equations-20250306-154834.png}
% \caption{}
\label{}
\end{figure}
\end{definition}
The value of the Bochner integral of $f$ is independent of the sequence $\left\{f_n\right\}$ of approximating simple functions, and
\[
\left\|\int_0^T f d t\right\| \leq \int_0^T\|f\| d t
\]
Moreover, if $A: X \rightarrow Y$ is a bounded linear operator between Banach spaces $X, Y$ and $f:(0, T) \rightarrow X$ is integrable, then $A f:(0, T) \rightarrow Y$ is integrable and
\[
A\left(\int_0^T f d t\right)=\int_0^T A f d t
\]
More generally, this equality holds whenever $A: \mathcal{D}(A) \subset X \rightarrow Y$ is a \textbf{closed} linear operator and $f:(0, T) \rightarrow \mathcal{D}(A)$, in which case $\int_0^T f d t \in \mathcal{D}(A)$.

\begin{definition}[closed linear operater]
\begin{figure}[H]
\centering
\includegraphics[width=\textwidth]{Parabolic equations-20250306-155321.png}
% \caption{}
\label{}
\end{figure}
\end{definition}
\begin{example}[$\left< \omega,\cdot \right>$ as an operator]
\begin{figure}[H]
\centering
\includegraphics[width=\textwidth]{Parabolic equations-20250306-155441.png}
% \caption{}
\label{}
\end{figure}
\end{example}
\begin{theorem}
\begin{figure}[H]
\centering
\includegraphics[width=\textwidth]{Parabolic equations-20250306-155608.png}
% \caption{}
\label{}
\end{figure}
\end{theorem}
The DCT for Bochner integrals with proof same as the scalar-valued case omitted.

\begin{theorem}
\begin{figure}[H]
\centering
\includegraphics[width=\textwidth]{Parabolic equations-20250306-155856.png}
% \caption{}
\label{}
\end{figure}
\end{theorem}
The definition and properties of $L^{p}$ -spaces of $X$ -valued functions are analogous to the case of real-valued functions.

\begin{definition}[$L^{p}(0,T;X)$]
\begin{figure}[H]
\centering
\includegraphics[width=\textwidth]{Parabolic equations-20250306-160128.png}
% \caption{}
\label{}
\end{figure}
\end{definition}
As usual, functions equal pointwise a.e. are regarded equivalent.

\begin{theorem}
If $X$ is a Banach space and $1\leq p\leq \infty$ then $L^{p}(0,T;X)$ is a Banach space.
\end{theorem}
Simple functions of the form $f(t)=\sum_{i=1}^{n}c_i\chi_{E_i}(t)$ are dense in $L^{p}(0,T;X)$. By mollifying these functions with respect to $t$, we get the following density result.

\begin{theorem}
\begin{figure}[H]
\centering
\includegraphics[width=\textwidth]{Parabolic equations-20250306-160610.png}
% \caption{}
\label{}
\end{figure}
\end{theorem}
\begin{theorem}
\begin{figure}[H]
\centering
\includegraphics[width=\textwidth]{Parabolic equations-20250306-160834.png}
% \caption{}
\label{}
\end{figure}
\end{theorem}
The proof is more complicated than in the scalar case and some condition on $X$ is required. Reflexivity is sufficient.

\subsubsection{Differentiability}

\begin{definition}[strong continuous and strong differentiable]
\begin{figure}[H]
\centering
\includegraphics[width=\textwidth]{Parabolic equations-20250306-161157.png}
% \caption{}
\label{}
\end{figure}
\end{definition}